\chapter{Morphologische und syntaktische Schreibprinzipien}

\label{sec:andereschrift}

\section{Wortbezogene Schreibungen}

\subsection{Spatien}

\label{sec:spatien}
\index{Spatium}

In diesem Kapitel geht es sehr gestrafft und überblickshaft um Schreibprinzipien, die zwischen Wort- und Satzebene operieren.
Das wahrscheinlich wichtigste Prinzip, das man ggf.\ leicht aus dem Auge verliert, ist, dass wir syntaktische Wörter (Wortformen) in der Schrift durch Leerzeichen (Spatien) trennen.

\Satz{Prinzip der Spatienschreibung}{\label{satz:spatien}
Syntaktische Wörter werden durch Spatien getrennt.
\index{Schreibprinzip!Spatienschreibung}
\index{Wort!graphematisch}
}

Dieses Prinzip galt historisch im Deutschen nicht immer, und auch viele moderne Sprachen kommen ohne Worttrennung durch Spatien aus (\zB Chinesisch und Japanisch).
Ein Beispiel wie (\ref{ex:msschr001}) zeigt, dass beim Verzicht auf die Spatien die Lesbarkeit nicht völlig zusammenbricht.
Trotzdem ist das Spatium ohne Zweifel eine wichtige Lesehilfe.

\begin{exe}
  \ex{\label{ex:msschr001} SiekönneneinenSatzohneSpatienwahrscheinlichproblemloslesen.}
\end{exe}

Offensichtlich ist, dass Elemente der Wortbildung (\ref{ex:msschr002a}) und Flexion (\ref{ex:msschr002b}) nicht getrennt werden.

\begin{exe}
  \ex\label{ex:msschr002} 
  \begin{xlist}
    \ex[*]{\label{ex:msschr002a} Vanessa hat Gelegen heit, die Schreib ung von Wörtern gründ lich zu unter suchen.}
    \ex[*]{\label{ex:msschr002b} Oma koch t der ausgekühlt en Vanessa ein en heiß en Tee.}
  \end{xlist}
\end{exe}

Andererseits werden Wörter nicht einfach so zusammengeschrieben, \zB weil sie zusammen eine analytische Tempusform ergeben (\ref{ex:msschr003a}) oder eine Phrase bilden (\ref{ex:msschr003b}).

\begin{exe}
  \ex\label{ex:msschr003} 
  \begin{xlist}
    \ex[*]{\label{ex:msschr003a} Vanessa istgeritten.}
    \ex[*]{\label{ex:msschr003b} Vanessa reitet imwald.}
  \end{xlist}
\end{exe}

Die Wörter \textit{ist} und \textit{geritten} behandeln wir \zB deshalb als getrennte syntaktische Wörter, weil sie durch einfache Umstellungen im Satz voneinander getrennt werden können.
Außerdem haben beide eine klar erkennbare Valenz (\textit{ist} verlangt ein Verb im dritten Status, \textit{geritten} eine NP im Nominativ).
Vor allem kann aber \textit{geritten} in diesem Satz durch beliebige andere intransitive Verben ersetzt werden, und es hat hier seine ganz normale Bedeutung im Sinne von \textit{reiten}.
Die Wörter verhalten sich erkennbar autonom und haben je eigene syntaktische Eigenschaften, weswegen wir sie als syntaktische Wörter bezeichnen, und weswegen sie eben auch nicht zusammengeschrieben werden.%
\footnote{Man könnte zusätzlich noch die Akzentverhältnisse bemühen, da Wörter typischerweise genau einen Hauptakzent haben.}

Das Prinzip aus Satz~\ref{satz:spatien} ist einfach, klar und scheinbar unverfänglich.
Trotzdem gibt es typische Schwierigkeiten in der Orthographievermittlung und der Normierung bezüglich der Zusammenschreibung und Getrenntschreibung.
Abfolgen von Wörtern, die sich im Gebrauch stark aneinander binden und dabei ihren semantischen Gehalt teilweise verlieren bzw.\ zu grammatischen Funktionswörtern werden, tendieren nämlich dazu, zu einfachen Wörtern zu verschmelzen.
Dieser Prozess wird Univerbierung genannt.
Typisch für das Deutsche sind \zB Bildungen von sekundären Präpositionen wie \textit{anstatt} oder \textit{zulasten}.
Typische Zweifelsfälle entstehen auch bei der potentiellen Bildung neuer Verbalpartikeln aus unabhängigen Wörtern wie in \textit{weichklopfen} oder \textit{schlechtreden} und im Bereich der Adjektive durch Verschmelzung mit einer vorangestellten Partikel (im Grunde eine Art von Komposition) wie in \textit{tiefrot} oder \textit{halbfest}.
Diesen Bildungen (bis auf die sekundären Präpositionen) ist sogar eine gewisse Produktivität eigen, und Sprecher können damit solche Wörter ggf.\ ad hoc bilden.
Die Normierung setzt daher viel zu spät an, wenn sie sich nur auf die Schreibung bezieht.
Vielmehr wird Schreibern mit einer Regelung, die \zB ausschließlich \textit{zu Lasten} oder \textit{zulasten} erlaubt, eine bestimmte grammatische Form vorgeschrieben.
Auch wenn man für die verschiedenen Typen von Univerbierungen gute heuristische Tests ansetzen kann, die die eine oder andere Variante sinnvoller erscheinen lassen, so wäre doch die orthographische Freigabe beider Varianten sinnvoll, da Sprecher und Schreiber damit (bewusst oder unbewusst) eine der beiden systematisch gesehen völlig adäquaten Varianten wählen könnten.

\subsection{Wortklassen}

\label{sec:wortklassschreib}
\index{Wortklasse!Schreibung}

Nachdem wir jetzt festgestellt haben, dass Wörter in der Schreibung am besten dadurch identifiziert werden können, dass sie durch Spatien getrennt sind, werden ausgewählte Phänomene auf Wortebene betrachtet.
Ein wichtiges Merkmal jedes Wortes ist seine Klassenzugehörigkeit.
Im Normalfall gibt es keine direkte Markierung der Wortklasse in der Schreibung.
Natürlich sind Schreibungen bestimmter Wörter gut geeignet, die Klassenzugehörigkeit der Wörter anzuzeigen, wie \zB Wörter, die auf \textit{ig} oder \textit{keit} enden.
Das kommt natürlich nur auf Umwegen zustande:
Die Buchstaben kodieren Segmente, die zusammen ein Wortbildungssuffix ergeben, welches wiederum ein Indikator für eine Wortklasse ist, weil es ein wortartveränderndes Suffix ist.
Das hat mit einer spezifischen graphematischen Markierung nichts zu tun.

Neben vereinzelten Markierungen von Wortklassenunterschieden durch Varianten der Schreibung (Komplementierer \textit{dass} und Artikel bzw.\ Pronomen \textit{das}) gibt es nur eine für das Deutsche charakteristische systematische Wortklassenmarkierung, nämlich die Großschreibung von Eigennamen und Substantiven.
Wie in vielen Sprachen, die sich der Lateinschrift bedienen (\zB alle germanischen und romanischen Sprachen) werden Eigennamen immer (in allen Position) großgeschrieben.
Die Substantivgroßschreibung ist eine Besonderheit des Deutschen.
Wie immer geht es hier im weiteren Verlauf überhaupt nicht darum, was die gültige Orthographieregelung zu sagen hat, sondern ausschließlich um die zugrundeliegenden Prinzipien.

\Satz{Positionsunabhängige Majuskelschreibung}{
\label{satz:grosschreib}
Eigennamen und Substantive werden unabhängig von ihrer Position immer mit einleitender Majuskel geschrieben.
\index{Majuskel}
\index{Substantiv!Großschreibung}
}

Der Kern der Substantivgrossschreibung ist völlig unproblematisch.
Was ein Substantiv ist, wird immer (auch mitten im Satz oder in der Nennform, in Listen usw.) großgeschrieben.
Nach Filter~\ref{wfilt:subst} auf S.~\pageref{wfilt:subst} wissen wir genau, was ein Substantiv ist, nämlich ein Nomen mit festem Genus.
Das beste diagnostische Kriterium dafür, ob die Substantivgroßschreibung greifen sollte oder nicht, ist also, ob ein genusspezifischer Artikel vor einem Wort steht oder stehen kann.

\index{Konversion}
\index{Substantivierung}
Problemfälle treten vor allem bei Konversionen von Adjektiven auf, vgl.\ (\ref{ex:msschr006}).

\begin{exe}
  \ex\label{ex:msschr006} 
  \begin{xlist}
    \ex{\label{ex:msschr006c} An der Nacht auf dem Land schätze ich vor allem das Dunkle.}
    \ex{\label{ex:msschr006b} Alle Pferde müssen geputzt werden. Vanessa putzt das schwarze.}
    \ex{\label{ex:msschr006a} Vanessa trägt in der Oper das Schwarze.}
  \end{xlist}
\end{exe}

Satz (\ref{ex:msschr006c}) ist der ganz typische Fall eines substantivierten Adjektivs, weil sich hier auf das Dunkle an sich (die Eigenschaft, dunkel zu sein) bezogen wird.
In (\ref{ex:msschr006b}) liegt eine typische Situation für eine Ellipse vor, also eine Auslassung eines oder mehrerer Wörter in einem Folgesatz (hier das Substantiv \textit{Pferd}), die ansonsten eine Wiederholung darstellen würden.
Daher ist das Adjektiv nicht substantiviert und wird nicht großgeschrieben.
In (\ref{ex:msschr006a}) wird \textit{das Schwarze} als Bezeichnung für \textit{das schwarze Kleid} benutzt, und das Adjektiv ist daher substantiviert und wird groß geschrieben.
Allerdings ist (\ref{ex:msschr006a}) ein sehr plakativ gewähltes Beispiel, und der Übergang von Fällen wie (\ref{ex:msschr006b}) zu solchen wie (\ref{ex:msschr006a}) ist normalerweise alles andere als scharf.
Das Schreibsystem bietet hier zwei Möglichkeiten, genau zwei syntaktische Strukturen zu kodieren, und ein Regelungsbedarf auf Seiten der Orthographie besteht eigentlich nicht.
Man muss sich vielmehr bewusst sein, dass jede orthographische Regelung hier (wie schon bei den Univerbierungen) eigentlich eine grammatische Regelung darstellt.

\index{Substantiv!Großschreibung}
Schwieriger wird es in Fällen wie \textit{im übrigen}, \textit{recht geben} bzw.\ \textit{rechtgeben}, \textit{im trüben fischen}.
Hier wird Kleinschreibung und ggf.\ sogar Zusammenschreibung sinnvoll, weil die Wortsequenzen ihre Eigenständigkeit und ursprüngliche Bedeutung verlieren, so dass man bei \textit{übrigen} und \textit{recht} usw.\ nicht mehr von Substantiven sprechen kann.
Das formale Kriterium, das hier leidlich gut funktioniert, ist die Frage nach der morphosyntaktischen Verwendbarkeit.
In \textit{im übrigen} kann das potentielle Substantiv \textit{das Übrige} \zB nicht mehr modifiziert werden, vgl.\ (\ref{ex:msschr007}).

\begin{exe}
  \ex\label{ex:msschr007} 
  \begin{xlist}
    \ex[*]{\label{ex:msschr007a} im literarischen Übrigen}
    \ex[*]{\label{ex:msschr007b} Im Übrigen\slash In dem Übrigen, von dem wir gestern schon gesprochen haben, ist dieses Buch langweilig.}
  \end{xlist}
\end{exe}

\index{Univerbierung}
In \textit{rechtgeben} behandelt man \textit{recht} wahrscheinlich am sinnvollsten als Verbpartikel, die aus einem Substantiv entstanden ist, also eigentlich eine Univerbierung.
Weder ein Artikel noch Adjektive oder Relativsätze können \textit{recht} noch modifizieren.
Formale Kriterien helfen hier aber nur begrenzt weiter, über semantische ist oft nur schwer Einigkeit zu erzielen, und eigentlich müsste die Normierung strikt statistisch vorgehen und den üblichen Gebrauch dieser Sequenzen (einschließlich der Gebrauchsschreibungen) vor einem Normierungsversuch berücksichtigen, um zu einer dem System angemessenen und von den meisten Schreibern akzeptierten Regelung zu gelangen.

\index{Eigenname!Schreibung}
Wir kommen zu den Eigennamen.
Für Personennamen wie \textit{Willy Brandt} ist Satz~\ref{satz:grosschreib} völlig unproblematisch, bei potentiellen Namen wie in (\ref{ex:msschr004}) kommen grammatisch, semantisch und damit auch oft graphematisch mehrere Möglichkeiten infrage.

\begin{exe}
  \ex\label{ex:msschr004} 
  \begin{xlist}
    \ex{\label{ex:msschr004a} der Deutsche Bundestag}
    \ex{\label{ex:msschr004b} eine Molekulare Ratsche}
    \ex{\label{ex:msschr004c} am Schwarzen Brett}
    \ex{\label{ex:msschr004d} Das ist Der Spiegel von letzter Woche.}
  \end{xlist}
\end{exe}

In diesen Fällen geht es vor allem darum, zu entscheiden, ob die jeweiligen Begriffe so speziell gelesen werden, dass sie Eigennamen darstellen.
Was genau ein Eigenname ist, ist allerdings (jenseits der Personennamen) nicht einfach zu entscheiden.
Das primäre Kriterium ist, dass mit einem Eigennamen immer genau ein Ding in der Welt bezeichnet wird, und nicht eine Sorte von Dingen.
Für den \textit{Deutschen Bundestag} ist dies der Fall, denn es gibt ja nicht mehrere deutsche Bundestage.
Bei der \textit{Molekularen Ratsche} stimmt dies nur in einem komplizierteren Sinn.
Molekulare Ratschen könnte es mehrere geben, wenn sie jemand bauen würde.
Allerdings handelt es sich nicht um irgendwelche Ratschen, die irgendetwas mit Molekülen zu tun haben, sondern der Begriff \textit{Molekulare Ratsche} ist auf einen sehr spezifischen Apparat festgelegt.
Es gibt also nur eine Sorte von Objekten, die \textit{Molekulare Ratsche} genannt werden können, und der Begriff referiert also auf genau eine (eng definierte) Sorte von Gegenständen.
Das gleiche gilt für das \textit{Schwarze Brett}, wobei hier hinzukommt, dass es nicht einmal schwarz und eigentlich auch kein Brett sein muss.
Die wahrscheinlich zuverlässigste Normierung für alle diese Fälle wäre bei echten Eigennamen mit singulärem Bezugsobjekt in der Welt die Großschreibung festzulegen (\textit{Deutscher Bundestag}) und alle anderen Adjektive prinzipiell kleinzuschreiben (\textit{schwarzes Brett}, \textit{molekulare Ratsche}).

Ein eher randständiges Problem sind Namen, bei denen der Artikel Teil des Namens ist wie in (\ref{ex:msschr004d}).
Hierbei ist es ungewöhnlich, den Artikel weiter großzuschreiben, wenn er flektiert.

\begin{exe}
  \ex\label{ex:msschr005} 
  \begin{xlist}
    \ex[?]{\label{ex:msschr005a} Ich lese Den Spiegel.}
    \ex[?]{\label{ex:msschr005b} Ich lese Der Spiegel.}
  \end{xlist}
\end{exe}

Während (\ref{ex:msschr005a}) wahrscheinlich von vielen Lesern als sehr auffällig empfunden wird, gibt es für (\ref{ex:msschr005b}) zumindest die Lesart, bei der \textit{Der Spiegel} ohne zu flektieren immer in der Zitatform verwendet wird.
Für dieses Problem gibt es sehr wohl keine absolut zufriedenstellende Lösung, was angesichts seiner Randständigkeit aber sicherlich auch nicht das gesamte System zum Einsturz bringt.

%\subsection[Die sogenannte NP-Kopf-Großschreibung]{\Opsional Die sogenannte NP-Kopf-Großschreibung}
%
%\label{sec:syntgrossschr}
%
%Manchmal (vgl.\ \citealp[269--275]{FuhrhopPeters2013} und die dort zitierte Literatur) wird vertreten, die Substantivgroßschreibung sei nicht lexikalisch, sondern syntaktisch motiviert, und es werde vielmehr der Kopf der NP großgeschrieben, nicht das Substantiv an sich.
%Es wird argumentiert, dass auch NP-Köpfe, die Schreiber gar nicht als Substantive kennen (\zB Phantasiewörter), großgeschgeschrieben würden.
%Daher könne die Großschreibung nicht lexikalisch gesteuert sein, weil diese Wörter ja im Lexikon der Sprecher nicht abgelegt seien und daher auch kein Wortklassenmerkmal im Lexikon hätten.
%Diese Auffassung wird hier aus den folgenden vier Gründen abgelehnt.
%
%Erstens liegt ein Missverständnis vor, was lexikalisch bedeutet.
%Wie in Abschnitt~\ref{sec:lexkat} und Kapitel~\ref{sec:wortbildung} argumentiert wurde, ist das Lexikon im linguistischen Sinn die Menge aller Wörter einer Sprache, und diese Menge ist offen und durch produktive Prozesse der Wortbildung erweiterbar.
%Das Lexikon ist im Sinne der grammatischen Systembeschreibung also nicht etwa bloß die Menge aller Wörter, die wir bisher in unserem Leben gelernt haben.
%Dass dies sehr angemessen ist, sehen wir täglich daran, dass wir selber neue Wörter durch Wortbildungsprozesse erschaffen.
%Es ist natürlich richtig, dass wir erkennen können, dass in bestimmten syntaktischen Kontexten nur Substantive stehen können.
%Mit dieser Erkenntnis wird die NP-Kopf-Großschreibung aber zur reinen Umbenennung.
%
%Zweitens erfordert die Theorie der NP-Kopf-Großschreibung eine besondere Definition der NP.
%Pronomina als Köpfe von NPs werden nämlich nicht großgeschrieben.
%Man müsste also dann getrennt von Pronominalphrasen und anderen Nominalphrasen sprechen.
%Das ist einerseits ungünstig, weil die NP ja gerade deswegen eingeführt wurde, weil das syntaktische Verhalten von allen NPs (einschließlich derer mit pronominalem Kopf) gleich ist (Kapitel~\ref{sec:konstituentenstruktur} und Abschnitt~\ref{sec:ngr}).
%Andererseits hätte man dann die anderen NPs heimlich zur Substantivphrasen gemacht und endet damit wieder bei der Substantivgroßschreibung.
%
%Drittens sind viele der angeführten Beispiele für die Angemessenheit der NP-Kopf-Großschreibung eigentlich genauso gute oder sogar bessere Beispiele für eine Analyse als Substantivgroßschreibung.
%Zum Beispiel ist \textit{das lyrische Ich} insofern charakteristisch, als das Pronomen \textit{ich} hier als Substantiv verwendet wird und dabei ein festes Genus (Neutrum) erhält.
%Als Pronomen hat es genau dies nicht, als Substantiv muss es das haben -- und wird großgeschrieben.
%Man sollte dazu beachten, dass bei solchen Konversionen i.\,d.\,R.\ eine deutliche Veränderung der Bedeutung stattfindet.
%
%Viertens werden Substantive auch in Isolation (Nennform) und in Listen großgeschrieben, wo sie nicht erkennbar einen NP-Kopf mit syntaktischem Kontext bilden.


\subsection{Wortbildung}

\label{sec:wortbildschreib}

Die in Abschnitt~\ref{sec:wortklassschreib} besprochene Substantivgroßschreibung gehört eigentlich gleichzeitig in diesen Abschnitt, in dem es um Phänomene der Wortbildung geht, die einen Effekt in der Schreibung haben.
Konversion zum Substantiv äußert sich in der Großschreibung.
Im Bereich der Konversion und Derivation sind sonst keine besonderen Anmerkungen innerhalb der Graphematik zu machen.

\index{Kompositum!Schreibung}
Für die Komposition gilt dies nicht ganz, denn im Bereich der Normschreibung und Gebrauchsschreibung gibt es verschiedene Varianten, um Substantiv-Komposita zu schreiben, s.\ (\ref{ex:msschr008}).%
\footnote{Bei anderen Komposita sind die Möglichkeiten der Schreibung eingeschränkt.
Wahrscheinlich findet man im adjektivischen Bereich \zB \textit{blaugrün} und \textit{blau-grün}, aber nicht \textit{blau grün}.
}

\begin{exe}
  \ex\label{ex:msschr008} 
  \begin{xlist}
    \ex{\label{ex:msschr008a} Abendausritt}
    \ex{\label{ex:msschr008b} Abend-Ausritt}
    \ex{\label{ex:msschr008c} Abend Ausritt}
    \ex{\label{ex:msschr008d} AbendAusritt}
  \end{xlist}
\end{exe}

Neben der Zusammenschreibung (\ref{ex:msschr008a}) findet man die Schreibung mit Bindestrich (\ref{ex:msschr008b}), in der Gebrauchsschreibung aber auch die Spatienschreibung (\ref{ex:msschr008c}) und die Schreibung mit der sogenannten Binnenmajuskel (\ref{ex:msschr008d}).
Da Komposita ein Wort bilden (vgl.\ Abschnitt~\ref{sec:komp}), ist (\ref{ex:msschr008a}) im Rahmen des Spatienprinzips (Abschnitt~\ref{sec:spatien}) unstrittig.
Bei \textit{Abendausritt} handelt es sich um ein Wort, und dementsprechend enthält es keine Spatien.

\index{Bindestrich}
Für die Beschreibung von (\ref{ex:msschr008b}) muss zunächst der Bindestrich -- ein Nicht-Buchstabe -- an sich eingeordnet werden.
Er tritt auf in Komposita, als Silbentrennzeichen und in Koordinationen mit Ellipse wie \textit{Luft- und Raumfahrt}.
Es ist schwer, eine einheitliche Funktion für den Bindestrich zu finden, aber er kommt ganz offensichtlich nur im Bereich der Wortschreibung zum Einsatz (also nicht als Satzzeichen) und ist damit ein Wortzeichen.

\Definition{Wortzeichen}{
\label{def:wortzeich}
Wortzeichen sind Nicht-Buchstaben, die im Bereich der Wortschreibungen verwendet werden.
\index{Zeichen!Wort--}
}

Aus Sicht der Grammatik kommt es infrage, den Bindestrich in Komposita als Markierung der Grenze zwischen den phonologischen Wörtern, die im Kompositum zusammen ein prosodisches Wort bilden, zu analysieren (s.\ Abschnitt~\ref{sec:prosphonwort}).
Dazu passt allerdings nicht, dass er nur sehr sporadisch in dieser Position eingesetzt wird, und dass es vermutlich ganz anders motivierte Faktoren gibt, die die Bindestrichschreibung begünstigen.
Es ist denkbar, dass die Länge und die Komplexität (Anzahl der Glieder) des Kompositums, Eigennamen- und Lehnwortbeteiligung (\textit{Brandt-Regierung} statt \textit{Brandtregierung} und \textit{Email-Ablage} statt \textit{Emailablage}), seine Häufigkeit bzw.\ die Produktivität seiner Bildung oder spezifische semantische Relationen zwischen seinen Gliedern eine Rolle spielen.
Spekulativ gesagt könnte \textit{Abend-Ausritt} eine bessere Schreibung sein als \textit{Morgen-Sonne}, und \textit{Email-Ablage} könnte nahezu obligatorisch mit Bindestrich geschrieben werden.

\index{Majuskel}
\index{Spatium}
Die Varianten mit Spatium (\ref{ex:msschr008c}) und Binnenmajuskel (\ref{ex:msschr008d}) sind bezüglich ihrer Einordnung problematisch.
Die Schreibung mit Spatium verletzt das Prinzip der Spatienschreibung (Satz~\ref{satz:spatien}), denn Wörter wie \textit{Abendausritt} sind nach der hier vertretenen Grammatik genau ein syntaktisches Wort.
Wie in Abschnitt~\ref{sec:komp} gezeigt wurde, haben solche Komposita nämlich eine grammatische Merkmalsausstattung, \zB nur einen Kasus, ein Genus, ein Numerus.
Das Element, das nicht der Kopf ist, verliert seine grammatischen Merkmale und ist damit in der Syntax keine Einheit.
Auch wenn sie gelegentlich vorkommt, passt die Spatienschreibung von Komposita also eigentlich nicht ins System.
Angesichts der Tendenz des Deutschen, sehr lange Komposita zu bilden, ist allerdings auch fraglich, ob sich eine solche Schreibung jemals in größerem Ausmaß durchsetzen würde.
Die Schreibung mit Binnenmajuskel (\ref{ex:msschr008d}) ist sehr idiosynkratisch.
Sie verletzt die Prinzipien der positionsunabhängigen Majuskelschreibung (Satz~\ref{satz:grosschreib}) und der Satzschreibung (Satz~\ref{satz:grunabhsatz} auf S.~\pageref{satz:grunabhsatz}) insofern, als sie eine neue Umgebung und Funktion für die Majuskel eröffnet.
Welche Funktion das genau ist, ist fraglich, aber vermutlich nah an der der Bindestrichschreibung.

Zur Wortbildung gehören eigentlich auch noch die Kurzwortbildungen.
Um diese geht es jetzt unter anderem in Abschnitt~\ref{sec:abkuerz}.

\subsection{Abkürzungen und Auslassungen}

\label{sec:abkuerz}

\index{Kurzwort}
\index{Substantiv!s-Flexion}
Zu den Abkürzungen und Auslassungen gehören zunächst echte Kurzwortbildung.
Einerseits gibt es sie von einem trunkierenden (abschneidenden) Typus wie \textit{Lok}, \textit{Vopo} oder \textit{Schweini} (vgl.\ Übung~\ref{u74} auf S.~\pageref{u74}).
Diese Wörter werden graphematisch nicht besonders markiert und verhalten sich grammatisch und graphematisch wie andere Wörter.%
\footnote{Mit der Besonderheit, dass viele von ihnen auf Vollvokal enden und damit nicht ganz perfekte Wörter des Kernwortschatzes sind.}
Den graphematisch interessanten Typus stellen Abkürzungen dar, die aus explizit gelesenen Anfangsbuchstaben von Wortfolgen oder Gliedern von Komposita bestehen, \zB \textit{LKW} \textipa{[PElkave:]} (\textit{Lastkraftwagen}), \textit{AU} \textipa{[Pa:Pu:]} (\textit{astronomische Einheit}), \textit{SHK} \textipa{[PEshaka:]} (\textit{studentische Hilfskraft}).
Hierbei handelt es sich um genuine Wörter, die sich zwar aus einer rein graphischen Abkürzungskonvention ergeben, die aber klare grammatische Eigenschaften haben, wie \zB die Betonung auf der letzten Silbe.
In Fällen, wo sich dies anbietet (typischerweise wenn sich eine Folge aus Vokal, Konsonant und Vokal ergibt), werden sie allerdings auch gerne mit Erstsilbenakzent und nicht buchstabierend gelesen, also \textit{ASU} \textipa{[Pa:zu]} (\textit{Abgassonderuntersuchung}).
Der besondere Charakter dieser Bildungen (und ihre Verankerung im grammatischen System) zeigt sich auch an der \textit{s}"=Plural-Bildung, die an die Stelle der Pluralbildung des Vollwortes tritt, also \textit{LKWs}, \textit{AUs}, \textit{SHKs}, \textit{ASUs}.
Gelegentliche gespreizte Schreibungen wie *\textit{den SHKen} für \textit{den studentischen Hilfskräften} sind insofern ungewöhnlich, als es sich bei den betreffenden Wörtern nicht um eine reine Buchstaben-Abkürzung handelt, denen man die Flexion des Vollwortes verpassen kann, sondern um Kurzwörter, die wie zu erwarten die \textit{s}"=Flexion nehmen.
Kaum hört man dementsprechend die Realisierung *\textipa{[PEshaka:@n]}.
Mit gleichem Recht könnte man sonst auch *\textit{den Loken} (für \textit{den Lokomotiven}) oder *\textit{den Fundin} (für \textit{den Fundamentalpolitikern}) schreiben.

Die echten, mit Punkten markierten Schreib-Abkürzungen wie \textit{Abk.}, \textit{usw.} oder \textit{z.H.} sind graphematisch vor allem interessant, weil sie das Funktionsspektrum des Punktes über seine Kernfunktion (Abschnitt~\ref{sec:hauptsatzschreib}) hinaus erweitern.
Sie haben in der Regel keine eigene phonologische Korrespondenz und werden beim lauten Lesen zu vollen Wörtern rekonstruiert.

Den interessantesten Fall von Abkürzungen im weitesten Sinn findet man in der Form von sogenannten Klitisierungsphänomenen und ihrer Verschriftung.
Klitisierung ist ein Prozess, bei dem Wörter typische Worteigenschaften verlieren und sich eher in die Richtung eines Affixes entwickeln, ohne jedoch (zunächst) ganz dort anzukommen.
Sie werden zum Klitikon, vgl.\ Definition~\ref{def:klitikon}.

\Definition{Klitisierung (Klitikon)}{
\label{def:klitikon}
Ein Wort klitisiert (wird zum Klitikon), wenn es seinen Wortakzent verliert und sich prosodisch einem vorangehenden (Enklise) oder folgenden Wort (Proklise) anschließt.
Morphologisch bleibt es selbständig, wird also nicht zum Affix.
\index{Klitikon}
}

\index{Gebrauchsschreibung}
Klitisierungen findet man im deutschen Standard wenige, in der gesprochenen Sprache und in Gebrauchsschreibungen aber durchaus mehr.
Relevante Beispiele sind in (\ref{ex:msschr010})--(\ref{ex:msschr012}) zusammengefasst.

\begin{exe}
  \ex{\label{ex:msschr010} im, zum, zur, ins}
  \ex\label{ex:msschr011} 
  \begin{xlist}
    \ex{\label{ex:msschr011a} durch's, auf's, mit'm, so'n, so'nen}
    \ex{\label{ex:msschr011b} ich's, geht's, hat's}
  \end{xlist}
  \ex\label{ex:msschr012} 
  \begin{xlist}
    \ex{\label{ex:msschr012a} durchs, aufs, mitm, son, sonen}
    \ex{\label{ex:msschr012b} ichs, gehts, hats}
  \end{xlist}
\end{exe}

\index{Apostroph}
In (\ref{ex:msschr010}) sind bereits vollständig standardisierte Klitisierungsprodukte als Endergebnis einer historischen Entwicklung zu sehen.
In diesen Fällen haben sich die Klitika \textit{m}, \textit{r} und \textit{s} (teilweise bereits unsegmentierbar) mit dem vorangehenden Wort verbunden und verhalten sich wie Affixe (s.\ auch Abschnitt~\ref{sec:syntaxflektierbareprp}).
Auf der Ebene der Schreibung wird dies dadurch abgebildet, dass sie auch wie Affixe (also ohne Spatien oder sonstige Kennzeichen) geschrieben werden.
In (\ref{ex:msschr011}) sind weniger stark standardisierte Klitisierungen mit Apostroph geschrieben.
Der Apostroph zeigt hier wahrscheinlich nicht nur das Fehlen von Buchstaben, sondern auch die Stammgrenze an.
Wenn wir diese Klitisierungen schreiben wie in (\ref{ex:msschr012}), sind vor allem Wörter wie \textit{mitm} auffällig, weil sie einen silbischen Nasal (Abschnitt~\ref{sec:silbnasal}) verschriften und dadurch eine Buchstabensequenz ergeben, die sonst nicht vorkommt, und bei der die morphologische Segmentierung recht unklar ist.
Obwohl der Apostroph also bei diesen nicht kanonischen Klitisierungen eine gut benennbare und wichtige Funktion hat, findet man in entsprechenden Registern durchaus Schreibungen wie in (\ref{ex:msschr012}).
Falsch sind diese allerdings insofern auch nicht, als die Entwicklung zu einer Situation wie in (\ref{ex:msschr010}) für einzelne Schreiber unterschiedlich stark fortgeschritten sein kann, so dass (\ref{ex:msschr012}) die angemessenere Schreibung für sie ist.
Wie so oft bieten das grammatische System und die Schreibprinzipien mehrere Möglichkeiten an, und Sprecher bzw.\ Schreiber bedienen sich ihrer individuell verschieden.

\index{Artikel!indefinit}
Bezüglich \textit{son} und dem verkürzten Indefinitartikel \textit{n}, \textit{ne}, \textit{nen} usw.\ hat man im Übrigen festgestellt, dass besondere Entwicklungen innerhalb der Sprecher- und Schreibergemeinschaft im Gange sind.
Bei \textit{son} ist vor allem die Ausbildung eines Plurals wie in \textit{sone Pferde} markant, der sich nicht auf eine einfache Klitisierung reduzieren lässt, weil \textit{ein} keinen Plural hat (*\textit{so eine Pferde}).
Bei \textit{son} handelt es sich also vielmehr um ein neues Pronomen als um das Ergebnis einer Klitisierung, weswegen Schreibungen wie \textit{so'n} eigentlich dispräferiert sein sollten.

Der Indefinitartikel \textit{n} kommt \zB auch satzinitial vor, kann also zumindest nicht immer eine Enklise darstellen.\label{abs:nen}%
\footnote{Proklise wird für das Deutsche weitgehend ausgeschlossen.}
Da die Apo\-stroph\-schrei\-bung mehr ist als eine einfache Markierung von fehlendem Material, sondern eben auch Stammgrenzen markiert, ist besonders die satzinitiale Verwendung von apostrophiertem \textit{n} dem System fremd, vgl. (\ref{ex:msschr013}).
Variante (\ref{ex:msschr013b}) ist gegenüber (\ref{ex:msschr013a}) die deutlich schlechtere Lösung.

\begin{exe}
  \ex\label{ex:msschr013} 
  \begin{xlist}
    \ex{\label{ex:msschr013a} Nen Ausritt hat Vanessa heute nicht mehr geplant.}
    \ex{\label{ex:msschr013b} 'Nen Ausritt hat Vanessa heute nicht mehr geplant.}
  \end{xlist}
\end{exe}

Außerdem findet man die aufgefüllte Form \textit{nen} wie in \textit{nen Kind}, die ebenfalls dafür spricht, dass es sich nicht mehr nur um einen einfachen Reduktionsprozess handelt (*\textit{einen Kind}).
Damit ist also auch bei diesem neuen Artikel die Apostrophschreibung nur noch begrenzt einschlägig.
Hinzu kommt, dass die Genitive \textit{nes} und \textit{ner} wie in *\textit{der Mustang nes Freundes} oder *\textit{die Corvette ner Freundin} nicht verwendet werden, obwohl sie im Rahmen eines Klitisierungsprozesses ja durchaus verfügbar sein sollten.
Es bildet sich hier vielmehr ein neuer Indefinitartikel heraus, der zunächst umgangssprachlich und in Gebrauchsschreibungen alternativ zum Artikel \textit{ein} existiert.

\index{Genitiv!sächsisch}
Angesichts der genannten Funktion des Apostrophs kann man nun auch das vieldiskutierte apostrophierte \textit{s} in Fällen wie (\ref{ex:msschr016}) bewerten.

\begin{exe}
  \ex\label{ex:msschr016} 
  \begin{xlist}
    \ex[*]{\label{ex:msschr016a} Emma's Lebkuchen}
    \ex[*]{\label{ex:msschr016b} legendäre Auto's}
  \end{xlist}
\end{exe}

Dass hier der Einfluss des Englischen am Werk ist, könnte für (\ref{ex:msschr016a}) -- den sogenannten \textit{sächsischen Genitiv} -- eventuell stimmen, aber für (\ref{ex:msschr016b}) definitiv nicht.
Selbst wenn das Englische beim Genitiv das Vorbild wäre, würde uns das nicht sagen, ob die Schreibung ins deutsche System passt oder nicht.
Sie passt nicht ins System, wie es hier beschrieben wurde.
Es handelt sich bei \textit{s} um Flexionsaffixe, die Genitiv und Plural der \textit{s}"=Flexion anzeigen.
Wie weiter oben in diesem Abschnitt argumentiert, werden sonst im Deutschen Flexionsaffixe nie durch Apostroph abgetrennt.
Es gibt in (\ref{ex:msschr016}) weder einen Bedarf, die Stammgrenze zu markieren, noch ist irgendwelches Material zu rekonstruieren, und der Apostroph wird damit jenseits seiner sonst im System verankerten Funktion verwendet.
Das Englische hat im Übrigen einen zusätzlichen Grund, das \textit{s} des Genitivs besonders zu markieren.
Es ist eigentlich kein Flexionsaffix, sondern eine Art Klitikon, dass auch an komplexere NPs angefügt werden kann, vgl.\ (\ref{ex:msschr017}).

\begin{exe}
  \ex\label{ex:msschr017} 
  \begin{xlist}
    \ex{\label{ex:msschr017a} Emma's gingerbread}
    \ex{\label{ex:msschr017b} the Queen of Denmark's gingerbread}
  \end{xlist}
\end{exe}

Die Situation ist also eine ganz andere als im Deutschen, wo \textit{-s} ein ganz normales Flexionsaffix ist.

\subsection{Konstantschreibungen}

\label{sec:konstanz}

Im Bereich der Wortschreibungen geht es in diesem Abschnitt abschließend um ein wichtiges Prinzip, das sich unter anderem nochmal auf Schärfungs- und Silbengelenkschreibungen sowie h-Schreibungen an der Silbengrenze (Abschnitte~\ref{sec:laengeschreib}--\ref{sec:silbengelenk}) rückbezieht.
Dort wurde besprochen, dass im Kernwortschatz der kurze geschlossene Einsilbler bis auf wenige Ausnahmen mit Schärfungsschreibung geschrieben wird (\textit{platt}, \textit{Rock}, \textit{Kamm}).
Andererseits wurde betont, dass die Schärfungsschreibung vor allem als Silbengelenkschreibung motiviert ist (\textit{platter}, \textit{Röcke}, \textit{Kämme}).
Wenn man das Prinzip der Konstantschreibung zugrundelegt, kann man erklären, warum die eigentlich überflüssige Schärfungsschreibung im Einsilbler erfolgt.

\Satz{Prinzip der Konstantschreibung}{
\label{satz:konstantschr}
Wortformen eines Stammes (in zweiter Näherung eines Wortes) werden möglichst konstant (also ähnlich) geschrieben.
\index{Schreibprinzip!Konstanz}
}

Da nun in Formen wie \textit{platter}, \textit{Röcke} und \textit{Kämme} die Schärfungsschreibung als Gelenkschreibung nötig ist, um die phonologischen Korrelate *\textipa{[pla:t5]}, *\textipa{[r\o:k@]} und *\textipa{[kE:m@]} zu verhindern, kann man dem Prinzip der Konstantschreibung bei diesen Stämmen nur gerecht werden, wenn die einsilbige Form die Schärfungsschreibung übernimmt.
Anders gesagt wird \textit{Rock} also vor allem deshalb nicht *\textit{Rok} geschrieben, weil *\textit{Röke} nur *\textipa{[r\o:k@]} und nicht \textipa{[r{\oe}k@]} gelesen werden kann.
Damit kann man die Schärfungsschreibung weitgehend als Silbengelenkschreibung umdeuten (aber siehe Übung~\ref{u151}), und die kurzen geschlossenen Einsilbler gehen auf das Konto der Konstantschreibung.

\index{Silbengelenk}
\Definition{Schärfungsschreibung (revidiert)}{
\label{def:kuerzschreib2}
Eine Schärfungsschreibung besteht in einem zusätzlichen, nicht segmental zu lesenden Konsonantenzeichen vor einem Vokal.
Sie markiert das Silbengelenk und zeigt damit indirekt die Kürze des Vokals an.
\index{Schärfungsschreibung}
}

Mit dem Prinzip der Konstantschreibung lassen sich auch ß-Schreibungen wie \textit{aß} (statt *\textit{as} wie in \textit{las}) erklären.
Zwar ist \textit{aß} nicht besonders konstant zum Präsensstamm \textit{ess}, aber dieser ist eben auch ein anderer Stamm.%
\footnote{Die Schreibprinzipien sind sozusagen nicht dafür verantwortlich, dass es starke und unregelmäßige Verben gibt.
Um ein extremeres Beispiel zu nennen:
Es ist sicherlich kein Bruch des Prinzips der Konstantschreibung, dass \textit{war} komplett inkonstant zu der des Stamms \textit{sei} geschrieben wird.}
Da innerhalb der Formen des Präteritalstamms \textit{aßen} nur mit \textit{ß} geschrieben werden kann, weil sowohl *\textit{asen} als auch *\textit{assen} nicht das gewünschte phonologische Korrelat haben (Abschnitt~\ref{sec:eszett}), ist \textit{aß} die konstanteste aller Möglichkeiten.
Schreibungen wie *\textit{Fluß} aus der Zeit vor der Orthographiereform von 1996 waren daher im Grunde die Normierung einer ungrammatischen Form, weil einerseits (untypisch vor \textit{ß}) ein kurzes /\textipa{U}/ vorliegt, und weil *\textit{Fluß} außerdem keine Konstantschreibung zu \textit{Flüsse} ist.

\index{Dehnungsschreibung}
Formen wie \textit{siehst} sind nun ebenfalls systematisch erklärbar.
Das \textit{ie} für /\textipa{i:}/ ist als einzige Dehnungsschreibung obligatorisch, und das \textit{h} stellt eine Konstantschreibung zu den Formen des Verbs dar, in denen \textit{h} die Silbengrenze markiert (Abschnitt~\ref{sec:intervokh}).
Es liegt also keine doppelte Dehnungsschreibung vor, sondern eine obligatorische Dehnungsschreibung und eine Konstantschreibung.

\index{Umlaut!Schreibung}
Üblicherweise wird auch die starke graphische Ähnlichkeit der Umlautvokale als Zeichen des Prinzips der Konstantschreibung gewertet.
Graphisch sind also Wörter wie \textit{öfter} zu \textit{oft} und \textit{brünstig} zu \textit{Brunst} stammkonstanter, als wenn eigene Buchstaben mit stark abweichender Form verwendet würden.
Besonders relevant wird dies bei \textit{ä} und \textit{äu}, weil im Sinne der Stammkonstanz hier nie auf \textit{e} und \textit{eu} ausgewichen wird.
Dass man also niemals *\textit{Reume} statt \textit{Räume} oder *\textit{leuft} statt \textit{läuft} schreibt, bestätigt das Prinzip der Konstantschreibung, weil mit diesen die Schreibung konstanter zu der der Stämme \textit{Raum} und \textit{lauf} ist.
Wenn man die Konstanz der Silbengelenkschreibung und der Umlautgraphien zusammen betrachtet, gäbe es ohne das Prinzip der Konstantschreibung im Paradigma eines Wortes also (auf Basis der anderen Schreibprinzipien) ungünstige Schreibungen wie *\textit{Kam} und *\textit{Kemme} statt \textit{Kamm} und \textit{Kämme}.

Damit endet der (alles andere als vollständige) Überblick über wortbezogene Schreibungen im Deutschen.
Wörter wurden in diesem Buch als die kleinsten Einheiten der Syntax behandelt, die Phrasen und Sätze bilden.
Ob und wie Phrasen und Sätze in der Schreibung kodiert werden, ist Thema von Abschnitt~\ref{sec:satzschreib}.

\section{Schreibung von Phrasen und Sätzen}

\label{sec:satzschreib}

\subsection{Phrasen}

\label{sec:phrasenschrift}
\label{sec:koordinschreib}

Mit Abschnitt~\ref{sec:satzschreib} kommen wir jetzt zum Bereich, der traditionell Interpunktion genannt wird, und innerhalb dessen Komma und Punkt die zentralen Zeichen sind.
Viel ist dabei im Bereich der Verschriftung von Phrasen nicht zu holen, zumindest wenn man wie hier die Schreibungen von Nebensätzen und ähnlichem in den Bereich der Satzschreibungen (Abschnitt~\ref{sec:satzschreib}) verschiebt.
Ein wichtiger Bereich, in dem das Komma eine seiner Kernfunktionen hat, sind allerdings Aufzählungen.
In Abschnitt~\ref{sec:koor} wurde argumentiert, dass Konjunktionen zwei Phrasen gleichen Typs (also zwei NPs, zwei APs usw.) zu einer Phrase desselben Typs verbinden.
Wie schon in Abschnitt~\ref{sec:adjektiveundartikelwoerter} bei der Koordination von APs angedeutet, kann man das Komma in Aufzählungsstrukturen wie eine rein graphische Konjunktion betrachten.
Wie in (\ref{ex:msschr014}) wird dabei die letzte Phrase üblicherweise mit einer normalen Konjunktion statt mit Komma angefügt.

\begin{exe}
  \ex{\label{ex:msschr014} Vanessa putzt Tarek, Bird Brain und Dragonfly.}
\end{exe}

\index{Koordination!Schreibung}
\index{Konjunktion}
\index{Komma}
Das Komma unterstützt in solchen Sätzen also den Aufbau einer völlig kanonischen syntaktischen Struktur, nämlich einer Aufzählung.
Zum Komma und seiner Funktion kommen wir in Abschnitt~\ref{sec:nebensatzschreib} nochmals zurück.
Es gibt diverse weitere Möglichkeiten, Aufzählungen graphisch zu kennzeichnen, die im Kern alle ähnlich wie das Komma funktionieren, aber graphisch besonders sind.
Hierzu zählen das Et-Zeichen \textit{\&}, der Listenstrich -- und standardfern auch das Pluszeichen \textit{+}.

Auf jeden Fall ist die Strukturebene, auf der das Komma angesiedelt ist, nicht das Wort, sondern die Ebene der Phrasen und Sätze.
Es ist ein syntaktisches Zeichen.

\Definition{Syntaktisches Zeichen}{
\label{def:syntzeich}
Syntaktische Zeichen sind Nicht-Buchstaben, die im Bereich der Phrasen- und Satzschreibungen verwendet werden.
\index{Zeichen!syntaktisch}
}

\index{Parenthese}
Im Bereich der Phrasen sind ansonsten noch Parenthesen zu berücksichtigen.
Unter Parenthesen verstehen wir hier Einschübe, die in den Satzzusammenhang gestellt werden, aber zu diesem keine normale syntaktische Beziehung haben.
Sie sind keine Konstituenten des Satzes.
Für Parenthesen kommen verschiedene Markierungen in Frage, vgl.\ (\ref{ex:msschr0015}).

\begin{exe}
  \ex\label{ex:msschr0015} 
  \begin{xlist}
    \ex[ ]{\label{ex:msschr0015a} Emma ist -- es war wohl gestern -- alleine ausgeritten.}
    \ex[ ]{\label{ex:msschr0015b} Emma ist (es war wohl gestern) alleine ausgeritten.}
    \ex[?]{\label{ex:msschr0015c} Emma ist, es war wohl gestern, alleine ausgeritten.}
  \end{xlist}
\end{exe}

\index{Gedankenstrich}
Der Gedankenstrich in (\ref{ex:msschr0015a}) und die Klammern in (\ref{ex:msschr0015b}) sind für diesen Zweck ähnlich oder gleich gut funktionierende Markierungen der Nicht-Integriertheit der Parenthese in die Syntax des Satzes.
Die Parenthese \textit{es war wohl gestern} ist immerhin selber ein vollständiger Verb-Zweit-Satz und lässt sich nach den Schemata aus Kapitel~\ref{sec:phrasen} und Kapitel~\ref{sec:saetze} nicht in den größeren Satz integrieren.
Das Komma in (\ref{ex:msschr0015c}) scheint hier nicht wirklich angemessen zu sein, wird aber von Schreibern durchaus in dieser Funktion verwendet.
Wie bei den Aufzählungen schon argumentiert wurde, ist das Komma eine Lesehilfe für den regulären syntaktischen Aufbau.
Als Indikator für eine Parenthese, die jenseits der normalen Satzsyntax angesiedelt ist und diese gleichsam stört, ist das Komma nicht geeignet.
Das Komma als wichtiges Element im syntaktischen Aufbau geschriebener Sätze wird im nächsten Abschnitt zu den Satzschreibungen, mit dem die Graphematik hier schließt, nochmals wiederkehren.

\subsection{Unabhängige Sätze}

\label{sec:hauptsatzschreib}

\index{Satz!Schreibung}
Unabhängige Sätze sind in erster Näherung das, was in Definition~\ref{def:satz} (S.~\pageref{def:satz}) definiert wurde, also ein finites Verb, das nicht von einer anderen Konstituente abhängt, mit all seinen Ergänzungen.
Während (\ref{ex:msschr0016}) also ein unabhängiger Satz ist, ist das eingeklammerte Material in (\ref{ex:msschr0017}) jeweils kein unabhängiger Satz.

\begin{exe}
  \ex{\label{ex:msschr0016} Die Freundinnen reiten aus.}
  \ex\label{ex:msschr0017}
  \begin{xlist}
    \ex[ ]{\label{ex:msschr0017a} Emma weiß, dass [die Freundinnen ausreiten].}
    \ex[ ]{\label{ex:msschr0017b} Emma behauptet, [die Freundinnen reiten aus].}
    \ex[ ]{\label{ex:msschr0017c} Emma reitet aus, weil [das Wetter schön ist].}
    \ex[*]{\label{ex:msschr0017d} [Reiten aus].}
  \end{xlist}
\end{exe}

\index{Punkt}
In (\ref{ex:msschr0017a}) und (\ref{ex:msschr0017c}) wird das eingeklammerte Material von dem Komplementierer regiert, in (\ref{ex:msschr0017b}) direkt vom Verb \textit{behaupten}, und in (\ref{ex:msschr0017d}) ist der Satz schlicht nicht vollständig, weil das Subjekt fehlt.
Die Unterschiede in der Schreibung zwischen (\ref{ex:msschr0016}) und (\ref{ex:msschr0017}) sind offensichtlich.
Einerseits wird im graphematischen unabhängigen Satz das erste Wort positionsbedingt immer mit Majuskel geschrieben, andererseits finden wir den Punkt als Satzende-Zeichen.
Alternativ zum Punkt kommen das optionale Ausrufezeichen und das in Fragen obligatorische Fragezeichen infrage.
Das Satzende-Zeichen hat als besondere und einzigartige Funktion, dass der syntaktische Aufbau endgültig und unwiderruflich beendet wird.
Leser erwarten nach einem Punkt nicht, dass noch Ergänzungen oder Angaben folgen, die zum abgeschlossenen Satz gehören.

\index{Satz!Koordination}
Das System der syntaktischen Zeichen erlaubt allerdings eine Doppeldeutigkeit.
Weil das Komma genauso wie Konjunktionen jede Art von Konstituente koordinieren kann, sind Alternativschreibungen wie in (\ref{ex:msschr0018}) möglich.

\begin{exe}
  \ex\label{ex:msschr0018} 
  \begin{xlist}
    \ex{\label{ex:msschr0018a} Emma reitet aus, Vanessa putzt Dragonfly.}
    \ex{\label{ex:msschr0018b} Emma reitet aus. Vanessa putzt Dragonfly.}
  \end{xlist}
\end{exe}

Zweifelsohne sind die beiden Sätze vor und nach dem Komma in (\ref{ex:msschr0018a}) nach unserer Definition unabhängig.
Außerdem zeigt (\ref{ex:msschr0018b}), dass der Satzende-Punkt hier durchaus eine gute Option ist.
Trotzdem kann das Koordinationskomma zum Einsatz kommen, also ein syntaktisches Zeichen, das eigentlich signalisiert, dass die syntaktische Struktur nicht vollständig abgeschlossen ist und der Strukturaufbau weitergeht.
Schreiber haben wahrscheinlich sehr gute Gründe, Variante (\ref{ex:msschr0018a}) oder (\ref{ex:msschr0018b}) zu wählen.
Zur Erklärung ist zu berücksichtigen, dass sich bestimmte Satzpaare besser für eine Kommaschreibung eignen als andere, vgl.\ (\ref{ex:msschr0019}).

\begin{exe}
  \ex\label{ex:msschr0019} 
  \begin{xlist}
    \ex[?]{\label{ex:msschr0019a} Gestern wurden die Pferde neu behuft, die Lichtgeschwindigkeit ist in allen Bezugssystemen konstant.}
    \ex[ ]{\label{ex:msschr0019b} Gestern wurden die Pferde neu behuft und die Lichtgeschwindigkeit ist in allen Bezugssystemen konstant.}
  \end{xlist}
\end{exe}

Aus ganz verschiedenen inhaltlichen bzw.\ semantischen und syntaktischen Gründen eignen sich die Sätze in (\ref{ex:msschr0019}) nicht zur Koordination, wobei die Koordination mit \textit{und} wenigstens formal grammatisch wirkt, was für die Version mit Komma stark angezweifelt wird.
Im Vergleich damit ist (\ref{ex:msschr0018}) ein ideales Beispiel für gut koordinierbare Sätze, weil sich eine Reihe aus zwei syntaktisch und semantisch gleich strukturierten Sätzen ergibt.
Der Sprecher oder Schreiber stellt durch die Reihung der Sätze den Kontrast zwischen dem, was Emma tut, und dem, was Vanessa tut, in den Vordergrund.
Als Erkenntnis muss also Satz~\ref{satz:grunabhsatz} festgehalten werden.

\Satz{Graphematisch unabhängiger Satz}{
\label{satz:grunabhsatz}
Der graphematisch unabhängige Satz wird mit einleitender Majuskel geschrieben und durch ein Satzende-Zeichen abgeschlossen.
Er besteht ggf.\ aus mehreren grammatisch unabhängigen Sätzen, die durch Komma getrennt sind.
\index{Satz!graphematisch}
}

Eine Beschreibung der Faktoren, die die verschiedenen möglichen Strukturen hier begünstigen, führt offensichtlich weit über die reine Formebene hinaus, auf die sich in diesem Buch gemäß Abschnitt~\ref{sec:sprachsystem} so weit wie möglich beschränkt wurde.
Zur umfassenden Beschreibung der Graphematik ist also tatsächlich ein Überblick über die Grammatik und eigentlich noch viele andere Teilgebiete der Linguistik erforderlich.
Das unterstreicht nochmals die in Abschnitt~\ref{sec:graphegrammatik} genannten Gründe dafür, dass die Graphematik in diesem Buch ganz an den Schluss gestellt wurde.

\subsection{Nebensätze und Verwandtes}

\label{sec:nebensatzschreib}

\index{Nebensatz!Schreibung}
Dass Nebensätze mit Komma abgetrennt werden, ist ein elementarer und jedem Absolventen einer Schule im deutschsprachigen Raum bekannter Grundsatz.
Egal, wo sie stehen, werden Nebensätze (zunächst alle Strukturen, die in Abschnitt~\ref{sec:nebensaetze} besprochen wurden) links und rechts (falls der linke Rand nicht mit dem Satzanfang bzw.\ der rechte Rand nicht mit dem Satzende zusammenfällt) mit einem Komma begrenzt, vgl.\ (\ref{ex:msschr0020}).

\begin{exe}
  \ex\label{ex:msschr0020} 
  \begin{xlist}
    \ex{\label{ex:msschr0020a} Jeder, der gerne reitet, mag eigentlich auch Pferde.}
    \ex{\label{ex:msschr0020b} Falls das Wetter gut ist, reiten Emma und Vanessa aus.}
    \ex{\label{ex:msschr0020c} Alle wissen, dass Tarek kein Pferd für Anfänger ist.}
  \end{xlist}
\end{exe}

Dadurch, dass Nebensätze oft im Vorfeld und Nachfeld stehen, ist für diese Feldergrenzen das Komma sehr charakteristisch, aber natürlich kein wirklich zuverlässiges Kennzeichen.
Ähnlich wie bei der Aufzählungsfunktion des Kommas scheint es so zu sein, dass auch bei den Nebensätzen das Komma die Integration einer in sich abgeschlossenen Struktur in eine Matrix signalisiert.
Die Frage ist, ob dasselbe in Sätzen wie in (\ref{ex:msschr0021}) vorliegt.

\begin{exe}
  \ex\label{ex:msschr0021} 
  \begin{xlist}
    \ex{\label{ex:msschr0021a} ob Vanessa öfter auszureiten scheint}
    \ex{\label{ex:msschr0021b} ob Vanessa wünscht(,) am Abend auszureiten}
    \ex{\label{ex:msschr0021c} ob Emma glaubt, dass Vanessa wünscht, am Abend auszureiten}
    \ex{\label{ex:msschr0021d} ob Vanessa Tarek putzt, um später auf ihm auszureiten}
  \end{xlist}
\end{exe}

\index{Kohärenz!Schreibung}
\index{Infinitiv}
\index{Komplementsatz}
In allen Sätzen aus (\ref{ex:msschr0021}) liegen zu-Infinitive vor (vgl.\ Abschnitte~\ref{sec:halbmodale} und~\ref{sec:kontrollinfinitive}).
In (\ref{ex:msschr0021a}) regiert das Halbmodalverb \textit{scheinen} diesen zu-Infinitiv, der obligatorisch kohärent konstruiert.
Im optional inkohärenten Fall mit Kontrollinfinitiven in (\ref{ex:msschr0021b}) scheint das Komma mehr oder weniger optional zu sein.
Wenn der zu-Infinitiv über eine Nebensatzgrenze nach rechts versetzt wird wie in (\ref{ex:msschr0021c}), was nur mit inkohärenten Infinitiven geht, steht das Komma obligatorisch.
Es kann natürlich nicht entschieden werden, ob hier die rechte Grenze des Komplementsatzes markiert wird oder die linke Grenze des Infinitivs.
Bei Angaben in Form des zu-Infinitivs wie in (\ref{ex:msschr0021d}) ist das Komma im Grunde nicht weglassbar.
Diese werden zudem gerne ins Vor- oder Nachfeld und nur selten ins Mittelfeld gestellt.

Der zu-Infinitiv hat eine Nähe zu Nebensätzen, weil er ähnliche Positionen im Satz einnimmt wie diese, und weil er oft eine eigene Phrase bilden kann, die unabhängig vom Verbalkomplex ist.
Die wichtigen Unterschiede des zu-Infinitivs zum Nebensatz sind, dass das Verb nicht finit ist und dass er prototypisch ohne Komplementierer auftritt.
Ob man daher die zu-Infinitive zu den Nebensätzen zählen will oder nicht, ist abhängig von der spezifischen Theorie und persönlichen Gewichtungen.
Die Interpunktion hat jedenfalls eine deutliche Tendenz, anhand der Kohärenz des zu-Infinitivs zu entscheiden, ob mit Komma getrennt wird oder nicht.

Viele Zeichen sind hier aus Platzgründen unberücksichtigt geblieben, \zB das Semikolon oder der Doppelpunkt.
Die weiterführende Literatur hat auch zu ihnen reiches Material und Theorien, und wie bei allen grammatischen Phänomenen haben alle Sprecher, Hörer, Schreiber und Leser jederzeit die Möglichkeit, selber nach Regularitäten im Gebrauch zu suchen.

\Zusammenfassung

\begin{enumerate}
  \item Spatien trennen syntaktische Wörter.
  \item Das Deutsche hat die Möglichkeit der Univerbierung, die sich systematisch auf die Getrennt- und Zusammenschreibung auswirkt.
  \item Wortklassen werden i.\,d.\,R.\ in der Schreibung nicht markiert, außer im Fall der positionsunabhängigen Substantiv- und Eigennamengroßschreibung.
  \item Ehemalige Substantive, die ihren Status als Substantiv verlieren, können kleingeschrieben werden.
  \item Eigennamen sind vor allem dadurch gekennzeichnet, dass sie genau ein bestimmtes Objekt in der Welt bezeichnen.
  \item Buchstaben-Abkürzungen wie \textit{LKW}, die komplett in Majuskeln geschrieben werden, sind keine einfachen Abkürzungen, sondern Kurzwörter mit eigenen grammatischen Eigenschaften. 
  \item Klitisierung wird im Deutschen durch den Apostroph markiert, kann bei voranschreitender Konventionalisierung aber auch zu einer Zusammenschreibung führen.
  \item Konstantschreibungen sichern die Wiedererkennbarkeit von Stämmen bei Flexions- und Wortbildungsprozessen.
  \item Das Komma ist eine Lesehilfe, die die Integration von aufgezählten Konstituenten und Nebensätzen in die syntaktische Struktur der Matrix anzeigt.
  \item Der graphematische Satz kann unter bestimmten Umständen aus mehreren unabhängigen Hauptsätzen bestehen, die durch Komma getrennt werden.
  \item Infinitive mit \textit{zu} nähern sich besonders bei inkohärenter Konstruktion bzw.\ Stellung ins Vor- und Nachfeld stark den Nebensätzen und werden entsprechend mehr oder weniger typisch mit Komma getrennt.
\end{enumerate}

\Uebungen

\Uebung[\tristar] \label{u151} Warum sind Wörter wie \textit{dann} und \textit{wenn} oder \textit{Mett}, \textit{Müll}, \textit{Suff} problematisch, wenn man die Schärfungsschreibung vollständig auf eine Silbengelenkschreibung und Konstantschreibungen reduzieren möchte?

\Uebung \label{u153} Setzen Sie im folgenden Text alle Kommas, die Sie für angemessen halten und bestimmen Sie die Funktion gemäß den in diesem Kapitel genannten Grundsätzen.%
\footnote{\url{http://de.wikipedia.org/wiki/Molekulare_Ratsche} vom 21.03.2015, editiert vom Verfasser.}
Bei Nebensätzen stellen Sie zudem fest, ob ein Adverbialsatz, Komplementsatz, Relativsatz eingeleitet wird.
Markieren Sie die Kommas, die eine Struktur beenden, gesondert, und geben Sie an, welche Struktur es jeweils ist.

\begin{quote}

Eine molekulare Ratsche oder auch Brownsche Ratsche ist eine Nanomaschine die aus brownscher Molekularbewegung (also aus Wärme) gerichtete Bewegung erzeugt.
Dies kann nur funktionieren wenn von außen Energie in das System gebracht wird.
Solche Systeme werden in der Literatur meistens Brownsche Motoren (siehe Literatur/Links) genannt.
Eine molekulare Ratsche ohne von außen zugeführter Energie wäre ein Perpetuum Mobile zweiter Art und funktioniert somit nicht.
Der Physiker Richard Feynman zeigte in einem Gedankenexperiment 1962 als erster wie eine molekulare Ratsche prinzipiell aussehen könnte und erklärte warum sie nicht funktioniert.

Eine molekulare Ratsche besteht aus einem Flügelrad und einer Ratsche mit Sperrzahn.
Die gesamte Maschine muss sehr klein sein (wenige Mikrometer) damit die Stöße des umgebenden Gases keinen nennenswerten Einfluss auf sie haben.
Die Funktionsweise ist denkbar einfach:
Ein Gasteilchen das das Flügelrad beispielsweise so trifft wie durch den grünen Pfeil markiert bewirkt ein Drehmoment das sich über die Achse auf die Ratsche überträgt und diese eine Stellung weiterdrehen kann.
Ein Teilchen das wie durch den roten Pfeil markiert auftrifft bewirkt keine Drehung da der Sperrzahn die Ratsche blockiert.
Die molekulare Ratsche sollte also aus Wärmeenergie eine gerichtete Bewegung erzeugen was aber nach dem zweiten Hauptsatz der Thermodynamik nicht möglich ist.

Der Sperrzahn funktioniert nur wenn er mit einer Feder gegen die Ratsche gedrückt wird.
Auch er unterliegt dem Bombardement der Brownschen Molekularbewegung.
Wird er durch diese ausgelenkt beginnt er auf die Ratsche zu schlagen was zu einem Nettodrehmoment entgegen der zuvor angenommen Drehrichtung führt.
Die Wahrscheinlichkeit für die Auslenkung des Sperrzahns die groß genug ist um eine Ratschenposition zu überspringen ist $exp(-\Delta E/k_BT)$ wobei $\Delta E$ die Energie die benötigt wird um die Feder des Sperrzahns auszulenken ist $T$ die Temperatur und $k_B$ die Boltzmann-Konstante.
Die Drehung über das Flügelrad muss aber auch die Feder spannen um in die nächste Position der Ratsche zu gelangen das heißt, die Wahrscheinlichkeit ist ebenfalls $exp(-\Delta E/k_BT)$.
Folglich dreht sich die Ratsche im Mittel nicht.

Anders sieht es aus wenn ein Temperaturunterschied zwischen Flügelscheibe und Ratsche vorliegt.
Ist die Umgebung des Flügelrades wärmer als die der Ratsche dreht sich die molekulare Ratsche wie zuvor angenommen.
Ist die Umgebung der Ratsche wärmer dreht sich die Maschine in die entgegengesetzte Richtung.

\end{quote}

\Uebung[\tristar] \label{u154} Vor Wörtern wie \textit{aber} und \textit{sondern} soll nach den Orthographieregeln ein Komma stehen.
Ordnen Sie die Wörter in eine Wortart ein und stellen Sie fest, welche syntaktischen Strukturen durch die Regel mit Komma getrennt werden.
Wird das Komma hier gemäß seiner in diesem Kapitel etablierten Funktionen verwendet?
Recherchieren Sie ggf., wie die Orthographieregel motiviert wird.

\WeitereLiteratur

\paragraph*{Einführungen}

\begin{sloppypar}

Grundlegend zur Einführung in die Schreibprinzipien des Deutschen ist der \textit{Grundriss} \citep[Kapitel~8]{Eisenberg1}.
Eine kurze Einführung ist \citet{Fuhrhop2009}, und \citet{FuhrhopPeters2013} verbindet eine ausführliche Einführung in die Phonologie mit einer ausführlichen Einführung in die theoretische Graphematik.
Speziell zur Interpunktion gibt es \citet{Bredel2012}.
In den angesprochenen Bereich der Eigennamen führt \citet{NueblingEa2012} ein.
Der hier vertretene Fremdwortbegriff (einschließlich des Begriffs des Kernwortschatzes) ist kompatibel zu \citet{Eisenberg2012}.
Einen Überblick über die Diskussionen hinter der letzten großen Reform der deutschen Orthographie bietet \citet{AugstEa1997}.
Einen Überblick über Schriftsysteme im Allgmeinen sowie die Geschichte der Schrift findet man in \citet{Coulmas1989}.

\paragraph*{Weiterführende Lesevorschläge}
\citet{Gallmann1995} zur Substantivgroßschreibung;
\citet{Eisenberg1981} zur Großschreibung von Eigennamen;
\citet{SchaeferSayatz2014} sowohl zu Klitisierungsphänomenen und deren Verschriftung als auch zur empirisch fundierten Erforschung von Gebrauchsschreibungen;
\citet{Jacobs2005} zur Getrennt- und Zusammenschreibung;
\citet{Buchmann2015} zu den Wortzeichen;
\citet{Primus1993} zum Komma;
\citet{Primus2008} zu einer klassischen Domäne des Gedankenstrichs;
\citet{Bredel2008} als aktuelle Theorie der Interpunktion, die in Ansätzen hier zugrundegelegt wurde.

\end{sloppypar}
