\chapter{Phonologische Schreibprinzipien}

\label{sec:phonschrift}

\section{Status der Graphematik}

\subsection{Graphematik als Teil der Grammatik}

\label{sec:graphegrammatik}

Der letzte Teil dieses Buches hat nur zwei Kapitel und wirkt in vieler Hinsicht vielleicht wie ein Anhang zu den anderen Teilen.
Es stellt sich die Frage, ob es legitim ist, die Beschreibung und Analyse der Schreibung so weit ans Ende zu stellen, und ihnen damit nur geringen Raum und scheinbar geringeres Gewicht zu geben.
Hinter dieser Frage verbirgt sich die theoretische Grundsatzentscheidung, ob das System der Schreibung als Teil eines allgemeinen Systems der Grammatik angesehen werden soll, oder ob es ein zur Grammatik externes System ist, das lediglich starke Verbindungen zur Grammatik aufweist und Phänomene der Grammatik ggf.\ nachbildet.
Dazu muss jetzt etwas ausgeholt werden, ohne dass zunächst Beispiele gegeben werden.
Die Beispiele am Ende dieses Abschnitts illustrieren aber dann die theoretischen Überlegungen.

Um diese Frage irgendwie beantworten zu können, muss zunächst geklärt werden, was prinzipiell zur Grammatik gehören soll und was nicht.
Man kann Grammatik so verstehen, dass sie die Erforschung der Regularitäten in sprachlichen Äußerungen ist, ohne dass man dabei unbedingt berücksichtigen muss, wie die Sprache im Gehirn produziert oder verstanden wird.
Dabei ist es relativ unproblematisch, sich auf eine (in letzter Konsequenz fiktive) Standardsprache oder Verkehrssprache zu beziehen und als Material auf Sätze aus Textkorpora zurückzugreifen.
Das ist typisch für die deskriptive Grammatik, und so wurde in diesem Buch auch vorgegangen.
Eine zweite Möglichkeit ist es, Grammatik mit einer Art von kognitivem Realismus zu betreiben.
Dabei möchte man eine Grammatik konstruieren, die so funktioniert wie das System im Gehirn individueller Sprecher, das für die Sprache zuständig ist.
Beide Auffassungen sind legitim und wichtig, wobei die kognitiv-realistische insofern die anspruchsvollere ist, als sie ohne aufwendige Experimente nicht sinnvoll zu betreiben ist.
Aus diesen zwei Auffassungen von Grammatik bzw.\ Grammatikforschung ergeben sich nun aber auch zwei Möglichkeiten, die Graphematik einzuordnen.

\newpage

Wenn die Graphematik unter der kognitiv-realistischen Sichtweise zur Grammatik gehören soll, dann müssten wir Evidenz dafür beschaffen, dass die Produktion von graphischen Einheiten (das Schreiben) und deren verstehendes Verarbeiten (das Lesen) im Gehirn nach denselben Prinzipien ablaufen wie grammatische Prozesse, also die Bildung von Flexionsformen, die Verarbeitung von verschiedenen Satzgliedstellungen usw.
Ein teilweise hierzu einschlägiges Argument wäre die Feststellung, dass es viele Sprachen gibt, die nicht verschriftet sind, aber keine Schrift ohne Sprache.
Außerdem lernen Kinder zunächst Sprache ohne Schrift, und die Schrift kommt erst später dazu.
Das lässt die Schrift wie ein Epiphänomen erscheinen, also als einen möglichen (nicht notwendigen) Nebeneffekt der Sprache, aber eben nichts, das auf die Sprache oder Sprachfähigkeit zurückwirkt oder gar für die Existenz von Sprache notwendig ist.
Entkräftet wird dieses Argument teilweise dadurch, dass dies ja nicht unbedingt bedeutet, dass die Schreibung nicht trotzdem nach denselben Prinzipien verarbeitet wird wie die Grammatik.
Im Gegenteil wäre es sogar nach allgemeinen Grundsätzen der wissenschaftlichen Reduktion die plausibelste Annahme, solange keine Evidenz gegen diese Annahme vorliegt.
Über solche einfachen Überlegungen hinaus muss man (wie oben schon gesagt) feststellen, dass dem kognitiven Anspruch in der Sprachbeschreibung schwer gerecht zu werden ist, weil er sich letztlich nur über aufwändige experimentelle Verfahren prüfen lässt.
Weder die kognitive Linguistik und Neurolinguistik noch die Graphematik sind sehr wahrscheinlich in der Lage, hierzu momentan mehr als vorläufige Antworten zu geben.

\index{Graphematik}
Daher ist für uns die zweite Möglichkeit der Einordnung der Graphematik interessanter.
Wie in Abschnitt~\ref{sec:normalsbeschreibung} argumentiert wurde, basiert dieses Buch auf der idealisierten Annahme, dass es eine vergleichsweise einheitliche verschriftete deutsche Verkehrssprache (eine standardnahe Varietät des Deutschen) gibt, die weitgehend unabhängig von den Gehirnen ihrer Sprecher untersuchbar ist.
Diese Idealisierung sollte keinen normativen bzw.\ präskriptiven Charakter haben und gerne auch Variation (\zB die zwei Formen \textit{deren} und \textit{derer} aus Abschnitt~\ref{sec:prondefart}) zulassen.%
\footnote{Würde es diese funktionierende Verkehrssprache nicht geben, wäre Sprach- und Grammatikunterricht für Erstsprecher sowie jegliche Form von Fremdsprachenunterricht ausgesprochen schwer, wenn nicht unmöglich.}
Die Grammatik dieses Konstrukts \textit{Standarddeutsch} haben wir näherungsweise beschrieben.
Dies geschah auf eine Weise, dass man (vor allem geschriebene) Sätze daraufhin prüfen kann, ob sie dem hier beschriebenen System von Regularitäten genügen (also relativ zu diesem grammatisch sind) oder nicht.
In diesem Teil des Buches wird nun gezeigt werden, dass die Schreibung dieses Standarddeutschen auf sehr systematische Weise der Grammatik folgt, und zwar auf den Ebenen der Phonologie, Morphologie und Syntax.
Die Schreibung bringt durchaus zusätzliche eigene Regularitäten mit und erlaubt in Details immer Abweichungen vom System.
Letzteres sehen wir aber in der Grammatik auch immer wieder (\zB echt unregelmäßige Verben wie in Abschnitt~\ref{sec:graduellunregelmäßig}) und zweifeln dennoch nicht an ihrem Systemcharakter.
Es wäre also überhaupt nicht zielführend, die Graphematik nicht als Teil der Grammatik zu betrachten.

Mit den Beispielen (\ref{ex:phonschrift1000}) kann man nun zeigen, dass eine Trennung von Grammatik und Graphematik ganz praktisch nicht ans Ziel führt, wenn eine Art der Sprachbeschreibung wie in diesem Buch angestrebt wird.
Ein erkenntnisleitendes Gedankenspiel ist bei allen diesen Beispielen im Übrigen, warum Programme zur Rechtschreibprüfung an diesen Sätzen nichts zu monieren hätten.

\begin{exe}
  \ex\label{ex:phonschrift1000} 
  \begin{xlist}
    \ex[*]{\label{ex:phonschrift1000a} Fine findet, das die Schuhe gut aussehen.}
    \ex[*]{\label{ex:phonschrift1000b} Wenn ich Geld hätte, nehme ich den Stax-Kopfhörer mit.}
    \ex[*]{\label{ex:phonschrift1000c} Um beruflich voranzukommen, nimmt Fine an der Fortbildung Teil.}
    \ex[*]{\label{ex:phonschrift1000d} Zurückbleibt der Schreibtisch nur, wenn der LKW randvoll ist.}
  \end{xlist}
\end{exe}

Relativ zu der in diesem Buch beschriebenen (nicht normativ verstandenen) Grammatik des (in gewissem Maß fiktiven) Standarddeutschen sind diese Sätze nicht in Ordnung.
Im Rahmen einer Grundschuldidaktik müsste man sich nun bei jedem dieser Sätze fragen, ob ein Schreibfehler oder ein Grammatikfehler vorliegt.
Das ist eigentlich die völlig falsche Fragestellung, denn man kann sie natürlich alle als simple Verschreibungen klassifizieren.
Genauso kann man sie aber wie folgt als ungrammatisch beschreiben, ohne einen Rechtschreibfehler zu diagnostizieren:
In (\ref{ex:phonschrift1000a}) steht der Artikel oder (Relativ-)Pronomen \textit{das} (Abschnitt~\ref{sec:artikelpronomen}) an einer Stelle, an der gemäß den Schemata für Komplementsätze (Abschnitt~\ref{sec:komplementsaetze}) der Komplementierer \textit{dass} stehen müsste.
In (\ref{ex:phonschrift1000b}) steht eine Indikativform \textit{nehme} \textipa{[ne:me]} statt der Konjunktivform \textit{nähme} \textipa{[nE:me]}.
Alternativ ist statt des Segments /\textipa{E:}/ das Segment /\textipa{e:}/ geschrieben worden, ggf.\ weil der Schreiber aus einem Dialektgebiet kommt, wo der Unterschied nicht gemacht wird.%
\footnote{Diese Formulierung ist absichtlich auf \textit{ein Segment schreiben} zugespitzt, vgl.\ Abschnitt~\ref{sec:buchstabensegmente}.}
In (\ref{ex:phonschrift1000c}) ist das Substantiv \textit{Teil} statt der in der Position korrekten Verbpartikel \textit{teil} verwendet worden.
Beispiel (\ref{ex:phonschrift1000d}) ist ein unabhängiger Aussagesatz mit ungefülltem Vorfeld, und das Partikelverb \textit{zurückbleiben} wurde komplett aus dem Verbalkomplex herausbewegt, obwohl die Partikel hätte zurückbleiben müssen (Abschnitt~\ref{sec:konstituentenstrukturinv2}, besonders Phrasenschema~\ref{str:v2} auf S.~\pageref{str:v2}).
Diese grammatischen Interpretationen ergeben sind nur, weil die Schreibung sehr engmaschig Merkmale auf allen grammatischen Ebenen kodiert.
Daher ist es unmöglich, von einer Trennung von Grammatik und Graphematik zu sprechen, sobald man geschriebene Daten berücksichtigt.
Dass die meisten Linguisten sich exzessiv auf geschriebene Daten kaprizieren, macht es umso wichtiger, die Prinzipien der Schreibung als Teil der Grammatik zu berücksichtigen.
Natürlich kann man für jedes Beispiel in (\ref{ex:phonschrift1000}) den Schreiber befragen und versuchen herauszufinden, ob in (\ref{ex:phonschrift1000a}) ein falsch geschriebener Komplementierer oder ein grammatisch falsch gewähltes Pronomen gemeint sind, usw.
Das würde aber an den möglichen Interpretationen für die Beispiele, wie sie da stehen, rein gar nichts ändern.%
\footnote{Abgesehen davon ist es ausgesprochen schwierig, diese Informationen von Schreibern durch explizites Fragen zu bekommen, vor allem ohne den Ausgang der Befragung erheblich zu beeinflussen.
Man landet dann sehr schnell wieder in einer Situation, in der ein ordentliches und damit in seiner Durchführung anspruchsvolles Experiment vonnöten wäre.}

Für Beispiel (\ref{ex:phonschrift200}) könnte man nun vermuten, dass hier klar eine einfache Verschreibung vorliegt, die nichts mit dem Verhältnis von Grammatik und Graphematik zu tun hat.

\begin{exe}
  \ex[*]{\label{ex:phonschrift200} Lingusitik ist uninteressant.}
\end{exe}

Auch das ist ein Trugschluss, denn hier ist regelhaft ein Wort /\textipa{lIngUzi:tIk}/ kodiert.
Dass es dieses Wort sehr wahrscheinlich nicht gibt, und dass wir das gerade wegen der klaren Beziehung von Buchstabenschrift und Phonologie im Deutschen sofort erkennen, ist prinzipiell unabhängig davon, dass beim Tastaturschreiben ohne Zehnfinger-System oft als reiner Unfall \textit{Lingusitik} statt \textit{Linguistik} herauskommt.
Dass Rechtschreibprogramme nur Beispiel (\ref{ex:phonschrift200}) und nicht die Beispiele in (\ref{ex:phonschrift1000}) als falsch klassifizieren würden, liegt eben genau daran, dass diese Programme keinerlei Wissen über Grammatik haben (ausgenommen evtl.\ eingeschränktes Wissen darüber, wie Komposita gebildet werden), sondern einen simplen Abgleich mit großen Datenbanken bekannter Wörter durchführen.

Um damit nun zur Ausgangsfrage zurückzukommen:
Der einzige Grund, warum die Graphematik ganz am Ende des Buches steht, ist, dass man einen sehr guten Überblick über die gesamte Grammatik haben muss, bevor man die Graphematik verstehen kann.
Damit soll also im Rahmen der deskriptiven Grammatik keine Degradierung der Graphematik an sich verbunden sein.
Das genaue Gegenteil ist der Fall.
Die weiteren Kapitel zeigen hoffentlich eindrücklich, dass dies so ist.

Das Verhältnis der gewachsenen Regularitäten des Schreibsystems und dessen expliziter Normierung -- also der Orthographie bzw.\ Rechtschreibung -- kann hier nicht hinreichend diskutiert werden.\index{Orthographie}
Auf keinen Fall ist es so, dass das Schreibsystem in irgendeiner Form geplant oder erdacht wurde.
Elemente der gegenwärtigen Schreibung wie die Dehnungs- und Schärfungsschreibungen (vgl.\ Abschnitt~\ref{sec:laengeschreib}), das Interpunktionssystem mit Punkt und Komma im Zentrum (vgl.\ Abschnitte~\ref{sec:hauptsatzschreib} und~\ref{sec:nebensatzschreib}), die Substantivgroßschreibung (vgl.\ Abschnitt~\ref{sec:wortklassschreib}) und selbst uns so elementar erscheinende Dinge wie die Worttrennung durch Spatien (vgl.\ Abschnitt~\ref{sec:spatien}) sind das Ergebnis jahrhundertelanger komplizierter Entwicklungen.
Es ist mitnichten alles genormt (und muss es auch nicht sein), und man kann sicherlich den meisten Autoren und Reformatoren von Rechtschreibregeln unterstellen, dass sie lediglich versuchen, unsystematischen historischen Ballast im Sinne der existierenden Schreibprinzipien zu systematisieren.%
\footnote{Das ist parallel zur Auffassung von grammatischer Norm als Beschreibung, die in Abschnitt~\ref{sec:normalsbeschreibung} vorgeschlagen wurde.}
Dass dabei manchmal Uneinigkeit darüber besteht, was die wichtigen Schreibprinzipien sind, und was als unsystematischer historischer Ballast angesehen wird, ist nicht zu ändern.
Wir halten uns hier aus Reformdiskussionen vollständig heraus.

\index{Gebrauchsschreibung}
In Ansätzen beziehen wir darüber hinaus auch Gebrauchsschreibungen in die Betrachtung mit ein.
In vielen Schreibsituationen (überwiegend Situationen der persönlichen Kommunikation) ist der Normdruck auf die Schreiber gelockert, und sie verwenden grammatische Formen inkl.\ deren Verschriftungen, die nicht der Norm entsprechen.
Ein Beispiel wäre \textit{n} als Indefinitartikel (statt \textit{ein}).
Dabei lassen sich besonders gut echte (nicht-normative) Eigenschaften des Schreibsystems beobachten, denn für alles, was morphosyntaktisch nicht dem Standard folgt (in dem es den Artikel \textit{n} ja gar nicht gibt), gibt es auch keine orthographische Norm.
Schreiber wählen dann zwangsläufig eine dem System entsprechende Verschriftung, wobei man im Fall von \textit{n} auch eine graphematisch durchaus erwartbare Variante \textit{nen} (wohlgemerkt statt \textit{ein}) findet.
Außerdem ist die Verwendung oder Nicht-Verwendung des Apostrophs graphematisch relevant, also ob \textit{'n} oder \textit{n} geschrieben wird.
In vielen Fällen kommt es auch zu Zusammenschreibungen wie \textit{istn} (statt \textit{ist ein}).
Für alle diese Varianten gibt es nicht voneinander zu trennende grammatische und graphematische Interpretationen, die helfen, auch das teilweise genormte Kernsystem zu verstehen (s.\ Abschnitt~\ref{sec:abkuerz}).%
\footnote{Bei solchen Gebrauchsschreibungen liegt es sehr nah, zu vermuten, dass hier einfach die gesprochene Sprache irgendwie verschriftet wird.
Sicherlich sind viele Gebrauchsschreibungen von gesprochener Sprache beeinflusst, aber es ist auf keinen Fall sinnvoll, hier einfach eine Gleichsetzung vorzunehmen.
Immerhin ist schon die Formulierung \textit{Verschriftung gesprochener Sprache} eigentlich ein Widerspruch in sich.
Sobald verschriftet wird, unterwirft man sich unausweichlich den Regularitäten des Schreibsystems.}

Abschließend erfolgt jetzt eine Einordnung des deutschen Schriftsystems in die Schriftsysteme der Welt.
Man unterscheidet drei primäre Typen von Schriftsystemen, nämlich Buchstaben-, Silben- und Wortschriften.
Bei der Buchstabenschrift entspricht im Prinzip jeder Buchstabe einem Laut.
Bei der Silbenschrift gibt es für jede Silbe ein Schriftzeichen, und bei der Wortschrift wird jedes Wort mit einem Zeichen (einem sogenannten Ideogramm) wiedergeben.
Die meisten existierenden Schriften sind allerdings kompliziertere Zwischenformen oder modifizierte Varianten eines der drei Haupttypen.
Die Schreibung des Deutschen basiert auf der lateinischen Buchstabenschrift.
Als dominantes Prinzip gilt dabei, dass ein Buchstabe ein zugrundeliegendes Segment wiedergibt.
Allerdings wird in diesem Kapitel gezeigt werden, dass einige Buchstaben auch ganz andere systematische Funktionen haben.
Außerdem gibt es systematische und idiosynkratische Phänomene, die auf morphologischen und syntaktischen Prinzipien beruhen (Kapitel~\ref{sec:andereschrift}).

\subsection{Ziele und Vorgehen in diesem Buch}

Hier wird methodisch ein anderer Weg gegangen, als es in vielen Einführungen in die Graphematik üblich ist.%
\footnote{Allerdings ist Kapitel~8 aus \citet{Eisenberg1} sehr ähnlich in seinem Herangehen.}
Alle Abschnitte in diesem und dem nächsten Kapitel fragen, wie bestimmte grammatische Phänomene, die im Buch vorher beschrieben wurden, verschriftet werden.
Es wird dabei keine fertige graphematische Theorie angenommen, sondern vielmehr der Erkenntnisprozess in den Vordergrund gestellt, mittels dessen man von den Daten zu einer minimal komplizierten Theorie mit maximalem Erklärungsanspruch gelangt.
Dementsprechend wird auf Themen wie \zB die Graph\slash Graphem-Unterscheidung nicht eingegangen, ebenso wie empirisch weniger offensichtliche Theorien wie die von der graphematischen Silbe bzw.\ dem graphematischen Fuß.
Auch über die Form der Buchstaben und sonstigen Zeichen sagen wir aus Platzgründen nichts, obwohl die existierende Literatur auch zu diesem Thema viel zu sagen hat.
Daraus folgt, dass uns der rein graphische Unterschied von Großbuchstaben (Majuskeln) und Kleinbuchstaben (Minuskeln) nicht interessiert.\index{Minuskel}\index{Majuskel}
Wir schreiben daher bald die Majuskel, bald die Minuskel, ohne einen Unterschied zu machen, außer wenn ausdrücklich grammatische Markierungen durch Majuskelschreibung erfolgen (Abschnitte~\ref{sec:hervorhebung}, \ref{sec:wortklassschreib} und \ref{sec:hauptsatzschreib}).
Wir verzichten hier auch darauf, Einheiten der Graphematik wie sonst üblich in <~> zu setzen, weil dies optisch sehr ungünstig ist.
Stattdessen nehmen wir den kursiven Schriftschnitt.

\index{Kernwortschatz}
Bezüglich der beschriebenen Phänomene beschränken wir uns auf den Kernwortschatz.
Der Kernwortschatz ist der Teil des Lexikons, der sich nach den primären, elementaren und i.\,d.\,R.\ weittragenden Regularitäten verhält.
In der Phonologie und damit zu einem großen Teil auch in diesem Kapitel zur Beziehung zwischen Phonologie und dem Schreibsystem bedeutet das, dass wir uns auf die Betrachtung einfacher trochäischer Wörter beschränken, die nicht erkennbar entlehnt sind.
Damit gilt das hier Gesagte vor allem für (in dieser Reihenfolge) Substantive, Verben und Adjektive, die überwiegend, aber längst nicht ausschließlich germanischen Ursprung sind.
Besonders in der Silben- und Fußphonologie und der Graphematik gibt es jenseits des trochäischen Kernwortschatzes stärkere Abweichungen in anderen Wortklassen.
Da die Substantive, Verben und Adjektive aber die offenen Wortklassen sind (also Wortklassen, in denen sehr viele und potentiell auch immer wieder neue Wörter enthalten sind), stellt die Beschränkung auf ihre Beschreibung kein nennenswertes Problem dar.
Dass sich Pronomina, Partikeln oder Präpositionen nicht immer nach diesen Regularitäten verhalten, spielt kaum eine Rolle, da sie sich kompakt und umfassend auflisten und ggf.\ auch lernen lassen.
Der Bedarf an großer Einheitlichkeit und Regularität entsteht also aus systematischen Gründen vor allem für Substantive, Verben und Adjektive.
Auf keinen Fall sollte angenommen werden, dass Wörter außerhalb des Kernwortschatzes irgendwie falsch sind, nicht in die Sprache gehören oder dem Kern angepasst werden sollten.
Genauso wie in der Morphologie die Präteritalpräsentien bzw.\ unregelmäßigen Verben oder die schwachen Substantive in kleinen Klassen (oder sogar in Einzelfällen) ein abweichendes Verhalten zeigen, gibt es auch Abweichungen in der Phonologie und Graphematik.

\section{Buchstaben und phonologische Segmente}

\label{sec:buchstabensegmente}

\subsection{Schreibung von konsonantischen Segmenten}

\label{sec:konssegschreib}

Die Frage soll hier sein, wie bestimmte grammatische Einheiten verschriftet werden, nicht umgekehrt.
Tabelle~\ref{tab:segschreibkons} fasst daher als Erstes zusammen, mit welchen Buchstaben (hier nur die Minuskeln) die konsonantischen Segmente aus Kapitel~\ref{sec:phonologie} (genauer Tabelle~\ref{tab:pholkonsmerk} auf S.~\pageref{tab:pholkonsmerk}) primär geschrieben werden.%
\footnote{Während in der Phonologie die Affrikaten weitgehend unbeachtet blieben, weil ihre phonologische Beschreibung zusätzliche Schwierigkeiten mit sich bringt, nehmen wir sie hier als eigenständige Segmente auf.}
Das heißt nicht, dass für die genannten Buchstaben nicht auch andere systematische oder unsystematische Verwendungen existieren.
Zu den Rändern und Ausnahmen der Schreibungen im Kernwortschatz kommen wir im Anschluss.
Wörter wie \textit{Garage} oder \textit{Chips}, die nicht den allgemeinen phonologischen Regularitäten folgen, werden aus dem gleichen Grund nicht beachtet.
Ebenso berücksichtigen wir atypische Schreibungen nicht, \zB \textit{Cäsar}, \textit{Charakter} oder \textit{Spaghetti}.
Wir wenden uns mit Tabelle~\ref{tab:segschreibkons} zunächst den Konsonanten zu.

\index{Buchstabe!konsonantisch}
\index{Konsonant!Schreibung}
\begin{table}
  \centering
    \begin{tabular}{lll}
      \lsptoprule
      \textbf{Segment} & \textbf{Buchstabe} & \textbf{Beispielwörter} \\
      \midrule
      /\textipa{p}/ & p & \textit{Plan} \\
      /\textipa{b}/ & b & \textit{Baum}, \textit{ab} \\
      /\textipa{\t{pf}}/ & pf & \textit{Pflaume} \\
      /\textipa{m}/ & m & \textit{Mus} \\
      /\textipa{f}/ & f & \textit{Fahne} \\
      /\textipa{v}/ & w & \textit{Wille} \\
      /\textipa{t}/ & t & \textit{Tau} \\
      /\textipa{d}/ & d & \textit{Daumen}, \textit{sind}\\
      /\textipa{\t{ts}}/ & z & \textit{Zunge} \\
      /\textipa{n}/ & n & \textit{nein}, \textit{Angabe} \\
      /\textipa{s}/ & s & \textit{raus} \\
      /\textipa{z}/ & s & \textit{sammeln} \\
      /\textipa{S}/ & sch & \textit{Schiff} \\
      /\textipa{l}/ & l & \textit{Lob} \\
      /\textipa{\c{c}}/ & ch & \textit{schlicht}, \textit{wacht} \\
      /\textipa{J}/ & j & \textit{Jahr} \\
      /\textipa{k}/ & k & \textit{klar} \\
      /\textipa{g}/ & g & \textit{gut}, \textit{Weg}, \textit{wenig} \\
      /\textipa{N}/ & ng & \textit{Klang} \\
      /\textipa{K}/ & r & \textit{rot}, \textit{klar} \\
      /\textipa{h}/ & h & \textit{Hafen} \\
      \lspbottomrule
    \end{tabular}
  \caption{Konsonantische Segmente des Deutschen und ihre primäre Buchstabenkorrespondenz}
  \label{tab:segschreibkons}
\end{table}

Bei der Betrachtung von Tabelle~\ref{tab:segschreibkons} sollte im Auge behalten werden, dass nur die zugrundeliegenden Segmente in /~/ aufgelistet sind, und nicht etwa alle möglichen Phone des Deutschen.
Für /\textipa{\c{c}}/ müssen also die beiden Realisierungen \textipa{[\c{c}]} und \textipa{[X]} berücksichtigt werden, usw.
Das können wir uns erlauben, weil die Buchstaben tendentiell den zugrundeliegenden Segmenten (bzw.\ den traditionellen Phonemen) entsprechen, und damit eher phonologisch als phonetisch sind.
Tabelle~\ref{tab:konsseginvarianz} zeigt Beispiele für die Unveränderlichkeit der Konsonanten-Buchstaben eines zugrundeliegenden Segments.

\begin{table}
  \centering
  \resizebox{\textwidth}{!}{
    \begin{tabular}{lllllll}
      \lsptoprule
      \textbf{Seg.} & \textbf{Buchst.} & \textbf{Realis.~1} & \textbf{Schreib.~1} & \textbf{Realis.~2} & \textbf{Schreib.~2} & \textbf{phonet.~Schreib.~2} \\
      \midrule
      /\textipa{b}/ & b & \textipa{[b\t{aO}m]} & \textit{Baum} & \textipa{[lo:p]} & \textit{Lob} & *\textit{Lop} \\
      /\textipa{d}/ & d & \textipa{[d\t{aO}m@n]} & \textit{Daumen} & \textipa{[zInt]} & \textit{sind} & *\textit{sint} \\
      /\textipa{n}/ & n & \textipa{[n\t{aE}n]} & \textit{nein} & \textipa{[PaNgab@]} & \textit{Angabe} & *\textit{Anggabe} \\
      /\textipa{\c{c}}/ & ch & \textipa{[SlI\c{c}t]} & \textit{schlicht} & \textipa{[vaXt]} & \textit{wacht} & -- \\
      /\textipa{g}/ & g & \textipa{[gu:t]} & \textit{gut} & \textipa{[ve:nI\c{c}]} & \textit{wenig} & *\textit{wenich} \\
      /\textipa{K}/ & r & \textipa{[Ko:t]} & \textit{rot} & \textipa{[kl\t{a@}]} & \textit{klar} & (*\textit{klae}) \\
      \lspbottomrule
    \end{tabular}
  }
  \caption{Invarianz zugrundeliegender Konsonanten-Segmente in der deutschen Buchstabenschrift}
  \label{tab:konsseginvarianz}
\end{table}

\index{Auslautverhärtung!Schreibung}
\index{r-Vokalisierung!Schreibung}
Segmentale Prozesse wie die Auslautverhärtung (Abschnitt \ref{sec:prozauslautverh}), die Verteilung von \textipa{[\c{c}]} und \textipa{[X]} (Abschnitt \ref{sec:prozichach}), die Frikativierung von /\textipa{g}/ (Abschnitt \ref{sec:prozgfrik}) oder Vokalisierungen von /\textipa{K}/ (Abschnitt \ref{sec:prozrvok}) werden offensichtlich ganz konsequent in der Buchstabenschrift nicht abgebildet.
Sonst müssten wir die Schreibungen in der letzten Spalte von Tabelle~\ref{tab:konsseginvarianz} beobachten.%
\footnote{Um den Unterschied von \textipa{[\c{c}]} und \textipa{[X]} abzubilden, hätten wir nicht einmal orthographische Möglichkeiten.
Auch die Schreibung *\textit{klae} für \textipa{[kl\t{a@}]} würde eine ausgesprochene Dehnung des graphematischen Systems des Deutschen darstellen.}

Auch im Kernwortschatz gibt es nun segmentale Schreibungen, die noch nicht erfasst wurden.
Eine kleiner Sonderfall im System ist die kanonische Schreibung \textit{qu} für \textipa{[kv]}, die historisch, aber nicht synchron im System begründbar ist.
Der Buchstabe \textit{q} ist vor /\textipa{v}/ die generelle Vertretung von \textit{k}, und \textit{u} ist die generelle Vertretung von \textit{v} nach /\textipa{k}/.
Das ist recht seltsam, denn das \textit{u} (ein Vokalzeichen) kommt sonst nicht im konsonantischen Bereich vor, und \textit{q} gibt es ansonsten gar nicht.
Die zwei zugrundeliegenden Segmente korrespondieren also jeweils mit zwei Buchstaben statt nur einem.
Die Verteilung ist aber klar (und komplementär, vgl.\ Abschnitt~\ref{sec:segmenteverteilungen}), und das phonologische Schreibprinzip wird dadurch nicht aufgehoben.
Das gilt ebenso für \textit{sp} und \textit{st} in Onsets, die statt der direkten Schreibungen *\textit{schp} und *\textit{scht} für /\textipa{Sp}/ und /\textipa{St}/ stehen.

\enlargethispage{1\baselineskip}
Weiterhin gibt es systematisch verschiedene Möglichkeiten, die Segmentfolge /\textipa{ks}/ zu schreiben.
Diese Abfolge kommt am Silbenanfang im Deutschen im Grunde nicht vor, und in Lehnwörtern wird die besondere Schreibung \textit{x} verwendet (\textit{Xenon} usw.).
In der Coda wird prinzipiell \textit{ch} für /\textipa{k}/ vor \textit{s} substituiert, vgl.\ \textit{Wachs} /\textipa{vaks}/ oder \textit{Echse} /\textipa{Eks@}/.
Die naheliegende Schreibung \textit{ks} kommt vor allem (aber nicht nur) in Form von \textit{cks} vor (zum \textit{ck} hier siehe Abschnitt~\ref{sec:laengeschreib} und Abschnitt~\ref{sec:konstanz}).
Eher selten ist sie in Simplizia wie \textit{Keks} oder \textit{zwecks} anzutreffen, häufig aber an der Morphgrenze wie in \textit{steckst} oder \textit{Glücks}.

Das Zeichen \textit{s} schließlich ist scheinbar als einziges unter den primären Konsonantenschreibungen doppelt belegt, weil es sowohl für /\textipa{s}/ als auch /\textipa{z}/ verwendet wird.
Diese Beobachtung gehört eng zu der Beobachtung des \textit{ß} (also des scharfen S oder Eszett), das in bestimmten Kontexten für /\textipa{s}/ verwendet wird, vgl.\ Abschnitt~\ref{sec:silbenschreib}.
Die beiden Segmente sind bezüglich des Wortanlauts und Wortauslauts komplementär verteilt (\textit{Sahne} \textipa{[za:n@]}, aber \textit{Eis} \textipa{[P\t{aE}s]}), was schon in (\ref{ex:phol6440}) auf S.~\pageref{ex:phol6440} festgestellt wurde.
Allerdings gibt es Positionen im Wort, in denen sie distinktiv sind, und in denen das \textit{ß} bei der Unterscheidung zwischen /\textipa{s}/ und /\textipa{z}/ hilft, \zB \textit{Muße} /\textipa{mu:s@}/ und \textit{Muse} /\textipa{mu:z@}/, was in Abschnitt~\ref{sec:silbengelenk} genauer erklärt wird.

\Satz{Phonologisches Schreibprinzip}{
\label{satz:phonschreibprinz}
Jedes zugrundeliegende Segment (Phonem) korrespondiert primär mit genau einem Buchstaben (mit sehr wenigen Ausnahmen).
Die Schreibung ist invariant, auch wenn phonologische Prozesse die Merkmale des Segments ändern.
Man kann also von einer phonologischen statt einer phonetischen Schreibung sprechen.
\index{Schreibprinzip!phonologisch}
}

\subsection{Schreibung von vokalischen Segmenten}

Was bei den Konsonanten in Gestalt des \textit{s} ein Sonderfall ist, nämlich dass ein Buchstabe mehreren zugrundeliegenden Segmenten entspricht, ist bei den Vokalen regelmäßig der Fall.
In Tabelle~\ref{tab:segschreibvok} sind die vokalischen Segmente aus Kapitel~\ref{sec:phonologie} (genauer Tabelle~\ref{tab:pholvokmerk} auf S.~\pageref{tab:pholvokmerk}) und ihre korrespondierenden Buchstaben aufgelistet.

\index{Buchstabe!vokalisch}
\index{Vokal!Schreibung}
\begin{table}
  \centering
    \begin{tabular}{lll}
      \lsptoprule
      \textbf{Segment} & \textbf{Buchstabe} & \textbf{Beispielwörter} \\
      \midrule
      /\textipa{i}/ & i (ie) & \textit{Igel}, \textit{Wiege} \\
      /\textipa{I}/ & i & \textit{richtig} \\
      /\textipa{y}/ & ü & \textit{über} \\
      /\textipa{Y}/ & ü & \textit{flüchtig} \\
      /\textipa{u}/ & u & \textit{gut} \\
      /\textipa{U}/ & u & \textit{Butter} \\
      /\textipa{\o}/ & ö & \textit{höher} \\
      /\textipa{\oe}/ & ö & \textit{öfter} \\
      /\textipa{o}/ & o & \textit{Ofen} \\
      /\textipa{O}/ & o & \textit{offen} \\
      /\textipa{e}/ & e & \textit{wenig} \\
      /\textipa{E}/ & e, ä & \textit{Ende}, \textit{spät} \\
      /\textipa{@}/ & e & \textit{Plane} \\
      /\textipa{a}/ & a & \textit{Wal} \\
      \lspbottomrule
    \end{tabular}
  \caption{Vokalische Segmente des Deutschen und ihre primäre Buchstabenkorrespondenz}
  \label{tab:segschreibvok}
\end{table}

\index{gespannt!Schreibung}
\index{Dehnungsschreibung}
\index{Schärfungsschreibung}
Wo im phonologischen System eine lange (gespannte) und eine kurze (ungespannte) Variante eines Vokals existieren, gibt es jeweils nur ein Vokalzeichen.
Das ist systematisch so, und Abschnitt~\ref{sec:laengeschreib} widmet sich diesem Phänomen nochmals aus Sicht der Silbenphonologie und ihrer Verschriftung.%
\footnote{Es existieren nur sehr wenige Wörter (im Kernwortschatz, vgl.\ Abschnitt~\ref{sec:nichtkernschreib}), die ein langes /\textipa{i:}/ mit einfachem \textit{i} verschriften.
Die Dehnungsschreibung (vgl.\ Abschnitt~\ref{sec:laengeschreib}) mit \textit{ie} (seltener \textit{ih}) ist quasi obligatorisch, weswegen in der Tabelle zumindest in Klammern das \textit{ie} als mögliche Korrespondenz zu /\textipa{i:}/ angegeben ist.}
Ein gewisses Gedrängel gibt es aus historischen Entwicklungen heraus bei den Buchstaben \textit{e} und \textit{ä}.
Das \textit{e} verschriftet das gespannte /\textipa{e}/ (\textit{wenig}), das ungespannte kurze /\textipa{E}/ (\textit{Endspiel}) sowie Schwa (\textit{Bande}).
Der Buchstabe \textit{ä} kann ebenfalls das ungespannte kurze /\textipa{E}/ (\textit{ändern}) verschriften, aber eben auch dessen lange Variante (\textit{Ähre}).

\newpage

\index{Diphthong!Schreibung}
Wie im Fall von \textit{chs} und \textit{qu} (Abschnitt~\ref{sec:konssegschreib}) gibt es auch bei den Vokalen kleine Spezialitäten zu berücksichtigen.
Vor allem sind die Diphthonge \textit{eu} (\textit{Heu}) und \textit{ei} (\textit{frei}) zu nennen.
Bei ihnen korrespondieren die Buchstaben des geschriebenen Diphthongs nicht direkt (gemäß der Korrespondenzen aus Tabelle~\ref{tab:segschreibvok}) mit Segmenten, und man muss sie ähnlich wie \textit{ch} als jeweils eine graphematisch nicht teilbare Einheit auffassen.
Bei den Diphthongen \textit{ai}, \textit{au} und \textit{oi} wird direkt auf Basis der einfachen Segmente verschriftet.
Allerdings kommen \textit{ai} und \textit{oi} fast nur in Lehnwörtern (\textit{Kaiser}, \textit{Joint}) oder Namen vor, die dialektal beeinflusst sind (\textit{Mainz}, \textit{Moik}).
Zu den wenigen Ausnahmen zählt \textit{Waise}.
Im Prinzip haben wir es bei \textit{ei} und \textit{eu} mit einer historisch begründeten Ausnahme zu tun, aber die zu \textit{eu} alternative Schreibung \textit{äu} hat einen besonderen Stellenwert (vgl.\ Abschnitt~\ref{sec:konstanz}).

Viel mehr muss man für die hier verfolgten Zwecke zu den Schreibungen der Segmente gar nicht sagen, könnte es aber natürlich.
Es gibt im Bereich der Verschriftung phonologischer Phänomene im Deutschen allerdings auch Fälle von Zeichen, die nicht Segmenten entsprechen wie \textit{e} in \textit{Knie} oder \textit{c} in \textit{Rock}.
Solche Schreibungen haben in den meisten Fällen eine Motivation in der Silbenphonologie, um die es jetzt in Abschnitt~\ref{sec:silbenschreib} geht.

\section{Silben und Wörter}

\label{sec:silbenschreib}

\subsection{Zielsetzung}

In diesem Abschnitt werden nicht Konzepte wie die graphematische Silbe oder der graphematische Fuß besprochen.
Solche Einheiten werden in der Literatur durchaus mit guten Gründen diskutiert.
Hier würde eine Diskussion dieser Theorien zu weit führen, und wir beschränken uns auf die Aspekte der Silbenphonologie, die auf die segmentale Phonologie (hier in der Regel Vokallänge) zurückwirken und systematisch verschriftet werden.
Diese Phänomene sind gut am konkreten Material zu zeigen, und sie interagieren direkt mit vieldiskutierten Fragen der Orthographie (\zB \textit{ß}-Schreibungen).
Darum geht es also in Abschnitt~\ref{sec:silbengelenk}, wobei viele Phänomene und Regularitäten der didaktischen Reduktion zum Opfer gefallen sind.
In Abschnitt~\ref{sec:hervorhebung} werden dann noch kurz einige Gebrauchsschreibungen besprochen, die ggf.\ mit dem phonologischen Akzent korrelieren.

\subsection{Dehnungsschreibungen und Schärfungsschreibungen}

\label{sec:laengeschreib}

Besonderheiten der Schreibung auf Silbenebene betreffen vor allem Längen und Kürzen von Vokalen, was letztlich eine Interaktion der phonotaktischen mit der segmentalen Ebene darstellt.
In Kapitel~\ref{sec:phonologie} wurde Länge als ein segmentales Merkmal besprochen.
Demnach sind bestimmte Vokalsegmente einfach über ein Merkmal als lang oder kurz spezifiziert.
Beim Silbenbau wurde dann nicht besonders auf die Vokallänge eingegangen, einerseits aus Platzgründen, andererseits, weil sich bestimmte Gesetzmäßigkeiten hier im Rahmen der Graphematik besonders gut zeigen lassen.
Wichtig sind hier die sogenannten Schärfungsschreibungen (Definition~\ref{def:kuerzschreib}) und Dehnungsschreibungen (Definition~\ref{def:dehnungsschreib}).
Weil später die Schärfungsschreibungen leicht umgedeutet werden, ist Definition~\ref{def:kuerzschreib} als vorläufig markiert.

\Definition{Schärfungsschreibung (vorläufig)}{
\label{def:kuerzschreib}
Eine Schärfungsschreibung besteht in einem zusätzlichen, nicht segmental zu lesenden Konsonantenbuchstaben nach einem Vokal und zeigt dessen Kürze an.
\index{Schärfungsschreibung}
}

\Definition{Dehnungsschreibung}{
\label{def:dehnungsschreib}
Eine Dehnungsschreibung besteht in einem zusätzlichen, nicht segmental zu lesenden Buchstaben nach einem Vokal und zeigt dessen Länge an.
\index{Dehnungsschreibung}
}

Bei den Schärfungsschreibungen fällt vor allem Doppelkonsonanz ins Auge (\textit{Kinn}, \textit{knapp}) und \textit{ck} (\textit{Rock}, \textit{Knick}).
Dehnungsschreibungen gibt es in Form von \textit{h} (\textit{Reh}, \textit{hohl}), Doppelung (\textit{Schnee}, \textit{Moor}, \textit{Aal}) und bei \textit{i} typisch \textit{ie} (\textit{Knie}, \textit{viel}).
Das deutsche Schriftsystem bemüht sich offensichtlich darum, Länge und Kürze möglichst eindeutig zu markieren, auch wenn vor allem die Markierung der Längen im Ergebnis nur sehr inkonsequent durchgeführt wird.
Das Lateinische, von dem das Deutsche seine Schrift übernommen hat, hat ebenfalls einen Unterschied von Vokallängen, markiert diesen aber überhaupt nicht in der Schrift.
Die Schärfungs- und Dehnungsschreibungen sind also eine historisch gewachsene Erweiterung des aus dem Lateinischen entlehnten Buchstabensystems.

Wie verteilen sich die Dehnungs- und Schärfungsschreibungen?
Zunächst betrachten wir Tabelle~\ref{tab:dehnkuerzschreib}.
In dieser Tabelle wird nach offenen und geschlossenen Silben gemäß Definition~\ref{def:offengeschlossen} klassifiziert.

\Definition{Offene und geschlossene Silben}{
\label{def:offengeschlossen}
Silben mit (konsonantischer) Coda sind geschlossene Silben, Silben ohne Coda sind offene Silben.
\index{Silbe!offen}
\index{Silbe!geschlossen}
}

\newpage

\index{Simplex}
Die Tabelle listet und gruppiert simplexe Wörter (also nicht flektierte und nicht durch Wortbildung abgeleitete) einsilbige und zweisilbige trochäische Wörter des Kernwortschatzes.%
\footnote{Simplexe Wörter werden auch \textit{Simplizia} oder \textit{Simplicia} (Singular: \textit{Simplex}) genannt.}
Eine Ausnahme bilden die eingeklammerten Zweisilbler mit langer geschlossener Erstsilbe, die alle nicht simplex sind, weil simplexe Wörter dieses Typs mit wenigen Ausnahmen (\zB Namen wie \textit{Liedtke} /\textipa{li:tk@}/ oder \textit{Wiebke} /\textipa{vi:pke}/) nicht existieren.\label{abs:wiebke}
Die zweisilbigen Wörter wurden absichtlich so ausgesucht, dass die zweite Silbe mit einem Konsonant anlautet, was auch der typische und häufige Fall ist (s.\ aber Abschnitt~\ref{sec:intervokh}).
Es interessiert jeweils nur die erste (bzw.\ einzige) Silbe, und ob sie einen langen Vokal oder sein kurzes Pendant enthält.%
\footnote{Die Vokale /\textipa{\o}/ und /\textipa{y}/ und ihre kurzen Varianten fehlen aus Gründen der Übersichtlichkeit.
Vgl. Übung~\ref{u141}.}

\index{Silbe!und Schreibung}
\begin{table}
  \centering
  \resizebox{\textwidth}{!}{
    \begin{tabular}{lllllllll}
      \lsptoprule
      & & & \textbf{/i/, /ɪ/} & \textbf{/u/, /ʊ/} & \textbf{/e/} & \textbf{/ε/} & \textbf{/o/, /ɔ/} & \textbf{/a/} \\ 
      \midrule
      \multirow{4}{*}{\rotatebox{90}{\textbf{kurz}}}

	& \multirow{2}{*}{\rotatebox{90}{\textbf{offen}}}
	  & \textbf{einsilb.}  & \textit{\Nono}  & \textit{\Nono}           & \textit{\Nono}        & \textit{\Nono}         & \textit{\Nono}        & \textit{\Nono}           \\
	&& \textbf{zweisilb.}  & \textit{Li.ppe} & \textit{Fu.tter}         & \textit{\Nono}        & \textit{We.cke}        & \textit{o.ffen}       & \textit{wa.cker}         \\

        & \multirow{2}{*}{\rotatebox{90}{\textbf{gesch.}}}
	  & \textbf{einsilb.}  & \textit{Kinn}   & \textit{Schutt}    & \textit{\Nono}        & \textit{Bett}           & \textit{Rock}         & \textit{Watt}            \\
        && \textbf{zweisilb.}  & \textit{Rin.de} & \textit{Wun.der}        & \textit{\Nono}        & \textit{Wen.de}         & \textit{pol.ter}      & \textit{Tan.te}          \\


	\midrule

	\multirow{4}{*}{\rotatebox{90}{\textbf{lang}}}

	& \multirow{2}{*}{\rotatebox{90}{\textbf{offen}}}
	  & \textbf{einsilb.}  & \textit{Knie}   & \textit{Schuh}       & \textit{Schnee, Reh}  & \textit{zäh}          & \textit{roh}          & (\textit{da})            \\
	&& \textbf{zweisilb.}  & \textit{Bie.ne} & \textit{Kuh.le, Schu.le} & \textit{we.nig}       & \textit{Äh.re, rä.kel} & \textit{oh.ne, O.fen} & \textit{Fah.ne, Spa.ten} \\

	& \multirow{2}{*}{\rotatebox{90}{\textbf{gesch.}}}
	  & \textbf{einsilb.}  & \textit{Biest}  & \textit{Ruhm, Glut}      & \textit{Weg}          & \textit{spät}           & \textit{rot}          & \textit{Tat}             \\
	&& \textbf{zweisilb.}  & (\textit{lieb.lich}) & (\textit{lug.te})   & (\textit{red.lich})   & (\textit{wähl.te})     & (\textit{brot.los})   & (\textit{rat.los})       \\

      \lspbottomrule
    \end{tabular}
  }
  \caption{Schreibung im Kernwortschatz von Vokallängen in Einsilblern und Erstsilben von trochäischen Zweisilblern mit konsonantisch anlautender Zweitsilbe}
  \label{tab:dehnkuerzschreib}
\end{table}

Wenn wir uns zuerst den kurzen Silben zuwenden, ist interessant, dass es keine kurzen offenen Einsilbler wie *\textipa{[knI]} oder *\textipa{[KO]} gibt.%
\footnote{Schwa-Silben sind nie betont und nie lang, bilden nie einen Einsilbler (eventuell mit Ausnahmen im Funktionswortbereich, vgl.\ die Diskussion der Formen des neuen Pronomens \textit{n}, \textit{ne} usw.\ auf S.~\pageref{abs:nen}) und kommen durchaus offen vor.\index{Schwa}
Sie sind damit für die hier erörterten Fragen irrelevant.}
Passend zum Fehlen der offenen kurzen Einsilbler sind auch Schreibungen wie *\textit{Kni} oder *\textit{Ro} im Prinzip inakzeptabel.
Als Erstsilbe eines mehrsilbigen Worts können kurze offene Silben allerdings vorkommen, \zB \textit{Li.ppe} oder \textit{o.ffen}, wozu unten noch mehr gesagt wird.
In diesen Fällen muss immer eine Schärfungsschreibung erfolgen.
Ebenso steht die Schärfungsschreibung bis auf Ausnahmen (\textit{das}, \textit{zum}) immer in kurzen geschlossenen Einsilblern mit einfacher Coda (\textit{Kinn}, \textit{Rock} usw.).
Wenn die Coda komplex ist (\textit{Kalk}, \textit{Kind} usw.), steht die Schärfungsschreibung gar nicht zur Diskussion, weswegen in der Tabelle auch nur Silben mit einfacher Coda aufgeführt sind.
Interessanterweise darf aber keine Schärfungsschreibung stehen, wenn eine kurze geschlossene Silbe von einer konsonantisch anlautenden Silbe gefolgt wird.
Es gibt Wörter wie \textit{Rin.de} und \textit{pol.ter}, aber Wörter wie *\textit{Rinn.de} oder \textit{poll.ter} sind als simplexe Wörter ungrammatisch.%
\footnote{Alle Beispiele, die man findet, sind komplexe Formen wie \textit{rann-te}.
Wenn die Silbengrenze mit einer Morphgrenze zusammenfällt, gelten andere Gesetzmäßigkeiten, vgl.\ Abschnitt~\ref{sec:konstanz}.}
Das lässt sich vor allem damit begründen, dass in der gegebenen Position keine lange geschlossene Silbe stehen kann und die Schärfungsschreibung daher nicht benötigt wird.

Im Bereich der langen Silben finden wir bei den langen offenen Einsilblern wie \textipa{[kni:]}, \textipa{[Su:]} und \textipa{[Ke:]} immer eine Dehnungsschreibung (\textit{Knie}, \textit{Schuh}, \textit{Reh}).
Ausnahmen findet man im Bereich jenseits der Substantive, Verben und Adjektive (\zB \textit{je}, \textit{zu}) oder in Fachwörtern (\zB \textit{Re}).
In allen anderen Fällen mit langem Vokal ist der Gebrauch der Dehnungsschreibung nicht obligatorisch (\textit{Kuh.le} vs.\ \textit{Schu.le}, \textit{Ruhm} vs.\ \textit{Glut} usw.).
Lediglich \textit{ie} ist eine obligatorische Dehnungsschreibung.

\index{Dehnungsschreibung}
\index{Schärfungsschreibung}
Um die Schärfungsschreibungen an der Silbengrenze geht es in Abschnitt~\ref{sec:silbengelenk} nochmals, nachdem Tabelle~\ref{tab:dehnkuerzschreibvert} die Verteilung der Dehnungs- und Schärfungsschreibungen zusammenfasst.
Die Tabelle bezieht sich bei den Schärfungsschreibungen nur auf Fälle, in denen diese überhaupt möglich ist, also bei einfacher Coda (nicht Fälle wie \textit{Kalk} oder \textit{Kind}).

\begin{table}
  \centering
    \begin{tabular}{llll}
      \lsptoprule
      \multirow{4}{*}{\rotatebox{90}{\textbf{kurz}}}

        & \multirow{2}{*}{\rotatebox{90}{\textbf{offen}}}
	  & \textbf{einsilb.}  & \Nono \\
	&& \textbf{zweisilb.}  & SS obligatorisch \\

        & \multirow{2}{*}{\rotatebox{90}{\textbf{gesch.}}}
	& \textbf{einsilb.}  & SS obligatorisch \\
        && \textbf{zweisilb.}  & SS ausgeschlossen \\

	\midrule

	\multirow{4}{*}{\rotatebox{90}{\textbf{lang}}}

	& \multirow{2}{*}{\rotatebox{90}{\textbf{offen}}}
	  & \textbf{einsilb.}  & DS obligatorisch \\
	&& \textbf{zweisilb.}  & DS fakultativ \\

	& \multirow{2}{*}{\rotatebox{90}{\textbf{gesch.}}}
	  & \textbf{einsilb.}  & DS fakultativ \\
	&& \textbf{zweisilb.}  & DS fakultativ \\

      \lspbottomrule
    \end{tabular}
  \caption{Verteilung der Schärfungsschreibungen (SS) und Dehnungsschreibungen (DS) nach Silbentyp und Folgesilbe}
  \label{tab:dehnkuerzschreibvert}
\end{table}

\subsection{\textit{h} zwischen Vokalen}

\label{sec:intervokh}

In Wörtern wie \textit{wehe} /\textipa{ve:@}/, \textit{Ruhe} /\textipa{Ku:@}/, \textit{fliehe} /\textipa{fli:@}/, \textit{Krähe} /\textipa{kKE:@}/ usw.\ wird jeweils ein \textit{h} geschrieben, das genau wie die Schärfungs- und Dehnungsschreibungen nicht segmental gelesen wird.
Es entspricht also in der Phonologie nicht einem /\textipa{h}/.
Da die Erstsilben in diesen Fällen alle lang sein müssen, weil sie offen sind, könnte man einfach davon ausgehen, dass es eine Dehnungsschreibung ist.
Die Tatsache, dass dieses \textit{h} allerdings mit der \textit{e}-Dehnung in \textit{fliehe}, \textit{wiehern} und anderen Wörtern zusammen vorkommt, ist ein Hinweis darauf, dass es als Zusatzfunktion die Silbengrenze zwischen zwei Vokalen markiert.
Außerdem ist dieses \textit{h} obligatorisch, wenn eine offene Silbe und eine vokalisch anlautende Silbe aufeinandertreffen, und Dehnungsschreibungen sind eigentlich nie obligatorisch, sondern fakultativ.%
\footnote{Das \textit{h} wird nach Diphthongen allerdings konsequent nicht geschrieben (\zB \textit{Reue} und \textit{Kleie} statt *\textit{Reuhe} und *\textit{Kleihe}).}

Man kann daher annehmen, dass die eigentliche Funktion des \textit{h} hier ist, den Anlaut der zweiten Silbe graphisch zu kennzeichnen.
In Schreibungen wie *\textit{wee} (statt \textit{wehe}), *\textit{Rue} (statt \textit{Ruhe}), *\textit{fliee} (statt \textit{fliehe}) und *\textit{Kräe} (statt \textit{Krähe}) wären sonst die Silbengrenzen nicht nur schlecht graphisch markiert, sondern es käme auch zu Ambiguitäten.
Zum Beispiel könnte \textit{wee} auch einfach mit \textit{e} als Dehnungsschreibung für /\textipa{ve:}/ stehen (parallel zu \textit{Schnee}).
Dafür, dass auch Schreibungen wie \textit{fliehst} dann nicht als doppelte Dehnungsschreibung (\textit{e} und \textit{h}) betrachtet werden müssen, wird in Abschnitt~\ref{sec:konstanz} argumentiert.

\subsection{Silbengelenke}

\label{sec:silbengelenk}

Um die Verteilung der kurzen Vokale auf die Silbentypen und einige Regularitäten der Schärfungsschreibungen zu erklären, gibt es eine spezielle phonologische Theorie, die wir hier als Ergänzung zu Kapitel~\ref{sec:phonologie} einführen.
Man schafft dabei die kurzen offenen Silben für das Deutsche ganz ab und führt das Silbengelenk in die Beschreibung ein.
Definition~\ref{def:silbengelenk} formuliert die Theorie und Abbildung~\ref{fig:silbgel001} zeigt eine illustrative Analyse mit Silbengelenk für das Wort \textit{Lippe} /\textipa{lIp@}/.

\index{Coda}
\index{Onset}
\Definition{Silbengelenk}{
\label{def:silbengelenk}
Das Silbengelenk ist ein Konsonant, der gleichzeitig die Coda einer Silbe und den Onset der im selben Wort folgenden Silbe füllt.
Die Annahme des Silbengelenks erlaubt es, alle scheinbaren kurzen offenen Silben im deutschen Kernwortschatz als geschlossen mit Silbengelenk-Coda zu analysieren.
\index{Silbengelenk}
}

\begin{figure}[h!]
  \centering
  \Tree{
    && &&& \K{Wort}\B{dlll}\B{drrr} \\
    && \K{Silbe}\B{dll}\B{d}\B{drr} &&&&&& \K{Silbe}\B{dll}\B{d} \\
    \K{Onset}\B{d} && \K{Nukleus}\B{d} && \K{Coda}\B{dr} && \K{Onset}\B{dl} && \K{Nukleus}\B{d} \\
    \K{\textipa{[l]}} && \K{\textipa{[I]}} &&& \K{\textipa{[p]}} &&& \K{\textipa{[@]}}\\
  }
  \caption{Beispiel einer Analyse mit Silbengelenk}
  \label{fig:silbgel001}
\end{figure}

Es ist zu beachten, dass am Silbengelenk eben nicht zwei Konsonanten vorliegen (also *\textipa{[lIp.p@]}), sondern ein einziger Konsonant, der in zwei Positionen einer Struktur steht.
Damit muss man keine kurzen offenen Silben mehr annehmen, sondern postuliert in der einzigen Konfiguration, in der sie scheinbar auftauchen, immer eine durch das Silbengelenk geschlossene Silbe.
Besonders elegant wird diese Analyse aus graphematischer Sicht dadurch, dass sie die Schärfungsschreibungen an der Silbengrenze besser motivieren kann.
An der Silbengrenze kann man die Schärfungsschreibung als Kennzeichen des Silbengelenks auffassen, womit Silbentrennungen wie \textit{Lip- pe} usw.\ plötzlich systematisch wirken.
Besonders verhalten sich Wörter mit \textit{ch} und \textit{sch} am Silbengelenk, da diese Di- und Trigraphen nicht verdoppelt werden (\textit{Tasche}, \textit{Küche}), wahrscheinlich weil sonst der optische Eindruck bzw.\ die Lesbarkeit leiden würde (*\textit{Taschsche}).
Bei den Affrikaten gibt es eine Zweiteilung.\index{Affrikate!Schreibung}
Einerseits wird /\textipa{\t{pf}}/ auch nicht gedoppelt (\textit{Schnepfe}), andererseits existiert für /\textipa{\t{ts}}/ eine besondere Silbengelenkschreibung \textit{tz} (\textit{Ritze}).

\index{Auslautverhärtung!am Silbengelenk}
Eine wichtige (bisher wohl weitgehend unbeachtete) Forderung ergibt sich aus der Theorie vom Silbengelenk.
Wenn der Konsonant, der das Silbengelenk darstellt, gleichzeitig in einer Coda und einem Onset steht, kann er nicht stimmhaft sein, denn in Codas wirkt die Auslautverhärtung.
\label{abs:robbe}Passend dazu gibt es auch nur sehr wenige Wörter mit stimmhaftem Silbengelenk, \zB \textit{Kladde}, \textit{Robbe} oder \textit{Bagger} (zu stimmhaften \textit{s}-Silbengelenken wie in \textit{quasseln} folgt in Abschnitt~\ref{sec:eszett} mehr).
Alle diese Wörter sind aus dem niederdeutschen Bereich entlehnt.
Auch das zunächst vielleicht unauffällige Wort \textit{Bagger} ist relativ frisch aus dem Niederländischen entlehnt.
Diese Wörter bilden (in diesem Fall bedingt durch ihre Herkunft) eine kleine Klasse, die sich nicht nach den allgemeinen Regularitäten verhält, und sie gehören damit nicht zum Kernwortschatz.

\newpage

Abschließend wird die wichtige Generalisierung zur Obligatorizität der Gelenkschreibungen in Satz~\ref{satz:gelenke} festgehalten.

\Satz{Prinzip der Gelenkschreibung}{\label{satz:gelenke}
Im trochäischen Simplex des Kernwortschatzes ist bei kurzvokalischer Erstsilbe immer genau eine Gelenkschreibung erforderlich.
Entweder treffen zwei Konsonanten aufeinander (\zB \textit{Rinde}) oder es steht eine explizite Silbengelenkschreibung (\zB \textit{Futter}).
Die beiden Schreibungen schließen einander prinzipiell aus.
\index{Schreibprinzip!Gelenkschreibung}
}

\subsection[Eszett an der Silbengrenze]{\Opsional Eszett an der Silbengrenze}

\label{sec:eszett}

Auch die Verwendung des Eszett \textit{ß} an der Silbengrenze ist jetzt relativ einfach einzuordnen.
Aus grammatischer Sicht bietet es sich an, die Frage nach \textit{ss} und \textit{ß} unter Hinzuziehung des einfachen \textit{s} zu erörtern.
Die Regel, dass nach langem Vokal \textit{ß} (\textit{Maß}) steht und nach kurzem Vokal \textit{ss} (\textit{krass}), ist nämlich prinzipiell nicht falsch.
Aus ihr lässt sich aber nicht ableiten, warum \zB \textit{Mus} nicht *\textit{Muß} (vgl. \textit{Fuß}) und \textit{was} nicht *\textit{wass} (vgl. \textit{Hass}) geschrieben wird.
Ganz konkret ist es für /\textipa{s}/ nach langem oder kurzem Vokal im Wortauslaut schlicht nicht ganz systematisch (wenn auch systematischer als vor der Reform von 1996) geregelt, ob einfaches \textit{s} steht oder auf \textit{ß} bzw.\ \textit{ss} ausgewichen wird (aber vgl.\ auch Abschnitt~\ref{sec:konstanz}).
Im Rahmen der Silbengelenkschreibungen ist die Betrachtung eines Kontextes, in dem die drei \textit{s}-Schreibungen jede eine eigene phonologische Variante kodieren, viel interessanter.
Es bieten sich die Wörter in (\ref{ex:phonschrift4000}) in Zusammenhang mit den Analysen in Abbildung~\ref{fig:busen} an.

\begin{exe}
  \ex\label{ex:phonschrift4000} 
  \begin{xlist}
    \ex{\label{ex:phonschrift4000a} Busen}
    \ex{\label{ex:phonschrift4000b} Bussen}
    \ex{\label{ex:phonschrift4000c} Bußen}
  \end{xlist}
\end{exe}

\begin{figure}[h!]
  \centering
  \Tree{
    &&& \K{Wort}\B{dll}\B{drr} \\
    & \K{Silbe}\B{dl}\B{dr} &&&& \K{Silbe}\B{dl}\B{dr} \\
    \K{Onset}\B{d} && \K{Nuk}\B{d} && \K{Onset}\B{d} & \K{Nuk}\B{d} & \K{Coda}\B{d} \\
    \K{\textipa{[b]}} && \K{\textipa{[u:]}} && \K{\textipa{[z]}} & \K{\textipa{[@]}} & \K{\textipa{[n]}} \\
  }
  \Tree{
    &&& \K{Wort}\B{dll}\B{drr} \\
    & \K{Silbe}\B{dl}\B{dr} &&&& \K{Silbe}\B{dl}\B{dr} \\
    \K{Onset}\B{d} & \K{Nuk}\B{d} & \K{Coda}\B{dr} && \K{Onset}\B{dl} & \K{Nuk}\B{d} & \K{Coda}\B{d} \\
    \K{\textipa{[b]}} & \K{\textipa{[U]}} && \K{\textipa{[s]}} && \K{\textipa{[@]}} & \K{\textipa{[n]}} \\
  }
  \Tree{
    &&& \K{Wort}\B{dll}\B{drr} \\
    & \K{Silbe}\B{dl}\B{dr} &&&& \K{Silbe}\B{dl}\B{dr} \\
    \K{Onset}\B{d} & \K{Nuk}\B{d} & \K{Coda}\B{dr} && \K{Onset}\B{dl} & \K{Nuk}\B{d} & \K{Coda}\B{d} \\
    \K{\textipa{[b]}} & \K{\textipa{[u:]}} && \K{\textipa{[s]}} && \K{\textipa{[@]}} & \K{\textipa{[n]}} \\
  }
  \caption{Analysen der Silbenstruktur Wörter \textit{Busen}, \textit{Bussen} und \textit{Bußen}}
  \label{fig:busen}
\end{figure}

\index{Silbengelenk!und Eszett}
Weiter oben wurde festgestellt, dass die Theorie vom Silbengelenk es erlaubt, anzunehmen, dass es im Deutschen gar keine offenen kurzen Silben gibt.
Diese Analyse für \textit{Busen} ist damit insofern in Einklang, als die Erstsilbe zwar offen, aber lang ist.
Die zweite Silbe \textipa{[z@n]} beginnt mit einem stimmhaften /\textipa{z}/.
Das ist eigentlich typisch, denn überwiegend ist /\textipa{z}/ ja auf den Silbenanfang und /\textipa{s}/ auf das Silbenende verteilt.

Im Wort \textit{Bussen} ist die Erstsilbe kurz und dank Silbengelenk geschlossen, worauf die Silbengelenkschreibung \textit{ss} hinweist.
In solchen Wörtern sollte ein /\textipa{z}/ nicht möglich sein, denn durch die Silbengelenkposition steht das Segment ja stets in einer Coda, in der (wie oben angemerkt) die Auslautverhärtung wirkt und jedes /\textipa{z}/ zu \textipa{[s]} macht.
Das ist der Grund, warum aus Dialekten kommende Wörter mit kurzer offener Erstsilbe und stimmhaftem /\textipa{z}/ anlautender Zweitsilbe so schlecht ins Gesamtsystem passen und sich auch schlecht verschriften lassen.
Ein gutes Beispiel ist \textit{quasseln}, das angesichts der Schreibung und den phonotaktischen Regularitäten (analog zu \textit{prasseln} usw.) \textipa{[kva:s@ln]} realisiert werden sollte, bei vielen Sprechern aber \textipa{[kvaz@ln]} realisiert wird.
Im Unterschied zu den oben erwähnten \textit{Bagger} und \textit{Robbe} hat man den in Abschnitt~\ref{sec:konssegschreib} beschriebenen Nachteil, dass für /\textipa{s}/ und /\textipa{z}/ nicht zwei Buchstaben verfügbar sind, für /\textipa{k}/ und /\textipa{g}/ usw.\ aber schon.
Es wäre daher günstig, wenn wir auch für das phonologische System ohne eine Opposition von /\textipa{s}/ und /\textipa{z}/ auskämen, was sich im nächsten Absatz abzeichnet.

\textit{Bußen} hat in der hier vertretenen Analyse eine geschlossene lange Erstsilbe (nach demselben Typus wie \textit{Mus}), und die zweite Silbe beginnt mit einem /\textipa{s}/ oder /\textipa{z}/.%
\footnote{Diese Wörter sind im Übrigen sehr selten.
Phonotaktisch sind sie Wörtern wie \textit{Wiebke} sehr ähnlich, die ebenfalls im Kernwortschatz nicht als Simplizia existieren bzw.\ selten sind (vgl\ S.~\pageref{abs:wiebke}).}
Die \textit{s}"=Laute fallen in einem Silbengelenk zusammen, und der Anlaut der zweiten Silbe wird in jedem Fall von der Auslautverhärtung erfasst.
Das Segment, das diesen Anlaut füllt, steht eben gleichzeitig in der Coda der Erstsilbe.
Bei dieser Analyse entfällt die Notwendigkeit, /\textipa{s}/ und /\textipa{z}/ als zwei unterschiedliche zugrundeliegende Segmente (zwei Phoneme) aufzufassen.
Es würde reichen, immer /\textipa{z}/ anzunehmen, und die Auslautverhärtung /\textipa{z}/ zu /\textipa{s}/ verhärten zu lassen, wenn es in einer Coda bzw.\ einem Silbengelenk steht.
Damit wäre im System der primären Konsonantenschreibung die letzte Doppelbelegung (\textit{s} für /\textipa{z}/ und /\textipa{s}/) auch beseitigt.
Es wird daher hier vorgeschlagen, dass man \textit{ß} als eine nicht-kürzende Silbengelenkschreibung oder eine kombinierte Dehnungs- und Silbengelenkschreibung auffassen kann.

\subsection{Betonung und Hervorhebung}

\label{sec:hervorhebung}
\index{Majuskel}
\index{Akzent!Schreibung}

Über Majuskeln und Minuskeln wurde noch nichts gesagt, weil die Unterscheidung zwischen ihnen für die phonologische Seite der Graphematik keine Rolle spielt (aber s.\ Kapitel~\ref{sec:andereschrift}).
Auf jeden Fall sind die meisten Buchstaben in deutschen Texten Minuskeln, und Majuskeln sind seltener und markieren stets besondere Funktionen.
Im Bereich der Gebrauchsschreibungen gibt es nun Phänomene, die möglicherweise einen phonologischen Effekt kodieren, der in der Standardschreibung niemals markiert wird.
Die Beispiele in (\ref{ex:phonschrift6000}) zeigen das Phänomen.%
\footnote{Alle Belege stammen aus dem Korpus DECOW14AX (\url{https://webcorpora.org}) und sind über die URL darin dauerhaft auffindbar.}

\begin{exe}
  \ex\label{ex:phonschrift6000} 
  \begin{xlist}
    \ex{\label{ex:phonschrift6000a} Genau DAS ist das Problem!\footnote{\url{http://forum.rundschau-online.de/archive/index.php/t-321.html}}}
    \ex{\label{ex:phonschrift6000b} ICH MUSS WEG!\footnote{\url{http://www.meinliebeskummer.de/forum/archive/index.php/t-41-p-28.html}}}
    \ex{\label{ex:phonschrift6000c} Glaubensasche - Fragetasche [\ldots] wenn ich nur nicht immer an die TRAGEtasche - gelb und von Ikea - denken müßte dabei.\footnote{\url{http://www.vonwolkenstein.de/forum/archive/index.php?t-1468.html}}}
  \end{xlist}
\end{exe}

In diesen Sätzen werden ganze Wörter (\textit{DAS}), ganze Sätze (\ref{ex:phonschrift6000b}) und Teile von Wörtern (\textit{TRAGEtasche}) in Majuskeln geschrieben.
Das ist höchst auffällig, weil sonst nur einzelne Buchstaben an Wortanfängen als Majuskel geschrieben werden können.
Hier findet offensichtlich eine Art von Hervorhebung statt, und zwar jenseits der orthographischen Norm, also als Gebrauchsschreibung.
Hervorhebung ist ein schlecht definierter Begriff, und man würde vielleicht gerne die Funktionen dieser Majuskelschreibungen genauer benennen.
In (\ref{ex:phonschrift6000a}) wird offensichtlich \textit{DAS} (bzw.\ das, worauf es sich anaphorisch bezieht) als der Gegenstand oder Sachverhalt hervorgehoben, über den dann gesagt wird, er sei das Problem.
In (\ref{ex:phonschrift6000b}) wird evtl.\ dem ganzen Satz Emphase verliehen, analog zu einem lauten Sprechen.
In (\ref{ex:phonschrift6000c}) findet sehr deutlich eine Kontrastierung statt, indem die \textit{Tragetasche} der \textit{Fragetasche} gegenübergestellt wird.
Ganz offensichtlich gibt es nicht eine einzige Funktion, die man Majuskelschreibungen zuordnen kann, sondern mehrere.

Zumindest wenn wie in (\ref{ex:phonschrift6000a}) und (\ref{ex:phonschrift6000c}) einzelne Wörter in Majuskeln stehen, kann man aber sehr wahrscheinlich eine Beziehung zur Phonologie herstellen.
Alle diese Wörter würden in der Aussprache eine prominente Betonung erhalten, um phonologisch eine ähnliche Funktion zu markieren, wie es durch die Majuskeln graphematisch geschieht.
So ergäbe sich, wenn auch nur sehr begrenzt systematisch, ein neuer Anknüpfungspunkt zwischen Graphematik und Prosodie.
Dieses Phänomen ist allerdings noch nicht hinreichend untersucht, und es gibt keine eindeutigen abgesicherten Ergebnisse.
Es wurde hier aufgenommen, um zu zeigen, dass unsere Schreibungen nicht nur mechanisch einer Norm folgen (oder eben gegen diese verstoßen), sondern dass Schreiber Möglichkeiten des graphematischen Systems auch kreativ ergreifen können, um ihre Sprache möglichst erfolgreich zu kodieren.

\section{Ausblick auf den Nicht-Kernwortschatz}

\label{sec:nichtkernschreib}
\index{Kernwortschatz}

Im Nicht-Kernwortschatz finden sich diverse phonologische und graphematische Abweichungen zum Kernwortschatz.
Dabei muss man bedenken, dass es keine scharfe Trennung zwischen zwei Extremen im Wortschatz gibt, sondern geringere und größere Nähe zum Kern.
Alle Wortformen von Wörtern, die nicht deriviert oder komponiert und dabei nicht einsilbig (\textit{Maus}, \textit{gehst}) oder trochäisch mit kurzer Zweitsilbe (\textit{backe}, \textit{alten}, \textit{Brüdern}) oder daktylisch mit kurzer Zweit- und Drittsilbe (\textit{ruderest}, \textit{älteren}) sind, sind zumindest näher am Rand als die Wörter, die diese Bedingungen erfüllen.
Mit einem so eng gefassten Kern wird man natürlich dem Gesamtsystem nicht wirklich gerecht.

Einen erweiterten Kern erhalten wir durch Hinzuziehen von derivierten und komponierten Wörtern.
Hier findet man dann vor allem unbetonte Präfixe vor trochäischen und daktylischen Füßen (\textit{veränderst}, \textit{überredetest}), wohingegen sich die Suffixe normalerweise unbetont nach den einsilbigen oder trochäischen Stämmen einsortieren und damit neue Trochäen und Daktylen erzeugen (\textit{Haltung}, \textit{Schreiber}, \textit{Gläubigkeit}).
Präfigierung und Suffigierung treten natürlich auch zusammen auf (\textit{Unterhaltung}).
Weiters gibt es dann überwiegend mehrfüßige Komposita, die Ergebnisse von Präfigierung und Suffigierung enthalten können (\textit{Häuserfronten}, \textit{Unterhaltungsführung}).
Die Verschriftung dieser Wörter folgt ganz einfach aus den Kernprinzipien, vor allem wegen der in Abschnitt~\ref{sec:konstanz} beschriebenen Prinzipien der Konstantschreibung.

Weiter vom Kern entfernt sind Simplizia, die mehrere lange gespannte Vokale enthalten, womit oft eine atypische Fußstruktur einhergeht (\textit{Oma}, \textit{Politik}, \textit{Organigramm}).
In dieser Gruppe finden wir auch die in Abschnitt~\ref{sec:schwachsubst} besprochenen entlehnten schwachen Substantive mit betonten Letztsilben wie \textit{Apologet}, \textit{Ignorant}, \textit{Demiurg} usw.
Zumindest vom Betonungsmuster ähnlich sind die in Abschnitt~\ref{sec:deutscherwortakzent} kurz diskutierten w-Adverben mit Endbetonung wie \textit{warum}, \textit{weshalb} usw.

Einen ganz eigenen dem Kern sehr fernen Bereich erhält man durch Hinzunahme von Wörtern, die Segmente enthalten, die es im Kern gar nicht gibt, oder die es in der jeweiligen Position nicht gibt.
Hierzu gehören Wörter wie in (\ref{ex:phonschrift777}), wo die Transkription sicherheitshalber phonetisch erfolgt, weil die Bestimmung der zugrundeliegenden Form weitere Probleme mitbringt.

\begin{exe}
  \ex\label{ex:phonschrift777}
  \begin{xlist}
    \ex{\label{ex:phonschrift777a} Chips \textipa{[\t{tS}Ips]}}
    \ex{\label{ex:phonschrift777b} Dschungel \textipa{[\t{dZ}UN@l]}}
    \ex{\label{ex:phonschrift777e} Chuzpe \textipa{[XU\t{ts}p@]}}
    \ex{\label{ex:phonschrift777c} Pteranodon \textipa{[ptEKanodOn]}}
    \ex{\label{ex:phonschrift777d} mailen \textipa{[m\t{Ee}l@n]}, \textipa{[m\t{EI}l@n]}}
  \end{xlist}
\end{exe}

In (\ref{ex:phonschrift777a}) steht \textipa{[\t{tS}]} in einer Position, in der es überwiegend nicht steht.
Einer der Gründe, phonologisch \textipa{[\t{tS}]} im Deutschen nicht als echte Affrikate zu klassifizieren, ist gerade, dass es zwar in der Coda vorkommt (\textit{Matsch}) aber eben nicht im Onset (s.\ Vertiefung~\ref{vert:affrikaten} aus S.~\ref{vert:affrikaten}).
Wenn nicht auf die angepasstere Realisierung \textipa{[SIps]} ausgewichen wird, steht \textit{Chips} also außerhalb des Kerns.
Noch mehr gilt dies für (\ref{ex:phonschrift777b}), weil \textipa{[\t{dZ}]} im Kern in gar keiner Position vorkommt.
Das Wort \textit{Chuzpe} hat \textipa{[X]} im Silbenanlaut, wo es nicht hingehört.
Das Plateau \textipa{[pt]} in (\ref{ex:phonschrift777c}) ist im Kern völlig ausgeschlossen (und eine typische Realisierung von deutschen Sprechern dürfte daher wahrscheinlich \textipa{[p@tEKanodOn]} sein).
Schließlich enthält \textit{mailen} (wenn nicht auch hier auf die kommodere Realisierung \textipa{[me:l@n]} ausgewichen wird) einen Diphthong, den es im Kernwortschatz nicht gibt.

Wichtig ist, dass man an diesen Beispielen gut zeigen kann, warum man sie nicht in die Beschreibung des Kernwortschatzes aufnehmen sollte.
Würde man \textit{Chuzpe} \zB als konform zu den allgemeinen Generalisierungen beschreiben wollen, müsste man diese Generalisierungen anpassen, und die ansonsten sehr gut funktionierende Beschreibung der Verteilung von \textipa{[\c{c}]} und \textipa{[X]} wäre dahin.
Gerade weil diese Wörter selten sind und oft nur in bestimmten Registern und Stilen vorkommen, wäre dies mehr als ungeschickt.

Wie man jetzt die oben in Abschnitt~\ref{sec:silbengelenk} beschriebenen Wörter wie \textit{Robbe}, \textit{Bagger} und \textit{quasseln} einordnen möchte, ist nicht von großer Tragweite.
Sie gehören auf jeden Fall aus gut benennbaren Gründen (vgl.\ S.~\pageref{abs:robbe}) nicht direkt zum Kern und bilden dabei aber eine eigene kleine Klasse.
Letztlich gilt genau das aber auch für die endbetonten schwachen Substantive wie \textit{Linguist}, für w-Adverben wie \textit{warum} und \textit{wieso} usw.
Die Nähe zum Kern auf einer absoluten Skala messen zu wollen, ist nicht zielführend.
Angemessener ist die Annahme, dass die Grammatik es erlaubt, dass für einzelne Wörter oder Wortklassen eigene Regularitäten existieren, die von den ganz großen Regularitäten abweichen.

Es sind nun nicht alle diese Arten von kernfernen Wörtern gleichermaßen anfällig für Anomalien in der Schreibung.
Ganz besonders sticht die Gruppe der zu (\ref{ex:phonschrift777}) ähnlichen Lehnwörter heraus, die oft die Schreibung der Gebersprache konservieren.
Hierbei ist zu beachten, dass viele Lehnwörter phonologisch Wörter des Kernwortschatzes sind, aber trotzdem eine kernferne Schreibung aufweisen.
Ein Wort wie \textit{Christen} (statt *\textit{Kristen}) ist phonologisch in keiner Form auffällig, sticht aber durch die Schreibung \textipa{[chr]} für /\textipa{kK}/ heraus.
Ähnliches gilt für \textit{Vase} (statt *\textit{Wase}) oder \textit{Beamer} (statt *\textit{Biemer}).
Im Bereich der irregulären Schreibungen gibt es eine breite Variation (mit und ohne phonologische Auffälligkeit), die hier nicht im Einzelnen besprochen werden soll (s.\ Übung~\ref{u146}).
Beispiele sind \textit{chthonisch}, \textit{Genre}, \textit{Gonorrhö}, \textit{Pendant}, \textit{Souvenir}, \textit{Shopping}, \textit{Theorie}, \textit{zynisch}.

\Zusammenfassung

\begin{enumerate}
  \item Wenn kein kognitiver Realismus angestrebt wird und empirisch überwiegend auf geschriebene Daten zurückgegriffen wird, sind Grammatik und Graphematik nicht voneinander zu trennen.
  \item Zu jedem zugrundeliegenden Segment des Deutschen korrespondiert eine primäre Buchstabenschreibung (bei den Vokalen jeweils eine für den kurzen und den langen Vokal zusammen).
  \item Dehnungsschreibungen sind nur in langen offenen Einsilblern zuverlässig anzutreffen, ansonsten eine fakultative Kennzeichnung der Vokallänge.
  \item Das Silbengelenk ist eine besondere strukturelle Position zwischen zwei Silben, die Coda und Onset vereint.
  \item Unter Annahme des Silbengelenks gibt es im Kernwortschatz keine betonten kurzen offenen Silben.
  \item Schärfungsschreibungen stehen immer in kurzen geschlossenen Einsilblern und am Silbengelenk, sonst nie.
  \item Wenn \textit{h} an der Silbengrenze zwischen Vokalen steht, markiert es primär den Anfang der zweiten Silbe (ohne Onset) und ist (wenn überhaupt) nur sekundär eine Dehnungsschreibung.
  \item Das \textit{ß} ist eine kombinierte Dehnungs- und Silbengelenkschreibung.
  \item Zugehörigkeit zum Kernwortschatz ist graduell, und typischerweise gibt es Gruppen von Wörtern (kleine Klassen), die auf gleiche Weise von den Regularitäten des Kernwortschatzes abweichen.
  \item Hauptquelle für anomale Schreibungen sind Lehnwörter, die die Schreibung der Gebersprache konservieren, was allerdings nicht notwendig mit einer anomalen Phonologie einhergehen muss.
\end{enumerate}

\Uebungen

\Uebung \label{u141} In Tabelle~\ref{tab:dehnkuerzschreib} (S.~\pageref{tab:dehnkuerzschreib}) fehlen die Vokale /\textipa{y}/, /\textipa{Y}/ und /\textipa{\o}/, /\textipa{\oe}/.
Finden Sie Beispiele für diese Vokale und jede mögliche Zeile der Tabelle.

\Uebung[\tristar] \label{u142} Argumentieren Sie dafür, dass die Diphthonge in Tabelle~\ref{tab:dehnkuerzschreib} (S.~\pageref{tab:dehnkuerzschreib}) nicht aufgeführt sein müssen.

\Uebung[\tristar] \label{u143} Warum ist es angesichts des phonologischen und graphematischen Systems des Deutschen folgerichtig, dass der glottale Plosiv wie in \textipa{[PEnd@]} nicht durch einen Buchstaben verschriftet wird.

\Uebung \label{u144} Finden Sie in den folgenden Beispielen alle Dehnungs- und Schärfungsschreibungen.
Welche Dehnungsschreibungen sind nach den allgemeinen Regularitäten optional?
Schreiben Sie die entsprechenden Wörter jeweils ohne Dehnungsschreibung.
Finden Sie außerdem alle Silben, in denen eine Dehnungsschreibung möglich wäre, aber keine steht.
Schreiben Sie die entsprechenden Wörter jeweils mit Dehnungsschreibung.

\begin{enumerate}\Lf
  \item Auf dem Wohnungsmarkt ist Entspannung eingekehrt.
  \item Der König von Schweden hatte angeblich Kontakte zur Unterwelt.
  \item Eine Leseprobe endete in einer wüsten Schlägerei.
  \item Unter einer einstweiligen Verfügung kann sich Ischariot nichts vorstellen.
  \item Mit Möhren kann Vanessa ihr Pferd glücklich machen.
  \item Sie fragen sich jetzt sicher, wer die Stallpflege übernimmt.
  \item Passen Sie beim Einsteigen auf Ihr Knie auf. 
\end{enumerate}

\Uebung[\tristar] \label{u145} Warum können wir davon ausgehen, dass innerhalb des Kernwortschatzes in trochäischen Simplizia außer denen vom Typ \textit{Wehe}, \textit{Ruhe}, \textit{Krähe} usw. (Abschnitt~\ref{sec:intervokh}) phonologisch der Onset der zweiten Silbe immer gefüllt ist?

\Uebung \label{u146} Was macht die folgenden Wörter zu Schreibungen jenseits des Kerns?

\begin{enumerate}\Lf
  \item chthonisch
  \item Genre
  \item Gonorrhö
  \item Pendant
  \item Souvenir
  \item Shopping
  \item Theorie
  \item zynisch
\end{enumerate}

