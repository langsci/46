\chapter{Phonologische Schreibprinzipien}

\label{sec:phonschrift}

\section{Status der Graphematik}

\subsection{Graphematik als Teil der Grammatik}

\label{sec:graphegrammatik}

Der letzte Teil dieses Buchs hat nur zwei Kapitel und wirkt eventuell wie ein Anhang zu den anderen Teilen.
Es stellt sich die Frage, ob es legitim ist, die \textit{Graphematik} als Beschreibung und Analyse der Schrift -- oder besser der \textit{Schreibung} -- so weit ans Ende zu stellen, und ihnen damit nur geringen Raum und scheinbar geringeres Gewicht zu geben.
Hinter dieser Frage verbirgt sich die theoretische Grundsatzentscheidung, ob das System der Schreibungen als Teil eines allgemeinen Systems der Grammatik angesehen werden soll, oder ob es ein zur Grammatik externes System ist, das lediglich starke Verbindungen zur Grammatik aufweist und Phänomene der Grammatik ggf.\ nachbildet.
Dazu muss jetzt etwas ausgeholt werden.
Die Beispiele am Ende dieses Abschnitts illustrieren dann die theoretischen Überlegungen.%
\footnote{Es handelt sich in diesem Abschnitt stärker als sonst in diesem Buch um eine persönliche Stellungnahme des Autors.
Der Rest des Kapitels kehrt dann zu einer möglichst neutralen Systembeschreibung zurück.}

Um diese Frage irgendwie beantworten zu können, muss zunächst geklärt werden, was prinzipiell zur Grammatik gehören soll und was nicht.
Man kann Grammatik so verstehen, dass sie die Erforschung der Regularitäten in sprachlichen Äußerungen ist, ohne dass man dabei unbedingt berücksichtigen muss, wie die Sprache im Gehirn produziert oder verstanden wird.
Dabei ist es relativ unproblematisch, sich auf eine (in letzter Konsequenz fiktive) Standardsprache oder Verkehrssprache zu beziehen und als Material auf Sätze aus Textkorpora zurückzugreifen (s.\ Kapitel~\ref{sec:grammatik}).
Dieses Vorgehen ist typisch für die deskriptive Grammatik, wie sie in diesem Buch verstanden wird.
Eine zweite Möglichkeit ist es, Grammatik mit einer Art von \textit{kognitivem Realismus} zu betreiben.
Dabei möchte man ein Grammatikmodell entwickeln, das zu dem System im Gehirn individueller Sprecher, das für die Sprache zuständig ist, äquivalent oder zumindest kongruent ist.
Beide Auffassungen sind legitim und wichtig, wobei die kognitiv-realistische insofern die anspruchsvollere ist, als sie ohne aufwendige Experimente nicht effektiv zu betreiben ist.
Aus diesen zwei Auffassungen von Grammatik bzw.\ Grammatikforschung ergeben sich nun aber auch zwei Möglichkeiten, die Graphematik einzuordnen.

Wenn die Graphematik unter der kognitiv-realistischen Sichtweise zur Grammatik gehören soll, dann müssten wir Evidenz dafür beschaffen, dass die Produktion von graphischen Einheiten (das \textit{Schreiben}) und deren verstehendes Verarbeiten (das \textit{Lesen}) im Gehirn nach denselben Prinzipien ablaufen wie grammatische Prozesse, also die Bildung von Flexionsformen, die Verarbeitung von verschiedenen Satzgliedstellungen usw.
Zu dieser Frage wäre ein scheinbar einschlägiges Argument, dass es viele Sprachen ohne Verschriftung gibt, aber keine Schrift ohne Sprache.
Außerdem lernen Kinder zunächst Sprache ohne Schrift, und die Schrift kommt erst später dazu.
Das lässt die Schrift wie ein Epiphänomen erscheinen, also als einen möglichen (nicht notwendigen) Nebeneffekt der Sprache, aber eben nichts, das auf die Sprache oder Sprachfähigkeit zurückwirkt oder gar für die Existenz von Sprache notwendig ist.
Entkräftet wird dieses Argument teilweise dadurch, dass dies ja nicht notwendigerweise bedeutet, dass die Schreibung nicht trotzdem nach denselben Prinzipien verarbeitet wird wie die Grammatik.
Im Gegenteil wäre es sogar nach allgemeinen Grundsätzen der wissenschaftlichen Reduktion die plausibelste Annahme, solange keine Evidenz gegen diese Annahme vorliegt.
Über solche einfachen Überlegungen hinaus muss man (wie oben schon gesagt) feststellen, dass dem kognitiven Anspruch in der Sprachbeschreibung schwer gerecht zu werden ist, weil er sich letztlich nur über aufwändige experimentelle Verfahren prüfen lässt.
Weder die kognitive Linguistik und Neurolinguistik sind sehr wahrscheinlich in der Lage, hierzu momentan mehr als vorläufige Antworten zu geben.

\index{Graphematik}
Daher ist für uns die zweite Möglichkeit der Einordnung der Graphematik interessanter.
Wie in Kapitel~\ref{sec:grammatik} argumentiert wurde, basiert dieses Buch auf der idealisierten Annahme, dass es eine vergleichsweise einheitliche verschriftete deutsche Verkehrssprache (eine \textit{standardnahe Varietät des Deutschen}) gibt, die weitgehend unabhängig von den Gehirnen ihrer Sprecher untersuchbar ist.
Diese Idealisierung sollte keinen normativen bzw.\ präskriptiven Charakter haben und gerne auch Variation (\zB die zwei Formen \textit{deren} und \textit{derer} aus Abschnitt~\ref{sec:prondefart}) zulassen.%
\footnote{Würde es diese funktionierende Verkehrssprache nicht geben, wäre Sprach- und Grammatikunterricht für Erstsprecher sowie jegliche Form von Fremdsprachenunterricht ausgesprochen schwer, wenn nicht unmöglich.}
Die Grammatik dieses Konstrukts \textit{Standarddeutsch} haben wir näherungsweise beschrieben.
Dies geschah auf eine Weise, dass man (vor allem geschriebene) Sätze daraufhin prüfen kann, ob sie dem hier beschriebenen System von Regularitäten genügen (also relativ zu diesem grammatisch sind) oder nicht.
In diesem Teil des Buchs wird nun gezeigt werden, dass die Schreibung dieses Standarddeutschen auf sehr systematische Weise der Grammatik folgt, und zwar auf den Ebenen der Phonologie, Morphologie und Syntax.
Die Schreibung bringt durchaus zusätzliche eigene Regularitäten mit und erlaubt in Details immer Abweichungen vom System.
Letzteres sehen wir aber in der Grammatik auch immer wieder (\zB echt unregelmäßige Verben wie in Abschnitt~\ref{sec:kleineverbklassen}) und zweifeln dennoch nicht an ihrem Systemcharakter.
Es wäre also überhaupt nicht zielführend, die Graphematik nicht als Teil der Grammatik zu betrachten.

\index{Medium!schriftlich}
Ganz unabhängig von diesen Überlegungen ist es nicht plausibel, die Erscheinungsform von Sprache in einem bestimmten Medium aus der Sprachbetrachtung auszuschließen.%
\footnote{Zu einer kurzen Diskussion der Medienspezifik von Sprache s.\ Abschnitt~\ref{sec:akustischemedium}.}
Strukturalistische Sprachwissenschaftler wie Ferdinand de Saussure (1857--1913) und Leonard Bloomfield (1887--1949) haben im zwanzigsten Jahrhundert die (bis heute oft affirmativ weitergegebene) Auffassung vertreten, die Linguistik habe sich nur mit der gesprochenen Sprache zu beschäftigen und die Schriftlichkeit außer Acht zu lassen.
Es wurde das Schlagwort vom \textit{Primat der gesprochenen Sprache} in die Welt gesetzt, vgl.\ \cite[Kapitel~0]{Duerscheid2012a} für einen Überblick.
Warum aber die Erscheinungsform von Sprache in einem Medium (akustische Symbole) gegenüber der Erscheinungsform in einem anderen Medium (graphische Symbole) höher gewichtet werden sollte, ist nur schwer zu begründen.
Ideen wie die von der \textit{größeren Spontaneität} der mündlichen Sprachproduktion und einer damit einhergehenden größeren \textit{Ursprünglichkeit}, größeren \textit{Unverfälschtheit} und \textit{Unabhängigkeit von Normen} ziehen nicht.
Erstens gibt es keinen Grund, anzunehmen, dass weniger spontan produzierte Sprache nicht auch eine Form natürlicher Sprache ist.
Zweitens müsste wenigstens der Nachweis erbracht werden, dass gesprochene Sprache nicht von Normierungsversuchen betroffen ist.
Dieser Nachweis ist meiner Überzeugung nach nicht zu erbringen.
Drittens sind geschriebene Texte den geringsten Teil der Schriftgeschichte über -- zumal in Ermangelung einer Norm -- sonderlich normnah gewesen, und schon Handschriften aus dem 19.\,Jahrhundert können den modernen Leser mit ausufernder Variation in Erstaunen versetzen.
Das Gleiche gilt für aktuelle spontan und unter geringem Normdruck produzierte Sprache in Foren, Kurznachrichten usw.
Eine umfassende Linguistik und Grammatik sollte kein Medium stigmatisieren, sei es das akustische, das graphische oder \zB das gesturale im Fall der Gebärdensprache.


Mit den Beispielen in (\ref{ex:phonschrift1000}) kann man nun zeigen, dass eine Trennung von Grammatik und Graphematik ganz praktisch nicht ans Ziel führt, wenn eine Art der Sprachbeschreibung wie in diesem Buch angestrebt wird.%
\footnote{Ein erkenntnisleitendes Gedankenspiel ist bei allen diesen Beispielen, warum Programme zur Rechtschreibprüfung an diesen Sätzen nichts zu monieren hätten.}

\begin{exe}
  \ex\label{ex:phonschrift1000} 
  \begin{xlist}
    \ex[*]{\label{ex:phonschrift1000a} Fine findet, das die Schuhe gut aussehen.}
    \ex[*]{\label{ex:phonschrift1000b} Wenn ich Geld hätte, nehme ich den Kopfhörer mit.}
    \ex[*]{\label{ex:phonschrift1000c} Um beruflich voranzukommen, nimmt Fine an der Fortbildung Teil.}
    \ex[*]{\label{ex:phonschrift1000d} Zurückbleibt der Schreibtisch nur, wenn der LKW randvoll ist.}
  \end{xlist}
\end{exe}

Relativ zu der in diesem Buch beschriebenen (nicht normativ verstandenen) Grammatik des (in gewissem Maß fiktiven) Standarddeutschen sind diese Sätze nicht in Ordnung.
Im Rahmen einer Grundschuldidaktik müsste man sich nun bei jedem dieser Sätze fragen, ob ein \textit{Schreibfehler} oder ein \textit{Grammatikfehler} vorliegt.
Das ist eigentlich die völlig falsche Fragestellung, denn man kann sie natürlich alle als simple Verschreibungen klassifizieren.
Genauso kann man sie aber wie folgt als ungrammatisch beschreiben, ohne einen einzigen Rechtschreibfehler zu diagnostizieren.
In (\ref{ex:phonschrift1000a}) steht der Artikel oder (Relativ-)Pronomen \textit{das} (Abschnitt~\ref{sec:artikelpronomen}) an einer Stelle, an der gemäß den Schemata für Komplementsätze (Abschnitt~\ref{sec:komplementsaetze}) der Komplementierer \textit{dass} stehen müsste.
In (\ref{ex:phonschrift1000b}) steht eine Indikativform \textit{nehme} \textipa{[ne:me]} statt der Konjunktivform \textit{nähme} \textipa{[nE:me]}.
Alternativ ist statt des Segments /\textipa{E}/ das Segment /\textipa{e}/ geschrieben worden, ggf.\ weil der Schreiber aus einem Dialektgebiet kommt, wo der Unterschied nicht gemacht wird.%
\footnote{Diese Formulierung ist absichtlich auf \textit{ein Segment schreiben} zugespitzt, vgl.\ Abschnitt~\ref{sec:buchstabensegmente}.}
In (\ref{ex:phonschrift1000c}) ist das Substantiv \textit{Teil} statt der in der Position korrekten Verbpartikel \textit{teil} verwendet worden.
Beispiel (\ref{ex:phonschrift1000d}) ist ein unabhängiger Aussagesatz mit ungefülltem Vorfeld, und das Partikelverb \textit{zurückbleiben} wurde komplett aus dem Verbkomplex herausbewegt, obwohl die Partikel hätte zurückbleiben müssen (Abschnitt~\ref{sec:konstituentenstrukturinv2}, besonders Phrasenschema~\ref{str:v2} auf S.~\pageref{str:v2}).
Diese grammatischen Interpretationen ergeben sich nur, weil die Schreibung sehr engmaschig Merkmale aller grammatischer Ebenen kodiert.
Daher ist es unmöglich, von einer Trennung von Grammatik und Graphematik zu sprechen, sobald man geschriebene Daten berücksichtigt.
Dass die meisten Linguisten sich exzessiv auf geschriebene Daten stützen, macht es umso wichtiger, die Prinzipien der Schreibung als Teil der Grammatik zu berücksichtigen.
Natürlich kann man für jedes Beispiel in (\ref{ex:phonschrift1000}) den Schreiber befragen und versuchen herauszufinden, ob in (\ref{ex:phonschrift1000a}) ein falsch geschriebener Komplementierer oder ein grammatisch falsch gewähltes Pronomen gemeint ist, usw.
Das würde aber an den möglichen Interpretationen für die Beispiele, wie sie da stehen, rein gar nichts ändern.%
\footnote{Abgesehen davon ist es ausgesprochen schwierig, diese Informationen von Schreibern durch explizites Fragen zu erhalten, zumal ohne den Ausgang der Befragung erheblich zu beeinflussen.
Man landet dann sehr schnell wieder in einer Situation, in der ein ordentliches und damit in seiner Durchführung anspruchsvolles Experiment vonnöten wäre.}

Für Beispiel (\ref{ex:phonschrift200}) könnte man nun vermuten, dass hier klar eine einfache Verschreibung vorliegt, die nichts mit dem Verhältnis von Grammatik und Graphematik zu tun hat.

\begin{exe}
  \ex[*]{\label{ex:phonschrift200} Lingusitik ist uninteressant.}
\end{exe}

Auch das ist ein Trugschluss, denn hier ist regelhaft ein phonologisches Wort /\textipa{lInguzitIk}/ kodiert worden.
Dass es dieses Wort sehr wahrscheinlich nicht gibt, und dass wir das gerade wegen der klaren Beziehung von Buchstabenschrift und Phonologie im Deutschen sofort erkennen, ist prinzipiell unabhängig davon, dass beim Tastaturschreiben ohne Zehnfinger-System oft als reiner Unfall \textit{Lingusitik} statt \textit{Linguistik} herauskommt.
Dass Rechtschreibprogramme nur Beispiel (\ref{ex:phonschrift200}) und nicht die Beispiele in (\ref{ex:phonschrift1000}) als falsch klassifizieren würden, liegt eben genau daran, dass diese Programme keinerlei Wissen über Grammatik haben (ausgenommen evtl.\ eingeschränktes Wissen darüber, wie Komposita gebildet werden), sondern einen simplen Abgleich mit großen Datenbanken bekannter Wörter durchführen.

Um damit nun zur Ausgangsfrage zurückzukommen:
Der einzige Grund, warum die Graphematik ganz am Ende des Buchs steht, ist, dass man einen sehr guten Überblick über die gesamte Grammatik haben muss, bevor man die Graphematik verstehen kann.
Damit soll also im Rahmen der deskriptiven Grammatik keine Degradierung der Graphematik an sich verbunden sein.
Das genaue Gegenteil ist der Fall.
Die weiteren Kapitel zeigen hoffentlich eindrücklich, dass dies so ist.

Das Verhältnis der gewachsenen Regularitäten des Schreibsystems und dessen expliziter Normierung -- also der \textit{Orthographie} bzw.\ \textit{Rechtschreibung} -- kann hier nicht hinreichend diskutiert werden.\index{Orthographie}
Auf keinen Fall ist es so, dass das Schreibsystem in irgendeiner Form geplant oder erdacht wurde.
Elemente der gegenwärtigen Schreibung wie die Dehnungs- und Schärfungsschreibungen (vgl.\ Abschnitt~\ref{sec:laengeschreib}), das Interpunktionssystem mit Punkt und Komma im Zentrum (vgl.\ Abschnitte~\ref{sec:hauptsatzschreib} und~\ref{sec:nebensatzschreib}), die Substantivgroßschreibung (vgl.\ Abschnitt~\ref{sec:wortklassschreib}) und selbst uns so elementar erscheinende Dinge wie die Worttrennung durch Spatien (vgl.\ Abschnitt~\ref{sec:spatien}) sind das Ergebnis jahrhundertelanger komplexer Entwicklungen.
Es ist mitnichten \textit{alles genormt} (und muss es auch nicht sein), und man kann sicherlich den meisten Autoren und Reformatoren von Rechtschreibregeln unterstellen, dass sie lediglich versuchen, unsystematischen historischen Ballast im Sinne der existierenden Schreibprinzipien zu systematisieren.%
\footnote{Das ist parallel zur Auffassung von grammatischer Norm als Beschreibung, die in Abschnitt~\ref{sec:normalsbeschreibung} vorgeschlagen wurde.}
Dass dabei manchmal Uneinigkeit darüber besteht, was die wichtigen Schreibprinzipien sind, und was als unsystematischer historischer Ballast angesehen wird, ist nicht zu ändern.
Wir halten uns hier aus Reformdiskussionen daher vollständig heraus.

\index{Gebrauchsschreibung}
In Ansätzen beziehen wir darüber hinaus auch sogenannte \textit{Gebrauchsschreibungen} in die Betrachtung mit ein.
In vielen Schreibsituationen (überwiegend Situationen der persönlichen Kommunikation) ist der Normdruck auf die Schreiber gelockert, und sie verwenden grammatische Formen inklusive deren Verschriftungen, die nicht der Norm entsprechen.
Ein Beispiel wäre \textit{n} als Indefinitartikel (statt \textit{ein}).
Dabei lassen sich besonders gut echte (nicht-normative) Eigenschaften des Schreibsystems beobachten, denn für alles, was morphosyntaktisch nicht dem Standard folgt (in dem es den Artikel \textit{n} ja gar nicht gibt), gibt es auch keine orthographische Norm.
Schreiber wählen dann zwangsläufig eine dem System entsprechende Verschriftung, wobei man im Fall von \textit{n} auch eine graphematisch durchaus erwartbare Variante \textit{nen} (wohlgemerkt statt \textit{ein}) findet.
Außerdem ist die Verwendung oder Nicht-Verwendung des Apostrophs graphematisch relevant, also ob \textit{'n} oder \textit{n} geschrieben wird.
In vielen Fällen kommt es auch zu Zusammenschreibungen wie \textit{istn} (statt \textit{ist ein}).
Für alle diese Varianten gibt es nicht voneinander zu trennende grammatische und graphematische Interpretationen, die helfen, auch das stärker genormte Kernsystem zu verstehen (s.\ Abschnitt~\ref{sec:abkuerz}).%
\footnote{Bei solchen Gebrauchsschreibungen liegt es sehr nah, zu vermuten, dass hier einfach die gesprochene Sprache irgendwie verschriftet wird.
Sicherlich sind viele Gebrauchsschreibungen von gesprochener Sprache beeinflusst, aber es ist auf keinen Fall zielführend, hier einfach eine Gleichsetzung vorzunehmen.
Immerhin ist schon die Formulierung \textit{Verschriftung gesprochener Sprache} eigentlich ein Widerspruch in sich.
Sobald verschriftet wird, unterwirft man sich unausweichlich den Regularitäten des Schreibsystems.}

Abschließend erfolgt jetzt eine Einordnung des deutschen Schriftsystems in die Schriftsysteme der Welt.
Man unterscheidet drei primäre Typen von Schriftsystemen, nämlich \textit{Buchstabenschriften}, \textit{Silbenschriften} und \textit{Wortschriften}.
Bei der Buchstabenschrift entspricht im Prinzip jeder Buchstabe einem Laut.
Bei der Silbenschrift gibt es für jede Silbe ein Schriftzeichen, und bei der Wortschrift wird jedes Wort mit einem Zeichen (einem sogenannten \textit{Ideogramm}) wiedergeben.
Die meisten existierenden Schriften sind allerdings kompliziertere Zwischenformen oder modifizierte Varianten eines der drei Haupttypen.
Die Schreibung des Deutschen basiert auf der lateinischen Buchstabenschrift.
Als dominantes Prinzip gilt dabei, dass ein Buchstabe ein zugrundeliegendes Segment wiedergibt.
Allerdings wird in diesem Kapitel gezeigt, dass einige Buchstaben auch ganz andere systematische Funktionen haben.
Außerdem gibt es sowohl systematische als auch idiosynkratische Phänomene, die auf morphologischen und syntaktischen Prinzipien beruhen (Kapitel~\ref{sec:andereschrift}).

\subsection{Ziele und Vorgehen in diesem Buch}

Hier wird methodisch ein anderer Weg gegangen, als es in vielen Einführungen in die Graphematik üblich ist.%
\footnote{Allerdings ist Kapitel~8 aus \citet{Eisenberg1} sehr ähnlich in seinem Herangehen.}
Alle Abschnitte in diesem und dem nächsten Kapitel fragen, wie bestimmte grammatische Phänomene, die im Buch vorher beschrieben wurden, verschriftet werden.
Es wird dabei keine fertige graphematische Theorie angenommen, sondern vielmehr der Erkenntnisprozess in den Vordergrund gestellt, mittels dessen man von den Daten zu einer minimal komplizierten Theorie mit maximalem Erklärungsanspruch gelangt.
Dementsprechend wird auf Themen wie \zB die Unterscheidung von \textit{Graphen} und \textit{Graphemen} nicht eingegangen, ebenso wie empirisch weniger offensichtliche Theorien wie die von der \textit{graphematischen Silbe} bzw.\ dem \textit{graphematischen Fuß}.
Auch über die Form der Buchstaben und sonstigen Zeichen sagen wir aus Platzgründen nichts, obwohl die existierende Literatur auch zu diesem Thema viel zu sagen hat.
Daraus folgt, dass uns der rein graphische Unterschied von Großbuchstaben (\textit{Majuskeln}) und Kleinbuchstaben (\textit{Minuskeln}) nicht interessiert.\index{Minuskel}\index{Majuskel}
Wir schreiben daher bald die Majuskel, bald die Minuskel, ohne einen Unterschied zu machen, außer wenn ausdrücklich grammatische Markierungen durch Majuskelschreibung erfolgen (Abschnitte~\ref{sec:hervorhebung}, \ref{sec:wortklassschreib} und \ref{sec:hauptsatzschreib}).
Wir verzichten hier auch darauf, Einheiten der Graphematik wie sonst üblich in <~> zu setzen, weil dies optisch sehr ungünstig ist.
Stattdessen nehmen wir den kursiven Schriftschnitt.

\index{Kernwortschatz}
Bezüglich der beschriebenen Phänomene beschränken wir uns auf den Kernwortschatz.
Der Kernwortschatz ist der Teil des Lexikons, der sich nach den primären, elementaren und i.\,d.\,R.\ weittragenden Regularitäten verhält (vgl.\ Abschnitt~\ref{sec:kernundperipherie}).
In der Phonologie und damit zu einem großen Teil auch in diesem Kapitel zur Beziehung zwischen Phonologie und dem Schreibsystem bedeutet das, dass wir uns auf die Betrachtung einfacher trochäischer Wörter beschränken, die nicht erkennbar entlehnt sind.
Damit gilt das hier Gesagte vor allem für (in dieser Reihenfolge) Substantive, Verben und Adjektive, die überwiegend, aber längst nicht ausschließlich germanischen Ursprungs sind.
Besonders in der Silben- und Fußphonologie und der Graphematik gibt es jenseits des trochäischen Kernwortschatzes stärkere Abweichungen in anderen Wortklassen.
Da die Substantive, Verben und Adjektive aber die offenen Wortklassen sind (also Wortklassen, in denen sehr viele und potentiell auch immer wieder neue Wörter enthalten sind), stellt die Beschränkung auf ihre Beschreibung kein nennenswertes Problem dar.
Dass sich Pronomina, Partikeln oder Präpositionen nicht immer nach diesen Regularitäten verhalten, spielt kaum eine Rolle, da sie sich kompakt und umfassend auflisten und ggf.\ auch lernen lassen.
Mit anderen Worten:
Sie haben eine sehr geringe Typenhäufigkeit (s.\ Abschnitt~\ref{sec:kernundperipherie}).
Der Bedarf an großer Einheitlichkeit und Regularität entsteht also aus systematischen Gründen vor allem für Substantive, Verben und Adjektive.
Auf keinen Fall sollte angenommen werden, dass Wörter außerhalb des Kernwortschatzes irgendwie \textit{falsch} sind, nicht in die Sprache gehören oder gar dem Kern angepasst werden sollten.
Genauso wie in der Morphologie die Präteritalpräsentien bzw.\ unregelmäßigen Verben (Abschnitt~\ref{sec:kleineverbklassen}) oder die schwachen Substantive (Abschnitt~\ref{sec:schwachsubst}) in kleinen Klassen ein abweichendes Verhalten innerhalb des Systems zeigen, gibt es auch Abweichungen in der Phonologie und Graphematik.


\Zusammenfassung{
In einer Sprache mit einer tief verwurzelten Schreibkultur sind Grammatik und Graphematik nicht voneinander zu trennen.
So wie die Phonetik sich mit der Kodierung von Sprache im \textit{akustischen Medium} beschäftigt, beschäftigt sich die Graphematik mit der Kodierung von Sprache im \textit{Schriftmedium}.
}


\section{Buchstaben und phonologische Segmente}

\label{sec:buchstabensegmente}

\subsection{Konsonantenschreibungen}

\label{sec:konssegschreib}

Anders als das in Kapitel~\ref{sec:phonetik} besprochene phonetische Alphabet (IPA), ist das deutsche Alphabet das Produkt einer weitgehend nicht geplanten und gesteuerten Entwicklung.\index{Alphabet!deutsch}
Die Frage soll hier sein, wie bestimmte grammatische Einheiten verschriftet werden, nicht umgekehrt.
Daher beginnen wir aber auch nicht mit einer Gesamtdarstellung des deutschen Alphabets, sondern gehen zunächst von den Segmenten der Phonologie des Deutschen (Kapitel~\ref{sec:phonologie}) aus.
Tabelle~\ref{tab:segschreibkons} fasst daher als Erstes zusammen, mit welchen Buchstaben (hier nur die Minuskeln) die zugrundeliegenden konsonantischen Segmente (s.\ Kapitel~\ref{sec:phonetik}, genauer Tabelle~\ref{tab:photkons} auf S.~\pageref{tab:photkons}) primär geschrieben werden.
Das heißt nicht, dass für die genannten Buchstaben nicht auch andere systematische oder unsystematische Verwendungen existieren.
Zu den Rändern und Ausnahmen der Schreibungen im Kernwortschatz kommen wir weiter unten.
Wörter wie \textit{Garage} oder \textit{Chips}, die nicht den allgemeinen phonologischen Regularitäten folgen, werden aus dem gleichen Grund nicht beachtet.
Ebenso berücksichtigen wir atypische Schreibungen zunächst nicht, \zB \textit{Cäsar}, \textit{Charakter} oder \textit{Spaghetti}.
(Siehe dazu Abschnitt~\ref{sec:nichtkernschreib}.)

\index{Buchstabe!konsonantisch}
\index{Konsonant!Schreibung}
\begin{table}[!htbp]
  \centering
    \begin{tabular}{lll}
      \lsptoprule
      \textbf{Segment} & \textbf{Buchstabe(n)} & \textbf{Beispielwörter} \\
      \midrule
     \textipa{p} & p & \textit{Plan} \\
     \textipa{b} & b & \textit{Baum}, \textit{Trab} \\
     \textipa{\t{pf}} & pf & \textit{Pfad} \\
     \textipa{f} & f & \textit{Fahrt} \\
     \textipa{v} & w & \textit{Wand} \\
     \textipa{m} & m & \textit{Mus} \\
     \textipa{t} & t & \textit{Tau} \\
     \textipa{d} & d & \textit{Dach}, \textit{Bild}\\
     \textipa{\t{ts}} & z & \textit{Zeit} \\
     \textipa{s} & s & \textit{Los} \\
     \textipa{z} & s & \textit{Sau} \\
     \textipa{S} & sch & \textit{Schiff} \\
     \textipa{n} & n & \textit{Not}, \textit{Klang} \\
     \textipa{l} & l & \textit{Lob} \\
     \textipa{\c{c}} & ch & \textit{Blech}, \textit{Wacht} \\
     \textipa{J} & j & \textit{Jahr} \\
     \textipa{k} & k & \textit{Kiel} \\
     \textipa{g} & g & \textit{Gans}, \textit{Weg}, \textit{König} \\
     \textipa{K} & r & \textit{Ritt}, \textit{Tür} \\
     \textipa{h} & h & \textit{Herz} \\
      \lspbottomrule
    \end{tabular}
  \caption{Konsonantische Segmente und ihre Buchstabenkorrespondenz}
  \label{tab:segschreibkons}
\end{table}

\begin{table}[!htbp]
  \centering
  \resizebox{\textwidth}{!}{
    \begin{tabular}{lp{0.15cm}lp{0.15cm}llp{0.15cm}llp{0.15cm}l}
      \lsptoprule
      \textbf{zugr.} && \textbf{Buch-} && \multicolumn{2}{l}{\textbf{phonetische}}    && \multicolumn{2}{l}{\textbf{phonologische}} && \textbf{phonetische} \\
      \textbf{Segm.} && \textbf{stabe(n)} && \multicolumn{2}{l}{\textbf{Realisierungen}} && \multicolumn{2}{l}{\textbf{Schreibungen}}  && \textbf{Schreibung} \\
      \midrule
      \textipa{b} && b && \textipa{b\t{aO}m} & \textipa{lo:p} && \textit{Baum} & \textit{Lob} && *\textit{Lop} \\
      \textipa{d} && d && \textipa{daX} & \textipa{KInt} && \textit{Dach} & \textit{Rind} && *\textit{Rint} \\
      \textipa{n} && n && \textipa{naXt} & \textipa{klaN} && \textit{Nacht} & \textit{Klang} && *\textit{Kla\textipa{N}} \\
      \textipa{\c{c}} && ch && \textipa{lI\c{c}t} & \textipa{vaXt} && \textit{Licht} & \textit{Wacht} && *\textit{Waχt} \\
      \textipa{g} && g && \textipa{gans} & \textipa{k\o:nI\c{c}} && \textit{Gans} & \textit{König} && *\textit{Könich} \\
      \textipa{K} && r && \textipa{Ku:m} & \textipa{t\t{o5}} && \textit{Ruhm} & \textit{Tor} && *\textit{Toe} \\
      \lspbottomrule
    \end{tabular}
  }
  \caption{Invarianz zugrundeliegender Konsonanten-Segmente}
  \label{tab:konsseginvarianz}
\end{table}

In Tabelle~\ref{tab:segschreibkons} sind in der ersten Spalte die zugrundeliegenden Segmente (der Übersicht halber ohne /~/) aufgelistet sind, und nicht etwa alle möglichen phonetischen Segmente des Deutschen.
Für /\textipa{\c{c}}/ müssen also die beiden Realisierungen \textipa{[\c{c}]} und \textipa{[X]} berücksichtigt werden, usw.
Das können wir uns genau deshalb erlauben, weil die Buchstaben gerade den \textit{zugrundeliegenden} Segmenten entsprechen und damit eine \textit{phonologische} und keine \textit{phonetische} Verschriftung darstellen.
Auch die Gruppen aus mehreren Buchstaben, die ein einziges Segment kodieren (\zB \textit{ch} und \textit{sch}), kodieren immer ein zugrundeliegendes Segment, nicht ein rein phonetisches. 
Tabelle~\ref{tab:konsseginvarianz} zeigt Beispiele für die Unveränderlichkeit (\textit{Invarianz}) der Konsonanten-Buchstaben eines zugrundeliegenden Segments.
\index{Auslautverhärtung!Schreibung}
\index{r-Vokalisierung!Schreibung}
Segmentale Anpassungen zugrundeliegender Formen wie die Auslautverhärtung (Abschnitt~\ref{sec:auslautverhaertungphonologie}), die Verteilung von \textipa{[\c{c}]} und \textipa{[X]} (Abschnitt~\ref{sec:verteilungvonichach}) oder Vokalisierungen von /\textipa{K}/ (Abschnitt~\ref{sec:rvokalisierungen}) werden offensichtlich ganz konsequent in der Buchstabenschrift nicht abgebildet.
Sonst müssten wir die fiktiven Schreibungen in der letzten Spalte von Tabelle~\ref{tab:konsseginvarianz} (oder ähnliche Schreibungen) beobachten können.
In *\textit{Lop} und *\textit{Rint} wird die Auslautverhärtung in der Schrift abgebildet, in *\textit{Könich} die Frikativierung des /g/ nach /\textipa{I}/. 
Für phonetische Realisierungen von \textipa{[N]}, \textipa{[X]} und \textipa{[5]} existieren allerdings keine Buchstaben oder Sequenzen von Buchstaben.
Die hypothetische Realisierung *\textit{Kla\textipa{N}} zeigt, dass für \textipa{[N]} ein Buchstabe eingeführt werden müsste.
Ebenso wird in *\textit{Waχt} das χ als möglicher Buchstabe zur Abbildung von \textipa{[X]} verwendet.
Die Schreibung *\textit{Toe} für \textipa{[t\t{o5}]} wäre aus mehreren Gründen ungünstig, stellt aber ebenfalls einen Versuch der Phonetisierung dar.
Ganz offensichtlich gibt es keinen Bedarf an solchen Lösungen, weil das phonologische Buchstabensystem, bei dem Buchstaben zugrundeliegenden Segmenten entsprechen, etabliert ist und einwandfrei funktioniert.

Auch im Kernwortschatz gibt es nun segmentale Schreibungen, die von dieser Beschreibung noch nicht erfasst werden.
Eine kleiner Sonderfall im System ist die kanonische Schreibung \textit{qu} für /\textipa{kv}/, die historisch, aber nicht synchron im System begründbar ist.
Der Buchstabe \textit{q} ist vor /\textipa{v}/ die generelle Vertretung von \textit{k}, und \textit{u} ist die generelle Vertretung von \textit{v} nach /\textipa{k}/.
Das ist recht seltsam, denn das \textit{u} (ein Vokalzeichen) kommt sonst nicht im konsonantischen Bereich vor, und \textit{q} gibt es ansonsten gar nicht.
Die zwei zugrundeliegenden Segmente korrespondieren also jeweils mit zwei Buchstaben statt nur einem.
Die Verteilung ist aber klar (und komplementär, vgl.\ Abschnitt~\ref{sec:segmentemerkmaleverteilungen} zum Begriff \textit{komplementär}), und das phonologische Schreibprinzip wird dadurch nicht aufgehoben.
Das gilt ebenso für \textit{sp} und \textit{st} am Silbenanfang, die statt der direkten Schreibungen *\textit{schp} und *\textit{scht} für /\textipa{Sp}/ und /\textipa{St}/ stehen.

Weiterhin gibt es systematisch verschiedene Möglichkeiten, die Segmentfolge /\textipa{ks}/ zu schreiben.
Diese Abfolge kommt am Silbenanfang im Deutschen im Grunde nicht vor, und in Lehnwörtern wird die besondere Schreibung \textit{x} verwendet (\textit{Xenon} usw.).
Am Wortende wird prinzipiell \textit{ch} für /\textipa{k}/ vor \textit{s} substituiert, vgl.\ \textit{Wachs} /\textipa{vaks}/ oder \textit{Echse} /\textipa{Eks@}/.
Die naheliegende Schreibung \textit{ks} kommt vor allem (aber nicht nur) in Form von \textit{cks} vor (zum \textit{ck} hier siehe Abschnitt~\ref{sec:laengeschreib} und Abschnitt~\ref{sec:konstanz}).
Eher selten ist sie in einfachen (nicht derivierten oder flektierten) Wörtern wie \textit{Keks} oder \textit{zwecks} (historisch nicht einfach) anzutreffen, häufig aber an der Morphgrenze wie in \textit{steckst} oder \textit{Glücks}.

Das Zeichen \textit{s} schließlich ist scheinbar als einziges unter den primären Konsonantenschreibungen doppelt belegt, weil es sowohl für /\textipa{s}/ als auch /\textipa{z}/ verwendet wird.
Diese Beobachtung gehört eng zu der Beobachtung des \textit{ß} (also des \textit{scharfen S} oder \textit{Eszett}), das in bestimmten Kontexten für /\textipa{s}/ verwendet wird, vgl.\ Abschnitt~\ref{sec:silbenschreib}.
Die beiden Segmente sind bezüglich des Wortanlauts und Wortauslauts komplementär verteilt (\textit{Sog} \textipa{[zo:k]}, aber \textit{fließ} \textipa{[fli:s]}), was schon in (\ref{ex:phol6440}) auf S.~\pageref{ex:phol6440} festgestellt wurde.
Allerdings gibt es Positionen im Wort, in denen sie distinktiv sind, und in denen das \textit{ß} bei der Unterscheidung zwischen /\textipa{s}/ und /\textipa{z}/ hilft, \zB \textit{Muße} /\textipa{mu:s@}/ und \textit{Muse} /\textipa{mu:z@}/, was in Abschnitt~\ref{sec:eszett} genauer erklärt wird.

\Satz{Phonologisches Schreibprinzip}{
\label{satz:phonschreibprinz}
Jedes zugrundeliegende Segment korrespondiert primär mit genau einem Buchstaben (mit sehr wenigen Ausnahmen).
Die Schreibung ist invariant, auch wenn die zugrundeliegende Form an Strukturbedingungen angepasst wird.
Die Schreibung des Deutschen ist also phonologisch und nicht phonetisch.
\index{Schreibprinzip!phonologisch}
}

\subsection{Vokalschreibungen}

Was bei den Konsonanten in Gestalt des \textit{s} ein Sonderfall ist, nämlich dass ein Buchstabe mehreren zugrundeliegenden Segmenten entspricht, ist bei den Vokalen regelmäßig der Fall.
In Tabelle~\ref{tab:segschreibvok} sind die vokalischen Segmente aus Kapitel~\ref{sec:phonologie} (genauer Abbildung~\ref{fig:vokalviereckmitgespannt} auf S.~\pageref{fig:vokalviereckmitgespannt}) und ihre korrespondierenden Buchstaben aufgelistet.

\index{Buchstabe!vokalisch}
\index{Vokal!Schreibung}
\begin{table}[!htbp]
  \centering
    \begin{tabular}{lp{0.5cm}llp{0.25cm}ll}
      \lsptoprule
      \multirow{2}{*}{\textbf{Buchstabe}} && \multicolumn{2}{l}{\textbf{Segment}} && \multicolumn{2}{l}{\textbf{Segment}} \\
       && \textbf{gespannt} & \textbf{Beispiel} && \textbf{ungespannt} & \textbf{Beispiel} \\
      \midrule
      i	&& \textipa{i}  & \textit{Igel} && \textipa{I} & \textit{Licht} \\
      ü	&& \textipa{y}  & \textit{Rübe} && \textipa{Y} & \textit{Rücken} \\
      u	&& \textipa{u}  & \textit{Mut} && \textipa{U} & \textit{Butter} \\
      e	&& \textipa{e}  & \textit{Mehl} && \textipa{\u{E}} & \textit{Bett} \\
      ö	&& \textipa{\o} & \textit{Höhle} && \textipa{\oe} & \textit{Löffel} \\
      o	&& \textipa{o}  & \textit{Ofen} && \textipa{O} & \textit{Motte} \\
      ä	&& \textipa{E}  & \textit{Gräte} && \textipa{\u{E}} & \textit{Säcke} \\
      a	&& \textipa{a}  & \textit{Wal} && \textipa{\u{a}} & \textit{Wall} \\
      \lspbottomrule
    \end{tabular}
  \caption{Vokalische Segmente und ihre Buch\-staben\-korres\-pondenz}
  \label{tab:segschreibvok}
\end{table}

\index{gespannt!Schreibung}
\index{Dehnungsschreibung}
\index{Schärfungsschreibung}
Wo im phonologischen System eine gespannte und eine ungespannte Variante eines Vokals existieren, gibt es jeweils nur ein Vokalzeichen.
Das ist systematisch so, und Abschnitt~\ref{sec:laengeschreib} widmet sich diesem Phänomen nochmals aus Sicht der Silbenphonologie und ihrer Verschriftung.
Besondere Aufmerksamkeit verdienen hier nur \textit{e} und \textit{ä}.
In Abschnitt~\ref{sec:gespanntheitbetonunglaenge} (besonders Abbildung~\ref{fig:vokalviereckmitgespannt} auf S.~\pageref{fig:vokalviereckmitgespannt}) wurde ein Vokalsystem vorgeschlagen, in dem sowohl dem gespannten /\textipa{E}/ als auch dem gespannten /e/ die ungespannte Variante /\textipa{\u{E}}/ zugeordnet ist.
Die Buchstaben \textit{e} (gespanntes /e/ und ungespanntes /\textipa{\u{E}}/) und \textit{ä} (gespanntes /\textipa{E}/ und ungespanntes /\textipa{\u{E}}/) verhalten sich entsprechend.
Es gibt folgerichtig zwei Varianten für die Verschriftung von /\textipa{\u{E}}/, nämlich \textit{e} (\textit{Bett}) und \textit{ä} (\textit{Säcke}).
Zusätzlich wird \textit{e} für /\textipa{@}/ verwendet.
Im Fall der gespannten /\textipa{e}/, /\textipa{E}/ und der zwei ungespannten /\textipa{\u{E}}/ ist die Buchstabenschreibung also hochgradig konsequent, und eventuelle Verwirrung kommt nur daher, dass im phonologische System zwei gespannte Vokale mit demselben ungespannten Vokal korrespondieren.

\index{Diphthong!Schreibung}
Wie im Fall von \textit{chs} und \textit{qu} (Abschnitt~\ref{sec:konssegschreib}) gibt es auch bei den Vokalen kleine Extravaganzen zu berücksichtigen.
Vor allem sind die Diphthonge \textit{eu} (\textit{Heu}) und \textit{ei} (\textit{frei}) zu nennen.
Bei ihnen korrespondieren die Buchstaben des geschriebenen Diphthongs nicht direkt (gemäß der Korrespondenzen aus Tabelle~\ref{tab:segschreibvok}) mit Segmenten, und man muss sie ähnlich wie \textit{ch} als jeweils eine graphematisch nicht teilbare Einheit auffassen.
Nur bei den Diphthongschreibungen \textit{ai}, \textit{au} und \textit{oi} wird im Sinn der Korrespondenzen der Einzelsegmente verschriftet.
Allerdings kommen \textit{ai} und \textit{oi} fast nur in Lehnwörtern (\textit{Kaiser}, \textit{Joint}) oder Namen vor, die dialektal beeinflusst sind (\textit{Mainz}, \textit{Moik}).
Zu den wenigen Ausnahmen zählt \textit{Waise}.
Im Prinzip haben wir es bei \textit{ei} und \textit{eu} mit einer historisch begründeten Sonderentwicklung zu tun, die synchron kaum der Erklärung bedarf.
Die zu \textit{eu} alternative Schreibung \textit{äu} hat allerdings einen besonderen Stellenwert, der in Abschnitt~\ref{sec:konstanz} besprochen wird.

Viel mehr muss man für die hier verfolgten Zwecke zu den Schreibungen der Segmente gar nicht sagen, könnte es aber natürlich.
Es gibt im Bereich der Verschriftung phonologischer Phänomene im Deutschen allerdings auch Fälle, in denen Buchstaben nicht Segmenten entsprechen, wie \textit{e} in \textit{Knie} oder \textit{c} in \textit{Rock}.
Solche Schreibungen haben in den meisten Fällen eine Motivation in der Silbenphonologie, um die es jetzt in Abschnitt~\ref{sec:silbenschreib} geht.


\Zusammenfassung{
Zu jedem zugrundeliegenden Segment des Deutschen korrespondiert eine primäre Buchstabenschreibung (bei den Vokalen jeweils eine für den kurzen und den langen Vokal zusammen).
Die Schreibung ist also \textit{phonologisch} und nicht \textit{phonetisch}.
}


\section{Silben und Wörter}

\label{sec:silbenschreib}

In diesem Abschnitt geht es nicht um theoretische Konzepte wie die \textit{graphematische Silbe} oder den \textit{graphematischen Fuß}.
Solche Einheiten werden in der Literatur aus guten Gründen diskutiert.
Hier würde eine Diskussion dieser Theorien zu weit führen, und wir beschränken uns auf die Aspekte der Silbenphonologie, die auf die segmentale Phonologie (vor allem Vokallänge) zurückwirken und systematisch verschriftet werden.
Diese Phänomene können gut am konkreten Material illustriert werden, und sie interagieren direkt mit vieldiskutierten Fragen der Orthographie, \zB \textit{ß}-Schreibungen.

\subsection{Dehnungs- und Schärfungsschreibungen}

\label{sec:laengeschreib}

Besonderheiten der Schreibung auf Silbenebene betreffen vor allem die Länge von Vokalen.
In Abschnitt~\ref{sec:gespanntheitbetonunglaenge} wurde festgestellt, dass zugrundeliegend das Merkmal \textsc{Länge} nicht spezifiziert werden muss, weil genau die Vokale, die gespannt und betont sind, lang sind.
Im Kernwortschatz tritt Gespanntheit nur mit Betonung zusammen auf, und alle gespannten Vokale sind lang und betont.
Wir besprechen jetzt das System der sogenannten \textit{Schärfungsschreibungen} (Definition~\ref{def:kuerzschreib}) und \textit{Dehnungsschreibungen} (Definition~\ref{def:dehnungsschreib}).%
\footnote{Weil später die Schärfungsschreibungen anders definiert werden, ist Definition~\ref{def:kuerzschreib} als vorläufig markiert.}
Obwohl in der in diesem Buch gewählten Darstellung die Gespanntheit gegenüber der Länge das zentrale Merkmal ist, zielt die etablierte Terminologie vor allem mit der Rede von der \textit{Dehnung} auf die Länge ab.
Durch den erweiterten Wortschatz, in dem auch unbetonte gespannte -- und damit kurze -- Vokale vorkommen (\zB \textit{Etymologie} als \textipa{[Petymologi:]} statt *\textipa{[PEtYmOlOgi:]}) ist dieser Sprachgebrauch durchaus brauchbar.

\Definition{Schärfungsschreibung (vorläufig)}{
\label{def:kuerzschreib}
Schärfungsschreibungen sind Doppelungen von Konsonantenbuchstaben (im Fall von \textit{k} steht \textit{ck}) nach ungespannten Vokalen.
Sie zeigen die Kürze des Vokals an und sind nicht segmental.
\index{Schärfungsschreibung}
}

\Definition{Dehnungsschreibung}{
\label{def:dehnungsschreib}
Dehnungsschreibungen sind nach einem Vokal eingefügte Buchstaben, die dessen Länge anzeigen.
Sie sind nicht segmental.
\index{Dehnungsschreibung}
}


Bei den Schärfungsschreibungen fällt vor allem \textit{Doppelkonsonanz} ins Auge (\textit{Kinn}, \textit{knapp}) und die \textit{ck}-Schreibung (\textit{Rock}, \textit{Knick}).
Dehnungsschreibungen gibt es in Form von \textit{h} (\textit{Reh}, \textit{hohl}), Doppelung (\textit{Schnee}, \textit{Moor}, \textit{Aal}) und bei \textit{i} typisch \textit{ie} (\textit{Knie}, \textit{viel}).
Das deutsche Schreibsystem bemüht sich offensichtlich darum, Länge und Kürze zu markieren, auch wenn vor allem die Markierung der Längen im Ergebnis nur sehr inkonsequent durchgeführt wird.
Das Lateinische, von dem das Deutsche seine Schrift übernommen hat, hat ebenfalls einen Unterschied von Vokallängen, markiert diesen aber überhaupt nicht in der Schrift.
Die Schärfungs- und Dehnungsschreibungen sind also eine historisch gewachsene Erweiterung des aus dem Lateinischen entlehnten Buchstabensystems.

Wie verteilen sich die Dehnungs- und Schärfungsschreibungen?
Zunächst betrachten wir Tabelle~\ref{tab:dehnkuerzschreib}.
In dieser Tabelle wird nach offenen und geschlossenen Silben gemäß Definition~\ref{def:offengeschlossen} klassifiziert.

\newcommand{\LocStrutGrph}{\hspace{0.1\textwidth}}

\index{Silbe!und Schreibung}
\begin{table}[!htbp]
  \centering
  \resizebox{\textwidth}{!}{
    \begin{tabular}{lllllllll}
      \lsptoprule

      & & & \textbf{ɪ} & \textbf{ʊ} & \multicolumn{2}{l}{\LocStrutGrph\textbf{\textipa{\u{E}}}} & \textbf{ɔ} & \textbf{\textipa{\u{a}}} \\ 
      \midrule

      \multirow{4}{*}{\rotatebox{90}{\textbf{ungespannt}}}

	& \multirow{2}{*}{\rotatebox{90}{\textbf{offen}}}
	  & \textbf{einsilb.}  & \textit{\Nono}  & \textit{\Nono}           & \multicolumn{2}{l}{\LocStrutGrph\textit{\Nono}}         & \textit{\Nono}        & \textit{\Nono}           \\
	&& \textbf{zweisilb.}  & \textit{Li.ppe} & \textit{Fu.tter}         & \multicolumn{2}{l}{\LocStrutGrph\textit{We.cke}}        & \textit{o.ffen}       & \textit{wa.cker}         \\
        & \multirow{2}{*}{\rotatebox{90}{\textbf{gesch.}}}
	  & \textbf{einsilb.}  & \textit{Kinn}   & \textit{Schutt}    & \multicolumn{2}{l}{\LocStrutGrph\textit{Bett}}           & \textit{Rock}         & \textit{Watt}            \\
        && \textbf{zweisilb.}  & \textit{Rin.de} & \textit{Wun.der}        & \multicolumn{2}{l}{\LocStrutGrph\textit{Wen.de}}        & \textit{pol.ter}      & \textit{Tan.te}          \\

	\midrule

	\multirow{4}{*}{\rotatebox{90}{\textbf{gespannt}}}

	& \multirow{2}{*}{\rotatebox{90}{\textbf{offen}}}
	  & \textbf{einsilb.}  & \textit{Knie}   & \textit{Schuh}       & \textit{Schnee, Reh}  & \textit{zäh}          & \textit{roh}          & (\textit{da})            \\
	&& \textbf{zweisilb.}  & \textit{Bie.ne} & \textit{Kuh.le, Schu.le} & \textit{we.nig}       & \textit{Äh.re, rä.kel} & \textit{oh.ne, O.fen} & \textit{Fah.ne, Spa.ten} \\

	& \multirow{2}{*}{\rotatebox{90}{\textbf{gesch.}}}
	  & \textbf{einsilb.}  & \textit{lieb}  & \textit{Ruhm, Glut}      & \textit{Weg}          & \textit{spät}           & \textit{rot}          & \textit{Tat}             \\
	&& \textbf{zweisilb.}  & (\textit{lieb.lich}) & (\textit{lug.te})   & (\textit{red.lich})   & (\textit{wähl.te})     & (\textit{brot.los})   & (\textit{rat.los})       \\

	\midrule
      & & & \textbf{i} & \textbf{u} & \textbf{e} & \textbf{ε} & \textbf{o} & \textbf{a} \\ 

      \lspbottomrule
    \end{tabular}
  }
  \caption{Schreibung von Vokallängen in Einsilblern und Erstsilben von trochäischen Zweisilblern mit konsonantisch anlautender Zweitsilbe (nur Kernwortschatz)}
  \label{tab:dehnkuerzschreib}
\end{table}

\index{Simplex}
Die Tabelle listet und gruppiert \textit{simplexe} Wörter (also nicht flektierte und nicht durch Wortbildung abgeleitete) einsilbige und trochäische Wörter des Kernwortschatzes.%
\footnote{Simplexe Wörter werden auch \textit{Simplizia} oder \textit{Simplicia} (Singular: \textit{Simplex}) genannt.}
Eine Ausnahme bilden die eingeklammerten Zweisilbler mit langer geschlossener Erstsilbe, die alle nicht simplex sind, weil simplexe Wörter dieses Typs mit wenigen Ausnahmen (\zB Namen wie \textit{Liedtke} \textipa{[li:tk@]} oder \textit{Wiebke} \textipa{[vi:pk@]}) nicht existieren.\label{abs:wiebke}
Die zweisilbigen Wörter wurden absichtlich so ausgesucht, dass die zweite Silbe mit einem Konsonant anlautet, was auch der typische und häufige Fall ist (s.\ aber Abschnitt~\ref{sec:intervokh}).
Es interessiert jeweils nur die erste (bzw.\ einzige) Silbe, und ob sie einen gespannten Vokal oder sein ungespanntes Pendant enthält.%
\footnote{Die Vokale /\textipa{\o}/ und /\textipa{y}/ und ihre kurzen bzw.\ ungespannten Varianten fehlen aus Gründen der Übersichtlichkeit.
Vgl. Übung~\ref{u141}.}

Wir wenden uns zunächst den Silben mit langem gespanntem Vokal zu.
In offenen Einsilblern findet man nahezu durchweg Dehnungsschreibung wie in \textit{Knie}, \textit{Schnee}, \textit{roh} usw.
Ausnahmen findet man vor allem jenseits der Substantive, Verben und Adjektive (\zB \textit{je}, \textit{zu}) oder in Fachwörtern (\zB \textit{Re}).
In allen anderen Fällen (geschlossene Silben mit langem Vokal und offene Erstsilben im Zweisilbler) ist der Gebrauch der Dehnungsschreibung nicht obligatorisch, vgl.\ \textit{Ruhm} vs.\ \textit{Glut} oder \textit{Kuh.le} vs.\ \textit{Schu.le}.
Bei geschlossenen Silben mit langem Vokal ist es unerheblich, ob es sich um einen Einsilbler handelt (\textit{lieb}) oder um einen Mehrsilbler (\textit{lieblich}).

Bei den Silben mit ungespanntem Vokal wird die Angelegenheit interessanter, weil die Schärfungsschreibungen ins Spiel kommen.
Wir wissen aus Kapitel~\ref{sec:phonologie}, dass es keine kurzen offenen (und damit einmorigen) Einsilbler wie *\textipa{[knI]} oder *\textipa{[KO]} gibt.
In Abschnitt~\ref{sec:einsilblerzweisilbler} wurden die entsprechenden Generalisierungen mit Bezug auf das Silbengewicht formuliert.
Kurz gesagt sind Silben mit betonbaren Vokalen (also alle Silben außer Schwa-Silben) immer mindestens zweimorig.
Passend zum Fehlen der offenen kurzen Einsilbler sind auch Schreibungen wie *\textit{Kni} oder *\textit{Ro} im Prinzip inakzeptabel, und die erste Zeile in Tabelle~\ref{tab:dehnkuerzschreib} bleibt leer.

Aus den Regularitäten des Silbengewichts kann man ebenfalls ableiten, warum in Wörtern vom Typ \textit{Lippe} die Schärfungsschreibung steht, in denen vom Typ \textit{Rinde} allerdings nicht (*\textit{Rinnde}).
Da Silben wir \textipa{[lI]}, \textipa{[fU]}, \textipa{[vE]} usw.\ nur einmorig und damit zu leicht wären, bildet der erste Konsonant der zweiten Silbe ein Silbengelenk und macht die erste Silbe zweimorig.
Die Schärfungsschreibung ist also eine Silbengelenkschreibung, und in den Beispielen in Tabelle~\ref{tab:dehnkuerzschreib} kodiert \textit{Lippe} \textipa{[lI\Sgel{p}@]}, \textit{Futter} \textipa{[fU\Sgel{t}5]}, \textit{Wecke} \textipa{[vE\Sgel{k}@]}, \textit{offen} \textipa{[O\Sgel{f}@n]} und \textit{wacker} \textipa{[va\Sgel{k}5]}.%
\footnote{Die Notation mit dem tiefgestellten Konsonanten als Zeichen des Silbengelenks wurde in Abschnitt~\ref{sec:einsilblerzweisilbler} eingeführt.}

In den Wörtern vom Typ \textit{Rinde}, \textit{Wunder}, \textit{Wende} usw.\ steht nun genau deshalb niemals eine Schärfungsschreibung, weil in diesen Wörtern kein Silbengelenk nötig und möglich ist.
Das Silbengelenk tritt ja nur genau dann auf, wenn die Erstsilbe ansonsten nicht schwer genug wäre und kein anderes Material verfügbar ist, um den Endrand zu füllen.
Die Silbifizierungen \textit{Rin.de}, \textit{Wun.der}, \textit{Wen.de} usw.\ stellen aber sicher, dass die Erstsilben zweimorig sind, und ein Silbengelenk kommt überhaupt nicht infrage.
Wollte man hier mit Silbengelenk silbifizieren, wäre nur *\textipa{[KI\Sgel{n}d@]} möglich, was als *\textit{Rinnde} zu verschriften wäre.
Die Silbe \textipa{[nd@]} ist aber nicht wohlgeformt, und sowohl die Silbifizierung als auch die Schreibung sind ausgeschlossen.
Es wäre also nicht nur unangemessen, die Schärfungsschreibung als \textit{Kürzungsschreibung} zu bezeichnen, sondern auch schlicht falsch, weil sie in zahlreichen Silben mit kurzem Vokal gar nicht stehen darf.
Es ergibt sich Definition~\ref{def:schaerfungsschreibung}.

\Definition{Schärfungsschreibung}{
\label{def:schaerfungsschreibung}
Schärfungsschreibungen sind Doppelungen von Konsonantenbuchstaben (im Fall von \textit{k} steht \textit{ck}) nach ungespannten Vokalen.
Sie markieren obligatorisch die Position von Silbengelenken.
\index{Schärfungsschreibung}
}

Es bleiben die Wörter vom Typ \textit{Kinn}, \textit{Schutt}, \textit{Bett} usw.
Diese stellen ein Problem dar, wenn Definition~\ref{def:schaerfungsschreibung} gelten soll, weil hier eine Schärfungsschreibung erfolgt, die strukturell kein Silbengelenk anzeigen kann, weil es sich um Einsilbler handelt.
Während die echte Silbengelenkschreibung aber obligatorisch ist, gibt es Einsilbler wie \textit{hin} \textipa{[hIn]}, \textit{was} \textipa{[vas]} oder \textit{um} \textipa{[PUm]}, in denen keine Schärfungsschreibung erfolgt.
In Abschnitt~\ref{sec:konstanz} werden die Verhältnisse auf das Prinzip der \textit{Konstantschreibung} zurückgeführt.
Es besagt, dass Formen wie \textit{Bett} deswegen mit Schärfungsschreibung geschrieben werden, weil in anderen Formen des Wortes -- wie \zB \textit{Bettes} -- eine Gelenkschreibung erfolgen \textit{muss}.
Die Schreibung *\textit{Betes} könnte auf jeden Fall nur als \textipa{[be:t@s]} interpretiert werden, es muss also \textit{Bettes} geschrieben werden.
Die Konstantschreibung verlangt nun als zusätzliches Prinzip, dass alle Formen eines Wortes einander möglichst ähnlich sein sollen.
Die Formen *\textit{Bet} und \textit{Bettes} wären dies aber nur eingeschränkt, so dass \textit{Bett} ohne graphematisch-phonologische Notwendigkeit mit Schärfungsschreibung geschrieben wird.
In Fällen wie \textit{hin}, \textit{was} und \textit{um} gibt es keine entsprechenden mehrsilbigen Formen, der Bedarf an einer Konstantschreibung entfällt, und die Schärfungsschreibung erfolgt nicht.

\subsection[Eszett an der Silbengrenze]{Eszett an der Silbengrenze}

\label{sec:eszett}

Auch die Verwendung des \textit{Eszett} \textit{ß} an der Silbengrenze können wir jetzt einzuordnen.
Aus grammatischer Sicht bietet es sich an, die Frage nach \textit{ss} und \textit{ß} unter Hinzuziehung des einfachen \textit{s} zu erörtern.
Die Regel, dass nach langem Vokal \textit{ß} (\textit{Maß}) und nach kurzem Vokal \textit{ss} (\textit{krass}) steht, ist nämlich prinzipiell gar nicht falsch.
Aus ihr lässt sich aber nicht ableiten, warum \zB \textit{Mus} nicht *\textit{Muß} (vgl. \textit{Fuß}) und \textit{was} nicht *\textit{wass} (vgl. \textit{Hass}) geschrieben wird.%
\footnote{Letzteres könnte mit Bezug auf Gelenk- und Konstantschreibung erklärt werden, wie in Abschnitt~\ref{sec:laengeschreib} angedeutet wurde.
Das Problem mit \textit{Mus} und \textit{Fuß} aber nicht.}
Vor allem für /\textipa{s}/ nach langem Vokal im Wortauslaut ist es schlicht nicht vollständig systematisch (wenn auch systematischer als vor der Reform von 1996) geregelt, ob einfaches \textit{s} steht oder auf \textit{ß} bzw.\ \textit{ss} ausgewichen wird (aber vgl.\ auch Abschnitt~\ref{sec:konstanz}).
Im Rahmen der Silbengelenkschreibungen ist die Betrachtung eines Kontextes, in dem die drei \textit{s}-Schreibungen jede eine eigene phonologische Variante kodieren, viel interessanter.
Es bieten sich die Wörter in (\ref{ex:phonschrift4000}) in Zusammenhang mit den Analysen in Abbildung~\ref{fig:busen} an.

\begin{exe}
  \ex\label{ex:phonschrift4000} 
  \begin{xlist}
    \ex{\label{ex:phonschrift4000a} Busen}
    \ex{\label{ex:phonschrift4000b} Bussen}
    \ex{\label{ex:phonschrift4000c} Bußen}
  \end{xlist}
\end{exe}

\begin{figure}[!htbp]
  \centering
  \Tree{
    & \K{PhW}\B{d}\B{drr} \\
    & \K{Silbe}\B{ddl}\B{d} && \K{Silbe}\B{ddl}\B{d} \\
	& \K{R}\B{d} && \K{R}\B{d}\B{dr} \\
    \K{A}\B{d} & \K{K}\B{d} & \K{A}\B{d} & \K{K}\B{d} & \K{E}\B{d} \\
    \K{\textipa{b}} & \K{\textipa{u:}} & \K{\textipa{z}} & \K{\textipa{@}} & \K{\textipa{n}} \\
  }\\
  \vspace{0.5cm}
  \Tree{
    & \K{PhW}\B{d}\B{drrr} \\
    & \K{Silbe}\B{ddl}\B{d} &&& \K{Silbe}\B{ddl}\B{d} \\
	& \K{R}\B{d}\B{dr} & && \K{R}\B{d}\B{dr} \\
    \K{A}\B{d} & \K{K}\B{d} & \K{E}\B{dr} & \K{A}\B{d} & \K{K}\B{d} & \K{E}\B{d} \\
    \K{\textipa{b}} & \K{\textipa{U}} && \K{\textipa{s}} & \K{\textipa{@}} & \K{\textipa{n}} \\
  }\\
  \vspace{0.5cm}
  \Tree{
    & \K{PhW}\B{d}\B{drrr} \\
    & \K{Silbe}\B{ddl}\B{d} &&& \K{Silbe}\B{ddl}\B{d} \\
	& \K{R}\B{d}\B{dr} & && \K{R}\B{d}\B{dr} \\
    \K{A}\B{d} & \K{K}\B{d} & \K{E}\B{d} & \K{A}\B{d} & \K{K}\B{d} & \K{E}\B{d} \\
    \K{\textipa{b}} & \K{\textipa{u:}} & \K{\textipa{s}} & \K{\textipa{s}} & \K{\textipa{@}} & \K{\textipa{n}} \\
  }
  \caption{Analysen der Silbenstruktur Wörter \textit{Busen}, \textit{Bussen} und \textit{Bußen}}
  \label{fig:busen}
\end{figure}

\index{Silbengelenk!und Eszett}
Die Theorie vom Silbengelenk erlaubt es uns, anzunehmen, dass es im Deutschen gar keine offenen Silben mit kurzem Vokal (einmorige Silben) gibt.
Diese Analyse für \textit{Busen} ist damit insofern in Einklang, als die Erstsilbe zwar offen, aber dank des langen Vokals trotzdem zweimorig ist.
Die zweite Silbe \textipa{[z@n]} beginnt mit einem stimmhaften /\textipa{z}/.
Das ist systemkonform, denn /\textipa{z}/ kann nur im Anfangsrand und /\textipa{s}/ nur im Endrand (oder extrasilbisch danach) stehen.

Im Wort \textit{Bussen} \textipa{[bU\Sgel{s}@n]} ist der Vokal der Erstsilbe kurz, die Silbe aber dank Silbengelenk geschlossen und damit zweimorig.
Die Silbengelenkschreibung \textit{ss} zeigt genau dies an.
In solchen Wörtern sollte ein /\textipa{z}/ nicht möglich sein, denn durch die Silbengelenkposition steht das Segment ja stets in einem Endrand, in dem die Auslautverhärtung wirkt und bewirkt, dass jedes /\textipa{z}/ als \textipa{[s]} realisiert wird (s.\ auch Abschnitt~\ref{sec:einsilblerzweisilbler}, besonders S.~\pageref{abs:silbengelenkstimmlos}).
Das ist der Grund, warum aus Dialekten stammende Wörter wie \textit{quasseln} mit stimmhaftem /\textipa{z}/ am Silbengelenk so schlecht ins Gesamtsystem passen.
\textit{quasseln} sollte angesichts der Schreibung und der phonotaktischen Regularitäten analog zu \textit{prasseln} \textipa{[pKa\Sgel{s}@ln]} als \textipa{[kva\Sgel{s}@ln]} realisiert werden, bei vielen Sprechern wird es aber als \textipa{[kva\Sgel{z}@ln]} mit einem bei strenger Auslautverhärtung aus Sicht des phonologischen Systems nicht möglichen stimmhaften Silbengelenk realisiert.
Im Unterschied zu den in Abschnitt~\ref{sec:einsilblerzweisilbler} erwähnten Wörtern wie \textit{Bagger} und \textit{Robbe} hat man bei \textit{quasseln} zusätzlich den erheblichen Nachteil, dass für /\textipa{s}/ und /\textipa{z}/ nur ein Buchstabe für die stimmhafte und die stimmlose Variante zur Verfügung steht, für /\textipa{k}/ und /\textipa{g}/ usw.\ aber zwei, die den Unterschied im Stimmton anzeigen.
Die Nicht-Verfügbarkeit von zwei Buchstaben für /s/ und /z/ können wir nun als starken Hinweis darauf werten, dass wir zumindest für den Kern des phonologischen Systems ohne eine Opposition von /\textipa{s}/ und /\textipa{z}/ auskommen sollten, zumal die Segmente eigentlich nahezu komplementär verteilt sind und damit eventuell auf ein zugrundeliegendes Segment reduziert werden können.
Hier wird argumentiert, dass dies möglich ist.

Das eigentliche Problem sind Wörter wie \textit{Bußen}.
Hier ist der Vokal lang, und in der Silbifizierung \textipa{[bu:.s@n]} erwarten wir daher prinzipiell kein Silbengelenk.
Das Ziel ist jetzt also, eine in das System passende Ableitung für \textipa{[bu:.s@n]} -- ggf.\ auch eine etwas andere Repräsentation -- für diese Wörter zu finden.%
\footnote{Die Analyse aus der ersten Auflage dieses Buchs, die ad hoc ein Silbengelenk in Wörtern wie \textit{Bußen} eingeführt hat, wurde in persönlicher Kommunikation mir gegenüber teilweise energisch kritisiert.
In der Tat lässt sich ein Silbengelenk nach langem Vokal nicht rechtfertigen, worauf mich Ulrike Sayatz eigentlich auch schon vor der Fertigstellung der ersten Auflage hingewiesen hatte.
Die jetzt präsentierte Analyse basiert auf der selben Überlegung, kommt aber ohne unplausible Ad hoc-Annahmen aus.}
Wie oben erwähnt, stört die Silbe \textipa{[s@n]} aber im System, weil stimmloses \textipa{[s]} hier im Anfangsrand steht, der dabei aber nicht an einem Silbengelenk beteiligt ist.
Das graphematische System scheint uns darauf hinzuweisen, dass hier etwas nicht stimmt, weil extra für die sehr spezielle Konstellation (langer Vokal vor der Silbengrenze mit folgendem /s/ im Anfangsrand) ein Buchstabe (nämlich \textit{ß}) existiert.

Die Lösung ist im Prinzip nicht schwierig.
Zunächst verbannen wir aus allen zugrundeliegenden Formen das /s/ und setzen /z/ ein.
Das funktioniert gut, weil sich zusammen mit der Auslautverhärtung das normale Verteilungsmuster von \textipa{[s]} und \textipa{[z]} automatisch ergibt, vgl.\ (\ref{ex:eszett001}).
Im Endrand wird /z/ durch die Auslautverhärtung zu \textipa{[s]}.
Das gilt auch am Silbengelenk, s.\ (\ref{ex:eszett003}).

\begin{exe}
  \ex{\label{ex:eszett001} /zog/ (\textit{Sog}) \phopro\ \textipa{[zo:k]}}
  \ex{\label{ex:eszett002} /\textipa{f\u{a}z}/ (\textit{Fass}) \phopro
  \begin{xlist}
  	\ex[ ]{\label{ex:eszett002a} \textipa{[fas]}}
  	\ex[*]{\label{ex:eszett002b} \textipa{[faz]}}
  \end{xlist}
  }
  \ex{\label{ex:eszett003} /\textipa{bUz@n}/ (\textit{Bussen}) \phopro
  \begin{xlist}
  	\ex[ ]{\label{ex:eszett003a} \textipa{[bU\Sgel{s}@n]}}
  	\ex[*]{\label{ex:eszett003b} \textipa{[bUz@n]}}
  \end{xlist}
  }
\end{exe}

Weiterhin nehmen wir für die problematischen Wörter wie \textit{Bußen} eine zugrundeliegende Form mit zwei /z/ an, also /\textipa{buzz@n}/.
Das ist -- wenn man unbedingt möchte -- der einzige Trick an unserer Analyse, der nicht zusätzlich motiviert werden kann.
Damit ergäbe sich für \textit{Bußen} eigentlich *\textipa{[bu:s.z@n]}, weil die beiden /z/ nicht zusammen in den Anfangsrand der zweiten Silbe passen (*\textipa{[zz@n]}) und auf den Endrand der ersten und den Anfangsrand der zweiten Silbe verteilt werden, hypothetisch als *\textipa{[bu:z.z@n]}.
Im Endrand der erste Silbe muss der Auslautverhärtung genügt werden, so dass *\textipa{[bu:s.z@n]} herauskommt.
Das letzt Problem stellt dann der Anlaut der zweiten Silbe dar, weil das Wort eben nicht als *\textipa{[bu:s.z@n]}, sondern als \textipa{[bu:.s@n]} oder aber -- wie hier angenommen wird -- \textipa{[bu:s.sen]} artikuliert wird.
Das Problem wird mit Blick auf Beispiele wie die in (\ref{ex:eszett004}) gelöst.

\begin{exe}
  \ex\label{ex:eszett004}
  \begin{xlist}
  	\ex{\label{ex:eszett004a} /\textipa{\u{E}kz@}/ \phopro\ \textipa{[PEk.s@]} (\textit{Echse})}
  	\ex{\label{ex:eszett004b} /\textipa{\u{E}Kbze}/ \phopro\ \textipa{[P\t{E@}p.s@]} (\textit{Erbse})}
  \end{xlist}
\end{exe}

Wörter wir die in (\ref{ex:eszett004a}) sind die einzigen anderen Wörter, in denen \textipa{[s]} im Anfangsrand vorkommt, und zwar in direktem Kontakt mit stimmlosen Plosiven in den Endrändern der vorausgehenden Silbe.
Offensichtlich wird /z/ hier an die Stimmlosigkeit des vorangehenden Endrands angepasst (\textit{assimiliert}).
Es gibt keinen Grund zur Annahme, dass im Fall von *\textipa{[bu:s.z@n]} nicht genau dasselbe passiert, so dass das untere Diagramm in Abbildung~\ref{fig:busen} die angemessene Analyse darstellt.
Die Opposition zwischen /\textipa{s}/ und /\textipa{z}/ ist abgeschafft.%
\footnote{Wir brauchen hier die Zusatzannahme, dass im Deutschen kein hörbarer Unterschied zwischen \textipa{[bu:s.s@n]} und \textipa{[bu:.s@n]} besteht.
Diese Zusatzannahme ist insofern gerechtfertigt, als es im gesamten System keinen Fall gibt, in dem doppelte bzw.\ lange Konsonanten (sogenannten \textit{Geminaten}) phonetisch realisiert werden.}
In (\ref{ex:eszett005}) wird die Analyse (unter Verzicht auf [~] und /~/) in Einzelschritten nochmals aufgerollt.

\begin{exe}
  \ex\label{ex:eszett005} \begin{tabular}[t]{lll}
  	 1. & \textipa{buzz@n} & zugrundeliegende Form \\
  	 2. & \textipa{buz.z@n} & Silbifizierung \\
  	 3. & \textipa{bu:z.z@n} & Längung gespannter Vokale \\
  	 4. & \textipa{bu:s.z@n} & Auslautverhärtung \\
  	 5. & \textipa{bu:s.s@n} & Assimilation des Anfangsrands \\
  \end{tabular}
\end{exe}

Das Eszett kodiert in dieser Analyse einerseits die zwei zugrundeliegenden /z/ und stellt gleichzeitig eine Dehnungsschreibung dar.
Alternativ käme die Schreibung mit Doppelkonsonant infrage, bei der dann eine Dehnungsschreibung die Gelenk-Lesart des Doppelkonsonanten blockiert, also *\textit{Buhssen}.
Intuitiv wird das von Sprechern auch wie beabsichtigt gelesen (ebenso *\textit{Strahssen} oder *\textit{Straassen} usw.).
Mit der \textit{e}-Dehnungsschreibung oder Diphthongen kommt man sogar in den Bereich von normativ zwar falschen, aber hochgradig akzeptablen Schreibungen, \zB \textit{fliessen} statt \textit{fließen} oder \textit{draussen} statt \textit{draußen}.
Der systematische Kern der \textit{ss}- und \textit{ß}-Schreibungen ist damit beschrieben.

\subsection{\textit{h} zwischen Vokalen}

\label{sec:intervokh}

Wir schließen den Abschnitt über Silben- und Wortschreibungen mit der Betrachtung einer Besonderheit aus dem Bereich der Dehnungsschreibungen.
In Wörtern wie \textit{wehe} /\textipa{ve@}/, \textit{Ruhe} /\textipa{Ku@}/, \textit{fliehe} /\textipa{fli@}/, \textit{Krähe} /\textipa{kKE@}/ usw.\ wird jeweils ein \textit{h} geschrieben, das genau wie die Schärfungs- und Dehnungsschreibungen nicht segmental gelesen wird.
Es entspricht also in der Phonologie nicht einem /\textipa{h}/.
Da die Erstsilben in diesen Fällen alle lang sein müssen, weil sie offen sind, könnte man einfach davon ausgehen, dass es eine Dehnungsschreibung ist.
Die Tatsache, dass dieses \textit{h} allerdings mit der \textit{e}-Dehnung in \textit{fliehe}, \textit{wiehern} und anderen Wörtern zusammen vorkommt, ist ein Hinweis darauf, dass es als Zusatzfunktion die Silbengrenze zwischen zwei Vokalen markiert.
Dieses \textit{h} ist zudem obligatorisch, wenn eine offene Silbe und eine vokalisch anlautende Silbe aufeinandertreffen.
Normale Dehnungsschreibungen sind allerdings sonst nie obligatorisch, sondern immer fakultativ.%
\footnote{Das \textit{h} wird nach Diphthongen allerdings konsequent nicht geschrieben (\zB \textit{Reue} und \textit{Kleie} statt *\textit{Reuhe} und *\textit{Kleihe}).}

Man kann daher annehmen, dass die eigentliche Funktion des \textit{h} hier ist, den Anlaut der zweiten Silbe graphisch zu kennzeichnen.
In Schreibungen wie *\textit{wee} (statt \textit{wehe}), *\textit{Rue} (statt \textit{Ruhe}), *\textit{fliee} (statt \textit{fliehe}) und *\textit{Kräe} (statt \textit{Krähe}) wären sonst die Silbengrenzen nicht nur schlecht graphisch markiert, sondern es käme auch zu Ambiguitäten.
Zum Beispiel könnte \textit{wee} auch einfach mit \textit{e} als Dehnungsschreibung für /\textipa{ve:}/ stehen (parallel zu \textit{Schnee}).
Es bleiben trotzdem Schreibungen wie \textit{fliehst}.
Hier steht das \textit{h} als redundante Dehnungsschreibung nicht an der Silbengrenze, und es kann demnach auch nicht die Silbengrenze markieren.
Dass es sich trotzdem nicht um eine doppelte Dehnungsschreibung handelt, wird in Abschnitt~\ref{sec:konstanz} gezeigt.


\Zusammenfassung{
\textit{Dehnungsschreibungen} sind nur in langen offenen Einsilblern zuverlässig anzutreffen, ansonsten eine fakultative Kennzeichnung der Vokallänge.
\textit{Schärfungsschreibungen} stehen immer in kurzen geschlossenen Einsilblern und am \textit{Silbengelenk}, sonst nie.
Wenn \textit{h} an der Silbengrenze zwischen Vokalen steht, markiert es primär den Anfang der zweiten Silbe (ohne Anfangsrand) und ist (wenn überhaupt) nur sekundär eine Dehnungsschreibung.
}


\section{Betonung und Hervorhebung}

\label{sec:hervorhebung}
\index{Majuskel}
\index{Akzent!Schreibung}

Über Majuskeln und Minuskeln wurde noch nichts gesagt, weil die Unterscheidung zwischen ihnen für die phonologische Seite der Graphematik keine Rolle spielt (aber s.\ Kapitel~\ref{sec:andereschrift}).
Auf jeden Fall sind die meisten Buchstaben -- der Normalfall -- in deutschen Texten Minuskeln.
Majuskeln sind seltener und markieren stets besondere Funktionen.
Im Bereich der Gebrauchsschreibungen gibt es nun Phänomene, die möglicherweise einen phonologischen Effekt kodieren, der in der Standardschreibung niemals markiert wird.
Die Beispiele in (\ref{ex:phonschrift6000}) zeigen das Phänomen.%
\footnote{Alle Belege stammen aus dem Korpus DECOW14AX (\url{https://webcorpora.org}) und sind über die URL darin dauerhaft auffindbar.}

\begin{exe}
  \ex\label{ex:phonschrift6000} 
  \begin{xlist}
    \ex{\label{ex:phonschrift6000a} Genau DAS ist das Problem!\footnote{\url{http://forum.rundschau-online.de/archive/index.php/t-321.html}}}
    \ex{\label{ex:phonschrift6000b} ICH MUSS WEG!\footnote{\url{http://www.meinliebeskummer.de/forum/archive/index.php/t-41-p-28.html}}}
    \ex{\label{ex:phonschrift6000c} Das war nicht nur auf JUNGpferde bezogen, sondern allgemein auf "`Pferde"'.\footnote{\url{http://www.stallboard.de/archive/t-341.html}}}
  \end{xlist}
\end{exe}

In diesen Sätzen werden ganze Wörter (\textit{DAS}), ganze Sätze (\ref{ex:phonschrift6000b}) und Teile von Wörtern (\textit{JUNGpferde}) in Majuskeln geschrieben.
Das ist höchst auffällig, weil sonst nur einzelne Buchstaben an Wortanfängen als Majuskel geschrieben werden können.
Hier findet offensichtlich eine Art von Hervorhebung statt, und zwar jenseits der orthographischen Norm, also als Gebrauchsschreibung.
Hervorhebung ist ein schlecht definierter Begriff, und man würde vielleicht gerne die Funktionen dieser Majuskelschreibungen genauer benennen.
In (\ref{ex:phonschrift6000a}) wird offensichtlich \textit{DAS} (bzw.\ das, worauf es sich anaphorisch bezieht) als der Gegenstand oder Sachverhalt hervorgehoben, über den dann gesagt wird, er sei das Problem.
In (\ref{ex:phonschrift6000b}) wird evtl.\ dem ganzen Satz Emphase verliehen, analog zu lautem Sprechen.
In (\ref{ex:phonschrift6000c}) findet sehr deutlich eine Kontrastierung statt, indem die Jungpferde allen Pferden gegenübergestellt werden.
Ganz offensichtlich gibt es nicht eine einzige Funktion, die man Majuskelschreibungen zuordnen kann, sondern mehrere.

Zumindest wenn wie in (\ref{ex:phonschrift6000a}) und (\ref{ex:phonschrift6000c}) einzelne Wörter in Majuskeln stehen, kann man aber sehr wahrscheinlich einen einheitlichen Bezug zur Phonologie herstellen.
Alle diese Wörter würden in der Aussprache eine prominente Betonung erhalten.
Die Majuskelschreibungen scheinen hier eine ähnliche Funktion zu haben, wie es die Betonung hätte.
So ergäbe sich -- wenn auch nur begrenzt systematisch -- ein neuer Anknüpfungspunkt zwischen Graphematik und Prosodie.
Dieses Phänomen ist allerdings noch nicht hinreichend untersucht, und es gibt keine eindeutigen abgesicherten Ergebnisse.
Es wurde hier aufgenommen, um zu zeigen, dass unsere Schreibungen nicht nur mechanisch einer Norm folgen (oder eben gegen diese verstoßen), sondern dass Schreiber Möglichkeiten des graphematischen Systems auch kreativ ergreifen können, um ihre Sprache möglichst erfolgreich zu kodieren.


\Zusammenfassung{
Schreiber machen kreativen Gebrauch von den Möglichkeiten des Schreibsystems.
\textit{Vollständige Majuskelschreibungen} sind ein Beispiel dafür, und sie korrelieren wahrscheinlich mit einer prosodischen Markierung (Betonung).
}


\section{Ausblick auf den Nicht-Kernwortschatz}

\label{sec:nichtkernschreib}
\index{Kernwortschatz}

Im Nicht-Kernwortschatz (vgl.\ auch Abschnitt~\ref{sec:kernundperipherie}) finden sich diverse phonologische und graphematische Abweichungen zum Kernwortschatz.
Dabei gilt, dass es keine scharfe Trennung zwischen zwei Extremen \textit{Kern} und \textit{Peripherie} im Wortschatz gibt, sondern geringere und größere Nähe zum Kern.
Alle Wortformen von Wörtern, die nicht deriviert oder komponiert und dabei nicht einsilbig (\textit{Maus}, \textit{gehst}) oder trochäisch mit Schwa-Zweitsilbe (\textit{backe}, \textit{alten}, \textit{Brüdern}) oder daktylisch mit Schwa-Zweit- und Drittsilbe (\textit{ruderest}, \textit{älteren}) sind, sind zumindest näher an der Peripherie als die Wörter, die diese Bedingungen erfüllen.%
\footnote{Zu den Fußtypen s.\ Abschnitt~\ref{sec:wortakzentimdeutschen}.}
Bei den Kern-Fußformen muss man zusätzlich zwischen Wortarten und Wortformen unterscheiden.
So sind \zB daktylische Wörter nur im Bereich bestimmter Adjektiv- und Verbformen wirklich im Kern.
Daktylische Substantive wie \textit{Charisma} \textipa{[ka.KIs.ma]} (bei vielen Sprechern mit Betonung auf der ersten Silbe) gehören nicht zum Kern.
Gleichsam ist bei den Kern-Substantiven der Singular zwar oft einsilbig, der Plural aber immer trochäisch (\textit{Tür} und \textit{Türen}, \textit{Tisch} und \textit{Tische} usw.).


Mit einem so eng gefassten Kern wird man natürlich dem Gesamtsystem nicht wirklich gerecht.
Einen erweiterten Kern erhalten wir durch Hinzuziehen von derivierten und komponierten Wörtern.
Hier findet man dann vor allem unbetonte Präfixe vor trochäischen und daktylischen Füßen (\textit{veränderst}, \textit{überredetest}), wohingegen sich die Suffixe normalerweise unbetont nach den einsilbigen oder trochäischen Stämmen einsortieren und damit neue Trochäen und Daktylen erzeugen (\textit{Haltung}, \textit{Schreiber}, \textit{Gläubigkeit}).
Präfigierung und Suffigierung treten natürlich auch zusammen auf (\textit{Unterhaltung}).
Weiters gibt es dann überwiegend mehrfüßige Komposita, die Ergebnisse von Präfigierung und Suffigierung enthalten können (\textit{Häuserfronten}, \textit{Unterhaltungsführung}).
Die Verschriftung dieser Wörter folgt ganz einfach aus den Kernprinzipien, vor allem wegen der in Abschnitt~\ref{sec:konstanz} noch zu beschreibenden Prinzipien der Konstantschreibung.

Weiter vom Kern entfernt sind Simplizia, die mehrere lange gespannte Vokale enthalten, womit oft eine atypische Fußstruktur einhergeht (\textit{Oma}, \textit{Politik}, \textit{Organigramm}).
In dieser Gruppe finden wir auch die in Abschnitt~\ref{sec:schwachsubst} besprochenen entlehnten schwachen Substantive mit betonten Letztsilben wie \textit{Apologet}, \textit{Ignorant}, \textit{Demiurg} usw.
Zumindest vom Betonungsmuster ähnlich sind die in Abschnitt~\ref{sec:wortakzentimdeutschen} kurz diskutierten w-Adverben mit Endbetonung wie \textit{warum}, \textit{weshalb} usw.

Einen ganz eigenen dem Kern sehr fernen Bereich erhält man durch Hinzunahme von Wörtern, die Segmente enthalten, die es im Kern gar nicht gibt, oder die es dort in der jeweiligen Position nicht gibt.
Hierzu gehören Wörter wie in (\ref{ex:phonschrift777}), wo die Transkription sicherheitshalber phonetisch erfolgt, weil die Bestimmung der zugrundeliegenden Form weitere Probleme mitbringt.

\begin{exe}
  \ex\label{ex:phonschrift777}
  \begin{xlist}
    \ex{\label{ex:phonschrift777a} Chips \textipa{[\t{tS}Ips]}}
    \ex{\label{ex:phonschrift777b} Dschungel \textipa{[\t{dZ}UN@l]}}
    \ex{\label{ex:phonschrift777e} Chuzpe \textipa{[XU\t{ts}p@]}}
    \ex{\label{ex:phonschrift777c} Pteranodon \textipa{[ptEKanodOn]}}
    \ex{\label{ex:phonschrift777d} mailen \textipa{[m\t{Ee}l@n]}, \textipa{[m\t{EI}l@n]}}
  \end{xlist}
\end{exe}

In (\ref{ex:phonschrift777a}) steht \textipa{[\t{tS}]} in einer Position, in der es überwiegend nicht steht.
Einer der Gründe, phonologisch \textipa{[\t{tS}]} im Deutschen nicht als echte Affrikate zu klassifizieren, ist gerade, dass es zwar im Endrand vorkommt (\textit{Matsch}) aber eben nicht im Anfangsrand.
Wenn nicht auf die angepasstere Realisierung \textipa{[SIps]} ausgewichen wird, steht \textit{Chips} also außerhalb des Kerns.
Noch mehr gilt dies für (\ref{ex:phonschrift777b}), weil \textipa{[\t{dZ}]} im Kern in gar keiner Position vorkommt.
Das Wort \textit{Chuzpe} hat \textipa{[X]} im Silbenanlaut, wo es nicht hingehört.
Das Plateau \textipa{[pt]} in (\ref{ex:phonschrift777c}) ist im Kern völlig ausgeschlossen (und eine typische Realisierung von deutschen Sprechern dürfte daher wahrscheinlich \textipa{[p@tEKanodOn]} sein).
Schließlich enthält \textit{mailen} (wenn nicht auch hier auf die kommodere Realisierung \textipa{[me:l@n]} ausgewichen wird) einen Diphthong, den es im Kernwortschatz nicht gibt.

Man kann an diesen Beispielen gut zeigen, warum man sie nicht in die Beschreibung des Kernwortschatzes aufnehmen sollte.
Würde man \textit{Chuzpe} \zB als konform zu den allgemeinen Generalisierungen beschreiben wollen, müsste man diese Generalisierungen anpassen, und die ansonsten sehr gut funktionierende Beschreibung der Verteilung von \textipa{[\c{c}]} und \textipa{[X]} wäre dahin.
Gerade weil diese Wörter per Definition eine geringe Typenhäufigkeit haben (s.\ Abschnitt~\ref{sec:kernundperipherie}) und oft nur in bestimmten Registern und Stilen vorkommen, wäre dies mehr als ungeschickt.

Es sind nun nicht alle diese Arten von kernfernen Wörtern gleichermaßen anfällig für Anomalien in der Schreibung.
Ganz besonders sticht die Gruppe der zu (\ref{ex:phonschrift777}) ähnlichen Lehnwörter heraus, die oft die Schreibung der Gebersprache konservieren.
Hierbei ist zu beachten, dass viele Lehnwörter phonologisch Wörter des Kernwortschatzes sind, aber trotzdem eine kernferne Schreibung aufweisen.
Ein Wort wie \textit{Christen} (statt *\textit{Kristen}) ist phonologisch in keiner Form auffällig, sticht aber durch die Schreibung \textipa{[chr]} für /\textipa{kK}/ heraus.
Ähnliches gilt für \textit{Vase} (statt *\textit{Wase}) oder \textit{Beamer} (statt *\textit{Biemer}).
Im Bereich der irregulären Schreibungen gibt es eine breite Variation (mit und ohne phonologische Auffälligkeit), die hier nicht in voller Breite besprochen wird (s.\ Übung~\ref{u146}).
Beispiele sind \textit{chthonisch}, \textit{Genre}, \textit{Gonorrhö}, \textit{Pendant}, \textit{Souvenir}, \textit{Shopping}, \textit{Theorie}, \textit{zynisch}.


\Zusammenfassung{
Zugehörigkeit zum \textit{Kernwortschatz} ist graduell, und typischerweise gibt es Gruppen von Wörtern (kleine Klassen), die auf gleiche Weise von den Regularitäten des Kernwortschatzes abweichen.
Hauptquelle für anomale Schreibungen sind \textit{Lehnwörter}, die die Schreibung der Gebersprache konservieren, was allerdings nicht notwendig mit einer anomalen Phonologie einhergehen muss.
}


\Uebungen

\Uebung \label{u141} In Tabelle~\ref{tab:dehnkuerzschreib} (S.~\pageref{tab:dehnkuerzschreib}) fehlen die Vokale /\textipa{y}/, /\textipa{Y}/ und /\textipa{\o}/, /\textipa{\oe}/.
Finden Sie Beispiele für diese Vokale und jede mögliche Zeile der Tabelle.

\Uebung[\tristar] \label{u142} Argumentieren Sie dafür, dass die Diphthonge in Tabelle~\ref{tab:dehnkuerzschreib} (S.~\pageref{tab:dehnkuerzschreib}) nicht aufgeführt sein müssen.

\Uebung[\tristar] \label{u143} Warum ist es angesichts des phonologischen und graphematischen Systems des Deutschen folgerichtig, dass der glottale Plosiv wie in \textipa{[PEnd@]} nicht durch einen Buchstaben verschriftet wird.

\Uebung \label{u144} Finden Sie in den folgenden Beispielen alle Dehnungs- und Schärfungsschreibungen.
Welche Dehnungsschreibungen sind nach den allgemeinen Regularitäten optional?
Schreiben Sie die entsprechenden Wörter jeweils ohne Dehnungsschreibung.
Finden Sie außerdem alle Silben, in denen eine Dehnungsschreibung möglich wäre, aber keine steht.
Schreiben Sie die entsprechenden Wörter jeweils mit Dehnungsschreibung.

\begin{enumerate}\Lf
  \item Auf dem Wohnungsmarkt ist Entspannung eingekehrt.
  \item Der König von Schweden hatte angeblich Kontakte zur Unterwelt.
  \item Eine Leseprobe endete in einer wüsten Schlägerei.
  \item Unter einer einstweiligen Verfügung kann sich Ischariot nichts vorstellen.
  \item Mit Möhren kann Vanessa ihr Pferd glücklich machen.
  \item Sie fragen sich jetzt sicher, wer die Stallpflege übernimmt.
  \item Passen Sie beim Einsteigen auf Ihr Knie auf. 
\end{enumerate}

\Uebung[\tristar] \label{u145} Warum können wir davon ausgehen, dass innerhalb des Kernwortschatzes in trochäischen Simplizia außer denen vom Typ \textit{Wehe}, \textit{Ruhe}, \textit{Krähe} usw. (Abschnitt~\ref{sec:intervokh}) phonologisch der Anfangsrand der zweiten Silbe immer gefüllt ist?

\Uebung \label{u146} Was macht die folgenden Wörter zu Schreibungen jenseits des Kerns?

\begin{enumerate}\Lf
  \item chthonisch
  \item Genre
  \item Gonorrhö
  \item Pendant
  \item Souvenir
  \item Shopping
  \item Theorie
  \item zynisch
\end{enumerate}

\Uebung[\tristar] \label{u147} Warum ist in Tabelle~\ref{tab:segschreibkons} /\textipa{N}/ nicht enthalten?
Argumentieren Sie phonologisch (s.\ Abschnitt\ref{sec:systematikderraender}) und graphematisch.

\Uebung[\tristar] \label{u148} Wie werden \textit{löblich} und \textit{Vertriebler} silbifiziert und warum?

\Uebung[\tristar] \label{u149} Diskutieren Sie, ob die unterschiedilchen Schreibungen von \textit{Tod} (Substantiv) und \textit{tot} (Adjektiv) systematisch sind.

\Uebung[\tristar] \label{u1410} Diskutieren Sie, ob die unterschiedlichen \textit{s}-Schreibungen \textit{Mus} und \textit{Fuß} anhand von Prinzipien des grammatischen Systems erklärt werden können, oder ob es sich bei einer von beiden Schreibungen um eine Unregelmäßigkeit handelt.
