\chapter{Grammatik}

\label{sec:grammatik}

\section{Sprache und Grammatik}

\label{sec:spracheundgrammatik}

In diesem Kapitel wird definiert, was der Auftrag und der Betrachtungsgegenstand der deskriptiven Grammatik ist.
Zunächst geht es um das Verhältnis von Sprache und Grammatik (Abschnitt~\ref{sec:spracheundgrammatik}), dann um den wichtigen Unterschied zwischen deskriptiver und präskriptiver Grammatik (Abschnitt~\ref{sec:deskriptivnormativ}).
Schließlich wird kurz auf empirische Methoden und die Abgrenzung der deskriptiven Grammatik von der Grammatiktheorie eingegangen (Abschnitt~\ref{sec:deskriptivtheoretisch}).

In diesem ersten Abschnitt soll vor allem ein genauerer Begriff von Grammatik eingeführt werden, bzw. die Mehrdeutigkeit des Begriffes erklärt werden.
Idealerweise sollte dieser Definition eine möglichst universelle Definition von Sprache vorausgehen, die allen möglichen Aspekten von Sprache gerecht wird.
Diese Definition gehört jedoch eher in den Bereich der theoretischen Linguistik.
Hier wird einleitend nur ein grober Überblick gegeben, der sich auf den für uns wichtigsten Aspekt von Sprache, nämlich ihren Systemcharakter, beschränkt.

\subsection{Sprache als Symbolsystem}

\label{sec:sprachsystem}

\index{Sprache}

Sprache kann unter sehr verschiedenen Blickwinkeln wissenschaftlich betrachtet werden.
Man kann Sprache als kognitive Aktivität des Menschen betrachten, denn offensichtlich bilden und verstehen Menschen sprachliche Äußerungen mittels kognitiver Vorgänge im Gehirn.
Mit gleichem Recht könnte man Sprache als soziale Interaktion (Kommunikation) charakterisieren und unter diesem Aspekt untersuchen.
Sprache wird tatsächlich in Teildisziplinen der Linguistik aus solchen und vielen anderen Perspektiven betrachtet, und jede Teildisziplin hat eine andere, dem Blickwinkel angepasste Definition von Sprache.

Hier beschränken wir uns so weit wie möglich auf einen ganz bestimmten, eng definierten Aspekt von Sprache, nämlich den Charakter von Sprache als symbolisches System.
Wir gehen dabei davon aus, dass Sprache unabhängig von der Art ihrer Verarbeitung im Gehirn, ihren sozialen Funktionen usw.\ einen solchen Charakter hat.
Damit ist gemeint, dass Sprache aus Symbolen und Symbolverbindungen (Lauten, Silben, Wörtern usw.) besteht, die in systematischen Beziehungen zueinander stehen, und die auf regelhafte Weise zusammengesetzt sind.

Als Sprecher des Deutschen kann man \zB sofort erkennen, dass (\ref{ex:gb2957a}) eine akzeptable Symbolfolge des Symbolsystems \textit{Deutsch} ist, dass (\ref{ex:gb2957b}) zwar aus Symbolen dieses Systems besteht, die aber falsch kombiniert sind, und dass (\ref{ex:gb2957c}) und (\ref{ex:gb2957d}) gar keine Symbole (zumindest Symbole im Sinne von Wörtern) dieses Systems enthalten.

\begin{exe}
  \ex
  \begin{xlist}
    \ex{\label{ex:gb2957a} Dies ist ein Satz.}
    \ex{\label{ex:gb2957b} Satz ein dies ist.}
    \ex{\label{ex:gb2957c} Kno kna knu.}
    \ex{\label{ex:gb2957d} This is a sentence.}
  \end{xlist}
\end{exe}

\index{Regularität}

Bezüglich (\ref{ex:gb2957a}) und (\ref{ex:gb2957b}) sind nun zwei Dinge bemerkenswert.
Erstens können wir sofort erkennen, dass die Symbole in (\ref{ex:gb2957a}) konform zu einem System von Regularitäten sind, auch wenn wir diese Regularitäten nicht immer (sogar meistens nicht) explizit benennen können.
Dass dies bei (\ref{ex:gb2957b}) nicht der Fall ist, erkennen wir auch unverzüglich und ohne explizit nachzudenken.
Als Sprecher des Deutschen haben wir also offensichtlich ein System von Regularitäten verinnerlicht, das es uns ermöglicht, zu beurteilen, ob eine Symbolfolge diesem System entspricht oder nicht.
Außerdem können wir aus den Bedeutungen der einzelnen bedeutungstragenden Symbole (der Wörter) und der Art, wie diese zusammengesetzt sind, unverzüglich die Bedeutung der Symbolfolge (des Satzes) erkennen.
Die zuletzt genannte Eigenschaft von Sprache nennt man \textit{Kompositionalität}.

\Definition{Kompositionalität}{
\label{def:kompositionalitaet}
Die Bedeutung komplexer sprachlicher Ausdrücke ergibt sich aus der Bedeutung ihrer Teile und der Art ihrer grammatischen Kombination.
\index{Kompositionalität}
}

Das Symbolsystem mit seinen Regularitäten und die Art der kompositionalen Konstruktion von Bedeutung sind dabei in gewissem Maß unabhängig voneinander, wie man an Satz (\ref{ex:dg7717}) zeigen kann.

\begin{exe}
  \ex{\label{ex:dg7717} Dies ist ein Kneck.}
\end{exe}

Satz (\ref{ex:dg7717}) hat sicherlich für keinen Leser dieses Buches eine vollständig er\-schließ\-bare Bedeutung.
Dies liegt aber nicht daran, dass die Symbolfolge inkorrekt konstruiert wäre, sondern nur daran, dass wir nicht wissen, was ein \textit{Kneck} ist.
Unter der Annahme (die wir implizit sofort machen, wenn wir den Satz lesen), dass es sich bei dem Wort \textit{Kneck} um ein Substantiv handelt, kategorisieren wir den Satz als akzeptabel.
Wir können sogar sicher sagen, dass wir die Bedeutung des Satzes kennen würden, sobald wir erführen, was ein \textit{Kneck} genau ist.
Ähnliches gilt für widersinnige oder widersprüchliche Sätze wie die in (\ref{ex:dg7718}), die ebenfalls grammatisch völlig in Ordnung sind.
Und gerade weil wir ein implizites Wissen davon haben, wie man aus Bedeutungen von Wörtern und der Art ihrer Zusammensetzung Bedeutungen von Sätzen ermittelt, können wir feststellen, dass die Sätze widersinnig bzw.\ widersprüchlich sind.

\begin{exe}
  \ex\label{ex:dg7718}
  \begin{xlist}
    \ex{Jede Farbe ist ein Kurzwellenradio.}
    \ex{Der dichte Tank leckt.}
  \end{xlist}
\end{exe}

Es zeigt sich also, dass die Sprachsymbole (Laute, Wörter usw.) ein eigenes Kombinationssystem (eine Grammatik) haben.
Dieses System ist dafür verantwortlich, dass wir Bedeutungen von komplexen Symbolfolgen verstehen (interpretieren) können.
Gleichzeitig ist das System selber aber bis zu einem gewissen Grad unabhängig davon, ob die Interpretation tatsächlich erfolgreich ist.
Wenn es dieses Sprachsystem und die Kompositionalität nicht gäbe, wäre es äußerst schwer, eine Sprache zu erlernen, sowohl als Erstsprache im Kindesalter als auch als Zweit- bzw.\ Fremdsprache.

Wegen der zumindest partiellen Unabhängigkeit des Symbolsystems von der Interpretation ist es legitim und sogar strategisch sinnvoll, zunächst nur das Symbolsystem zu beschreiben, ohne sich über die Bedeutung zu viele Gedanken zu machen.
Daher wird in diesem Buch die Bedeutung aus der grammatischen Analyse so weit wie möglich ausgeklammert.%
\footnote{Es gibt kognitiv ausgerichtete linguistische Theorien, die davon ausgehen, dass das formale System und die Bedeutung nur eingeschränkt die hier angenommene Unabhängigkeit aufweisen.
Es wird dabei \zB experimentell gezeigt, dass die Entscheidung zwischen wohlgeformten und nicht-wohlgeformten Symbolfolgen nicht allein von formal-grammatischen Bedingungen abhängt, sondern zu einem Großteil auch von der Bedeutung.
Für viele der in diesem Buch beschriebenen Phänomene kommt man allerdings recht weit, auch ohne Einbeziehung der Bedeutungsseite.
In Abschnitt~\ref{sec:deskriptivtheoretisch} wird die hier vertretene reduktionistische Position nochmals hinterfragt.}
Definitionen und Beschreibungen, die sich an der Form orientieren, sind meist viel einfacher nachzuvollziehen und anzuwenden sind, als solche, die semantische Beurteilungen erfordern.
Besonders für das Lehramt, in dem primär die Grammatikvermittlung und nicht die Grammatiktheorie im Vordergrund steht, sind klare formale Unterscheidungskriterien wahrscheinlich von höherem Wert als Überlegungen zur Interaktion von Form und Interpretation, die schwerer im Schulunterricht zu vermitteln sind -- auch wenn dies in Schulbüchern sehr oft versucht wird.
Trotzdem wird auch hier öfters die Bedeutungsseite der Sprache berücksichtigt, entweder weil sich bei einem gegebenen Phänomen die Trennung von Grammatik und Bedeutung als besonders schwierig erweist, oder weil die Berücksichtigung der Bedeutung die Argumentation wesentlich verkürzt und vereinfacht.
Dieses pragmatische Vorgehen deutet darauf hin, dass die starke Reduktion auf die Form (bzw.\ auf einen engen Begriff von Grammatik im Sinne einer Formgrammatik) in keiner Weise einer theoretischen Position des Autors entspricht (s.\ auch Abschnitt~\ref{sec:deskriptivtheoretisch}).

Abschließend sei angemerkt, dass Menschen andere externe Systeme von Symbolen normalerweise anders verarbeiten als sprachliche.
Als einfaches Beispiel sei die Gleichung in (\ref{ex:dg9700}) gegeben.

\begin{exe}
  \ex{\label{ex:dg9700} $\sqrt{a^3}\cdot a=a^2$}
\end{exe}

Ob die Gleichung eine Lösung hat (bzw.\ wieviele Lösungen), können die meisten Leser wahrscheinlich in einer überschaubaren Zeit entscheiden.
Das Lösen von Gleichungen ist aber ein bewusster und ggf.\ mühsamer Prozess.
Ob ein Satz ein Satz unserer Erstsprache ist und was er bedeutet, erschließt sich im Normalfall ohne Mühe und ohne Nachdenken unbewusst.
Während wir bei der Produktion und Rezeption von Sprache implizites Wissen anwenden, erfordert die Verarbeitung mathematischer Symbolfolgen explizites Nachdenken.
Genau das ist damit gemeint, dass unser Wissen über Sprache weitgehend implizit statt explizit ist.
Das macht Sprache noch nicht automatisch zu einer exotischen Fähigkeit, da Menschen sehr viele (auch komplexere) Tätigkeiten automatisiert und unbewusst ausführen.

Interessanterweise wird in verschrifteten Sprachen, die nationale Standardvarietäten ausbilden (wie \zB\ Chinesisch, Deutsch, Hindi oder Spanisch), teilweise von diesem Prinzip abgewichen. 
Einerseits führt das Medium der Schrift zu einer Verlangsamung der Sprachproduktion und -verarbeitung gegenüber der gesprochenen Sprache und damit zu einem erhöhten Grad an Reflexion über das Geschriebene.
Andererseits entsteht früher oder später meist eine Diskussion um den \textit{richtigen} oder \textit{guten} Sprachgebrauch, es setzt also eine Normierungsdiskussion ein. 
Diese Diskussion führt dazu, dass sprachliche Strukturen von Sprechern und Schreibern reflektiert werden.
Dies ist dann aber eine Art sekundärer Tätigkeit, die nur begrenzt mit gewöhnlicher Sprachverarbeitung (beim Sprechen und Hören sowie weitgehend auch beim Schreiben und Lesen) zu tun hat.

\subsection{Grammatik}

Wie verhält sich nun der Begriff Grammatik zu dem oben beschriebenen Verständnis von Sprache?
Er wird stark mehrdeutig verwendet, und wir legen die relativ neutrale Definition~\ref{def:grammatik} zugrunde.

\Definition{Grammatik (Sprachsystem)}{
\label{def:grammatik}
Eine Grammatik ist ein System von Regularitäten, nach denen aus einfachen Einheiten komplexe Einheiten einer Sprache gebildet werden.
\index{Grammatik!Sprachsystem}
\index{Regularität}
}

Wir gehen also davon aus, dass die zugrundeliegende Grammatik (das System von Regularitäten) für die Form der sprachlichen Äußerungen (\zB Sätzen) verantwortlich ist, und dass Grammatiker diese Regularitäten durch Beobachtungen dieser Äußerungen zu erkennen versuchen.
Wenn man diese Regularitäten aufschreibt bzw.\ formalisiert, liegt eine wissenschaftliche Grammatik als Modell für die beobachteten Daten vor.

Davon grundsätzlich zu unterscheiden wäre natürlich der Begriff der Grammatik als ein Artefakt (\zB ein Buch), in dem grammatische Regeln festgehalten werden.
Ebenso verschieden ist die Annahme einer mentalen Grammatik in verschiedenen Richtungen der Linguistik, also einer Repräsentation der sprachlichen Regularitäten im Gehirn.
Abgesehen davon bezeichnet der Begriff Grammatik natürlich auch die Wissenschaft, die sich mit grammatischen Regularitäten einzelner oder aller Sprachen beschäftigt.
Vor dem Hintergrund unserer Definition von Grammatik wird im nächsten Abschnitt der zentrale Begriff der Grammatikalität eingeführt.

\subsection{Grammatikalität}

\label{sec:grammungramm}
\label{sec:grammatikalitaet}

\index{Grammatikalität}

Der Begriff der Grammatikalität ist zentral für die Grammatik und die theoretische Linguistik.
Man kann ihn zunächst so definieren, dass man von einem kompetenten Sprecher (oder besser kompetenten \textit{Sprachbenutzer}) ausgeht.

\Definition{Grammatikalität (Sprachbenutzer)}{
  \label{def:grammatikalitaet1} Jede sprachliche Einheit (\zB jeder Satz), die von einem kompetenten Sprachbenutzer als konform zur eigenen Grammatik (als \textit{akzeptabel}) eingestuft wird, ist grammatisch, alle anderen sind ungrammatisch.
Grammatikalität ist die Eigenschaft einer Einheit, grammatisch zu sein.}

\enlargethispage{1\baselineskip}
Oft spricht man nur bei Sätzen und nicht bei kleineren Einheiten von Grammatikalität.
Ein kompetenter Sprachbenutzer muss also gemäß dieser Definition entscheiden können, ob ein Satz, den man ihm präsentiert, ein akzeptabler Satz ist oder nicht.
Kompetent ist ein Sprachbenutzer, wenn er die betreffende Sprache im frühen Kindesalter gelernt hat, sie seitdem kontinuierlich benutzt hat und an keiner Sprachstörung (Aphasie) leidet.
Dass kompetente Sprachbenutzer wirklich immer in der Lage sind, ein eindeutiges Akzeptabilitätsurteil abzugeben, ist mit Sicherheit zu verneinen.
Sehr oft sind Sprachbenutzer angesichts komplexerer Strukturen im Zweifel, ob diese Strukturen akzeptabel sind.
Eine typische Reihe von Sätzen, die dies demonstriert, wird in (\ref{ex:dg9113}) gegeben.%
\footnote{Für die Zusammenstellung der Sätze danke ich Felix Bildhauer.}

\begin{exe}
  \ex\label{ex:dg9113}
  \begin{xlist}
    \ex{\label{ex:dg9113a} Bäume wachsen werden hier so schnell nicht wieder.}
    \ex{\label{ex:dg9113b} Touristen übernachten sollen dort schon im nächsten Sommer.}
    \ex{\label{ex:dg9113c} Schweine sterben müssen hier nicht.}
    \ex{\label{ex:dg9113d} Der letzte Zug vorbeigekommen ist hier 1957.}
    \ex{\label{ex:dg9113e} Das Telefon geklingelt hat hier schon lange nicht mehr.}
    \ex{\label{ex:dg9113f} Häuser gestanden haben hier schon immer.}
    \ex{\label{ex:dg9113g} Ein Abstiegskandidat gewinnen konnte hier noch kein einziges Mal.}
    \ex{\label{ex:dg9113h} Ein Außenseiter gewonnen hat hier erst letzte Woche.}
    \ex{\label{ex:dg9113i} Die Heimmannschaft zu gewinnen scheint dort fast jedes Mal.}
    \ex{\label{ex:dg9113j} Ein Außenseiter gewonnen zu haben scheint hier noch nie.}
    \ex{\label{ex:dg9113k} Ein Außenseiter zu gewinnen versucht hat dort schon oft.}
    \ex{\label{ex:dg9113l} Einige Außenseiter gewonnen haben dort schon im Laufe der Jahre.}
  \end{xlist}
\end{exe}

Bei (\ref{ex:dg9113a}) sind sich die meisten Sprecher des Deutschen sicher (und untereinander einig), dass der Satz akzeptabel ist.
Genauso wird die Entscheidung, dass (\ref{ex:dg9113l}) nicht akzeptabel ist, meist eindeutig gefällt.
Die Sätze dazwischen führen in unterschiedlichem Maß zu Unsicherheiten bezüglich ihrer Akzeptabilität, und größere Gruppen von Sprachbenutzern sind sich selten über die genauen Urteile einig.
Dennoch ist es aus Sicht der Grammatik sinnvoll, zumindest als Arbeitshypothese davon auszugehen, dass eine eindeutige Entscheidung möglich ist.
Diese Vereinfachung kann jederzeit dadurch aufgehoben werden, dass man die Regularitäten nicht mehr als strikt interpretiert, sondern ihnen \zB unterschiedliches statistisches Gewicht gibt.
Anders als in der Physik, die mit tatsächlich ausnahmslos geltenden Naturgesetzen arbeitet, sind in der Grammatik bzw.\ Linguistik Generalisierung, die keine Ausnahmen oder Überlappungen mit anderen Generalisierungen haben, unüblich.

Die zweite Definition der Grammatikalität abstrahiert vom Sprachbenutzer und bezieht sich nur auf eine Grammatik als System von Regularitäten.

\Definition{Grammatikalität (System von Regularitäten)}{
\label{def:grammatikalitaet2}
Jede von einer Grammatik (im Sinne von Definition~\ref{def:grammatik}) beschriebene Symbolfolge ist grammatisch bezüglich dieser Grammatik, alle anderen Symbolfolgen sind ungrammatisch bezüglich dieser Grammatik.
\index{Grammatik}
}

Die Grammatik ist in dieser Definition ein explizit spezifiziertes System von Regularitäten, das definiert, wie aus einfachen Elementen (Symbolen) komplexere Strukturen (Symbolfolgen) zusamengesetzt werden.
Mit Symbol können dabei Laute, Buchstaben, Wörter, Satzteile oder sonstige Größen der Grammatik gemeint sein.
Wo und wie die Grammatik definiert ist oder sein kann, sagt Definition~\ref{def:grammatikalitaet2} nicht.
Es könnte sein, dass es sich wiederum um eine im Gehirn verankerte Sammlung von Regularitäten handelt, also eine Grammatik, die das in Definition~\ref{def:grammatikalitaet1} beschriebene Verhalten eines Sprachbenutzers steuert.
Definition~\ref{def:grammatikalitaet2} kann aber auch auf eine Grammatik bezogen sein, die ein Linguist definiert und niedergeschrieben hat, so wie es in diesem Buch getan wird.

Man verwendet * (den \textit{Asterisk}), um zu markieren, dass eine Struktur ungrammatisch ist.
Angesichts der Mehrdeutigkeit des Begriffes der Grammatik und damit der Grammatikalität müsste man eigentlich zusätzlich angeben, auf welche Grammatik bzw.\ welche Definition von Grammatikalität sich der Asterisk bezieht.
Wenn der Satz in (\ref{ex:grammi2types}) von den Sprechern, die wir befragen, nicht akzeptiert wird, wäre es korrekt, ihn mit einem Asterisk zu markieren, der sich auf Sprecherurteile bezieht.%
\footnote{Zum hier illustrierten Phänomen vgl.\ Abschnitt~\ref{sec:ersiofu}.}

\begin{exe}
  \ex\label{ex:grammi2types}
  \begin{xlist}
    \ex{$\ast^{\mathrm{\textnormal{Sprecher}}}$ Ich glaube, dass Alma die Bücher lesen gewollt hat.}
  \end{xlist}
\end{exe}

Wenn man markieren will, dass eine theoretische Grammatik den entsprechenden Satz nicht beschreibt (unabhängig davon, was Sprachbenutzer dazu sagen), weil sie vielleicht noch nicht vollständig oder nicht exakt genug formuliert ist, wäre eine Markierung wie in (\ref{ex:dg9108}) korrekt.
Diese Markierung sagt nur, dass die Theorie den Satz als ungrammatisch einstuft, auch wenn dies bedeutet, dass die Beurteilung durch die formale Theorie von den Urteilen der Sprachbenutzer abweicht.
Solchen Situationen begegnet man gerade bei der Entwicklung sehr genau formalisierter Grammatiken häufiger.

\begin{exe}
  \ex{\label{ex:dg9108} $\ast^{\mathrm{\textnormal{Theorie}}}$ Ich glaube, dass Alma die Bücher lesen gewollt hat.}
\end{exe}

In diesem Buch markiert der Asterisk, wenn es nicht anders gekennzeichnet wird, immer die Ungrammatikalität bezüglich zu erwartender Grammatikalitätsurteile von kompetenten Sprachbenutzern einer Varietät des Deutschen, die sehr nah am Standard ist.
Dass die Annahme von einheitlich urteilenden kompetenten Sprachbenutzern genauso wie die Annahme einer wohldefinierten standardnahen Varietät des Deutschen Illusionen sind, sollte nach der bisherigen Definition bereits offensichtlich sein.%
\footnote{Abgesehen davon orientieren wir uns hier sehr stark an der geschriebenen Sprache, die sich wesentlich von der gesprochenen unterscheidet.
Das ist teilweise der methodisch-didaktischen Reduktion, teilweise aber auch dem Forschungsstand in der Grammatik und Linguistik geschuldet.
Die Eigenheiten der gesprochenen Sprache sind (immer noch) ein Spezialgebiet innerhalb der Linguistik.}
Unter den in diesem Buch beschriebenen Phänomenen sind allerdings hoffentlich wenige, bei denen in größeren Gruppen von Sprechern der in Deutschland gesprochenen deutschen Verkehrssprache Uneinigkeit bezüglich der angenommenen Grammatikalitätsurteile herrscht.%
\footnote{Die Reduktion auf den in Deutschland verwendeten Standard ist aus Sicht des Autors bedauerlich, zumal (neben dialektaler Variation) in Österreich und der Schweiz auch etablierte abweichende Standards bestehen.
Der Platz reicht aber schlicht nicht aus, um andere Standards und dialektale Variation zu berücksichtigen.}

Grammatikalität betrifft nun viele verschiedene Faktoren (\zB die lautliche Gestalt, die Form der Wörter, den Satzbau), die man meistens verschiedenen Ebenen der Grammatik zuordnet.
Im folgenden Abschnitt werden kurz die verschiedenen Teilbereiche (Ebenen) der Grammatik vorgestellt, um die es im weiteren Verlauf des Buches geht.

\subsection{Ebenen der Grammatik}

\label{sec:grebenen}

\index{Ebene}

Es wird sich in den folgenden Kapiteln zeigen, dass die Teile der Grammatik, die \zB die Kombination von Sprachlauten regeln, ganz eigenen Gesetzmäßigkeiten folgen.
Genauso verhält es sich mit den Komponenten der Grammatik, die die Form von Wörtern und den Aufbau von Sätzen regeln.
Man spricht dabei von den Ebenen der Grammatik, die trotz gewisser Interaktionen eine große Unabhängigkeit voneinander haben.
Die Ebenen, mit denen wir uns in diesem Buch beschäftigen, sind diejenigen, die die rein formalen Eigenschaften von Sprache beschreiben, also die Eigenschaften, von denen wir gesagt haben, dass sie zumindest zu einem größeren Teil auch ohne Kenntnis der Bedeutung und der Gebrauchsbedingungen erkannt werden könnten.

Die Phonetik (Kapitel~\ref{sec:phonetik}) beschreibt die rein lautliche Ebene der Sprache.
Die typische Fragestellung der Phonetik ist:
Welche Laute kommen überhaupt in einer Sprache vor, und wie werden sie mit den Sprechorganen gebildet?
Die Phonologie (Kapitel~\ref{sec:phonologie}) beschreibt die systematischen Zusammenhänge in Lautsystemen sowie die lautlichen Regularitäten, die zur Anwendung kommen.
So eine Regularität kann sich \zB darauf beziehen, in welchen Reihenfolgen die Laute einer Sprache vorkommen können.
Die Morphologie (Teil~\ref{part:wort}) analysiert sowohl den Aufbau von Wörtern als auch die Beziehungen zwischen verschiedenen Wörtern und verschiedenen Formen eines Wortes.
Die Morphologie teilt sich in zwei Gebiete, die dann getrennt behandelt werden:
Die Wortbildung (Kapitel~\ref{sec:wortbildung}) beschreibt, wie aus bestehenden Wörtern neue Wörter gebildet werden (\zB \textit{Fußball} aus \textit{Fuß} und \textit{Ball} oder \textit{fraulich} aus \textit{Frau} und \textit{lich}).
Die Flexion (Kapitel~\ref{sec:nomina} und \ref{sec:verben}) beschäftigt sich mit der Bildung der Formen eines Wortes (also \zB \textit{gehen} und \textit{ging}). 
Die Syntax (Teil~\ref{part:syntax}) beschäftigt sich mit der Frage, wie Wörter zu größeren Gruppen und schließlich zu Sätzen zusammengefügt werden.
In der Graphematik (Teil~\ref{part:schrift}) geht es darum, wie die Schrift sprachliche Einheiten der anderen Ebenen kodiert.
Warum die Graphematik hier ganz am Ende des Buches steht, wird dort einleitend diskutiert.

Auch wenn in der Linguistik andere Ebenen wie Semantik (Bedeutungslehre), Pragmatik (Lehre vom Sprachgebrauch und sprachlichen Handeln) usw.\ intensiv erforscht werden, ist die Beschreibung der formalen Kern-Ebenen ein guter Ausgangspunkt jeder Sprachbetrachtung.
Wir beschränken uns hier explizit auf diesen Kern.
Damit ist nicht gesagt, dass es sich um den wichtigsten Teil der Sprachbeschreibung bzw.\ Linguistik handelt, wohl aber um den, der zielführenderweise zuerst behandelt werden sollte.
Es wäre schwierig, zum Beispiel den Aufbau von Texten zu erforschen, bevor geklärt ist, wie die Bestandteile des Textes (die Sätze) zu analysieren sind.
Bevor diese Ebenen der Grammatik einzeln am Deutschen durchgesprochen werden, sind allerdings in diesem und dem nächsten Kapitel noch einige Grundbegriffe zu klären.

\section{Deskriptive und präskriptive Grammatik}

\label{sec:deskriptivnormativ}

\subsection{Beschreibung und Vorschrift}

In diesem und dem nächsten Abschnitt soll die deskriptive Grammatik von jeweils anderen Arten der Grammatik unterschieden werden.
Als erstes soll hier eine Definition der deskriptiven Grammatik als Ausgangsbasis gegeben werden.

\Definition{Deskriptive Grammatik}{
\label{def:deskgr}
Deskriptive Grammatik ist die wissenschaftliche und wertneutrale Beschreibung von Sprachsystemen.
Sie beschreibt Sprachen so, wie sie beobachtbar sind.
\index{Grammatik!deskriptiv}
}

Wichtig ist nun die Abgrenzung zur präskriptiven Grammatik.
Die Du\-den-Gram\-ma\-tik \citep{Duden8} wird in ihrer aktuellen Auflage mit dem Slogan \textit{Unentbehrlich für richtiges Deutsch} beworben.
Dieser Slogan könnte so verstanden werden, dass in der Du\-den-Gram\-ma\-tik Vorschriften für die korrekte Bildung von grammatischen Strukturen des Deutschen beschrieben werden.
Während im Duden zur Rechtschreibung also die Schreibung der Wörter in ihrer verbindlich korrekten Form festgelegt ist, so soll offensichtlich auch im Grammatik-Band der Duden-Redaktion der korrekte Bau von Wörtern, Sätzen und vielleicht sogar größeren Einheiten wie Texten verbindlich festgelegt sein.
Der Slogan markiert einen normativen oder präskriptiven Anspruch:
Was in dieser Grammatik steht, definiert richtiges Deutsch.
Dieser Anspruch unterscheidet die präskriptive Grammatik prinzipiell von der deskriptiven, die stets nur möglichst genau beschreiben möchte, wie bestimmte Sprachen oder alle Sprachen beschaffen sind.
Betrachtet man die Liste der Autoren der Duden-Grammatik, die durchweg renommierte Linguisten sind, die keine stark präskriptiven Ansichten vertreten, ist im übrigen davon auszugehen, dass der hier diskutierte Slogan vom Verlag und nicht von den Autoren stammt. 
Es handelt sich bei der Duden-Grammatik um eine der wichtigsten und umfangreichsten \textit{deskriptiven} Grammatiken des Deutschen.

\Definition{Präskriptive Grammatik}{
\label{def:praegram}
Die präskriptive (normative) Grammatik will verbindliche Regeln festlegen, die korrekte von inkorrekter Sprache trennen.
Sie beschreibt eine Sprache, die erwünscht ist bzw.\ gefordert wird.
\index{Grammatik!präskriptiv}
}

Definition~\ref{def:praegram} verlangt bei genauem Hinsehen sofort nach einem Zusatz.
Während es bei Gesetzen meistens klar geregelt ist, wer das Recht hat, sie zu erlassen, in welchem Bereich sie gelten, und was bei Zuwiderhandlung geschieht, ist dies bei normativen grammatischen Regeln überhaupt nicht klar.
Wir halten also fest, dass die präskriptive Grammatik versucht, dasselbe für Sprache zu tun, was Gesetze für das menschliche Verhalten gegenüber anderen Menschen und gegenüber dem Staat zu tun versuchen.
Es sollen Regeln für den Sprachgebrauch aufgestellt werden.
Bevor wir uns der Frage widmen, wer die Autorität hat, solche Regeln aufzustellen, werden in Abschnitt~\ref{sec:regulgen} zunächst einige Begriffe wie Regel und Regularität genau getrennt.

\subsection{Regel, Regularität und Generalisierung}
\label{sec:regulgen}

In einer Grammatik der gegenwärtigen deutschen Standardsprache, die einen präskriptiven Anspruch erhebt, würde man vielleicht Regeln wie in (\ref{ex:gb002}) erwarten.

\begin{exe}
  \ex\label{ex:gb002}
  \begin{xlist}
    \ex{Relativsätze und eingebettete \textit{w}-Sätze werden nicht durch Komplementierer eingeleitet.}
    \ex{\textit{fragen} ist ein schwaches Verb.}
    \ex{\label{ex:gb002c} \textit{zurückschrecken} bildet das Perfekt mit dem Hilfsverb \textit{sein}.}
    \ex{Im Aussagesatz steht vor dem finiten Verb genau ein Satzglied.}
    \ex{In Kausalsätzen mit \textit{weil} steht das finite Verb an letzter Stelle.}
  \end{xlist}
\end{exe}

Man kann sich nun fragen, ob man den Regeln in (\ref{ex:gb002}) irgendwie ansieht, dass sie präskriptiv sein sollen.
Wohl kaum, denn es könnte sich auch einfach um die Beschreibungen von Beobachtungen handeln, die Grammatiker, die das Deutsche untersuchen, gemacht haben.
Im Kontext einer präskriptiven Grammatik werden solche Sätze allerdings nicht als Beobachtungen, sondern als Regeln mit Verbindlichkeitscharakter vorgetragen.
Ob die Beschreibung eines grammatischen Phänomens deskriptiv (als Beschreibung) oder präskriptiv (als Regel) verstanden werden soll, kann man nicht an der Art ihrer Formulierung ablesen, sondern nur an dem Kontext, in dem sie vorgetragen wird.
Zunächst benötigen wir jetzt Definitionen der Begriffe \textit{Regel} und \textit{Regularität}.%
\footnote{Es gibt auch andere nicht-präskriptive Verwendungen des Regelbegriffs in der Linguistik.
Oft wird einfach \textit{Regel} für \textit{Regularität} gebraucht, weil die Verwechslungsgefahr mit einem präskriptiven Vorgehen sowieso nicht besteht.
Außerdem gibt es technische Definitionen davon, was Regeln sind, die aber in entsprechenden Texten auch hinreichend eingeführt werden.}


\Definition{Regularität}{
\label{def:regularitaet}
Eine grammatische Regularität innerhalb eines Sprachsystems liegt dann vor, wenn sich Klassen von Symbolen unter vergleichbaren Bedingungen gleich (und damit vorhersagbar) verhalten.
\index{Regularität}
}

\Definition{Regel}{
\label{def:regel}
Eine grammatische Regel ist die Beschreibung einer Regularität, die in einem normativen Kontext geäußert wird.
\index{Regel}
}

Dem Begriff der Regel gegenüber steht der Begriff der Generalisierung.

\Definition{Generalisierung}{
\index{Generalisierung}Eine grammatische Generalisierung ist eine durch Beobachtung zustandegekommene Beschreibung einer Regularität.}

Eine Regularität ist also ein Phänomen des Betrachtungsgegenstandes Sprache, das Vorhandensein von Regularitäten in sprachlichen Daten ergibt sich aus dem Systemcharakter von Sprache (Definition~\ref{def:grammatik}).
Dagegen sind Regel und Generalisierung vom Menschen bewusst gemacht und werden im Prinzip auf identische Weise formuliert, vgl.\ (\ref{ex:gb002}).
Während eine Regel dabei Ansprüche an die Eigenschaften einer Sprache stellt, stellt die Generalisierung das Vorhandensein von Eigenschaften nur fest.

Interessant ist nun, dass es sowohl zu Regeln als auch zu Generalisierungen immer Abweichungen gibt.
Im Fall der Regel handelt es sich bei jeder Abweichung um eine Zuwiderhandlung, im Fall der Generalisierung ist eine Abweichung nur eine Beobachtungstatsache, die von der Generalisierung nicht adäquat vorhergesagt wird.
Die Sätze in (\ref{ex:gb003}) wurden in verschiedenen Formen von Sprechern des Deutschen gesprochen oder geschrieben.
Sie stellen jeweils eine Abweichung zu (\ref{ex:gb002}) dar.

\begin{exe}
  \ex\label{ex:gb003}
  \begin{xlist}
    \ex{\label{ex:gb003a} Dann sieht man auf der ersten Seite wann, wo und wer dass kommt.%
      \footnote{\raggedright{\url{http://www.caliberforum.de/}, 25.01.2010}}}
    \ex{\label{ex:gb003b} Er frägt nach der Uhrzeit. \footnote{\raggedright{DeReKo, A99\slash NOV.83902}}}
    \ex{\label{ex:gb003c} Man habe zu jener Zeit nicht vor Morden zurückgeschreckt.%
      \footnote{\raggedright{DeReKo, A98\slash APR.20499}}}
    \ex{\label{ex:gb003d} Der Universität zum Jubiläum gratulierte auch Bundesminister Dorothee Wilms, die in den fünfziger Jahren in Köln studiert hatte.%
      \footnote{\raggedright{Kölner Universitätsjournal, 1988, S.~36 (zitiert nach \citealp{Mueller03})}}}
    \ex{\label{ex:gb003e} Das ist Rindenmulch, weil hier kommt noch ein Weg.%
      \footnote{\raggedright{RTL2, Big Brother VI, 20.04.2005}}}
  \end{xlist}
\end{exe}

Aus einer präskriptiven Perspektive kann man feststellen, dass diese Sätze in (\ref{ex:gb003}) alle falsch sind, wenn man (\ref{ex:gb002}) als Regeln aufgestellt hat.%
\footnote{Wir nehmen hier im Sinne der Argumentation an, dass dies der Fall ist.
Es soll damit nicht unterstellt werden, dass irgendeine auf dem Markt befindliche Grammatik solche Regeln aufstellt.
Es ist jedoch davon auszugehen, dass für jeden Satz Sprecher zu finden sind, die ihn für normativ falsch halten.}
Aus Sicht der deskriptiven Grammatik fängt mit dem Auffinden solcher Sätze (also mit der Feststellung von grammatischer Variation) die eigentliche Arbeit und der Erkenntnisprozess erst an, denn keiner der Sätze ist willkürlich falsch, so wie es \zB ein simpler Tippfehler oder ein Versprecher sind.
Viele Abweichungen von der Norm oder von bereits aufgestellten Generalisierungen zeigen nämlich vielmehr interessante strukturelle Möglichkeiten auf, die das Sprachsystem anbietet.

Beispiel (\ref{ex:gb003a}) zeigt die Konstruktion eines eingebetteten \textit{w}-Fragesatzes mit einem Komplementierer (\textit{dass}), die nicht nur systematisch in vielen südlichen regionalen Varietäten des Deutschen vorkommt, sondern die auch aus grammatiktheoretischen Überlegungen durchaus interessant ist.\index{w-Satz}
Die Häufung von Fragepronomina ist davon unabhängig, macht den Satz aber umso interessanter.
Beispiel (\ref{ex:gb003b}) zeigt \textit{fragen} als starkes Verb mit Umlaut in der 2.\ und 3.\ Person Singular Präsens Indikativ.
Aus deskriptiver Sicht hat man hier das Privileg, beobachten zu können, wie ein Verb im gegenwärtigen Sprachgebrauch zwischen starker und schwacher Flexion schwankt (s.\ dazu Abschnitt~\ref{sec:vvflex}).
Weiterhin ist die häufig vorkommende Alternation von \textit{sein} und \textit{haben} bei der Perfektbildung wie in (\ref{ex:gb003c}) ein theoretisch relevantes Phänomen, weil es bei der Beantwortung der Frage hilft, welche grundsätzliche Systematik hinter der Wahl des Hilfsverbs (abhängig vom Vollverb) steckt.
Beispiel (\ref{ex:gb003d}) illustriert ein syntaktisches Phänomen, nämlich das der doppelten Vorfeldbesetzung.\index{Vorfeld}
Hier stehen scheinbar zwei Satzglieder vor dem finiten Verb (\textit{der Universität} und \textit{zum Jubiläum}), wobei die etablierte Generalisierung eigentlich die ist, dass dort nur ein Satzglied stehen kann (vgl.\ Abschnitt~\ref{sec:vorfeldtest} und Kapitel~\ref{sec:saetze}).
Die Beschreibung dieser Sätze in bestehende Theorien zu integrieren, ist aber durchaus möglich, und man erhält dabei eine hervorragende Möglichkeit, die Flexibilität und Adäquatheit der entsprechenden Theorien zu überprüfen.%
\footnote{Das Phänomen der doppelten Vorfeldbesetzung wird in \citet{Mueller03} diskutiert, wo auch auf Lösungsansätze verwiesen wird.
Es wird in dem vorliegenden Buch wegen seiner Komplexität nicht ausführlich besprochen.}
Dass Sätze wie (\ref{ex:gb003e}) schließlich als falsch wahrgenommen werden, liegt oft daran, dass sie in der geschriebenen Sprache selten, dafür in der gesprochenen Sprache umso häufiger sind.
Nach Komplementierern (\textit{obwohl}, \textit{dass}, \textit{damit} usw.) steht im Nebensatz sonst das finite Verb (hier \textit{kommt}) an letzter Stelle, was in (\ref{ex:gb003e}) nicht der Fall ist.
Aus deskriptiver Perspektive fällt vor allem auf, dass hier \textit{weil} nach dem Muster von \textit{denn} verwendet wird.
Es wird also wieder eine strukturelle Möglichkeit genutzt, die im System ohnehin verfügbar ist, und es handelt sich nicht etwa um grammatisches Chaos.
Außerdem hat \textit{weil} mit der Nebensatz-Wortstellung wie in (\ref{ex:gb003e}) Verwendungsbesonderheiten, die es auch funktional plausibel machen, zwischen zwei verschiedenen syntaktischen Mustern in \textit{weil}-Nebensätzen zu unterscheiden.
In all diesen Fällen einfach von falschem oder richtigem Sprachgebrauch zu sprechen, wäre ganz einfach nicht angemessen.

Es sollte klar geworden sein, warum für eine wissenschaftliche Betrachtung die normative Vorgehensweise nicht in Frage kommt.
Stattdessen widmen wir uns in diesem Buch der deskriptiven Grammatik und beschreiben, welche sprachlichen Konstrukte Sprecher systematisch produzieren, einschließlich eventueller systematischer Alternativen und Schwankungen.
Durch genau diesen Anspruch handeln wir uns allerdings gleich ein ganzes Bündel von praktischen Problemen ein.
Welche systematischen Phänomene suchen wir aus?
Wie systematisch muss ein Phänomen beobachtbar sein, damit es in die Beschreibung aufgenommen wird?
Welche regionalen Varianten des Deutschen wollen wir mit unserer Beschreibung abdecken?
Beschreiben wir auch Konstruktionen, die zwar systematisch vorkommen, aber nur in der gesprochenen Sprache?
Es gäbe noch eine ganze Reihe mehr solcher Fragen.

Weil bei genauem Hinsehen Sprache ein ausuferndes Maß an Variation (Dialekte, unterschiedliche Sprechstile, Unterschiede zwischen gesprochener und geschriebener Sprache, sogar systematische Unterschiede zwischen einzelnen Sprechern) aufweist, sind diese Probleme fast unlösbar.
Konkret wäre es bei einem gesteigerten deskriptiven Anspruch ein zum Scheitern verurteiltes Projekt, auf einigen hundert Seiten eine Sprache auch nur in Ansätzen darstellen zu wollen.
Paradoxerweise orientieren wir uns daher bei unserer Beschreibung an einer quasi normierten deutschen Standardsprache, wie sie zum Beispiel in der Duden-Grammatik oder in Peter Eisenbergs \textit{Grundriss der deutschen Grammatik} \citep{Eisenberg1,Eisenberg2} beschrieben wird.
Nur so ist überhaupt ein systematischer Einstieg in die Sprachbeschreibung möglich.
Der nächste Abschnitt gibt die Gründe dafür an, warum dieser vermeintliche Rückzug nach allem, was wir kritisch über normative Grammatik gesagt haben, trotzdem zulässig ist.

\subsection{Norm als Beschreibung}

\label{sec:normalsbeschreibung}

Die bisherige Darstellung hat suggeriert, es gäbe Institutionen, die für das Deutsche Sprachnormen (also Regeln für den zulässigen Gebrauch von Grammatik) erlässt.
Es ist allerdings äußerst schwer, eine solche normierende Instanz zu finden.
Während es \zB in Frankreich die Französische Akademie (Académie française) gibt, die einen staatlich legitimierten Normierungsauftrag hat, existiert eine vergleichbare Institution in Deutschland nicht.%
\footnote{\raggedright{\url{http://www.academie-francaise.fr/}}}
Die Kultusministerkonferenz (das Gremium, das für die bundesweite Normierung von Bildungsfragen zuständig ist) beschäftigt sich nicht intensiv mit Fragen der Grammatik, wohl aber mit Fragen der Orthographie.%
\footnote{\url{http://www.rechtschreibrat.com/}}
Das staatlich finanzierte Institut für Deutsche Sprache (IDS) könnte zunächst für eine normative Organisation gehalten werden, aber schon der zweite Satz der Selbstdarstellung des IDS lässt erkennen, dass dies nicht der Fall ist:

\begin{quote}
  "`[Das IDS] ist die zentrale außeruniversitäre Einrichtung zur Erforschung und Dokumentation der deutschen Sprache in ihrem gegenwärtigen Gebrauch und in ihrer neueren Geschichte."'%
    \footnote{\raggedright{\url{http://www.ids-mannheim.de/}, 21.09.2010}}
\end{quote}

Außerdem wird oft, wie bereits erwähnt, die Duden-Grammatik als normierend angesehen, auch wenn dem Duden-Verlag dafür kein staatlicher oder gesellschaftlicher Auftrag erteilt wurde.
Der Verlag selbst erweckt diesen Eindruck mit dem Slogan von der Unentbehrlichkeit für richtiges Deutsch.
Die aktuelle Duden-Grammatik wurde von Linguisten verfasst, die selber deskriptiv arbeiten und sehr wahrscheinlich den Anspruch haben, diejenige Sprache zu beschreiben, die von den Sprechern mehrheitlich als Standard akzeptiert wird (mit allen oben angedeuteten unvermeidbaren Unschärfen).
Insofern ist die Duden-Grammatik (bzw.\ jede gute deskriptive Grammatik) auch durchaus \textit{unentbehrlich für richtiges Deutsch}.
Eine solche Grammatik beschreibt eine Sprache, die von vielen Sprechern des Deutschen als natürlich und wenig dialektal geprägt empfunden wird.
Unentbehrlich ist eine solche Beschreibung, wenn Deutsch zum Beispiel als Fremdsprache gelernt wird, oder wenn in formeller Kommunikation eine möglichst neutrale Sprache erforderlich ist.
Von einer zweifelsfreien Unterscheidung von falsch und richtig in allen Details kann aber keine Rede sein.
Insofern richten wir unsere Beschreibung an einer Quasi-Norm aus, die durch Beobachtung zustande gekommen ist.

Der zweite Grund, warum wir so verfahren wie oben erwähnt, liegt in der Unaufhaltsamkeit des sprachlichen Wandels.
Selbst wenn in einer Sprachgemeinschaft eine normierende Organisation vorhanden ist, kann diese den beständigen Sprachwandel nicht aufhalten.
Als Beispiel könnten die Französische Akademie oder die Schwedische Akademie herangezogen werden, die es selbstverständlich nicht verhindern können, dass \zB neue Wörter in die Sprache übernommen werden oder die grammatischen Möglichkeiten des Systems auf eine Weise gedehnt werden, dass dies irgendwann zu einer Veränderung des Systems führt.
Schon die Lektüre einhundert Jahre alter Texte in diesen Sprachen wird jeden Leser überzeugen, dass auch eine scheinbar streng normierte Sprache ständigen Änderungen unterworfen ist.
Dass alle diese Sprachen (genau wie das Deutsche) nicht mehr mit ihren fünfhundert oder tausend Jahre alten Vorformen identisch sind, ist dann trivial.

Als Fazit bleibt die Feststellung, dass eine brauchbare präskriptive Grammatik im Grunde eine deskriptive Grammatik ist, die einen möglichst breiten Grundkonsens beschreibt, der sich leicht im Laufe der Zeit ändern kann.
Von falschem und richtigem Gebrauch kann dabei nicht gesprochen werden, sondern nur von typischem und atypischem.
Da der typische Gebrauch in vielen Situationen von großem Vorteil ist, bleibt dabei der Wert einer Grammatikvermittlung (\zB an Schulen) anhand von mehr oder weniger kanonisierten Standardwerken unbestritten.

\subsection{Kern und Peripherie}

\label{sec:kern}

Bisher ging es um das grammatische System als System von Regularitäten  \textbf{?? WEITER}

Der Unterschied von Kern und Peripherie darf nicht mit der weiter oben besprochenen grammatischen Variation und der mit ihr einhergehenden Unsicherheit über Akzeptabilität verwechselt werden.
Wenn nicht alle Sprecher eine bestimmte grammatische Konstruktion als akzeptabel einstufen oder einzelne Sprecher sich nicht sicher sind, ob eine Konstruktion akzeptabel ist, ist diese Konstruktion nicht unbedingt peripher.
Im Fall von Variation geht es darum, ob eine Einheit oder eine Regularität überhaupt zu einem grammatischen System gehört.
Peripher sind Einheiten und Regularitäten, wenn sie zwar zum grammatischen System gehören, aber innerhalb dessen einen mehr oder weniger großen Sonderstatus haben.

Besonders im Bereich der Phonologie (Kapitel~\ref{sec:phonologie}), der Flexion (Kapitel~\ref{sec:nomina} und~\ref{sec:verben}) und der Graphematik (Teil~\ref{part:schrift}) wird immer wieder auf den \textit{Kern} und die \textit{Peripherie} des grammatischen Systems Bezug genommen.
Die besondere Betonung dieser drei Bereiche hat dabei lediglich den Grund, dass die Unterscheidung von Kern und Peripherie dort früher eine Rolle spielt als \zB\ in der Syntax.
Im Grunde zieht sich die Unterscheidung durch alle Ebenen und Teilgebiete.

\subsection{Empirie und Theorie}

\label{sec:deskriptivtheoretisch}

\index{Empirie}

In jeder Wissenschaft stellt sich die Frage:
\textit{Woher wissen wir das alles?}
Naturwissenschaften können diese Frage in der Regel mit dem Verweis auf eine Jahrhunderte alte Tradition in Theoriebildung \textit{und} experimenteller Überprüfung der Theorien beantworten.%
\footnote{Für die Physik gibt Harald Lesch in der 72.\ Sendung von \textit{Alpha Centauri} (BR\slash ARD alpha, Erstausstrahlung 14. Juni 2001) eine populärwissenschaftlich aufbereitete, aber höchst souveräne Antwort: \url{http://www.br.de/fernsehen/ard-alpha/sendungen/alpha-centauri/alpha-centauri-wissen-2001_x100.html}}
Es gilt dann \zB\ in der Physik, dass Theorien wie die Allgemeine Relativitätstheorie oder die Quantenmechanik in ihrem jeweiligen Gültigkeitsbereich angemessene Beschreibungen der Wirklichkeit darstellen.
Die Feinheit der Experimente und Beobachtungen sowie die mathematische Präzision aktueller Theorien erlaubt es Physikern, mit sehr hoher Sicherheit anzunehmen, dass die Theorien tatsächlich in diesem Sinn adäquat sind.

Verglichen mit den Naturwissenschaften scheint die hier vorgestellte Art, Linguistik zu betreiben, zunächst einmal schlecht dazustehen.
Immerhin basiert die Argumentation in diesem Buch auf einem teilweise synthetischen Grundkonsens, den wir existierenden Grammatiken entnehmen, und eine Empirie im eigentlichen Sinn gibt es nicht.
Das hat natürlich in erster Linie damit zu tun, dass dieses Buch eine Einführung in grammatische Beschreibung ist und kein eigenständiges wissenschaftliches Werk.
Damit ist es legitim, auf einen empirischen Grundkonsens zurückzugreifen (den ich hoffentlich auch getroffen habe).
Es muss aber trotzdem erwähnt werden, wie eine deskriptive Grammatik zum gegenwärtigen Stand der Forschung empirisch betrieben werden kann und wie echte linguistische Theorien beschaffen sind.
Eine solide deskriptive Grammatik muss schließlich immer auf empirisch gewonnenen Daten basieren, und linguistische Theorien müssen anhand solcher Daten überprüfbar sein. 

Es bieten sich zunächst einmal verschiedene Möglichkeiten an, Daten zu gewinnen.
Die drei wichtigsten in der Linguistik sind Experiment, Befragung und Korpusstudie.
Bei einem \textit{Experiment} werden Sprecher unter kontrollierten Bedingungen mit sprachlichen Reizen konfrontiert oder zur Sprachproduktion animiert, ohne dass sie normalerweise explizit wissen, welcher Aspekt ihrer Sprache untersucht werden soll.
Die Reaktionen der Teilnehmer des Experiments werden ausgewertet und schließlich linguistisch interpretiert.
Bei der \textit{Befragung} werden mehr oder weniger direkt Urteile über sprachliche Phänomene von Sprechern erbeten, \zB\ ob ein Ausdruck akzeptabel ist oder ob zwei Ausdrücke im gegebenen Kontext gleichermaßen verwendbar sind.
Die Methode, bei der die größten Datenmengen berücksichtigt werden können, ist die \textit{Korpusstudie}.
Ein \textit{Korpus} (fachsprachlich immer ein Neutrum, im Plural \textit{Korpora}) ist ganz allgemein gesprochen eine Sammlung von Texten aus einer oder mehreren Sprachen, ggf.\ auch aus verschiedenen Epochen und Regionen.
Man könnte \zB Korpora mit folgenden Inhalten erstellen:

\index{Korpus}

\begin{itemize}\Lf
  \item möglichst alle Texte aus Berliner Lokalzeitungen von 1890--1910,
  \item Interviews von Bundesliga-Fußballerinnen aus der Spielzeit 2010\slash 2011,
  \item eine Stichprobe von Texten deutscher Webseiten,%
		\footnote{Ein sehr großes Korpus aus deutschen Internettexten (21 Mrd.\ Wörter und Satzzeichen), die naturgemäß viel mehr nicht-standardsprachliche Variation enthalten, ist DECOW14.
		Es kann online eingesehen werden \citep{SchaeferBildhauer2012a}: \url{http://corporafromtheweb.org/}}
  \item eine nach genau definierten Kriterien zusammengestellte Auswahl deutscher Texte aus den Gattungen Belletristik, Gebrauchstext, wissenschaftlicher Text und Zeitungstext aus dem zwanzigsten Jahrhundert.%
		\footnote{Ein solches Korpus wird von den Machern des Digitalen Wörterbuchs der deutschen Sprache (DWDS) erstellt: \url{http://www.dwds.de/}.}
\end{itemize}

In solchen Korpora kann man gezielt nach Material zu bestimmten grammatischen Phänomenen suchen und sowohl die Variation innerhalb des Phänomens beschreiben, aber natürlich auch die statistisch dominanten Muster herausarbeiten.
Letztere eignen sich dann zur Darstellung in einer deskriptiven (wenn man möchte auch normativ interpretierbaren) Grammatik.
Zusätzlich erlauben es Korpora oft, den Sprachgebrauch mit bestimmten Texttypen in Beziehung zu setzen, \zB\ Zeitungsartikel, wissenschaftliche Texte, gesprochene Sprache.
Es ist den meisten Sprechern wahrscheinlich bewusst, dass in Zeitungen die Grammatik und der Wortschatz anders gebraucht als in Internet-Foren usw.

Nur zur Illustration werden in diesem Buch werden gelegentlich Beispiele aus dem Deutschen Referenz-Korpus (DeReKo) des Instituts für Deutsche Sprache (IDS) in Mannheim zitiert.
Dieses Korpus enthält vor allem Zeitungstexte jüngeren Datums und kann online benutzt werden.%
\footnote{\url{http://www.ids-mannheim.de/cosmas2/}}
Gelegentlich wird das DeReKo fälschlicherweise als COSMAS bezeichnet.
Bei COSMAS (bzw.\ COSMAS2) handelt es sich aber nur um das Recherchesystem, nicht um das Korpus selber.

\index{Theorie}

Jenseits der reinen deskriptiven Grammatik ist die grundsätzlichere Frage ist, was für theoretische Fragen durch die empirisch gewonnenen Daten eigentlich gestützt oder widerlegt werden sollen.
Wissenschaften versuchen normalerweise, nicht nur Phänomene zu beschreiben, sondern diese Phänomene auch auf zugrundeliegende Gesetzmäßigkeiten zurückzuführen.
Wichtig ist dabei der Begriff der \textit{Kausalität}, denn es sollen \textit{notwendige und relevante Ursachen} für beobachtete Phänomene gefunden werden.%
\footnote{Notwendige Ursachen sind solche, ohne die ein Phänomen überhaupt nicht erst auftritt.
Wenn es keine Wörter gibt, gibt es \zB\ auch keine Sätze.
Eine relevante Ursache ist eine, die im engeren Sinn ein konkretes Phänomen steuert und nicht zu weit durch lange Ketten von Ursachen und Wirkungen von diesem Phänomen entfernt ist.
Ohne Urknall gibt es \zB\ keine Sprache, aber die Ursache \textit{Urknall} bringt uns für die theoretische Beschreibung von Sprache keinerlei konkreten Erkenntnisgewinn.
Interessanter sind hier die Beschaffenheit des menschlichen Sprechapparats und des Gehirns.}
In der Physik begnügt man sich \zB\ nicht damit, die Bewegungen von Himmelskörpern mit einer angemessenen Mathematik zu beschreiben, sondern man sucht nach den relevanten Einflusskräften (\zB\ Gravitation und Bahndrehimpuls), die als elementare Faktoren die beobachteten Bewegungsmuster verursachen.

Wie kann man sich das in der Linguistik vorstellen?
Die klassische \textit{Grammatiktheorie} bzw.\ \textit{theoretische Linguistik} unterscheidet sich von der deskriptiven Grammatik zunächst nur durch einen stärkeren Formalisierungsgrad.
Sie benutzt Formalismen, die aus der Informatik, Logik und Mathematik stammen, um einen hohen Grad an Exaktheit zu erzielen.
Grammatiktheorie setzt die deskriptive Grammatik also als ersten Generalisierungsschritt voraus und versucht, präzisere formalisierte Modelle darauf aufzubauen.
Die Grammatiktheorie steht also nicht im Gegensatz zur deskriptiven Grammatik, sondern geht über sie hinaus.
In dieser Einführung gehen wir zwar auch möglichst exakt vor und versuchen, explizite Definitionen für alle wichtigen Begriffe zu geben sowie gewisse Formalismen zu benutzen, aber wir bleiben dabei streng genommen im informellen Bereich.
Die meiste definitorische Arbeit wird in diesem Buch aber durch natürliche Sprache geleistet, wohingegen in der Grammatiktheorie die genannten formale Systeme zum Einsatz kommen müssen.

Jedoch sind solche formaleren Modelle im Kern noch keine kausalen, sondern nur präzisere deskriptive Modelle.
Meistens gehen sie auch davon aus, dass man scharf zwischen grammatischen und ungrammatischen Konstruktionen trennen kann, genauso wie es in diesem Buch aus deskriptiver Sicht getan wird.
In der kognitiv ausgerichteten Linguistik wird allerdings zunehmend empirische Evidenz dafür gesammelt, dass Akzeptabilität graduell ist und auch scharfe Trennungen von Grammatik und Bedeutung nicht angemessen ist.
Zum Abschluss dieses Kapitels möchte ich daher meine Überzeugung äußern, dass echte kausale Modelle in der Linguistik die kognitiven Mechanismen beschreiben müssen, nach denen sprachliches Wissen gespeichert und angewendet wird.
Dazu gehört die Beschreibung aller externen Bedingungen unter denen Sprache angewendet und gelernt wird.

Warum und wie sich \textbf{?? die Sprachen der Naturvölker} oder nationale Standardsprachen herausbilden, wie sie gesprochen und verschriftet werden, muss von einem angemessenen kausalen Modell erklärt werden.
Genauso müssen die Ursachen für Makro- und Mikrovariation innerhalb dieser Sprachen erklärt werden, wofür die Art der Repräsentation von Sprache im Gehirn notwendigerweise berücksichtigt werden muss.


\Zusammenfassung

\begin{enumerate}
  \item Wir betrachten Sprache (im Sinne einer vereinfachenden Arbeitshypothese) rein formal als Symbolystem.
  \item Ein Sprachsystem besteht aus Symbolen und den Regularitäten ihrer Anordnung und Manipulation (der Grammatik).
  \item Die Grammatikalität einer Symbolfolge ist die Konformität zu einer bestimmten Grammatik.
  \item Symbolfolgen werden von Erstsprechern typischerweise spontan als akzeptable oder nicht akzeptable Symbolfolgen ihrer Erstsprache erkannt.
  \item Wenn Sprecher explizit über die Akzeptabilität von Symbolfolgen (\zB Sätzen) nachdenken, entstehen allerdings oft Zweifelsfälle.
  \item Das Sprachsystem variiert zwischen Regionen (dialektal), Zeiträumen (diachron) und auch zwischen einzelnen Sprechern.
  \item Eine Verkehrssprache wie Deutsch kann nur als vergleichsweise abstrakter und sich ständig wandelnder Grundkonsens zwischen vielen individuellen Systemen beschrieben werden.
  \item Sprachnormierung kann nur als eine Suche nach solch einem Konsens betrieben werden.
  \item Eine zielführende Sprachnormierung ist immer eine Art von Sprachbeschreibung.
  \item Es gibt im deutschsprachigen Raum keine verbindliche normierende Instanz für Fragen der Grammatik.
\end{enumerate}
