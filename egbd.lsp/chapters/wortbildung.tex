\chapter{Wortbildung}

\label{sec:wortbildung}
 
\section{Komposition}

\label{sec:komp}

\subsection{Eingrenzung der Komposition}

Dieses Kapitel ist sehr einfach strukturiert.
Zuerst wird die Komposition besprochen (Abschnitt~\ref{sec:komp}), dann die Konversion (Abschnitt~\ref{sec:konv}) und abschließend die Derivation (Abschnitt~\ref{sec:deriv}).

Eine im Deutschen besonders häufige Art der Wortbildung ist die Komposition, eine Zusammenfügung existierender Wörter.
Die Definition bezieht sich auf den Kopf-Begriff, der gleich im Anschluss an die Definition (bis einschließlich Abschnitt~\ref{sec:wbkopf}) erklärt wird.

\Definition{Komposition}{
\label{def:komp}
Komposition ist ein Wortbildungsmuster, bei dem lexikalische Wörter gebildet werden, deren Stamm aus zwei Stämmen anderer lexikalischer Wörter zusammengesetzt ist, die die Glieder des Kompositums genannt werden.
Das gebildete Wort erhält seine grammatischen und semantischen Merkmale auf produktive oder zumindest transparente Weise von den beiden Gliedern.
\index{Komposition}
}

Erste Beispiele für Komposita sind \textit{Haus.meister} oder \textit{Rot.barsch}.
Wir markieren die Stelle der morphologischen Zusammenfügung bei der Komposition mit einem Punkt und nicht mit einem Bindestrich wie bei der Flexion.

\enlargethispage{1\baselineskip}
Die Definition besagt, dass Komposition Wortbildung ist.
Es müssen also statische Merkmale geändert, Merkmale gelöscht oder hinzugefügt werden (vgl.\ Definitionen~\ref{def:inhmerkmal} auf S.~\pageref{def:inhmerkmal} und \ref{def:wortbild} auf S.~\pageref{def:wortbild}).
Wie immer in diesem Buch sagt die Definition~\ref{def:komp} nichts Konkretes über die Bedeutung, aber bei Wortbildungsprozessen ist die Betrachtung der Bedeutungsseite trotzdem außerordentlich instruktiv.
Die Bedeutung von \textit{Haus.meister} ist im Grunde nicht produktiv aus \textit{Haus} und \textit{Meister} zusammenzusetzen, da Sprecher des Gegenwartsdeutschen \textit{Meister} für sich genommen nicht unbedingt in der hier gebrauchten Bedeutung benutzen.
Dennoch spielen die beiden Glieder eine vergleichsweise transparente Rolle bei der Bestimmung der Bedeutungen des Kompositums, und wir können annehmen, dass die Werte von \textsc{Bedeutung} beider Glieder im Kompositum erhalten bleiben.

Sehr eindeutig zu sehen ist aber der Verlust sämtlicher grammatischer Eigenschaften von \textit{Haus} im Kompositum \textit{Haus.meister}.
Zum Beispiel ist (\textit{das}) \textit{Haus} [\textsc{Genus}: \textit{neut}].
Das Wort (\textit{der}) \textit{Haus.meister} ist allerdings nur und in allen Fällen [\textsc{Genus}: \textit{mask}].
Von dem ursprünglichen Genus des ersten Gliedes ist im Kompositum nichts mehr zu erkennen.

\subsection{Produktivität und Transparenz}

\label{sec:produktivitaettransparenz}

Die Begriffe Produktivität und Transparenz werden hier jetzt nur sehr kurz angesprochen.
In Abschnitt~\ref{sec:komp} wurde eine Regularität definiert, und zwar diejenige, nach der im Deutschen zwei Stämme zusammen einen neuen Stamm bilden können (Komposition).
Wenn dies eine Regularität ist, dann bedeutet es automatisch, dass ein Sprachbenutzer sich jederzeit dieser Regularität bedienen kann, um Komposita zu bilden.
Es ist tatsächlich der Fall, dass im Deutschen täglich neue Komposita gebildet werden, wie vielleicht \textit{Sprechstundenverlegungsbenachrichtigung} oder ähnliche Komposita.
Komposition ist also produktiv im Deutschen.

\Definition{Produktivität}{
\label{def:produktivitaet}
Eine Regularität ist produktiv, wenn sie jederzeit und nahezu uneingeschränkt angewendet werden kann, um grammatische Strukturen aufzubauen.
Resultierende Strukturen sind produktiv gebildet.
\index{Produktivität}
}

Mit \textit{Hausmeister}, \textit{Kindergarten} usw. verhält es sich aber etwas anders.
Diese Wörter sehen aus, als seien sie produktiv nach der Regularität der Komposition gebildet.
Allerdings ist ihre Bedeutung spezieller, als es die produktive Regularität vermuten lassen würde.
Genau das ist ein Fall von Transparenz.

\Definition{Transparenz}{
\label{def:transparenz}
Eine transparente Struktur ist eine, die erkennbar einem produktiven Muster entspricht, die aber in ihrer Bedeutung oder Funktion spezialisiert wurde und damit im Lexikon abgelegt werden muss (Lexikalisierung).
\index{Transparenz}
}

Diejenigen Komposita, bei denen also die Bedeutung der Glieder nicht ausreichend ist, um die Bedeutung des Kompositums zu erschließen, sind zwar transparent, aber nicht produktiv gebildet.
Die Übergänge sind fließend, und man muss einen großen empirischen Aufwand betreiben, um bei der Beurteilung von Produktivität zu vernünftigen Ergebnissen zu kommen.
Völlig eindeutig ist bei Komposita auch ohne größeren empirischen Aufwand eine produktive Bildung nur dann auszuschließen, wenn eins der Glieder nicht (mehr) alleine vorkommen kann, wie \textit{*Him} und \textit{*Brom} in \textit{Himbeere} und \textit{Brombeere}.

\subsection{Köpfe}

\label{sec:wbkopf}

Die obige Analyse von \textit{Haus.meister} führt uns zu der nächsten wichtigen Definition.

\Definition{Kopf (Kompositum)}{
\label{def:kompkopf}
Der Kopf eines Kompositums ist das Glied, das die Werte der statischen grammatischen Merkmale und die gesamte grammatische Merkmalsaustattung des Kompositums bestimmt.
\index{Kopf!Komposition}
}

In dem im letzten Abschnitt gegebenen Beispiel \textit{Rot.barsch} ist \textit{rot} [\textsc{Klasse}: \textit{\textbf{adj}}] und \textit{Barsch} ist [\textsc{Klasse}: \textit{\textbf{subst}}] sowie [\textsc{Genus}: \textit{\textbf{mask}}].
Das Ergebnis der Wortbildung (also (\textit{der}) \textit{Rot.barsch}) ist [\textsc{Klasse}: \textit{\textbf{subst}}, \textsc{Genus}: \textit{\textbf{mask}}] genau wie \textit{Barsch}.
Damit muss \textit{Barsch} der Kopf des Kompositums sein.
Die Erkennung des Kopfes des Kompositums wird im Deutschen dadurch erleichtert, dass immer das rechte Glied der Kopf ist.

Die Komposita in (\ref{ex:wb2471}) und unzählige andere haben jeweils als rechtes Glied einen Kopf.

\begin{exe}
  \ex\label{ex:wb2471}
  \begin{xlist}
    \ex{Rot.barsch}
    \ex{Haus.meister}
    \ex{Studenten.werk}
    \ex{Lehr.veranstaltung}
    \ex{Grün.kohl}
    \ex{Fertig.gericht}
    \ex{feuer.rot}
  \end{xlist}
\end{exe}

An \textit{feuer.rot} ist gut zu sehen, dass Komposita nicht immer [\textsc{Klasse}: \textit{\textbf{subst}}] sein müssen, sondern auch \zB Adjektive in Frage kommen.
In diesem Kapitel werden überwiegend Substantiv-Komposita besprochen, weil die Prinzipien leicht auf andere Fälle erweiterbar sind.

\subsection{Kompositionstypen}

\label{sec:detrekkomp}

\index{Kompositum!Rektions--}
\index{Kompositum!Determinativ--}

Die in diesem Abschnitt besprochenen Komposita haben ein klar definierbares semantisches Zentrum.
Da wir hier hauptsächlich über Grammatik und nicht über Semantik sprechen, fehlt uns ein Vokabular, das exakt genug für eine Definition dieses Begriffs wäre.
Überlegen wir uns aber intuitiv, wie sich der Kopf in folgenden Gruppen von Komposita bezüglich seiner Bedeutung verhält (Beispiele aus \citealp[217ff.]{Eisenberg1}):

\begin{exe}
  \ex{\label{ex:detkomp} Schul.heft, Staats.finanzen, Gebraucht.möbel}
  \ex{\label{ex:rekkomp} Kandidaten.nennung, Managerinnen.schulung, Geld.wäsche}
\end{exe}

In den Gruppen in (\ref{ex:detkomp}) und (\ref{ex:rekkomp}) bildet der Kopf (das rechte Glied) das semantische Zentrum auf eine charakteristische Weise.
Wir können nämlich für jedes dieser Komposita einen Satz wie in (\ref{ex:kompimpl}) formulieren, der in jedem Fall wahr ist.

\begin{exe}
  \ex{\label{ex:kompimpl} Die folgenden Sätze sind immer wahr:}
  \begin{xlist}
    \ex{Ein Schulheft ist ein Heft. }
    \ex{Staatsfinanzen sind Finanzen.}
    \ex{Eine Managerinnenschulung ist eine Schulung.}
  \end{xlist}
\end{exe}

Die von dem Kompositum bezeichneten Gegenstände können also auch immer von dem Kopf-Wort bezeichnet werden, das Kompositum bezeichnet also eine Untermenge der Menge, die vom Kopf bezeichnet wird.
Mit dem Nicht-Kopf (dem linken Glied) verhält es sich niemals so, wie man an (\ref{ex:kompimplfail}) leicht sieht.
Die Sätze zeigen, was mit semantischen Zentrum (manchmal auch Kern) gemeint ist, denn der Kopf dominiert das Kompositum nicht nur grammatisch, sondern auch in der Bedeutung.

\begin{exe}
  \ex{\label{ex:kompimplfail} Die folgenden Sätze sind immer falsch (durch \# angezeigt):}
  \begin{xlist}
    \ex[\#]{Ein Schulheft ist eine Schule.}
    \ex[\#]{Staatsfinanzen sind Staaten. }
    \ex[\#]{Eine Managerinnenschulung ist eine Managerin.}
  \end{xlist}
\end{exe}

Zwischen den Gruppen (\ref{ex:detkomp}) und (\ref{ex:rekkomp}) lässt sich ebenfalls ein Unterschied feststellen.
Die Testsätze in (\ref{ex:kompimpl}) funktionieren für beide, aber es gibt eine zusätzliche Testsatzkonstruktion, die nur für die Gruppe in (\ref{ex:rekkomp}) funktioniert, nämlich die Tests in (\ref{ex:rekkompsucc}).

\begin{exe}
  \ex\label{ex:rekkompsucc}
  \begin{xlist}
    \ex{Bei einer Kandidatennennung wird ein Kandidat genannt.}
    \ex{Bei einer Managerinnenschulung wird eine Managerin geschult.}
    \ex{Bei einer Geldwäsche wird Geld gewaschen.}
  \end{xlist}
\end{exe}

Für die Gruppe aus (\ref{ex:detkomp}) lassen sich die entsprechenden Sätze meist nicht vernünftig bilden, vgl.\ (\ref{ex:rekkompfail}).
Selbst wenn man ihre Bildung forciert, sind die Sätze prinzipiell falsch.
  
\begin{exe}
  \ex\label{ex:rekkompfail}
  \begin{xlist}
    \ex[\#]{Bei einem Schulfheft wird eine Schule geheftet.}
    \ex[\#]{Bei Staatsfinanzen wird ein Staat finanziert.}
    \ex[\#]{Bei Gebrauchtmöbeln wird ein Gebrauch möbliert.}
  \end{xlist}
\end{exe}

Die Komposita, für die Testsätze wie in (\ref{ex:rekkompsucc}) funktionieren, nennt man Rektionskomposita, weil ihrem Kopf-Substantiv ein Verb wie hier \textit{nennen}, \textit{schulen} oder \textit{waschen} zugrundeliegt (zur Ableitung vom Verb zum Substantiv vgl.\ Abschnitt~\ref{sec:deriv}), und in einem Satz mit diesem Verb das linke Glied (der Nicht-Kopf) das direkte Objekt (im Akkusativ) wäre -- also genau solche Sätze wie in (\ref{ex:rekkompsucc}).
Wie in Abschnitt~\ref{sec:rektion} (Definition~\ref{def:rektion}, S.~\pageref{def:rektion}) definiert, regieren die entsprechenden Verben den Akkusativ.
Daher der Name Rektionskompositum.

Die Gruppe aus (\ref{ex:detkomp}), also \textit{Schul.heft}, \textit{Staats.finanzen}, \textit{Gebraucht.möbel} usw. werden Determinativkomposita genannt, weil der Nicht-Kopf den Kopf semantisch näher bestimmt, aber keine Rektionsbeziehung gegeben ist.
Zusammenfassend kann also Satz~\ref{satz:detrekkomp} aufgestellt werden.

\Satz{Determinativ- und Rektionskomposita}{\label{satz:detrekkomp}
Wenn der Test aus (\ref{ex:kompimpl}) funktioniert und die Tests aus (\ref{ex:kompimplfail}) und (\ref{ex:rekkompsucc}) misslingen, liegt ein Determinativkompositum vor.
Wenn die Tests aus (\ref{ex:kompimpl}) und (\ref{ex:rekkompsucc}) funktionieren und der Test aus (\ref{ex:kompimplfail}) misslingt, liegt ein Rektionskompositum vor.
}

Aus grammatischer Sicht kann festgestellt werden, dass das Determinativkompositum der Prototyp des Kompositums ist, also auch ohne den semantischen Begriff des Zentrums (oder Kerns) eine große Rolle in der Grammatik spielt.
Das Rektionskompositum ist ebenfalls ein relevantes grammatisches Phänomen, da seine durchaus produktive Bildung mit einem bestimmten Valenzmuster (Verben mit Akkusativ) zusammenfällt.

\subsection{Rekursion}

\label{sec:rekursion}

\index{Rekursion}

Definition~\ref{def:komp} war so formuliert, dass jeweils zwei Wörter (bzw. ihre Stämme) zu einem Kompositum zusammengefügt werden.
In diesem Zusammenhang muss man sich nun fragen, wie es sich mit Wörtern wie \textit{Lang.strecken.lauf} verhält.
An diesem Kompositum sind offensichtlich drei Glieder beteiligt, und die Definition scheint diesen Fall zunächst nicht abzudecken.

Wenn man aber überlegt, ob die Glieder dieses Kompositums in einem jeweils gleichen Verhältnis zueinander stehen, dann sieht man schnell, dass dies nicht so ist.
Ein Langstreckenlauf ist semantisch betrachtet wahrscheinlich in den meisten Fällen der Lauf einer Langstrecke, denn das Wort \textit{Lang.strecke} ist nicht nur bildbar, sondern wird auch von Sprechern häufig verwendet.
Seltener wird wahrscheinlich der lange Lauf einer Strecke bezeichnet, denn das Wort \textit{Strecken.lauf} ist bildbar, wird aber kaum verwendet.

Trotzdem sehen wir sofort, dass beide Interpretationsmöglichkeiten existieren, und sie rühren daher, dass man die Glieder des Kompositums in verschiedene Zweiergruppen zusammenfassen kann.
Man kann dies mit Klammern sehr gut verdeutlichen, s.\ (\ref{ex:kompbrack}), oder das morphologische Strukturformat aus Abschnitt~\ref{sec:morphstrukform} benutzen, wie in den Abbildungen~\ref{fig:langstreckenlauf1} und \ref{fig:langstreckenlauf2}.
Zur Verdeutlichung werden hier die Wortklassen im Baum annotiert.

\begin{exe}
  \ex\label{ex:kompbrack}\begin{xlist}
  \ex{(Lang.strecken).lauf}
  \ex{Lang.(strecken.lauf)}
  \end{xlist}
\end{exe}

\begin{figure}[!h]
  \centering
  \Tree[1.5]{
    && \K{Subst-Stamm}\B{dl}\B{dr} \\ 
    & \K{Subst-Stamm}\B{dl}\B{dr} && \K{Subst-Stamm}\B{d} \\
    \K{Adj-Stamm}\B{d} && \K{Subst-Stamm}\B{d} & \K{\textit{lauf}} \\
    \K{\textit{Lang}} && \K{\textit{strecken}} \\
  }
  \caption{Eine Analyse von \textit{Langstreckenlauf}}
  \label{fig:langstreckenlauf1}
\end{figure}

\begin{figure}[!h]
  \centering
  \Tree[1.5]{
    & \K{Subst-Stamm}\B{dl}\B{dr} \\
    \K{Adj-Stamm}\B{d} && \K{Subst-Stamm}\B{dl}\B{dr} \\
    \K{\textit{Lang}} & \K{Subst-Stamm}\B{d} && \K{Subst-Stamm}\B{d} \\
    & \K{\textit{strecken}} && \K{\textit{lauf}} \\
  }
  \caption{Eine alternative Analyse von \textit{Langstreckenlauf}}
  \label{fig:langstreckenlauf2}
\end{figure}

Je nachdem, welche Reihenfolge von Kompositionsprozessen man annimmt, ergeben sich andere Bedeutungen.
Es gibt in der Regel aber keine grammatischen Kriterien für oder gegen eine bestimmte Analyse.
Die Grammatik (in diesem Fall die Regularitäten der Komposition) sagt uns lediglich, dass alle denkbaren Strukturanalysen aus geschachtelten Zweiergruppen von Gliedern möglich sind, nicht aber, welche plausibel oder am häufigsten sind.
Die Entscheidung wird immer aufgrund von mehr oder weniger subjektiven semantischen Erwägungen im Einzelfall gefällt.

\BVertiefung{Wahrscheinliche Analysen von Komposita}{Man kann durch Analysen der Häufigkeit der beteiligten Wörter bestimmte Analysen plausibilisieren.
Im DeReKo findet man zum Beispiel für \textit{Langstrecke} 3.804 Belege, für \textit{Streckenlauf} hingegen nur 18 (bei Anfragen mit Wortformenoperator am 26.12.2009 im Archiv W-Öffentlich.)
Der einfache Vergleich dieser absoluten Häufigkeiten zeigt, dass die Wahrscheinlichkeit für die Analyse (\textit{Lang.strecken})\textit{.lauf} deutlich höher ist als die für \textit{Lang.}(\textit{strecken.lauf}), ganz einfach weil das Wort \textit{Lang.strecke} für sich genommen stärker im Wortschatz des Deutschen vertreten ist.
Im Einzelfall muss natürlich trotzdem damit gerechnet werden, dass die unwahrscheinlichere Analyse je nach Kontext doch die zutreffende ist.
}

Unabhängig von Problemen bei der konkreten Analyse im Einzelfall ist aus grammatischer Sicht aber auf jeden Fall interessant, dass die Komposition ein Prozess ist, bei dem das Ergebnis des Prozesses wieder als Ausgangsbasis des gleichen Prozesses verwendet werden kann.
Wurden also einmal \textit{lang} und \textit{Strecke} zu \textit{Lang.strecke} komponiert, kann das dabei entstehende Kompositum wie jedes andere Substantiv auch erneut in einem Kompositionsprozess verwendet werden.
Diese Eigenschaft mancher produktiver Prozesse nennt man Rekursion.
Innerhalb der Morphologie muss beachtet werden, dass Flexion die Eigenschaft der Rekursion nicht hat.
Wenn ein Substantiv einmal nach Kasus und Numerus flektiert wurde, kann dies nicht nochmal geschehen.
Gleiches gilt für ein Verb, das nach Modus, Tempus, Person und Numerus flektiert wurde.
Es kommt also eine weitere Unterscheidung zwischen Flexion und Wortbildung hinzu, s.\ Satz~\ref{satz:reknrek}.

\newpage

\Satz{Rekursion in der Morphologie}{
\label{satz:reknrek}
Wortbildung ist ein (eingeschränkt) rekursiver morphologischer Prozess.
Flexion ist ein nicht-rekursiver morphologischer Prozess.
\index{Rekursion!in der Morphologie}
}

Bei Satz~\ref{satz:reknrek} ist zu beachten, dass allgemein von Wortbildung gesprochen wird, nicht nur von Komposition.
In etwas eingeschränktem Maß sind Konversion (Abschnitt~\ref{sec:konv}) und Derivation (Abschnitt~\ref{sec:deriv}) ebenfalls rekursiv.
Im nächsten Abschnitt geht es aber zunächst um weitere spezielle morphologische Eigenschaften von Komposita, die sogenannten Fugenelemente.

\subsection{Kompositionsfugen}

\index{Kompositionsfuge}
\index{Fugenelement}

Besonders in den hier in erster Linie betrachteten Komposita aus Substantiv und Substantiv gibt es in vielen, aber nicht allen Fällen eine morphologische Markierung, die an der so genannten Fuge (der Grenze zwischen den beiden Gliedern des Kompositums) auftritt.
Betrachtet man Wörter wie (\textit{Lang.strecke-n})\textit{.lauf}, so sieht man, dass nicht einfach die Stämme der beiden Glieder das Komposition bilden, sondern dass das Suffix \textit{-n} an das Vorderglied angefügt wird.
In diesem Fall ist das so genannte Fugenelement formal identisch mit der Pluralmarkierung des Wortes \textit{Langstrecke}.
Ist die Annahme plausibel, dass \textit{-n} hier tatsächlich als Markierungsfunktion [\textsc{Numerus}: \textit{pl}] hat?
Die Antwort ist einfach:
Bei einem Langstreckenlauf werden nicht zwangsläufig mehrere Strecken gelaufen, es kann sich nicht um die Pluralmarkierung handeln.
Das Suffix \textit{-n} ist vielmehr ein bei der Wortbildung an der Fuge auftretendes spezielles Affix ohne semantische oder grammatische Markierungsfunktion.
Diese Annahme wird weiter gestützt durch das zwischen Verb und Substantiv auftretende Fugen-Schwa wie in \textit{Bad-e.hose}, wobei \textit{bad-e} zwar eine Wortform des Verbs \textit{bad-en} ist (\zB die 1.~Person Singular Präsens), aber die Bedeutung im Kompositum garantiert nicht dieser Verbform entspricht.
Alternativ könnte es auch der Dativ Singular des Substantivs \textit{Bad} sein.
Dafür würde die gleiche Argumentation gelten.
Warum sollte im Kompositum ausgerechnet der Dativ stehen?
Außerdem gibt es Fälle, in denen wie bei \textit{\Ast Schmerz-ens} in \textit{Schmerz-ens.geld} oder \textit{\Ast Heirat-s} in \textit{Heirat-s.antrag} das Fugenelement keiner Kasus-Numerus-Form des Vordergliedes entspricht.

Diese sogenannten Fugenelemente treten in verschiedener Form, aber nicht immer und nur schwer vorhersagbar auf.
Weil sie natürlich nicht paradigmatisch sind, können wir sie eigentlich nicht als Flexion bezeichnen.
Wegen der großen formalen Nähe vieler (nicht aller) Fugenelemente zu Flexionsaffixen trennen wir sie trotzdem mit dem Strich - vom vorangehenden Stamm ab.
Die wichtigsten Fugenelemente sind in Tabelle~\ref{tab:fugen} mit Beispielen angegeben.

\begin{table}[!h]
  \centering
  \begin{tabular}{ll}
    \lsptoprule
    \textbf{Fuge} & \textbf{Beispiel} \\
    \midrule
    -n & Blume-n.vase \\
    -s & Zweifel-s.fall \\
    -ns & Glaube-ns.frage \\
    -e & Pferd-e.wagen, Bad-e.hose \\
    -er & Kind-er.garten \\
    -en & Held-en.mut \\
    -es & Sieg-es.wille \\
    -ens & Schmerz-ens.schrei \\
    \lspbottomrule
  \end{tabular}
  \caption{Wichtige Fugenelemente}
  \label{tab:fugen}
\end{table}

\begin{figure}[!h]
  \centering
  \Tree[1.5]{
    && \K{Subst-Stamm}\B{dl}\B{dr} \\
    & \K{[Subst-Stamm]}\B{dl}\B{dr} && \K{Subst-Stamm}\B{d} \\
    \K{Subst-Stamm}\B{d} && \K{Fugenelement}\B{d} & \K{\textit{schrei}} \\
    \K{\textit{Schmerz}} && \K{\textit{-ens}} \\
  }
  \caption{Kompositionsstrukturen mit Fugenelement}
  \label{fig:kompfugstruk}
\end{figure}

\index{Kompositionsfuge}
Das Gegenteil zur Fugenbildung mit Fugenelementen gibt es in einigen Fällen auch, nämlich die Suffixtilgung an der Fuge.
Manche produktiven oder historischen Wortbildungssuffixe werden an der Kompositionsfuge gelöscht.
Beispielsweise entfällt das alte Ableitungssuffix für feminine Substantive \textit{:e} (wie in \textit{Wolle}) in Komposita wie \textit{Woll.decke}.
Genauso wie das Auftreten der Fugenelemente ist diese Tilgung allerdings nicht auf einfache Weise systematisch beschreibbar.

Damit sind viele der wesentlichen grammatischen Besonderheiten der Komposition beschrieben.
Die in \ref{sec:konv} und \ref{sec:deriv} diskutierten Wortbildungstypen gehen anders als die Komposition immer von nur einem einzelnen Stamm aus.

\section{Konversion}

\label{sec:konv}

\subsection{Definition und Übersicht}

\label{sec:konvdef}

Es wurde im letzten Abschnitt gezeigt, dass der Wortschatz einer Sprache durch Kompositionsbeziehungen zwischen Wörtern besonders strukturiert sein kann.
Ähnliche Prinzipien kann man auch in einem anderen Bereich der Wortbildung beobachten.
Vergleichen wir dazu die folgenden Beispiele (\ref{ex:wobimotiv}).

\begin{exe}
  \ex\label{ex:wobimotiv}
  \begin{xlist}
    \ex{Simone geht gerne einkaufen.}
    \ex{Das Einkaufen macht Simone Spaß.}
  \end{xlist}
\end{exe}

Im ersten Satz kommt \textit{einkauf-en} als Infinitiv des Verbs (also als Verbform) vor.
Im zweiten Satz steht \textit{Einkaufen} mit dem bestimmten Artikel als Subjekt des Satzes, es handelt sich also um ein Substantiv.
Die Orthographie verlangt genau wegen dieses Wechsels in die Klasse der Substantive, dass das Wort groß geschrieben wird (mehr in Abschnitt~\ref{sec:wortklassschreib}).

Da [\textsc{Klasse}: \textit{\textbf{subst}}] und [\textsc{Klasse}: \textit{\textbf{verb}}] statische Merkmale sind, kann die Beziehung zwischen den Wortformen \textit{einkauf-en} und \textit{Einkaufen} keine Flexionsbeziehung sein, sondern es muss sich um Wortbildung handeln (vgl.\ Definition~\ref{def:wortbild}, S.~\pageref{def:wortbild}).
Es handelt sich also jeweils um die Wortform eines eigenen Wortes (Substantiv bzw.\ Verb).
Trotzdem ist die Beziehung zwischen diesen beiden Wörtern vollständig vorhersagbar, denn fast jedes Verb in seiner Infinitivform kann auf diese Weise als Substantiv mit [\textsc{Genus}: \textit{\textbf{neut}}] verwendet werden.

Wir definieren deshalb einen neuen Typ von Wortbildungsprozess, wobei wir das Wort, das dem Prozess unterzogen wird, als Ausgangswort bezeichnen und das Ergebnis als Zielwort.

\Definition{Konversion}{
\label{def:konversion}
Konversion ist ein Wortbildungsprozess, bei dem ein Stamm (Stammkonversion) oder eine Wortform (Wortformenkonversion) eines Ausgangswortes als Stamm eines Zielwortes verwendet wird, wobei Wortklassenwechsel stattfindet.
\index{Konversion}\index{Wortklasse}
}

Diese Definition erfasst zwei verschiedene Fälle, von denen erst einer an Beispielen eingeführt wurde.
Der erste ist der, bei dem ein Stamm der Ausgangspunkt des Wortbildungsprozesses ist, und der zweite ist der, bei dem der Ausgangspunkt eine Wortform ist.
Illustrieren kann man den Unterschied, wenn man die Beispiele aus (\ref{ex:wobimotiv}) um einen Satz erweitert, vgl.\ (\ref{ex:wobimotiv2}).

\begin{exe}
  \ex{\label{ex:wobimotiv2} Der Einkauf an Heiligabend hat vier Stunden gedauert.}
\end{exe}

In diesem Beispiel wird ein zweites Wort verwendet, welches offensichtlich auch in einer Wortbildungsbeziehung zu dem Verb \textit{einkauf-en} steht.
Dass \textit{Einkauf} nicht dasselbe Substantiv wie \textit{Einkaufen} sein kann, sieht man leicht daran, dass das Genus der Wörter unterschiedlich ist (Maskulinum beziehungsweise Neutrum).
Außerdem unterscheiden sich die beiden Substantive darin, ob sie einen Plural bilden können.
\textit{Einkauf} kann einen Plural bilden (\textit{Einkäuf-e}), \textit{Einkaufen} hingegen nicht, vgl.\ Tabelle~\ref{tab:einkauf-en}.

\begin{table}
  \centering
  \begin{tabular}{ll}
    \lsptoprule
    \textbf{Stammkonversion} & \textbf{Wortformenkonversion} \\
    \midrule
    der Einkauf & das Einkaufen\\
    den Einkauf & das Einkaufen\\
    dem Einkauf & dem Einkaufen\\
    des Einkauf-s & des Einkaufen-s\\
    die Einkäuf-e & --- \\
    \lspbottomrule
  \end{tabular}
  \caption{Ausschnitt der Paradigmen von \textit{Einkauf} und \textit{Einkaufen}}
  \label{tab:einkauf-en}
\end{table}

\index{Wort!Stamm}

Die beiden Wörter sind also voneinander verschieden, haben unterschiedliche Stämme (\textit{Einkauf} und \textit{Einkaufen}).
Wir gehen hier daher davon aus, dass sie durch unterschiedliche Konversionsprozesse aus dem Verb gebildet wurden.
Im Fall von \textit{Einkaufen} wurde eine Wortform zugrundegelegt, nämlich der Infinitiv.
Es handelt sich also um den ersten Fall aus der Definition, nämlich Wortformenkonversion.
Im Gegensatz dazu ist bei \textit{Einkauf} der Verbalstamm in einen Substantivstamm konvertiert worden, wobei standardmäßig ein maskulines Substantiv entsteht.
Dies entspricht dem zweiten Fall aus der Definition, also der Stammkonversion.
Die Subklassifikation als Stammkonversion und Wortformenkonversion richtet sich dabei nach dem Ausgangspunkt der Konversion.
Das Ergebnis der Konversion ist selbstverständlich immer ein Stamm, denn es verhält sich wie ein gewöhnliches Wort der Wortklasse, zu der es gehört.
Es flektiert also wie jedes andere Verb oder Nomen, oder es ist unveränderlich (falls das Zielwort \zB ein Adverb ist).

Es muss terminologisch beachtet werden, dass im Falle unregelmäßiger Bildungen, bei denen \zB im Konversionsprodukt Ablautstufen vorliegen, die es sonst nicht gibt, nicht von Konversion gesprochen werden sollte.
Ein Beispiel dafür wäre \textit{schieß-en} zu \textit{Schuss}.
Diese Fälle behandeln wir als unregelmäßige, nicht-pro\-duk\-ti\-ve Bildungen, und betrachten die Stämme in unserer synchronen Grammatik als nicht aufeinander bezogen.
In diesem Fall gibt es trotz der lautlichen Ähnlichkeit und dem eindeutigen semantischen Bezug zwischen \textit{Schuss} und \textit{schieß-en} keine grammatische Beziehung.
Im nächsten Abschnitt folgen nun Beispiele für eindeutige Konversionsprozesse im Deutschen.

\subsection{Konversion im Deutschen}

\label{sec:konvdeutsch}

Spezielle Bezeichnungen für Konversionsprozesse werden normalerweise nach der Wortklasse des Zielwortes mit \textit{-ierung} gebildet.

\begin{table}
  \centering
  \resizebox{\textwidth}{!}{
    \begin{tabular}{lll}
      \lsptoprule
      \textbf{Typ} & \textbf{Ausgangswort} & \textbf{Zielwort} \\
      \midrule
      Adjektivierung & (Der Zaun wurde) ge-strich-en. & (der) gestrichen-e (Zaun) \\
      Substantivierung & (der) gestrichen-e (Zaun) & (der/die/das) Gestrichen-e \\
      \lspbottomrule
    \end{tabular}
  }
  \caption{Beispiele für Wortformenkonversion}
  \label{tab:wfkonv}
\end{table}

Eine Konversion, bei der das Zielwort zur Klasse der Adjektive gehört, wird also als Adjektivierung bezeichnet.

\begin{table}
  \centering
  \begin{tabular}{lll}
    \lsptoprule
    \textbf{Typ} & \textbf{Ausgangswort} & \textbf{Zielwort} \\
    \midrule
    Substantivierung & (Wir sollen) lauf-en. & (der) Lauf \\
    Verbalisierung & (der) grün-e (Rasen) & (Der Rasen) grün-t.\\
    Adverbierung & (das) schnell-e (Auto) & (Das Auto fährt) schnell. \\
    \lspbottomrule
  \end{tabular}
  \caption{Beispiele für Stammkonversion}
  \label{tab:stammkonv}
\end{table}

In den Tabellen~\ref{tab:wfkonv} und \ref{tab:stammkonv} finden sich einige Beispiele (übernommen aus \citealp[280]{Eisenberg1}), geordnet nach Wortformenkonversion und Stammkonversion sowie der Wortklasse des Zielwortes in eindeutigen syntaktischen Kontexten.
Die Wortformenkonversion vom Adjektiv \textit{gestrichen-e} zu dem Substantiv \textit{Gestrichen-e} ist eigentlich ein Sonderfall, weil hier im Grunde voll flektierte adjektivische Wortformen als Substantiv verwendet werden, denn das Zielwort flektiert nicht wie ein Substantiv, sondern wie ein Adjektiv.
Es handelt sich also um eine Konversion von einer Wortform zu einer Wortform und nicht von einer Wortform zu einem Stamm, worauf hier aber nicht weiter eingegangen werden kann.

Zur Notation der Wortanalysen muss noch Folgendes angemerkt werden.
Ist vom Infinitiv des Verbs die Rede, handelt es sich um eine Wortform aus einem Verbstamm und einem Flexionssuffix, weswegen der Bindestrich zwischen den Bestandteilen Wortstamm und Suffix stehen muss: \textit{kauf-en}.
Sobald die Wortformenkonversion zum Substantiv erfolgt ist, verhält sich das Resultat morphologisch immer wie ein Substantivstamm, und der Bindestrich muss entfallen: (\textit{das}) \textit{Kaufen}.

An den Beispielen in Tabelle (\ref{tab:wfkonv}) kann man erkennen, dass auch der Prozess der Konversion prinzipiell (aber gegenüber der Komposition eingeschränkt) rekursiv durchführbar ist, denn vom Partizip \textit{ge-strich-en} (zur Bildung der Form des Partizips s.\ Abschnitt~\ref{sec:infinflex}) kann ein Adjektiv \textit{gestrichen} gebildet werden, und von diesem Adjektiv kann wiederum durch Konversion ein Substantiv (\textit{der\slash die\slash das}) \textit{Gestrichen-e} gebildet werden.
Eine Darstellung in Strukturbäumen findet sich in den Abbildungen~\ref{fig:konv1} und \ref{fig:konv2}.

\begin{figure}[!h]
  \centering
  \Tree{
    \K{Subst-Stamm}\B{d} \\
    \K{V-Stamm}\B{d} \\
    \K{\textit{lauf}} \\
  }
  \caption{Einfache Stammkonversion}
  \label{fig:konv1}
\end{figure}

\begin{figure}[!h]
  \centering
  \Tree[3]{
    && \K{Subst-Wortform}\B{d} \\
    && \K{Adj-Wortform}\B{dr}\B{dl} \\
    & \K{Adj-Stamm}\B{d} && \K{Flexionssuffix}\B{d} \\
    & \K{V-Wortform}\B{dr}\B{d}\B{dl} && \K{\textit{-e}} \\
    \K{Flexionszirkumfix}\B{d} & \K{V-Stamm}\B{d} & \K{Flexionszirkumfix}\B{d} \\
    \K{\textit{ge-}} & \K{\textit{strich}} & \K{\textit{-en}} \\
  }
  \caption{Schrittweise Wortformenkonversionen}
  \label{fig:konv2}
\end{figure}

\section{Derivation}

\label{sec:deriv}

\subsection{Definition und Überblick}

Bei der Konversion findet typischerweise ein Wortklassenwechsel statt, es gibt aber kein Affix, das eine spezifische semantische Veränderung formal markiert.
Daher sind die semantischen Folgen eines bestimmten Konversionstypus normalerweise konventionalisiert. 
Das bedeutet \zB, dass im Fall der Wortformenkonversion vom verbalen Infinitiv zum Substantiv (\textit{lauf-en} zu \textit{Laufen}) und bei der Stammkonversion (\textit{lauf} zu \textit{Lauf}) ziemlich genau festgelegt ist, wie die Bedeutung der beiden Ziel-Substantive aus der Bedeutung des Verbs erschlossen werden kann, obwohl keine formalen Mittel (Affixe) dies markieren.
In den genannten Fällen bezeichnen die Ziel-Substantive die entsprechende Handlung bzw.\ den Vorgang (bei dem jemand läuft).
Man erwartet daher als kompetenter Sprachbenutzer, dass ein durch Konversion vom Verb gebildetes Substantiv \zB nicht im Einzelfall die handelnde (hier also laufende) Person bezeichnet.

Wenn man sich aber die Bildungen in (\ref{ex:derivmotiv}) ansieht, kann man Ableitungen beobachten, die unter Verwendung bestimmter Affixe und nicht durch Konversion zustandekommen.
Das Affix markiert in diesen Fällen immer auch eine bestimmte Änderung der Bedeutung im Vergleich mit dem Ausgangswort.
Die Doppelpunkte markieren die Grenzen zwischen dem Stamm des Ausgangswortes und den Derivationsaffixen.

\begin{exe}
  \ex\label{ex:derivmotiv}
  \begin{xlist}
    \ex{Der Läuf:er erreichte das Ziel.}
    \ex{Die Zielmarke ist aus dieser Entfernung schlecht erkenn:bar.}
    \ex{Die Auszehrung beim Marathon ist schreck:lich.}
    \ex{Ullis schreck:haft-er Hund hat einen japanischen Namen.}
  \end{xlist}
\end{exe}

Man kann an diesen Beispielen sofort erkennen, dass der Beitrag des Affixes zur Bedeutung des Zielwortes meistens sehr eindeutig ist.
Mit \textit{Läuf:er} bezeichnet man den Ausführenden einer Handlung des Laufens, und man kann alle (oder zumindest eine große Gruppe von) Verbalstämmen (hier \textit{lauf}) durch Suffigierung von \textit{\~:er} zu einem Substantiv derivieren, das den Ausführenden der Handlung bezeichnet.%
\footnote{Bei genauem Hinsehen ist der Fall von \textit{\~:er} eigentlich komplizierter, wenn man an Bildungen wie (\textit{Früh.blüh})\textit{:er} oder (\textit{Ver:lier})\textit{:er} in Zusammenhang mit der Formulierung \textit{Ausführender der Handlung} denkt.
Diese Verben beschreiben eigentlich keine Handlungen, die einen intentionalen Ausführer haben.}

Bei \textit{erkenn:bar} wurde ein Verbalstamm \textit{erkenn} durch das Suffix \textit{:bar} zu einem Adjektiv deriviert, das die Eigenschaft ausdrückt, die Rolle des Erkannten bei einem Prozess des Erkennens spielen zu können.
Weiterhin ist \textit{schreck:lich} ein mit \textit{\~:lich} deriviertes Adjektiv zum Substantivstamm \textit{Schreck}, dass die Eigenschaft angibt, etwas zu sein, das gewöhnlicherweise Schrecken hervorruft.
Im Fall von \textit{schreck:haft} (mit \textit{:haft}) ergibt sich hingegen die Bezeichnung der Eigenschaft eines belebten Wesens, sehr leicht die Rolle des Erleiders eines Schreckens zu spielen.

Allgemein können wir Definition~\ref{def:deriv} aufstellen.

\Definition{Derivation}{
\label{def:deriv}
Derivationen sind Wortbildungsprozesse, bei denen ein neuer Stamm unter Affigierung eines Affixes an einen anderen Stamm gebildet wird, wobei das Resultat zu einem neuen lexikalischen Wort gehört (also im Vergleich zum ursprünglichen Stamm andere statische Merkmale hat).
\index{Derivation}
}

Die Definition der Affixe (Definition~\ref{def:affix}, S.~\pageref{def:affix}) beinhaltet die Bedingung, dass sie nicht selbständig auftreten können (sie sind gebunden).
Der Unterschied der Derivation zur Komposition ist also der, dass bei der Derivation nicht zwei unabhängig vorkommende Stämme den Stamm des Zielworts bilden, sondern ein Stamm, der auch unabhängig vorkommen kann, zusammen mit einem Affix, das nicht selbständig vorkommen kann.

Die Definition beruft sich auf die Definition der Wortbildung (Definition~\ref{def:wortbild}, S.~\pageref{def:wortbild}).
Wir müssen also bei allen Prozessen, die wir als Derivation einstufen, statische (unveränderliche) Merkmale des Ausgangswortes angeben können, die im Zielwort in ihrem Wert geändert, hinzugefügt oder gelöscht werden.

Bei den in (\ref{ex:derivmotiv}) angegebenen Beispielen ist dies sehr leicht, da sich in allen Fällen das Merkmal \textsc{Klasse} ändert.
Dies muss aber nicht so sein.
Im nächsten Abschnitt werden kurz solche Derivationsaffixe vorgestellt, bei denen scheinbar kein Wortklassenwechsel eintritt.
Danach erfolgt ein Überblick über Derivationsaffixe mit Wortklassenwechsel und abschließend eine Betrachtung von Einschränkungen bei der rekursiven Derivation per Suffigierung.

\subsection{Derivation ohne Wortklassenwechsel}

\index{Wortklasse}
\label{sec:derivohnewaw}

\subsubsection{Nominale wortklassenerhaltende Affixe}

\index{Nomen}
\label{sec:nomwklasserhaltaff}

Wir betrachten zunächst ein Beispiel für ein nominales wortklassenerhaltendes Präfix, nämlich genau das oben erwähnte \textit{un:} als Adjektiv- und Substantiv-Präfix.
Das Präfix \textit{un:} hat Negationscharakter, vgl. (\ref{ex:underiv}).

\begin{exe}
  \ex\label{ex:underiv}
  \begin{xlist}
    \ex{Un:mensch, Un:glaube, Un:tiefe}
    \ex{un:bedeutend, un:selig, un:wirsch}
  \end{xlist}
\end{exe}

Es ist allerdings nicht voll produktiv und in vielen Fällen lexikalisiert.
Vor allem bei den Substantiven ist die Produktivität eingeschränkt, und bei den Adjektiven gilt, dass es nur bei solchen Adjektiven voll produktiv ist, die selbst einem erkennbaren Muster der Adjektivbildung folgen, vgl.\ (\ref{ex:unadj}).
Trotzdem gibt es Fälle, in denen auch ohne solch ein erkennbares Muster Präfigierung mit \textit{un:} möglich ist.

\begin{exe}
  \ex\label{ex:unadj}
  \begin{xlist}
    \ex Kein erkennbares Bildungsmuster beim Ausgangswort
    \begin{xlist} 
      \ex[*]{un:rot}
      \ex[*]{un:schnell}
      \ex[]{un:wirsch}
    \end{xlist}
    \ex Erkennbares Bildungsmuster beim Ausgangswort
    \begin{xlist}
      \ex[]{un:(glaub:lich)}
      \ex[]{un:(gläub:ig)}
      \ex[]{un:(beschreib:bar)}
    \end{xlist}
  \end{xlist}
\end{exe}

Viele Bildungen mit \textit{un:} sind auch insofern auf jeden Fall lexikalisiert, als die Stämme der Ausgangswörter selber nicht mehr existieren, wie in (\ref{ex:unlex}).

\begin{exe}
  \ex\label{ex:unlex}
  \begin{xlist}
    \ex[]{un:gestüm}
    \ex[*]{gestüm}
    \ex[]{un:bedarft}
    \ex[*]{bedarft}
  \end{xlist}
\end{exe}

Es ist in den eindeutig lexikalisierten Fällen natürlich fraglich, ob der Doppelpunkt überhaupt noch stehen sollte.
Es wäre ebenso legitim, \textit{unwirsch}, \textit{ungestüm}, \textit{unbedarft} (statt \textit{un:wirsch} usw.) zu schreiben.
Wem allerdings Transparenz als Kriterium für die Analyse ausreicht, der kann hier gerne den Doppelpunkt setzen.

\subsubsection{Verbale wortklassenerhaltende Affixe}

\index{Verb}
\label{sec:wkleaffixe}

Die verbalen wortklassenerhaltenden Präfixe sind im wesentlichen die Verbpartikeln und die Verbpräfixe.
Auf einen Unterschied bei der Akzentuierung wurde im Rahmen der Phonologie schon kurz eingegangen (Satz~\ref{satz:pholvprtprf}, S.~\pageref{satz:pholvprtprf}).

Die Unterschiede zwischen Verbpartikel und Verbpräfix liegen aber im morphologischen, syntaktischen und phonologischen Bereich.
Die Verbpartikel erlaubt den Einschub des Partizip-Präfixes \textit{ge-}, ist syntaktisch trennbar und zieht den Akzent auf sich.
Das Verbpräfix blockiert den Einschub des Partizip-Präfixes, ist nicht trennbar und ist nicht betonbar.
Diese drei Eigenschaften sind in (\ref{ex:vpart}) und (\ref{ex:vpre}) zusammengefasst, wobei als Trennzeichen für die Verbpartikeln \v: verwendet wird.

\begin{exe}
  \ex{\label{ex:vpart}Verbpartikel
  \begin{xlist}
    \ex{\label{ex:vpart-a} Das Auto hat den Pfosten um\v:ge-fahr-en.}
    \ex{\label{ex:vpart-b} Das Auto fähr-t den Pfosten um\v:.}
    \ex{\label{ex:vpart-c} Ich möchte den Pfosten \Akz um\v:fahr-en.}
  \end{xlist}}
  \ex{\label{ex:vpre}Verbpräfix
  \begin{xlist}
    \ex{\label{ex:vpre-a} Das Auto hat den Pfosten um:fahr-en.}
    \ex{\label{ex:vpre-b} Das Auto um:fähr-t den Posten.}
    \ex{\label{ex:vpre-c} Ich möchte den Pfosten um:\Akz fahr-en.}
  \end{xlist}}
\end{exe}

\label{abs:praefixundflex}Offensichtlich sind beide Arten der Bildung für die Flexion transparent, denn sowohl die Unterdrückung des Partizip-Präfixes in (\ref{ex:vpre-a}) als auch der Einschub des Partizip-Präfixes zwischen Verbpartikel und Verbstamm in (\ref{ex:vpart-a}) erfordern es, dass die Flexion die Grenze zwischen Verbpartikel bzw.\ Verbpräfix und Stamm sehen kann.
Das könnte ein Hinweis darauf sein, dass die Bildung der Partizipien besser als Wortbildung statt als Flexion beschrieben werden kann.
In morphologischen Theorien wird oft angenommen, dass erst nach dem vollständigen Abschluss der Wortbildungsprozesse die Flexionsprozesse stattfinden, so dass solche Mischungen von Wortbildungsaffixen und Flexionsaffixen nicht auftreten sollten.
Auf die Idee, die Bildung von Partizip und Infinitiv als Wortbildung statt als Flexion zu betrachten, gehen wir auf S.\ \ref{abs:infinwortbild} aus unabhängigen Gründen noch einmal ein.

Die Benennung der Verbpartikeln deutet darauf hin, dass die Verbindung zum Verb bei ihnen deutlich weniger eng ist als bei den Verbpräfixen.
Immerhin bilden gemäß den Wortklassenfiltern \ref{wfilt:subjunktion} (S.~\pageref{wfilt:subjunktion}) und \ref{wfilt:advpart} (S.~\pageref{wfilt:advpart}) Partikeln normalerweise eine Wortklasse (sind also selbständige syntaktische Einheiten), während Affixe lediglich (unselbständige) morphologische Einheiten sind.

Es soll noch hinzugefügt werden, dass die Klasse der Verbpräfixe stark eingeschränkt, also geschlossen ist.
Es gibt nur sehr wenige echte (typischerweise valenzändernde) Präfixe, während die Partikeln eine sehr offene Klasse an der Grenze zur Komposition bilden.
Neben präpositionalen Elementen wie in \textit{über\v:lauf-en} oder \textit{nach\v:schick-en} gibt es \zB auch Adjektive, die in Partikelfunktion vorkommen, \zB \textit{frei\v:sprech-en}. 

\subsection{Derivation mit Wortklassenwechsel}

Wir wenden uns nun der Derivation mit Wortklassenwechsel zu, oder besser gesagt: der Derivation, bei der das Derivations-Affix eine vom Ausgangswort verschiedene Wortklasse beisteuert.
Diese Fälle sind auf Suffixe und wenige Zirkumfixe beschränkt.
Ein Beispiel mit Verben als Ausgangswort und Substantiven als Zielwort ist \textit{Ge::e}.
Zu vielen Verben bildet dieses Zirkumfix ein Substantiv, das eine nicht zielgerichtete Ausführung der Handlung bezeichnet und einen abschätzigen Charakter hat, \zB \textit{Ge:red:e} zum Verb \textit{red-en}.

Die wortklassenändernden Suffixe werden oft (ähnlich wie schon bei Konversionsprozessen, vgl.\ Abschnitt~\ref{sec:konvdeutsch}) als \textit{-isierungs}-Suffixe bezeichnet.
Beispielsweise wäre \textit{:haft} ein Adjektivierungs-Suffix oder adjektivierendes Suffix für substantivische Ausgangswörter.
Eine Übersicht über die wichtigsten Suffixe dieser Art findet sich im nächsten Abschnitt, mit dem die Darstellung der Wortbildung endet.

\subsection{Mehrfachsuffigierung}

\label{sec:doppelsuff}

Nach \citet[267]{Eisenberg1} fassen wir in Tabelle~\ref{tab:derivaffixe} zunächst einige wichtige Derivationsaffixe des Deutschen sowie die Wortklasse ihrer Ausgangswörter (Zeilen) und Zielwörter (Spalten) zusammen.

\index{Substantiv}\index{Adjektiv}\index{Verb}
\begin{table}
  \centering
  \begin{tabular}{llll}
    \lsptoprule
    \textbf{Ausgangsklasse} & \textbf{Substantiv-Affix} & \textbf{Adjektiv-Affix} & \textbf{Verb-Affix} \\
   \midrule
   \multirow{4}{*}{\textbf{Substantiv}} & \~:chen & :haft & \\
   & :in & :ig & \\
   & :ler & \~:isch & \\
   & :schaft & \~:lich & \\
   \midrule
   \multirow{3}{*}{\textbf{Adjektiv}} & :heit & \~:lich & \\
    & :keit && \\
    & :igkeit && \\
   \midrule
   \multirow{3}{*}{\textbf{Verb}} & :er & :bar & \~:el \\
   & :erei && \\
   & :ung && \\
   \lspbottomrule
  \end{tabular}
  \caption{Derivationsaffixe nach Ausgangs- und Zielklasse}
  \label{tab:derivaffixe}
\end{table}

Die Tabelle deutet (durch die relative Anzahl der genannten Affixe) an, dass als Wortklasse für Zielwörter \textit{Substantiv} sehr häufig, \textit{Adjektiv} bereits seltener und \textit{Verb} fast gar nicht vorkommt.
Tabelle~\ref{tab:derivaffixex} zeigt parallel dazu Beispiele.

\begin{table}
  \centering
  \begin{tabular}{llll}
    \lsptoprule
    \textbf{Ausgangsklasse} & \textbf{Substantiv-Affix} & \textbf{Adjektiv-Affix} & \textbf{Verb-Affix} \\
   \midrule
   \multirow{4}{*}{\textbf{Substantiv}} & Äst:chen & schreck:haft & \\
   & (Arbeit:er):in & fisch:ig & \\
   & (Volk-s:kund):ler & händ:isch & \\
   & Wissen:schaft & häus:lich & \\
   \midrule
   \multirow{3}{*}{\textbf{Adjektiv}} & Schön:heit & röt:lich & \\
    & Heiter:keit && \\
    & Neu:igkeit && \\
   \midrule
   \multirow{3}{*}{\textbf{Verb}} & Arbeit:er & bieg:bar & kreis:el-n \\
   & Arbeit:erei && \\
   & Les:ung && \\
   \lspbottomrule
  \end{tabular}
  \caption{Beispiele für Derivationsaffixe}
  \label{tab:derivaffixex}
\end{table}

Weiter oben (Satz~\ref{satz:reknrek}, S.~\pageref{satz:reknrek}) wurde festgestellt, dass Wortbildung rekursiv ist.
Im Falle der Derivation ist dies prinzipiell auch der Fall, allerdings ist die Kombinierbarkeit der Affixe eingeschränkt.
Es sind aber nur bestimmte Abfolgen möglich, und die möglichen Reihenfolgen der Suffixe sind ebenfalls vergleichsweise festgelegt.
Die Gründe hierfür sind überwiegend semantischer Natur, abgesehen davon, dass natürlich \zB ein einmal zu einem Substantiv abgeleitetes Adjektiv (\textit{Neu:heit}) wie ein substantivisches Ausgangswort fungiert und nicht weiter wie ein Adjektiv abgeleitet werden kann.
Jeweils eine (nach Eisenberg) mögliche und eine nicht mögliche Bildung finden sich beispielhaft in (\ref{ex:adjmultisuff}) bis (\ref{ex:substmultisuff}) für verschiedene Wortklassen von Ausgangswörtern.

\begin{exe}
  \ex{\label{ex:adjmultisuff}
  \begin{xlist}
    \ex[]{(Schön:heit):chen}
    \ex[*]{(Schön:heit):haft}
  \end{xlist}}
  \ex{\label{ex:vbmultisuff}
  \begin{xlist}
    \ex[]{(Verzeih:ung):chen}
    \ex[*]{(Verzeih:ung):schaft}
  \end{xlist}}
  \ex{\label{ex:substmultisuff}
  \begin{xlist}
    \ex[]{(Gärtn:er):in}
    \ex[*]{Garten:in}
  \end{xlist}}
\end{exe}

\index{Diminutiv}

Die Darstellung bei Eisenberg suggeriert, dass die Suffigierung des sog.\ Diminutivs (\textit{\~:chen}) an Wörter wie \textit{Schön:heit} (Abstrakta) möglich sei.
Dies klingt zunächst zweifelhaft, und eine Recherche nach Bildungen auf \textit{:heit:chen} oder \textit{:keit:chen} im DeReKo ergibt auch, dass im gesamten Korpus lediglich die Wortformen (\textit{Krank:heit})\textit{:chen} und (\textit{Begeben:heit})\textit{:chen} vorkommen, und dies nur jeweils einmal.%
\footnote{Anfrage \texttt{*heitchen ODER *keitchen} am 03.01.2010 im Archiv W-Öffentlich.}
Man kann daher nicht sagen, dass diese Bildungen produktiv sind.
Allerdings sind sie als strukturelle Möglichkeit auch nicht ganz ausgeschlossen.

Damit sind wir am Ende der Wortbildung angelangt.
Im nächsten Kapitel geht es um die genauen Flexionsmuster bei den flektierbaren Wörtern.
Die genaue Diskussion der Flexion ist hier der Wortbildung nachgeordnet, weil es damit möglich wird, ggf.\ zu diskutieren, ob bestimmte Bildungen tatsächlich Flexion oder doch eher Wortbildung sind (\zB die Komparation, s.\ Abschnitt~\ref{sec:kompform}).

\Zusammenfassung

\begin{enumerate}
  \item Ein morphologischer Prozess ist umso produktiver, je weniger Einschränkung es bezüglich seiner Anwendbarkeit auf die Wörter einer Wortklasse gibt.
  \item Ein Prozess ist transparent (ggf.\ aber nicht produktiv), wenn die Art seiner Bildung deutlich erkennbar ist, wie bei \textit{Hausmeister}.
  \item Komposita sind Neubildungen eines Worts aus zwei existierenden Wörtern, von denen eins als Kopf die grammatischen Merkmale der Neubildung bestimmt.
  \item In der Komposition werden immer zwei Wörter zusammengesetzt, ggf.\ aber rekursiv.
  \item Fugenelemente haben keine einfach zu bestimmende grammatische Funktion wie die Markierung von Kasus oder Numerus.
  \item Bei der Konversion werden neue Wörter ohne Formveränderung aus bestehenden Wörtern gebildet, \zB das Substantiv \textit{Blau} aus dem Adjektiv \textit{blau}.
  \item Derivation ist die Bildung neuer Wörter aus existierenden Wörtern unter Anfügung von Affixen, \zB \textit{bläu:lich} aus \textit{blau}.
  \item Verben mit Verbpartikeln und Verbpräfixen unterscheiden sich in ihrer Syntax und ihrer Flexion, \zB \textit{übersetzt} und \textit{übergesetzt}.
  \item Bei der Derivation kann sich die Wortart ändern, muss aber nicht, \zB \textit{Leser:schaft} und \textit{leser:lich}.
  \item Wortbildungssuffixe sind nur in bestimmten Abfolgen kombinierbar.
\end{enumerate}

\Uebungen

\Uebung \label{u71} Bestimmen Sie für die folgenden Komposita (a) die vollständige morphologische Struktur einschließlich der Fugenelemente (als Baum oder in der linearen Notation, ggf.\ mit Klammern), (b) den Kopf, (c) den Typus. (d) Welche sind Ihrer Meinung nach produktiv gebildet und welche lexikalisiert? (e) Stellen Sie fest, ob die Ausgangswörter morphologisch komplex sind (\zB deriviert).

\begin{enumerate}\Lf
  \item Wesenszugsanalyse
  \item Einschuböffnung
  \item Esstisch
  \item Räderwerksreparatur
  \item Einschiebeöffnung
  \item Großrechner
  \item Banknotenfälschung
  \item Bergbauwissenschaftsstudium
  \item Anschlagsvereitelung
  \item Bioladen
  \item Kindergarten
  \item Mitbewohner
  \item Absichtserklärungsverlesung
  \item Monatsplanung
  \item feuerrot
  \item Notlaufprogramm
\end{enumerate}

\Uebung \label{u72} Bestimmen Sie für die folgenden Derivations- und Konversionsprodukte (a) die morphologische Struktur (als Baum oder in der linearen Notation), (b) die Wortklassen der Ausgangs- und Zielwörter, (c) den Typus (Derivation, Stamm- oder Wortformenkonversion). (d) Liegt Umlaut vor? (e) Welche sind Ihrer Meinung nach produktiv gebildet und welche lexikalisiert? 

\begin{enumerate}\Lf
  \item verkäuflich
  \item unterwander(n)
  \item alternativlos
  \item (der) Lauf
  \item aufsteig(en)
  \item Gebell
  \item beschließ(en)
  \item begegn(en)
  \item Röhrchen
  \item (das) Schlingern
  \item Geruder
  \item Überzocker
  \item Gebrüder
  \item Mündel
  \item schweigsam
\end{enumerate}

\Uebung[\tristar] \label{u73} Beschreiben Sie folgende Fälle als Wortbildung.
Was könnte ein Problem bezüglich der Struktur des Lexikons im Rahmen des Gesamtsystems der Grammatik sein?

\begin{enumerate}\Lf
  \item (das) Sich-in-die-kosmische-Unendlichkeit-Einfügen
  \item (die) Ethanol-haltige-Gefahrstoff-Kennzeichnung
  \item (eine) Mehr-als-Beliebigkeit
\end{enumerate}

\Uebung[\tristar] \label{u74} Wie sind folgende Fälle gebildet?
Wie passen sie in das System der Wortbildung?

\index{Kurzwort}
\begin{enumerate}\Lf
  \item Lok (Lokomotive)
  \item Fundi (eine Person aus dem fundamentalpolitischen Flügel der Partei Bündnis 90\slash Die Grünen)
  \item Vopo (Volkspolizist)
  \item Kotti (Kottbusser Tor)
  \item Schweini (Schweinsteiger)
  \item Poldi (Podolski)
\end{enumerate}

