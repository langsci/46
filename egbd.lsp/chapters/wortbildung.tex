\chapter{Wortbildung}

\label{sec:wortbildung}

Die Wortbildung beschäftigt sich, wie in Abschnitt~\ref{sec:defwb} beschrieben, mit der Bildung neuer Wörter aus existierenden Wörtern.
In diesem Kapitel werden die verschiedenen Arten der Wortbildung der Reihe nach besprochen, zuerst die \textit{Komposition} (Abschnitt~\ref{sec:komp}), dann die \textit{Konversion} (Abschnitt~\ref{sec:konv}) und abschließend die \textit{Derivation} (Abschnitt~\ref{sec:deriv}).
 
\section{Komposition}

\label{sec:komp}

\subsection{Definition und Überblick}

Eine im Deutschen besonders häufige Art der Wortbildung ist die \textit{Komposition}, eine Aneinanderfügung existierender Wörter.
Erste Beispiele für Komposita sind \textit{Haus.meister} oder \textit{Rot.barsch}.
Wir markieren die Stelle zwischen den aneinandergefügten Wörtern in Komposita mit einem Punkt und nicht mit einem Bindestrich wie bei der Flexion.

\Definition{Komposition}{
\label{def:komp}
Komposition ist ein Wortbildungsmuster, bei dem lexikalische Wörter gebildet werden, deren Stamm aus zwei Stämmen anderer lexikalischer Wörter zusammengesetzt ist, die die \textit{Glieder} des \textit{Kompositums} genannt werden.
Das Kompositum erhält seine grammatischen und semantischen Merkmale auf produktive oder zumindest meistens transparente Weise von den beiden Gliedern.
\index{Komposition}
}

Laut Definition~\ref{def:komp} soll Komposition nun Wortbildung sein.
Es müssen also statische Merkmale geändert, Merkmale gelöscht oder hinzugefügt werden (vgl.\ Definitionen~\ref{def:inhmerkmal} auf S.~\pageref{def:inhmerkmal} und \ref{def:wortbild} auf S.~\pageref{def:wortbild}).
Wir beginnen mit Überlegungen zu Bedeutungsmerkmalen und wenden uns dann formalen grammatischen Merkmalen im Kompositum zu.
Bezüglich der Bedeutung ist in Komposita vor allem die Relation zwischen seiner Bedeutung und den Bedeutungen seiner Bestandteile interessant.
In Wörtern wie \textit{Haus.meister} erkennen wir sofort die Bestandteile \textit{Haus} und \textit{Meister}, und die Gesamtbedeutung des Kompositums hat mit den Bedeutungen dieser Bestandteile auch erkennbar zu tun.
Würden wir aber das Wort \textit{Hausmeister} zum ersten Mal hören, wäre es fraglich, ob wir nur durch das Wort sofort einen präzisen Eindruck von der Tätigkeit eines Hausmeisters erhielten.

Diese Überlegungen lassen sich mit den Begriffen \textit{Produktivität} und \textit{Transparenz} auf den Punkt bringen.
Wenn die Bildung von Komposita nämlich eine Regularität des grammatischen Systems ist, müsste das bedeuten, dass ein Sprachbenutzer sich jederzeit dieser Regularität bedienen kann, um neue Komposita zu bilden.
Das scheint auch so zu sein, denn bei Bedarf können kurze Komposita wie \textit{Flaschenkiste} oder lange wie \textit{Sprechstundenverlegungsbenachrichtigung} jederzeit gebildet und verstanden werden.
Komposition ist also \textit{produktiv} im Sinn von Definition~\ref{def:produktivitaet}.%
\footnote{Einige typische semantische Beziehungen zwischen den Gliedern eines produktiv gebildeten Kompositums werden in Abschnitt~\ref{sec:detrekkomp} besprochen.}

\Definition{Produktivität}{
\label{def:produktivitaet}
Eine Regularität ist produktiv, wenn sie jederzeit und nahezu uneingeschränkt angewendet werden kann, um grammatische Strukturen aufzubauen.
Resultierende Strukturen sind produktiv gebildet und ihre Bedeutung ist kompositional.
\index{Produktivität}\index{Kompositionalität}
}

Mit Wörtern wie \textit{Hausmeister}, \textit{Kindergarten} und vielen anderen verhält es sich aber etwas anders.
Diese Wörter sehen aus, als seien sie produktiv nach der Regularität der Komposition gebildet -- und sie waren dies auch sicherlich irgendwann einmal.
Allerdings ist ihre Bedeutung spezieller, als es die produktive Regularität vermuten lassen würde.
Die Bedeutung ist nicht \textit{kompositional} (s.\ Abschnitt~\ref{sec:sprachealssymbolsystem}), sie ergibt sich also nicht einfach aus der Bedeutung seiner Bestandteile.
Da die Bestandteile aber einwandfrei identifizierbar sind, würde man von einem \textit{transparent} gebildeten Kompositum gemäß Definition~\ref{def:transparenz} sprechen.

\Definition{Transparenz}{
\label{def:transparenz}
Eine transparente Struktur ist eine, die erkennbar einem produktiven Muster entspricht, die aber in ihrer Bedeutung oder Funktion spezialisiert wurde und damit im Lexikon abgelegt werden muss (Lexikalisierung).
Transparente Bildungen haben demnach keine vollständig kompositionale Bedeutung.
\index{Transparenz}
}

Die Übergänge sind fließend, und man muss einen großen empirischen Aufwand betreiben, um bei der Frage, ob ein bestimmtes Kompositum produktiv gebildet ist oder nicht, zu vernünftigen Ergebnissen zu kommen.
Völlig eindeutig ist bei Komposita auch ohne größeren empirischen Aufwand eine produktive Bildung nur dann auszuschließen, wenn eins der Glieder nicht oder nicht mehr alleine vorkommen kann, wie \textit{*Him} und \textit{*Brom} in \textit{Himbeere} und \textit{Brombeere}.

Im Gegensatz zur Analyse und Beschreibung der Bedeutungsmerkmale eines Kompositums ist die Beschreibung seiner grammatischen Merkmale einfach.
In \textit{Haus.meister} ist \zB der Verlust sämtlicher grammatischer Merkmale von \textit{Haus} offensichtlich.
Das erste Glied \textit{Haus} ist \zB ein Neutrum, das zweite Glied \textit{Meister} ist ein Maskulinum.
Das Kompositum \textit{Haus.meister} ist immer und ohne Ausnahme ein Maskulinum, und vom ursprünglichen Genus des ersten Gliedes ist im Kompositum nichts mehr zu erkennen.
Das lässt sich auf alle Komposita generalisieren, denn das hintere Glied setzt seine grammatischen Merkmale immer durch, und das vordere verliert die seinen immer.
Es werden also auf jeden Fall Werte statischer Merkmale überschrieben bzw.\ Merkmale gelöscht.

\Definition{Kopf (Kompositum)}{
\label{def:kompkopf}
Der Kopf eines Kompositums ist das Glied, das die Werte der statischen grammatischen Merkmale und die gesamte grammatische Merkmalsaustattung des Kompositums bestimmt.
Der Kopf ist immer das rechte Glied.
\index{Kopf!Komposition}
}

Es ist nicht nur das Genus betroffen.
In \textit{Rot.barsch} ist \textit{rot} ein Adjektiv und \textit{Barsch} ein maskulines Substantiv.
Das Ergebnis der Wortbildung -- also \textit{Rot.barsch} -- ist ein maskulines Substantiv, genau wie \textit{Barsch}.

Wir schließen mit weiteren Beispielen in (\ref{ex:wb2471}).
Das Erstglied kommt hier aus den Klassen der Substantive (\textit{Kopf}, \textit{Student}, \textit{Feuer}), Adjektive (\textit{laut}, \textit{rot}, \textit{fertig}) und Verben (\textit{laufen}, \textit{essen}).
Bei \textit{Lehr.veranstaltung} ist nicht ohne weiteres erkennbar, ob \textit{Lehre} oder \textit{lehren} zugrundeliegt.
Im Fall von \textit{feuer.rot} ist der Kopf ein Adjektiv.
Komposita sind also nicht nur Substantive, und der Fokus auf Substantivkomposita in diesem Kapitel hat nur pragmatische Gründe.

\begin{exe}
  \ex\label{ex:wb2471}
  \begin{xlist}
    \ex{Kopf.hörer}
    \ex{Laut.sprecher}
    \ex{Studenten.werk}
    \ex{Lehr.veranstaltung}
    \ex{Rot.eiche}
    \ex{Lauf.schuhe}
	\ex{Ess.besteck}
    \ex{Fertig.gericht}
    \ex{feuer.rot}
  \end{xlist}
\end{exe}

\subsection{Kompositionstypen}

\label{sec:detrekkomp}

\index{Kompositum!Rektions--}
\index{Kompositum!Determinativ--}

In diesem Abschnitt werden Komposita diskutiert, die ein klar definiertes semantisches Zentrum haben.
Überlegen wir uns intuitiv, wie sich der Kopf und der Nicht-Kopf der Komposita in (\ref{ex:detkomp}) und (\ref{ex:rekkomp}) semantisch zueinander verhalten.%
\footnote{Die Beispiele sind \citealp[217ff.]{Eisenberg1} entnommen.}

\begin{exe}
  \ex\label{ex:detkomp}
  \begin{xlist}
  	\ex Schul.heft
  	\ex Staats.finanzen
  \end{xlist}
  \ex\label{ex:rekkomp}
  \begin{xlist}
    \ex Kandidaten.nennung
    \ex Managerinnen.schulung
    \ex Geld.wäsche
  \end{xlist}
\end{exe}

In den Gruppen in (\ref{ex:detkomp}) und (\ref{ex:rekkomp}) bildet der Kopf (das rechte Glied) das semantische Zentrum auf eine charakteristische Weise.
Wir können nämlich für jedes dieser Komposita einen Satz wie in (\ref{ex:kompimpl}) formulieren, der in jedem Fall wahr ist.
Bei \textit{Geldwäsche} klingt das Ergebnis evtl.\ leicht dubios.
Dies dürfte daran liegen, dass \textit{Geldwäsche} von den gegebenen Beispielen am wenigsten produktiv gebildet ist und eine stark spezialisierte Gesamtbedeutung aufweist.

\begin{exe}
  \ex\label{ex:kompimpl}
  \begin{xlist}
    \ex{Ein Schulheft ist ein Heft. }
    \ex{Staatsfinanzen sind Finanzen.}
    \ex{Eine Kandidatennennung ist eine Nennung.}
    \ex{Eine Managerinnenschulung ist eine Schulung.}
    \ex{Eine Geldwäsche ist eine Wäsche.}
  \end{xlist}
\end{exe}

Die vom Kompositum bezeichneten Gegenstände können also auch immer von dem Kopf bezeichnet werden.
Anders gesagt bezeichnet das Kompositum eine Untermenge der Menge, die vom Kopf bezeichnet wird.
Mit dem Nicht-Kopf (dem linken Glied) verhält es sich niemals so, wie man an (\ref{ex:kompimplfail}) leicht sieht.
Das Zeichen \# signalisiert, dass diese Sätze niemals wahr sein können.
Sie zeigen, was mit semantischen Zentrum (manchmal auch \textit{Kern}) gemeint ist, denn der Kopf dominiert das Kompositum nicht nur grammatisch, sondern auch in der Bedeutung.

\begin{exe}
  \ex\label{ex:kompimplfail}
  \begin{xlist}
    \ex[\#]{Ein Schulheft ist eine Schule. }
    \ex[\#]{Staatsfinanzen sind ein Staat.}
    \ex[\#]{Eine Kandidatennennung ist ein Kandidat.}
    \ex[\#]{Eine Managerinnenschulung ist eine Managerin.}
    \ex[\#]{Eine Geldwäsche ist Geld.}
  \end{xlist}
\end{exe}

Zwischen den Gruppen (\ref{ex:detkomp}) und (\ref{ex:rekkomp}) lässt sich ebenfalls ein Unterschied feststellen.
Die Testsätze in (\ref{ex:kompimpl}) funktionieren für beide, aber es gibt eine zusätzliche Testsatzkonstruktion, die nur für die Gruppe in (\ref{ex:rekkomp}) funktioniert, nämlich die Tests in (\ref{ex:rekkompsucc}).

\begin{exe}
  \ex\label{ex:rekkompsucc}
  \begin{xlist}
    \ex{Bei einer Kandidatennennung wird ein Kandidat genannt.}
    \ex{Bei einer Managerinnenschulung wird eine Managerin geschult.}
    \ex{Bei einer Geldwäsche wird Geld gewaschen.}
  \end{xlist}
\end{exe}

Für die Gruppe aus (\ref{ex:detkomp}) lassen sich die entsprechenden Sätze meist nicht vernünftig bilden, vgl.\ (\ref{ex:rekkompfail}).
Selbst wenn man ihre Bildung forciert, sind die Sätze prinzipiell falsch.
  
\begin{exe}
  \ex\label{ex:rekkompfail}
  \begin{xlist}
    \ex[\#]{Bei einem Schulfheft wird eine Schule geheftet.}
    \ex[\#]{Bei Staatsfinanzen wird ein Staat finanziert.}
  \end{xlist}
\end{exe}

Die Komposita, für die Testsätze wie in (\ref{ex:rekkompsucc}) funktionieren, nennt man \textit{Rektionskomposita}, weil ihrem Kopf-Substantiv ein Verb wie hier \textit{nennen}, \textit{schulen} oder \textit{waschen} zugrundeliegt (zur Ableitung vom Verb zum Substantiv vgl.\ Abschnitt~\ref{sec:deriv}), und in einem Satz mit diesem Verb das linke Glied (der Nicht-Kopf) das direkte Objekt (im Akkusativ) wäre -- so wie in den Sätzen in (\ref{ex:rekkompsucc}).
In einem Satz mit einem dem Kopf entsprechenden Verb würde dieses Verb den Akkusativ regieren. 
Daher der Name Rektionskompositum.

Die Gruppe aus (\ref{ex:detkomp}), also \textit{Schul.heft}, \textit{Staats.finanzen} usw. werden \textit{Determinativkomposita} genannt, weil der Nicht-Kopf den Kopf semantisch näher bestimmt (\textit{determiniert}), aber eben keine Rektionsbeziehung gegeben ist.
Zusammenfassend kann also Satz~\ref{satz:detrekkomp} aufgestellt werden.

\Satz{Determinativ- und Rektionskompositum}{\label{satz:detrekkomp}
Wenn der Test aus (\ref{ex:kompimpl}) funktioniert und die Tests aus (\ref{ex:kompimplfail}) und (\ref{ex:rekkompsucc}) misslingen, liegt ein \textit{Determinativkompositum} vor.
Wenn die Tests aus (\ref{ex:kompimpl}) und (\ref{ex:rekkompsucc}) funktionieren und der Test aus (\ref{ex:kompimplfail}) misslingt, liegt ein \textit{Rektionskompositum} vor.
}

Aus grammatischer Sicht kann festgestellt werden, dass das Determinativkompositum der Prototyp des Kompositums ist.
Das Rektionskompositum ist ebenfalls ein relevantes grammatisches Phänomen, da seine durchaus produktive Bildung mit einem bestimmten Valenzmuster (Verben mit Akkusativ) zusammenfällt.

\subsection{Rekursion}

\label{sec:rekursion}

\index{Rekursion}

Definition~\ref{def:komp} besagte, dass in einem Kompositum jeweils zwei Wörter (bzw. ihre Stämme) zusammengefügt werden.
In diesem Zusammenhang muss man sich nun fragen, wie es sich mit Wörtern wie \textit{Lang.strecken.lauf} verhält.
An diesem Kompositum sind offensichtlich drei Glieder beteiligt, und die Definition scheint diesen Fall zunächst nicht abzudecken.
Wenn man aber überlegt, ob die Glieder dieses Kompositums in einem jeweils gleichen Verhältnis zueinander stehen, dann erkennt man, dass dies nicht so ist.
Ein Langstreckenlauf ist semantisch betrachtet wahrscheinlich in den meisten Fällen der Lauf einer Langstrecke, denn das Wort \textit{Lang.strecke} ist nicht nur bildbar, sondern wird auch von Sprechern häufig verwendet.
Seltener wird wahrscheinlich der lange Lauf einer Strecke bezeichnet, denn das Wort \textit{Strecken.lauf} ist durchaus bildbar, wird aber kaum verwendet.

Trotzdem existieren zweifelsfrei beide Interpretationsmöglichkeiten.
Sie rühren daher, dass man die Glieder des Kompositums in verschiedene Zweiergruppen zusammenfassen kann und sich die Bedeutung im Sinn der Kompositionalität entsprechend ändert.
Man kann die unterschiedlichen Strukturen mit Klammern sehr gut verdeutlichen, s.\ (\ref{ex:kompbrack}).
Alternativ kann das morphologische Strukturformat aus Abschnitt~\ref{sec:morphstrukform} benutzt werden, s.\ Abbildung~\ref{fig:langstreckenlauf}.
Zur Verdeutlichung werden hier die Wortklassen im Baum annotiert.

\begin{exe}
  \ex\label{ex:kompbrack}\begin{xlist}
  \ex{(Lang.strecken).lauf}
  \ex{Lang.(strecken.lauf)}
  \end{xlist}
\end{exe}

\begin{figure}[!htbp]
  \centering
  \resizebox{0.45\textwidth}{!}{
    \Tree[1.5]{
      && \K{Subst-Stamm}\B{dl}\B{dr} \\ 
      & \K{Subst-Stamm}\B{dl}\B{dr} && \K{Subst-Stamm}\B{d} \\
      \K{Adj-Stamm}\B{d} && \K{Subst-Stamm}\B{d} & \K{\textit{lauf}} \\
      \K{\textit{Lang}} && \K{\textit{strecken}} \\
    }
  }\hspace{0.09\textwidth}\resizebox{0.45\textwidth}{!}{
     \Tree[1.5]{
      & \K{Subst-Stamm}\B{dl}\B{dr} \\
      \K{Adj-Stamm}\B{d} && \K{Subst-Stamm}\B{dl}\B{dr} \\
      \K{\textit{Lang}} & \K{Subst-Stamm}\B{d} && \K{Subst-Stamm}\B{d} \\
      & \K{\textit{strecken}} && \K{\textit{lauf}} \\
    }
  }
  \caption{Zwei mögliche Analysen von \textit{Langstreckenlauf}}
  \label{fig:langstreckenlauf}
\end{figure}

Je nachdem, welche Reihenfolge von Kompositionsprozessen man annimmt, ergeben sich die verschiedenen Bedeutungen.
Es gibt in der Regel aber keine grammatischen Kriterien für oder gegen eine bestimmte Analyse.
Die Grammatik (in diesem Fall die Regularitäten der Komposition) sagt uns lediglich, dass alle denkbaren Strukturanalysen aus geschachtelten Zweiergruppen von Gliedern möglich sind, nicht aber, welche plausibel oder am häufigsten sind.
Die Entscheidung wird immer aufgrund von mehr oder weniger subjektiven semantischen Erwägungen im Einzelfall gefällt.

\begin{Vertiefung}{Wahrscheinliche Analysen von Komposita}

\noindent Man kann durch Analysen der Häufigkeit der beteiligten Wörter bestimmte Analysen plausibilisieren.
Im DeReKo findet man zum Beispiel für \textit{Langstrecke} 3.804 Belege, für \textit{Streckenlauf} hingegen nur 18 (bei Anfragen mit Wortformenoperator am 26.12.2009 im Archiv W-Öffentlich.)
Der einfache Vergleich dieser absoluten Häufigkeiten zeigt, dass die Wahrscheinlichkeit für die Analyse (\textit{Lang.strecken})\textit{.lauf} deutlich höher ist als die für \textit{Lang.}(\textit{strecken.lauf}), ganz einfach weil das Wort \textit{Lang.strecke} für sich genommen stärker im Wortschatz des Deutschen vertreten ist.
Im Einzelfall muss natürlich trotzdem damit gerechnet werden, dass die unwahrscheinlichere Analyse je nach Kontext doch die zutreffende ist.

\end{Vertiefung}

Unabhängig von Problemen bei der konkreten Analyse im Einzelfall ist aus grammatischer Sicht aber auf jeden Fall interessant, dass die Komposition ein Prozess ist, bei dem das Ergebnis des Prozesses wieder als Ausgangsbasis des gleichen Prozesses verwendet werden kann.
Wurden also einmal \textit{lang} und \textit{Strecke} zu \textit{Lang.strecke} komponiert, kann das dabei entstehende Kompositum wie jedes andere Substantiv erneut in einem Kompositionsprozess verwendet werden.
Diese Eigenschaft mancher produktiver Prozesse nennt man \textit{Rekursion}.
Innerhalb der Morphologie muss beachtet werden, dass Flexion die Eigenschaft der Rekursion nicht hat.
Wenn ein Substantiv einmal nach Kasus und Numerus flektiert wurde, kann dies nicht nochmal geschehen.
Gleiches gilt für ein Verb, das nach Modus, Tempus, Person und Numerus flektiert wurde.
Es kommt also eine weitere Unterscheidung zwischen Flexion und Wortbildung hinzu, s.\ Satz~\ref{satz:reknrek}.

\Satz{Rekursion in der Morphologie}{
\label{satz:reknrek}
Wortbildung ist ein (eingeschränkt) rekursiver morphologischer Prozess.
Flexion ist ein nicht-rekursiver morphologischer Prozess.
\index{Rekursion!in der Morphologie}
}

Bei Satz~\ref{satz:reknrek} ist zu beachten, dass allgemein von Wortbildung gesprochen wird, nicht nur von Komposition.
In eingeschränkterem Maß sind Konversion (Abschnitt~\ref{sec:konv}) und Derivation (Abschnitt~\ref{sec:deriv}) ebenfalls rekursiv.

\subsection{Kompositionsfugen}

\index{Kompositionsfuge}
\index{Fugenelement}

Besonders in den hier in erster Linie betrachteten Komposita aus Substantiv und Substantiv gibt es in vielen, aber nicht allen Fällen eine morphologische Markierung, die an der so genannten \textit{Fuge} (der Grenze zwischen den beiden Gliedern des Kompositums) auftritt.
Betrachtet man Wörter wie (\textit{Lang.strecke-n})\textit{.lauf}, so sieht man, dass nicht einfach die Stämme der beiden Glieder das Komposition bilden.
Vielmehr wird das Suffix \textit{-n} an das Vorderglied angefügt.
In diesem Fall ist das so genannte \textit{Fugenelement} \textit{-n} formal identisch mit der Pluralmarkierung des Wortes \textit{Langstrecke}.
Man könnte nun vermuten, dass \textit{-n} hier tatsächlich die Markierungsfunktion [\textsc{Numerus}: \textit{pl}] hat.
Gegen diese Vermutung spricht vor allem ein semantischer Grund.
Bei einem Langstreckenlauf werden nicht zwangsläufig mehrere Strecken gelaufen, es kann sich nicht um die Pluralmarkierung handeln.
Das Suffix \textit{-n} ist vielmehr ein bei der Wortbildung an der Fuge auftretendes spezielles Affix ohne semantische oder grammatische Markierungsfunktion.
Diese Annahme wird weiter gestützt durch das zwischen Verb und Substantiv auftretende Fugen-Schwa wie in \textit{Bad-e.hose}, wobei \textit{bad-e} zwar eine Wortform des Verbs \textit{bad-en} ist (\zB die 1.~Person Singular Präsens), aber die Bedeutung im Kompositum garantiert nicht dieser Verbform entspricht.
Alternativ könnte es auch der Dativ Singular des Substantivs \textit{Bad} sein.
Dafür würde die gleiche Argumentation gelten.
Immerhin gibt es keinen Grund dafür, dass im Kompositum ausgerechnet der Dativ stehen sollte.
Außerdem gibt es Fälle, in denen wie bei \textit{\Ast Schmerz-ens} in \textit{Schmerz-ens.geld} oder \textit{\Ast Heirat-s} in \textit{Heirat-s.antrag} das Fugenelement keiner Kasus-Numerus-Form des Vordergliedes entspricht.

Diese sogenannten Fugenelemente treten in verschiedener Form, aber nicht immer und nur schwer vorhersagbar auf.
Weil sie natürlich nicht paradigmatisch sind, können wir sie eigentlich nicht als Flexion bezeichnen.
Wegen der großen formalen Nähe vieler (nicht aller) Fugenelemente zu Flexionsaffixen trennen wir sie trotzdem mit dem Bindestrich vom vorangehenden Stamm ab.
Die wichtigsten Fugenelemente sind in Tabelle~\ref{tab:fugen} mit Beispielen angegeben.

\begin{table}[!htbp]
  \centering
  \begin{tabular}{ll}
    \lsptoprule
    \textbf{Fuge} & \textbf{Beispiel} \\
    \midrule
    -n & Blume-n.vase \\
    -s & Zweifel-s.fall \\
    -ns & Glaube-ns.frage \\
    -e & Pferd-e.wagen, Bad-e.hose \\
    -er & Kind-er.garten \\
    -en & Held-en.mut \\
    -es & Sieg-es.wille \\
    -ens & Schmerz-ens.schrei \\
    \lspbottomrule
  \end{tabular}
  \caption{Wichtige Fugenelemente}
  \label{tab:fugen}
\end{table}

\begin{figure}[!htbp]
  \centering
  \Tree[1.5]{
    && \K{Subst-Stamm}\B{dl}\B{dr} \\
    & \K{[Subst-Stamm]}\B{dl}\B{dr} && \K{Subst-Stamm}\B{d} \\
    \K{Subst-Stamm}\B{d} && \K{Fugenelement}\B{d} & \K{\textit{schrei}} \\
    \K{\textit{Schmerz}} && \K{\textit{-ens}} \\
  }
  \caption{Kompositionsstrukturen mit Fugenelement}
  \label{fig:kompfugstruk}
\end{figure}

\index{Kompositionsfuge}
Das Gegenteil zur Fugenbildung mit Fugenelementen gibt es in einigen Fällen auch, nämlich die \textit{Suffixtilgung} an der Fuge.
Manche produktiven oder historischen Wortbildungssuffixe werden an der Kompositionsfuge gelöscht.
Beispielsweise entfällt das alte Ableitungssuffix für feminine Substantive \textit{:e} (wie in \textit{Wolle}) in Komposita wie \textit{Woll.decke}.
Genauso wie das Auftreten der Fugenelemente ist diese Tilgung allerdings nicht auf einfache Weise systematisch beschreibbar.

Damit sind viele der wesentlichen grammatischen Besonderheiten der Komposition beschrieben.
Die in \ref{sec:konv} und \ref{sec:deriv} diskutierten Wortbildungstypen gehen anders als die Komposition immer von nur einem einzelnen Stamm aus.


\Zusammenfassung{
  Ein morphologischer Prozess ist umso \textit{produktiver}, je weniger Einschränkung es bezüglich seiner Anwendbarkeit auf die Wörter einer Wortklasse gibt.
  Ein Prozess ist \textit{transparent} (ggf.\ aber nicht produktiv), wenn die Art seiner Bildung deutlich erkennbar ist, wie bei \textit{Hausmeister}.
  \textit{Komposita} sind Neubildungen eines Worts aus zwei existierenden Wörtern, von denen eins als Kopf die grammatischen Merkmale der Neubildung bestimmt.
  In der Komposition werden immer zwei Wörter zusammengesetzt, ggf.\ aber \textit{rekursiv}.
  \textit{Fugenelemente} haben keine einfach zu bestimmende grammatische Funktion wie die Markierung von Kasus oder Numerus.
}


\section{Konversion}

\label{sec:konv}

\subsection{Definition und Überblick}

\label{sec:konvdef}

Es wurde im letzten Abschnitt gezeigt, dass der Wortschatz einer Sprache durch Kompositionsbeziehungen zwischen Wörtern besonders strukturiert sein kann.
Ähnliche Prinzipien kann man auch in einem anderen Bereich der Wortbildung beobachten.
Vergleichen wir dazu die folgenden Beispiele (\ref{ex:wobimotiv}).

\begin{exe}
  \ex\label{ex:wobimotiv}
  \begin{xlist}
    \ex{Simone geht gerne einkaufen.}
    \ex{Das Einkaufen macht Simone Spaß.}
  \end{xlist}
\end{exe}

Im ersten Satz kommt \textit{einkauf-en} als Infinitiv des Verbs (also als Verbform) vor.
Im zweiten Satz steht \textit{Einkaufen} mit dem bestimmten Artikel als Subjekt des Satzes, es handelt sich also um ein Substantiv.
Die Orthographie verlangt genau wegen dieses Wechsels in die Klasse der Substantive, dass das Wort groß geschrieben wird (mehr in Abschnitt~\ref{sec:wortklassschreib}).
Da [\textsc{Klasse}: \textit{\textbf{subst}}] und [\textsc{Klasse}: \textit{\textbf{verb}}] statische Merkmale sind, kann die Beziehung zwischen den Wortformen \textit{einkauf-en} und \textit{Einkaufen} keine Flexionsbeziehung sein, sondern es muss sich um Wortbildung handeln (vgl.\ Definition~\ref{def:wortbild}, S.~\pageref{def:wortbild}).
Es handelt sich also jeweils um die Wortform eines eigenen Wortes (Substantiv bzw.\ Verb).
Trotzdem ist die Beziehung zwischen diesen beiden Wörtern vollständig vorhersagbar, denn fast jedes Verb in seiner Infinitivform kann auf diese Weise als Substantiv mit [\textsc{Genus}: \textit{\textbf{neut}}] verwendet werden.

Wir führen deshalb mit Definition~\ref{def:konversion} einen neuen Typ von Wortbildungsprozess -- die \textit{Konversion} -- ein, wobei wir das Wort, das dem Prozess unterzogen wird, als \textit{Ausgangswort} bezeichnen und das Ergebnis als \textit{Zielwort}.

\Definition{Konversion}{
\label{def:konversion}
Konversion ist ein Wortbildungsprozess, bei dem ein Stamm (\textit{Stammkonversion}) oder eine Wortform (\textit{Wortformenkonversion}) eines Ausgangswortes als Stamm eines Zielwortes verwendet wird, wobei Wortklassenwechsel stattfindet.
\index{Konversion}\index{Wortklasse}
}

Diese Definition erfasst zwei verschiedene Fälle, von denen erst einer an Beispielen eingeführt wurde.
Der erste ist der, bei dem ein Stamm der Ausgangspunkt des Wortbildungsprozesses ist, und der zweite ist der, bei dem der Ausgangspunkt eine Wortform ist.
Illustriert wird der Unterschied durch Satz (\ref{ex:wobimotiv2}) als Ergänzung zu (\ref{ex:wobimotiv}).

\begin{exe}
  \ex{\label{ex:wobimotiv2} Der Einkauf an Heiligabend hat vier Stunden gedauert.}
\end{exe}

In diesem Beispiel wird ein zweites Wort verwendet, welches offensichtlich auch in einer Wortbildungsbeziehung zu dem Verb \textit{einkauf-en} steht.
Dass \textit{Einkauf} nicht dasselbe Substantiv wie \textit{Einkaufen} sein kann, sieht man leicht daran, dass das Genus der Wörter unterschiedlich ist (Maskulinum beziehungsweise Neutrum).
Außerdem unterscheiden sich die beiden Substantive darin, ob sie einen Plural bilden können.
\textit{Einkauf} kann einen Plural bilden (\textit{Einkäuf-e}), \textit{Einkaufen} hingegen nicht, vgl.\ Tabelle~\ref{tab:einkauf-en}.

\begin{table}[!htbp]
  \centering
  \begin{tabular}{llll}
    \lsptoprule
    \textbf{Numerus} & \textbf{Kasus} & \textbf{Stammkonversion} & \textbf{Wortformenkonversion} \\
    && \textbf{(Maskulinum)} & \textbf{(Neutrum)} \\
    \midrule
    \multirow{4}{*}{\textbf{Singular}} & \textbf{Nominativ} & Einkauf & Einkaufen\\
     & \textbf{Akkusativ} & Einkauf & Einkaufen\\
     & \textbf{Dativ} & Einkauf & Einkaufen\\
     & \textbf{Genitiv} & Einkauf-s & Einkaufen-s\\
    \midrule
    \multirow{4}{*}{\textbf{Plural}} & \textbf{Nominativ} & Einkäuf-e & --- \\
     & \textbf{Akkusativ} & Einkäuf-e & --- \\
     & \textbf{Dativ} & Einkäuf-e-n & --- \\
     & \textbf{Genitiv} & Einkäuf-e & --- \\
    \lspbottomrule
  \end{tabular}
  \caption{Kasus-Numerus-Paradigma für \textit{Einkauf} und \textit{Einkaufen}}
  \label{tab:einkauf-en}
\end{table}

\index{Wort!Stamm}

Die beiden Wörter sind also voneinander verschieden, haben unterschiedliche Stämme (\textit{Einkauf} und \textit{Einkaufen}) und eine unterschiedliche Formenbildung.
Wir gehen hier daher davon aus, dass sie durch unterschiedliche Konversionsprozesse aus dem Verb gebildet wurden.
Im Fall von \textit{Einkaufen} wurde eine Wortform zugrundegelegt, nämlich der Infinitiv.
Es handelt sich also um den zweiten Fall aus der Definition, nämlich \textit{Wortformenkonversion}.
Im Gegensatz dazu ist bei \textit{Einkauf} der Verbalstamm in einen Substantivstamm konvertiert worden.
Bei diesem Konversionstyp entsteht immer ein maskulines Substantiv.
Dies entspricht dem ersten Fall aus der Definition, also der \textit{Stammkonversion}.
Die Subklassifikation als Stammkonversion und Wortformenkonversion richtet sich dabei nach dem Ausgangspunkt der Konversion.
Das Ergebnis der Konversion ist selbstverständlich immer ein Stamm, denn es verhält sich wie ein gewöhnliches Wort der Wortklasse, zu der es gehört.
Es flektiert also wie jedes andere Verb oder Nomen, oder es ist unveränderlich (falls das Zielwort \zB ein Adverb ist).

Es muss terminologisch beachtet werden, dass im Falle unregelmäßiger Bildungen, bei denen \zB im Konversionsprodukt Ablautstufen vorliegen, die es sonst nicht gibt, nicht von Konversion gesprochen werden sollte.
Ein Beispiel dafür wäre \textit{schieß-en} zu \textit{Schuss}.
Diese Fälle behandeln wir als unregelmäßige, nicht-pro\-duk\-ti\-ve Bildungen, und betrachten die Stämme in unserer synchronen Grammatik als nicht aufeinander bezogen.
In diesem Fall gibt es trotz der lautlichen Ähnlichkeit und dem eindeutigen semantischen Bezug zwischen \textit{Schuss} und \textit{schieß-en} keine grammatische Beziehung.
Im nächsten Abschnitt folgen nun Beispiele für eindeutige Konversionsprozesse im Deutschen.

\subsection{Konversion im Deutschen}

\label{sec:konvdeutsch}

Spezielle Bezeichnungen für Konversionsprozesse werden normalerweise nach der Wortklasse des Zielwortes mit \textit{-ierung} gebildet.
Eine Konversion, bei der das Zielwort zur Klasse der Adjektive gehört, wird also als \textit{Adjektivierung} bezeichnet usw.
In den Tabellen~\ref{tab:wfkonv} und \ref{tab:stammkonv} finden sich einige Beispiele, geordnet nach Wortformenkonversion und Stammkonversion sowie der Wortklasse des Zielwortes in eindeutigen syntaktischen Kontexten.%
\footnote{Die Beispiele wurden aus \citealp[280]{Eisenberg1} übernommen.}

\begin{table}[!htbp]
  \centering
  \resizebox{\textwidth}{!}{
    \begin{tabular}{lll}
      \lsptoprule
      \textbf{Typ} & \textbf{Ausgangswort} & \textbf{Zielwort} \\
      \midrule
      Adjektivierung & (Der Zaun wurde) ge-strich-en. & (der) gestrichen-e (Zaun) \\
      Substantivierung & (der) gestrichen-e (Zaun) & (der/die/das) Gestrichen-e \\
      \lspbottomrule
    \end{tabular}
  }
  \caption{Beispiele für Wortformenkonversion}
  \label{tab:wfkonv}
\end{table}

\begin{table}[!htbp]
  \centering
  \begin{tabular}{lll}
    \lsptoprule
    \textbf{Typ} & \textbf{Ausgangswort} & \textbf{Zielwort} \\
    \midrule
    Substantivierung & (Wir sollen) lauf-en. & (der) Lauf \\
    Verbalisierung & (der) grün-e (Rasen) & (Der Rasen) grün-t.\\
    \lspbottomrule
  \end{tabular}
  \caption{Beispiele für Stammkonversion}
  \label{tab:stammkonv}
\end{table}

Die Wortformenkonversion vom Adjektiv \textit{gestrichen-e} zum Substantiv \textit{Gestrichenes} ist eigentlich ein Sonderfall.
Hier wird eine voll flektierte adjektivische Wortformen als Substantiv verwendet, denn das Zielwort flektiert nicht wie ein Substantiv, sondern wie ein Adjektiv (vgl.\ Kapitel~\ref{sec:nominalflexion}).
Man könnte sagen, dass es sich um eine Konversion von einer Wortform zu einer Wortform handelt und nicht um eine Konversion von einer Wortform zu einem Stamm.
Eine andere Lösung wäre es, gar nicht von Konversion auszugehen, sondern von einem Adjektiv, das mit einem nicht ausgedrückten Substantiv oder vor einer leeren Substantiv-Position in der Nominalphrase auftritt (vgl.\ Abschnitt~\ref{sec:ngr}).
Welche Beschreibung man auch wählt, ist für unsere Belange nicht sehr zentral.

Zur Notation der Wortanalysen muss noch Folgendes angemerkt werden.
Ist vom Infinitiv des Verbs die Rede, handelt es sich um eine Wortform aus einem Verbstamm und einem Flexionssuffix, weswegen der Bindestrich zwischen den Bestandteilen Wortstamm und Suffix stehen muss: \textit{kauf-en}.
Sobald die Wortformenkonversion zum Substantiv erfolgt ist, verhält sich das Resultat morphologisch immer wie ein Substantivstamm, und der Bindestrich muss entfallen: \textit{Kaufen}.

An den Beispielen in Tabelle \ref{tab:wfkonv} kann man erkennen, dass auch der Prozess der Konversion prinzipiell (aber gegenüber der Komposition eingeschränkt) rekursiv durchführbar ist, denn vom Partizip \textit{ge-strich-en} (zur Bildung der Form des Partizips s.\ Abschnitt~\ref{sec:infinflex}) kann ein Adjektiv \textit{gestrichen} gebildet werden, und von diesem Adjektiv kann wiederum durch Konversion ein Substantiv (\textit{der\slash die\slash das}) \textit{Gestrichen-e} gebildet werden.
Eine Darstellung in Strukturbäumen findet sich in den Abbildungen~\ref{fig:konv1} und \ref{fig:konv2}.

\begin{figure}[!htbp]
  \centering
  \Tree{
    \K{Subst-Stamm}\B{d} \\
    \K{V-Stamm}\B{d} \\
    \K{\textit{lauf}} \\
  }
  \caption{Einfache Stammkonversion}
  \label{fig:konv1}
\end{figure}

\begin{figure}[!htbp]
  \centering
  \Tree[3]{
    && \K{Subst-Wortform}\B{d} \\
    && \K{Adj-Wortform}\B{dr}\B{dl} \\
    & \K{Adj-Stamm}\B{d} && \K{Flexionssuffix}\B{d} \\
    & \K{V-Wortform}\B{dr}\B{d}\B{dl} && \K{\textit{-e}} \\
    \K{Flexionszirkumfix}\B{d} & \K{V-Stamm}\B{d} & \K{Flexionszirkumfix}\B{d} \\
    \K{\textit{ge-}} & \K{\textit{strich}} & \K{\textit{-en}} \\
  }
  \caption{Schrittweise Wortformenkonversionen}
  \label{fig:konv2}
\end{figure}


\Zusammenfassung{
  Bei der \textit{Konversion} werden neue Wörter ohne Formveränderung aus bestehenden Wörtern gebildet, \zB das Substantiv \textit{Blau} aus dem Adjektiv \textit{blau}.
}



\section{Derivation}

\label{sec:deriv}

\subsection{Definition und Überblick}

Bei der Konversion findet typischerweise ein Wortklassenwechsel statt, es gibt aber kein Affix, das eine spezifische semantische Veränderung formal markiert.
Trotzdem sind die semantischen Folgen eines bestimmten Konversionstypus normalerweise konventionalisiert. 
Das bedeutet, dass \zB im Fall der Wortformenkonversion vom verbalen Infinitiv zum Substantiv (\textit{lauf-en} zu \textit{Laufen}) und bei der Stammkonversion (\textit{lauf} zu \textit{Lauf}) per Konvention gut vohersagbar ist, wie die Bedeutung der jeweiligen Ziel-Substantive aus der Bedeutung des Verbs erschlossen werden kann.
In den genannten Fällen bezeichnen die Ziel-Substantive die entsprechende Handlung bzw.\ den Vorgang (bei dem jemand läuft).
Man erwartet daher als kompetenter Sprachbenutzer, dass ein durch Konversion vom Verb gebildetes Substantiv \zB nicht im Einzelfall die handelnde (hier also laufende) Person bezeichnet.
Die Bildungen in (\ref{ex:derivmotiv}) sind hingegen Ableitungen, die unter Verwendung bestimmter Affixe -- per \textit{Derivation} -- zustandekommen.
In diesen Fällen kodiert das konkrete Affix immer eine ganz bestimmte Änderung der Bedeutung bezogen auf das Ausgangswort.
Die Doppelpunkte markieren die Grenzen zwischen dem Stamm des Ausgangswortes und den Derivationsaffixen.

\begin{exe}
  \ex\label{ex:derivmotiv}
  \begin{xlist}
    \ex{Der Läuf:er erreichte das Ziel.}
    \ex{Die Zielmarke ist aus dieser Entfernung schlecht erkenn:bar.}
    \ex{Die Auszehrung beim Marathon ist schreck:lich.}
    \ex{Ullis schreck:haft-er Hund hat einen japanischen Namen.}
  \end{xlist}
\end{exe}

Man erkennt an diesen Beispielen, dass der Beitrag des Affixes zur Bedeutung des Zielwortes recht eindeutig ist.
Mit \textit{Läuf:er} bezeichnet man den Ausführenden einer Handlung des Laufens, und man kann sehr viele Verbalstämme durch Suffigierung von \textit{\~:er} zu einem Substantiv derivieren, das den Ausführenden der Handlung bezeichnet.%
\footnote{Bei genauem Hinsehen ist der Fall von \textit{\~:er} eigentlich komplizierter, wenn man an Bildungen wie (\textit{Früh.blüh})\textit{:er} oder (\textit{Ver:lier})\textit{:er} in Zusammenhang mit der Formulierung \textit{Ausführender der Handlung} denkt.
Diese Verben beschreiben eigentlich keine Handlungen eines absichtlich handelnden Menschen.}
Bei \textit{erkenn:bar} wurde ein Verbalstamm \textit{erkenn} durch das Suffix \textit{:bar} zu einem Adjektiv deriviert, das die Eigenschaft ausdrückt, die Rolle des Erkannten bei einem Prozess des Erkennens spielen zu können.
Weiterhin ist \textit{schreck:lich} ein mit \textit{\~:lich} deriviertes Adjektiv zum Substantivstamm \textit{Schreck}, das die Eigenschaft angibt, etwas zu sein, das gewöhnlicherweise Schrecken hervorruft.
Im Fall von \textit{schreck:haft} (mit \textit{:haft}) ergibt sich die Bezeichnung der Eigenschaft eines belebten Wesens, sehr leicht erschreckbar zu sein.
Allgemein können wir Definition~\ref{def:deriv} aufstellen.

\Definition{Derivation}{
\label{def:deriv}
Derivationen sind Wortbildungsprozesse, bei denen ein neuer Stamm unter Affigierung eines Affixes an einen anderen Stamm gebildet wird, wobei das Resultat zu einem neuen lexikalischen Wort gehört (also im Vergleich zum ursprünglichen Stamm andere statische Merkmale hat).
\index{Derivation}
}

Die Definition des Affixes (Definition~\ref{def:affix}, S.~\pageref{def:affix}) beinhaltet die Bedingung, dass es nicht selbständig auftreten kann.
Es ist \textit{gebunden}.
Der Unterschied der Derivation zur Komposition ist also der, dass bei der Derivation nicht zwei unabhängig vorkommende Stämme den Stamm des Zielworts bilden, sondern ein Stamm, der auch unabhängig vorkommen kann, zusammen mit einem Affix, das nicht selbständig vorkommen kann.
Definition~\ref{def:deriv} beruft sich auf die Definition der Wortbildung (Definition~\ref{def:wortbild}, S.~\pageref{def:wortbild}).
Wir müssen also bei allen Prozessen, die wir als Derivation einstufen, statische Merkmale des Ausgangswortes angeben können, die im Zielwort in ihrem Wert geändert, hinzugefügt oder gelöscht werden.
Bei den in (\ref{ex:derivmotiv}) angegebenen Beispielen ist dies sehr leicht, da sich in allen Fällen das Merkmal \textsc{Klasse} ändert.
Dies muss aber nicht so sein.
Im nächsten Abschnitt werden kurz solche Derivationsaffixe vorgestellt, bei denen scheinbar kein Wortklassenwechsel eintritt.
Danach erfolgt ein Überblick über Derivationsaffixe mit Wortklassenwechsel und Überlegungen zur Rekursivität von Derivationsprozessen.

\subsection{Derivation ohne Wortklassenwechsel}

\label{sec:derivohnewaw}

\index{Wortklasse}
\index{Nomen}

Wir betrachten zunächst ein Beispiel für ein nominales wortklassenerhaltendes Präfix, nämlich genau das oben erwähnte \textit{un:} als Adjektiv- und Substantiv-Präfix.
Das Präfix \textit{un:} hat Negationscharakter, vgl. (\ref{ex:underiv}).

\begin{exe}
  \ex\label{ex:underiv}
  \begin{xlist}
    \ex{Un:mensch, Un:glaube, Un:tiefe}
    \ex{un:bedeutend, un:selig, un:wirsch}
  \end{xlist}
\end{exe}

Dieses Präfix ist allerdings nicht voll produktiv, und in vielen Fällen ist das Ergebnis der Derivation lexikalisiert.
Vor allem bei Substantiven ist die Produktivität eingeschränkt.
Bei Adjektiven gilt, dass es nur bei solchen Adjektiven voll produktiv ist, die selbst einem erkennbaren Muster der Adjektivbildung folgen, vgl.\ (\ref{ex:unadj}) und (\ref{ex:unadj2}).
Trotzdem gibt es Fälle, in denen auch ohne solch ein erkennbares Muster Präfigierung mit \textit{un:} möglich ist wie in (\ref{ex:unadjc}).
Andere Bildungen mit \textit{un:} müssen als lexikalisiert gelten, weil die Stämme der Ausgangswörter selber nicht mehr existieren, wie in (\ref{ex:unlex}).%
\footnote{Es ist in den eindeutig lexikalisierten Fällen natürlich fraglich, ob der Doppelpunkt überhaupt immer gesetzt werden sollte.
Angesichts der nicht produktiven Bildung dieser Wörter wäre ebenso legitim, \textit{unwirsch}, \textit{ungestüm}, \textit{unbedarft} (statt \textit{un:wirsch} usw.) zu schreiben.
Wenn Transparenz als Kriterium für die Analyse ausreicht, kann hier der Doppelpunkt gesetzt werden.}

\begin{exe}
  \ex\label{ex:unadj}
    \begin{xlist} 
      \ex[*]{un:rot}
      \ex[*]{un:schnell}
      \ex[]{\label{ex:unadjc} un:wirsch}
  \end{xlist}
  \ex\label{ex:unadj2}
  \begin{xlist}
    \ex[]{un:(glaub:lich)}
    \ex[]{un:(gläub:ig)}
    \ex[]{un:(beschreib:bar)}
  \end{xlist}
  \ex\label{ex:unlex}
  \begin{xlist}
    \ex[]{un:gestüm}
    \ex[*]{gestüm}
    \ex[]{un:bedarft}
    \ex[*]{bedarft}
  \end{xlist}
\end{exe}

\index{Verb}

Die verbalen wortklassenerhaltenden Präfixe sind im wesentlichen die \textit{Verbpartikeln} und viele (aber nicht alle) \textit{Verbpräfixe}.
Auf einen Unterschied bei der Akzentuierung von Verbpräfixen und Verbpartikeln wurde im Rahmen der Phonologie schon kurz eingegangen (Satz~\ref{satz:pholvprtprf}, S.~\pageref{satz:pholvprtprf}).
Die Unterschiede liegen aber nicht nur im phonologischen, sondern auch im morphologischen und syntaktischen Bereich.
Die Verbpartikel erlaubt den Einschub des Partizip-Präfixes \textit{ge-} und ist syntaktisch trennbar.
Das Verbpräfix blockiert den Einschub des Partizip-Präfixes und ist nicht trennbar.
Diese drei Eigenschaften sind in (\ref{ex:vpart}) und (\ref{ex:vpre}) bebeispielt, wobei als Trennzeichen für die Verbpartikeln = verwendet wird.

\begin{exe}
  \ex{\label{ex:vpart}
  \begin{xlist}
    \ex{\label{ex:vpart-a} Das Auto hat den Pfosten um=ge-fahr-en.}
    \ex{\label{ex:vpart-b} Das Auto fähr-t den Pfosten um=.}
    \ex{\label{ex:vpart-c} Ich möchte den Pfosten \Akz um=fahr-en.}
  \end{xlist}}
  \ex{\label{ex:vpre}
  \begin{xlist}
    \ex{\label{ex:vpre-a} Das Auto hat den Pfosten um:fahr-en.}
    \ex{\label{ex:vpre-b} Das Auto um:fähr-t den Posten.}
    \ex{\label{ex:vpre-c} Ich möchte den Pfosten um:\Akz fahr-en.}
  \end{xlist}}
\end{exe}

\label{abs:praefixundflex}Offensichtlich sind aber beide Arten der Bildung für die Flexion transparent, denn sowohl die Unterdrückung des Partizip-Präfixes in (\ref{ex:vpre-a}) als auch der Einschub des Partizip-Präfixes zwischen Verbpartikel und Verbstamm in (\ref{ex:vpart-a}) erfordern es, dass die Flexion auf die Grenze zwischen Verbpartikel bzw.\ Verbpräfix und Stamm zugreifen kann.
Das könnte ein Hinweis darauf sein, dass die Bildung der Partizipien besser als Wortbildung statt als Flexion beschrieben werden kann.
In morphologischen Theorien wird nämlich oft angenommen, dass erst nach dem vollständigen Abschluss der Wortbildungsprozesse die Flexionsprozesse stattfinden, so dass Mischungen von Wortbildungsaffixen und Flexionsaffixen nicht auftreten sollten.
Um es genauer zu machen, müsste hier ein opulenteres Theorieangebot gemacht werden, wofür der Platz fehlt.
Auf die Möglichkeit, die Bildung von Partizip und Infinitiv als Wortbildung statt als Flexion zu betrachten, gehen wir aber in Abschnitt~\ref{sec:finit} (S.~\pageref{abs:infinwortbild}) aus unabhängigen Gründen noch einmal ein.
Die Benennung als Verb\textit{partikeln} deutet jedenfalls darauf hin, dass die Verbindung zum Verb bei ihnen weniger morphologischer (und mehr syntaktischer) Natur ist als bei den Verbpräfixen.
Immerhin bilden gemäß den Wortklassenfiltern \ref{wfilt:subjunktion} (S.~\pageref{wfilt:subjunktion}) und \ref{wfilt:advpart} (S.~\pageref{wfilt:advpart}) Partikeln normalerweise eine \textit{Wortklasse} (sind also selbständige syntaktische Einheiten), während Affixe laut Definition~\ref{def:affix} (S.~\pageref{def:affix}) unselbständige morphologische Einheiten sind.

Ein Bereich, in den wir hier nicht umfassend einführen können, ist der der Valenzänderungen i.\,w.\,S.\ bei Verbpräfixen und Verbpartikeln.
Valenzänderungen i.\,w.\,S.\ findet man bei Präfixverben, vgl.\ (\ref{ex:wb983460}) und (\ref{ex:wb983461}).

\begin{exe}
  \ex \label{ex:wb983460}
  \begin{xlist}
  	\ex{\label{ex:wb983460a} Nadezhda klagt über den schlechten Grip der Hantel.}
  	\ex{\label{ex:wb983460b} Nadezhda beklagt den schlechten Grip der Hantel.}
  \end{xlist}
  \ex \label{ex:wb983461}
  \begin{xlist}
  	\ex{\label{ex:wb983461a} Jean-Pierre bricht durch die Schallmauer.}
  	\ex{\label{ex:wb983461b} Jean-Pierre durchbricht die Schallmauer.}
  \end{xlist}
\end{exe}

In (\ref{ex:wb983460a}) hat das Ausgangsverb \textit{klagen} einen Valenzrahmen aus einem Nominativ -- hier \textit{Nadezhda} -- und einer Präposition bzw.\ einer \textit{Präpositionalphrase} (vgl.\ die Abschnitte~\ref{sec:prpgr} und \ref{sec:ppergang}) -- hier \textit{über den schlechten Grip der Hantel}.
Die Präfigierung mit \textit{be:} ändert diesen Valenzrahmen.
Die Präpositionalphrase von \textit{klagen} taucht als Akkusativ von \textit{beklagen} wieder auf, s.\ (\ref{ex:wb983460b}).
Durch \textit{be:} entsteht hier also ein transitives Verb.
Ganz ähnlich verhält es sich mit \textit{brechen} und \textit{durchbrechen} in (\ref{ex:wb983461}).
Der wesentliche Unterschied ist, dass die von \textit{brechen} regierte Präposition formal dem Präfix entspricht.

Typisch für Verbpartikeln ist hingegen das Tilgen einer Ergänzung wie in (\ref{ex:wb983462}). 
Das Verb \textit{schreiben} kann mit einer präpositionalen Ergänzung mit \textit{auf} stehen -- in (\ref{ex:wb983462a}) \textit{auf ein Blatt Papier}.%
\footnote{Ob es eine fakultative Ergänzung oder eine Angabe ist, ist für unsere Zwecke nicht ausschlaggebend.}
Das Verb \textit{auf=schreiben} wird mit einer gleichlautenden Partikel \textit{auf=} gebildet und tilgt damit gleichsam die präpositionale Ergänzung.
Das systematische Bild wird getrübt durch Fälle wie (\ref{ex:wb983462c}), in denen Partikel und Präposition zusammen auftreten.
Solche Sätze sind vielleicht stilistisch nicht als herausragend einzustufen, aber alles andere als ungrammatisch.

\begin{exe}
  \ex \label{ex:wb983462}
  \begin{xlist}
  	\ex{\label{ex:wb983462a} Die Trainerin hat alle Ergebnisse [auf ein Blatt Papier] geschrieben.}
  	\ex{\label{ex:wb983462b} Die Trainerin hat alle Ergebnisse aufgeschrieben.}
  	\ex{\label{ex:wb983462c} Die Trainerin hat alle Ergebnisse [auf ein Blatt Papier] aufgeschrieben.}
  \end{xlist}
\end{exe}

Dieser Bereich der Valenzänderungen durch Präfixe und Tilgung von Ergänzungen durch Partikeln ist komplex, und es gibt sowohl Unterschiede zwischen Unterklassen der Präfixe und Partikeln als auch individuelle Unterschiede sowie diverse nicht oder nur eingeschränkt produktive Fälle.
Man kann nicht erwarten, mit den genannten drei Mustern jedem Präfix- oder Partikelverb beizukommen.
Vollständigere Grammatiken wie \citet{Eisenberg1} bieten eine gründlichere Gesamtschau.

\subsection{Derivation mit Wortklassenwechsel}

Wir wenden uns abschließend der Derivation mit Wortklassenwechsel zu.
Zunächst müssen hierzu einige Verbpräfixe gezählt werden, bei denen der Stamm, vor den sie treten, sonst nicht als Verb, aber als Substantiv (\ref{ex:wb012346}) oder Adjektiv (\ref{ex:wb012347}) existiert.
Die auftretenden Präfixe können \idR ebensogut als wortklassenerhaltende Präfixe vor Verben treten.
Alternativ könnte für die Fälle mit Wortklassenwechsel angenommen werden, dass die nominalen Stämme zunächst per Stammkonversion zu Verbstämmen abgeleitet werden und dann das Präfix hinzutritt.
Der Beschreibungsaufwand würde dadurch unnötig erhöht. 

\begin{exe}
  \ex\label{ex:wb012346}
  \begin{xlist}
  	\ex{\label{ex:wb012346a} bebeispielen, bestuhlen, bevölkern}
  	\ex{\label{ex:wb012346b} entvölkern, entgräten, entwanzen}
  	\ex{\label{ex:wb012346c} verholzen, vernageln, verwanzen, verzinnen}
  \end{xlist}
  \ex\label{ex:wb012347}
  \begin{xlist}
  	\ex{\label{ex:wb012347a} ergrauen, ermüden, erneuern}
  	\ex{\label{ex:wb012347b} befreien, beengen, begrünen}
  \end{xlist}
\end{exe}

Die weiteren Fälle sind auf Suffixe und wenige Zirkumfixe beschränkt.
Ein Beispiel mit Verben als Ausgangswort und Substantiven als Zielwort ist \textit{Ge:~:e}.
Zu vielen Verben bildet dieses Zirkumfix ein Substantiv, das eine nicht zielgerichtete Ausführung der Handlung bezeichnet und einen abschätzigen Charakter hat, \zB \textit{Ge:red:e} zum Verb \textit{red-en}.
Die wortklassenändernden Affixe werden oft (ähnlich wie schon bei Konversionsprozessen, vgl.\ Abschnitt~\ref{sec:konvdeutsch}) als \textit{-isierungs}-Suffixe bezeichnet.
Beispielsweise wäre \textit{:haft} ein \textit{Adjektivierungs}-Suffix oder \textit{adjektivierendes} Suffix für substantivische Ausgangswörter.
Nach \citet[267]{Eisenberg1} fassen wir in Tabelle~\ref{tab:derivaffixe} zunächst einige wichtige Derivationsaffixe des Deutschen sowie die Wortklasse ihrer Ausgangswörter (Zeilen) und Zielwörter (Spalten) zusammen.
Die Tabelle deutet durch die relative Anzahl der genannten Affixe an, dass Derivationsaffixe häufig Substantive, seltener Adjektive und noch seltener Verben bilden.
Tabelle~\ref{tab:derivaffixex} zeigt parallel dazu Beispielwörter.

\index{Substantiv}\index{Adjektiv}\index{Verb}
\begin{table}[!htbp]
  \centering
  \begin{tabular}{llll}
    \lsptoprule
    \textbf{Ausgangsklasse} & \textbf{Substantiv-Affix} & \textbf{Adjektiv-Affix} & \textbf{Verb-Affix} \\
   \midrule
   \multirow{4}{*}{\textbf{Substantiv}} & \~:chen & :haft & \\
   & :in & :ig & \\
   & :ler & \~:isch & \\
   & :schaft & \~:lich & \\
   \midrule
   \multirow{3}{*}{\textbf{Adjektiv}} & :heit & \~:lich & \\
    & :keit && \\
    & :igkeit && \\
   \midrule
   \multirow{3}{*}{\textbf{Verb}} & :er & :bar & \~:el \\
   & :erei && \\
   & :ung && \\
   \lspbottomrule
  \end{tabular}
  \caption{Derivationsaffixe nach Ausgangs- und Zielklasse}
  \label{tab:derivaffixe}
\end{table}

\begin{table}[!htbp]
  \centering
  \begin{tabular}{llll}
    \lsptoprule
    \textbf{Ausgangsklasse} & \textbf{Substantiv-Affix} & \textbf{Adjektiv-Affix} & \textbf{Verb-Affix} \\
   \midrule
   \multirow{4}{*}{\textbf{Substantiv}} & Äst:chen & schreck:haft & \\
   & (Arbeit:er):in & fisch:ig & \\
   & (Volk-s:kund):ler & händ:isch & \\
   & Wissen:schaft & häus:lich & \\
   \midrule
   \multirow{3}{*}{\textbf{Adjektiv}} & Schön:heit & röt:lich & \\
    & Heiter:keit && \\
    & Neu:igkeit && \\
   \midrule
   \multirow{3}{*}{\textbf{Verb}} & Arbeit:er & bieg:bar & kreis:el-n \\
   & Arbeit:erei && \\
   & Les:ung && \\
   \lspbottomrule
  \end{tabular}
  \caption{Beispiele für Derivationsaffixe}
  \label{tab:derivaffixex}
\end{table}

Weiter oben (Satz~\ref{satz:reknrek}, S.~\pageref{satz:reknrek}) wurde nun festgestellt, dass Wortbildung im Prinzip rekursiv sei.
Im Falle der Derivation ist dies prinzipiell auch der Fall, allerdings ist die Kombinierbarkeit der Affixe eingeschränkt.
Es sind nur bestimmte Abfolgen möglich, und die möglichen Reihenfolgen der Suffixe sind ebenfalls vergleichsweise festgelegt.
Die Gründe hierfür sind überwiegend semantischer Natur, abgesehen davon, dass natürlich \zB ein einmal zu einem Substantiv abgeleitetes Adjektiv (\textit{Neu:heit}) wie ein substantivisches Ausgangswort fungiert und nicht weiter wie ein Adjektiv abgeleitet werden kann.
Jeweils eine -- weiterhin nach \citealp{Eisenberg1} -- mögliche und eine nicht mögliche Bildung finden sich beispielhaft in (\ref{ex:adjmultisuff}) bis (\ref{ex:substmultisuff}) für verschiedene Wortklassen von Ausgangswörtern.

\begin{exe}
  \ex{\label{ex:adjmultisuff}
  \begin{xlist}
    \ex[]{(Schön:heit):chen}
    \ex[*]{(Schön:heit):haft}
  \end{xlist}}
  \ex{\label{ex:vbmultisuff}
  \begin{xlist}
    \ex[]{(Verzeih:ung):chen}
    \ex[*]{(Verzeih:ung):schaft}
  \end{xlist}}
  \ex{\label{ex:substmultisuff}
  \begin{xlist}
    \ex[]{(Gärtn:er):in}
    \ex[*]{Garten:in}
  \end{xlist}}
\end{exe}

\index{Diminutiv}

Die Darstellung bei Eisenberg suggeriert, dass die Suffigierung des sog.\ Diminutivs (\textit{\~:chen}) an Wörter wie \textit{Schön:heit} (Abstrakta) möglich sei.
Dies klingt zunächst zweifelhaft, und eine Recherche nach Bildungen auf \textit{:heit:chen} oder \textit{:keit:chen} im DeReKo ergibt auch, dass im gesamten Korpus lediglich die Wortformen (\textit{Krank:heit})\textit{:chen} und (\textit{Begeben:heit})\textit{:chen} vorkommen, und dies nur jeweils einmal.%
\footnote{Anfrage \texttt{*heitchen ODER *keitchen} am 03.01.2010 im Archiv W-Öffentlich.}
Man kann daher nicht behaupten, dass diese Bildungen sonderlich produktiv sind.
Allerdings sind sie als strukturelle Möglichkeit auch nicht ganz ausgeschlossen.

Im nächsten Kapitel geht es um die genauen Flexionsmuster bei den flektierbaren Wörtern.
Die ausführliche Diskussion der Flexion ist hier der Wortbildung unter anderem deshalb nachgeordnet, weil es so möglich wird, ggf.\ zu diskutieren, ob bestimmte Bildungen tatsächlich Flexion oder doch eher Wortbildung sind (\zB die Komparation, s.\ Abschnitt~\ref{sec:komparation}).


\Zusammenfassung{
  \textit{Derivation} ist die Bildung neuer Wörter aus existierenden Wörtern unter Anfügung von Affixen, \zB \textit{bläu:lich} aus \textit{blau}.
  Verben mit \textit{Verbpartikeln} und \textit{Verbpräfixen} unterscheiden sich in ihrer Syntax und ihrer Flexion, \zB \textit{übersetzt} und \textit{übergesetzt}.
Bei der Derivation kann sich die Wortart ändern, muss aber nicht, \zB \textit{Leser:schaft} und \textit{leser:lich}.
Wortbildungssuffixe sind nur in bestimmten Abfolgen kombinierbar.
}


\begin{Vertiefung}{Rückbildung und Univerbierung}

  \label{vert:rueckbildunguniverbierung}

\noindent Manchmal werden auch Bildungen wie die in (\ref{ex:wb35486620}) im Rahmen der Wortbildung diskutiert.

\begin{exe}
  \ex\label{ex:wb35486620} 
  \begin{xlist}
    \ex{\label{ex:wb35486620a} Notlandung \Folgt\ notlanden}
    \ex{\label{ex:wb35486620b} Zwangsräumung \Folgt\ zwangsräumen}
    \ex{\label{ex:wb35486620c} sanftmütig \Folgt\ Sanftmut}
  \end{xlist}
\end{exe}

Es handelt sich um sogennante \textit{Rückbildungen}, bei denen ein Ausgangswort um ein Suffix (hier \textit{-ung} und \textit{-ig}) verkürzt wird.
Der verkürzte Stamm dient dann als Basis für ein neues Wortbildungssuffix oder wird als Wortstamm in der entsprechenden Klasse des Zielworts verwendet.
Dieses Phänomen illustriert, wie schwierig es zu sein scheint, in der Wortbildung sauber zwischen produktiven und nicht produktiven Prozessen zu trennen.
In (\ref{ex:wb35486620}) muss in \textit{allen} Fällen entschieden werden, welches Wort historisch zuerst im Sprachgebrauch war, um überhaupt sicherzustellen, dass nicht eigentlich ganz regulär \textit{Notlandung} aus \textit{notlanden} usw.\ entstanden ist.%
\footnote{Bei \textit{sanftmütig} könnte man versuchen, über die Semantik zu argumentieren.
Man würde dabei feststellen, dass \textit{Sanftmut} nicht produktiv auf \textit{Mut} bezogen werden kann, und dass deshalb \textit{Sanftmut} nicht direkt produktiv aus \textit{sanft} und \textit{Mut} gebildet worden sein kann.
Es handelt sich hier systematisch gesehen um irrelevante Einzelfälle.
Damit ist von vornherein ausgeschlossen, dass sie das Ergebnis eines systematischen produktiven Prozesses sind.}
Damit haben wir es bei Rückbildungen mit einem sprachgeschichtlichen nicht mit einem produktiven Prozess zu tun.

Produktiv gesehen gehören \zB nahezu alle Bildungen mit \textit{-ung} entweder transparent zu einem Verbstamm (wie \textit{anfügen} und \textit{Anfügung}) oder sind intransparente lexikalisierte Wörter wie \textit{Brüstung} oder \textit{Zeitung}.
Sprecher bilden in den intransparenten Fällen gerade nicht produktiv Verben wie *\textit{zeiten} oder *\textit{brüsten}.
Es ist nicht einmal klar, was diese Wörter dann bedeuten sollten.
Selbst wenn in Fällen wie \textit{notlanden} zuerst \textit{Landung} aus \textit{landen} deriviert, dann mit \textit{Not} zu \textit{Notlandung} komponiert wurde, um schließlich zu \textit{notlanden} rückgebildet zu werden, bedeutet das für das produktive System der Sprecher nichts.
Wir haben am Ende wieder eine Situation, in der das Verb und die Bildung mit \textit{-ung} existieren, und Sprecher haben keinen offensichtlichen Anlass, hier eine Rückbildung zu vermuten.

In (\ref{ex:wb35486621}) sind \textit{Univerbierungen} bebeispielt.
Dieser hypothetische Wortbildungsprozess bildet aus zwei oft nebeneinander stehenden Wörtern ein neues.

\begin{exe}
  \ex\label{ex:wb35486621} 
  \begin{xlist}
    \ex{\label{ex:wb35486621a} kennen lernen \Folgt\ kennenlernen}
    \ex{\label{ex:wb35486621b} auf Grund \Folgt\ aufgrund}
    \ex{\label{ex:wb35486621c} wild geworden \Folgt\ wildgeworden}
  \end{xlist}
\end{exe}

Auch hier gilt, dass es sich um einen sprachgeschichtlichen Prozess handelt.
Ob Univerbierung stattfindet oder nicht hängt in starkem Ausmaß von der Häufigkeit der Wortverbindung ab.
Größere Häufigkeit führt dann (besonders deutlich \zB bei neu gebildeten Präpositionen wie \textit{aufgrund}) typischerweise zu einem Verblassen der Semantik der einzelnen Wörter.
Im Fällen wie \textit{aufgrund} spricht man von \textit{Grammatikalisierung}, weil das Substantiv \textit{Grund} seine Bedeutung vollständig verliert und sich ein grammatisches Funktionswort bildet.\index{Grammatikalisierung}
Solche Prozesse sind allerdings diachrone Prozesse, und Sprecher können nicht spontan -- also produktiv -- Univerbierungen vornehmen.
Genauso wie die Rückbildungen gehören die Univerbierungen nicht in die Beschreibung des grammatischen Systems.

\end{Vertiefung}



\Uebungen

\Uebung \label{u71} Bestimmen Sie für die folgenden Komposita (a) die vollständige morphologische Struktur einschließlich der Fugenelemente (als Baum oder in der linearen Notation, ggf.\ mit Klammern), (b) den Kopf, (c) den Typus. (d) Welche sind Ihrer Meinung nach produktiv gebildet und welche lexikalisiert? (e) Stellen Sie fest, ob die Ausgangswörter morphologisch komplex sind (\zB deriviert).

\begin{enumerate}\Lf
  \item Wesenszugsanalyse
  \item Einschuböffnung
  \item Esstisch
  \item Räderwerksreparatur
  \item Einschiebeöffnung
  \item Großrechner
  \item Banknotenfälschung
  \item Bergbauwissenschaftsstudium
  \item Anschlagsvereitelung
  \item Bioladen
  \item Kindergarten
  \item Mitbewohner
  \item Absichtserklärungsverlesung
  \item Monatsplanung
  \item feuerrot
  \item Notlaufprogramm
\end{enumerate}

\Uebung \label{u72} Bestimmen Sie für die folgenden Derivations- und Konversionsprodukte (a) die morphologische Struktur (als Baum oder in der linearen Notation), (b) die Wortklassen der Ausgangs- und Zielwörter, (c) den Typus (Derivation, Stamm- oder Wortformenkonversion). (d) Liegt Umlaut vor? (e) Welche sind Ihrer Meinung nach produktiv gebildet und welche lexikalisiert? 

\begin{enumerate}\Lf
  \item verkäuflich
  \item unterwander(n)
  \item alternativlos
  \item (der) Lauf
  \item aufsteig(en)
  \item Gebell
  \item beschließ(en)
  \item begegn(en)
  \item Röhrchen
  \item (das) Schlingern
  \item Geruder
  \item Überzocker
  \item Gebrüder
  \item Mündel
  \item schweigsam
\end{enumerate}

\Uebung[\tristar] \label{u73} Beschreiben Sie folgende Fälle als Wortbildung.
Was könnte ein Problem bezüglich der Struktur des Lexikons im Rahmen des Gesamtsystems der Grammatik sein?

\begin{enumerate}\Lf
  \item (das) Sich-in-die-kosmische-Unendlichkeit-Einfügen
  \item (die) Ethanol-haltige-Gefahrstoff-Kennzeichnung
  \item (eine) Mehr-als-Beliebigkeit
\end{enumerate}

\Uebung[\tristar] \label{u74} Wie sind folgende Fälle gebildet?
Wie passen sie in das System der Wortbildung?

\index{Kurzwort}
\begin{enumerate}\Lf
  \item Lok (Lokomotive)
  \item Fundi (eine Person aus dem fundamentalpolitischen Flügel der Partei Bündnis 90\slash Die Grünen)
  \item Vopo (Volkspolizist)
  \item Kotti (Kottbusser Tor)
  \item Schweini (Schweinsteiger)
  \item Poldi (Podolski)
\end{enumerate}

