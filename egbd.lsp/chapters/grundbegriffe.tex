\chapter{Grundbegriffe der Grammatik}

\label{sec:grundbegriffe}

\section{Merkmale und Werte}

\label{sec:merkmaleundwerte}

Im Folgenden wird der Begriff der \textit{grammatischen Einheit} verwendet.
Zunächst folgt also die sehr durchsichtige Definition dieses Begriffs in Definition~\ref{def:einheit}.

\index{Einheit}

\Definition{Grammatische Einheit}{\label{def:einheit}%
Eine \textit{grammatische Einheit} ist jedes systematisch beobachtbare sprachliche Objekt (\zB Laute, Buchstaben, Wörter, Sätze).
Eine Einheit ist jeweils auf einer der sprachlichen (bzw.\ grammatischen) Ebenen (\zB Phonologie, Morphologie, Syntax) verortet.
}

Laute sind Einheiten der Phonologie, Wortbestandteile und Wörter sind Einheiten der Morphologie.
Wörter sind gleichzeitig neben Gruppen von Wörtern und Sätzen die Einheiten der Syntax.
Sprachliche Einheiten auf allen Ebenen haben \textit{Merkmale} (man könnte auch von \textit{Eigenschaften} sprechen), so wie wir allen Dingen in der Welt Merkmale zusprechen können.
Umgangssprachlich würde man sagen, dass der Eiffelturm das Merkmal \textit{stählern} hat, weil er aus Stahl gebaut ist, oder man würde sagen, dass eine Erdbeere das Merkmal \textit{rot} hat.
Im Grunde wollen wir hier mit sprachlichen Einheiten (wie Lauten oder Wörtern) nicht anders vorgehen als mit dem Eiffelturm oder mit Erdbeeren:
Die Merkmale sprachlicher Einheiten sollen ermittelt werden.
Allerdings werden dabei das Merkmal und sein Wert genau getrennt.

\index{Merkmal}
\index{Wert}

Am Beispiel der roten Erdbeere lässt sich gut zeigen, dass das Merkmal des Rotseins auch anders angegeben werden kann.
Statt zu sagen, die Erdbeere habe das Merkmal \textit{rot}, könnten wir auch sagen, dass die Erdbeere das Merkmal \textit{Farbe} hat, welches den Wert \textit{rot} hat.
Was ist der Vorteil von dieser Trennung?
Würde man sagen, der Eiffelturm habe das Merkmal \textit{325m}?
Wahrscheinlich nicht, denn es könnte sich bei der Angabe \textit{325m} um die Breite oder Tiefe handeln, genausogut über die Höhe seines Sockels über dem Meeresspiegel oder die Gesamtlänge der in ihm verbauten Stahlträger.
Eindeutiger und korrekter wäre es, zu sagen der Eiffelturm hat das Merkmal \textit{Höhe} mit dem Wert \textit{325m}.
Das Trennen von Merkmal und Wert hat aber nicht nur den Vorteil der Eindeutigkeit.
Nicht alle Dinge, die wir wahrnehmen, haben ein Merkmal \textit{Höhe}.
Höhe ist nur ein gültiges Merkmal von physikalisch konkreten Dingen wie Erdbeeren oder Türmen, nicht aber von abstrakteren Dingen wie Ideen, Verträgen oder Gesprächen.
Allein durch die Anwesenheit oder die Abwesenheit eines Merkmals werden also Dinge klassifiziert, unabhängig von den jeweiligen Werten der Merkmale.

\index{Kategorie}

Merkmale, so wie sie oben definiert wurden, helfen also, Dinge zu kategorisieren.
Im grammatischen Bereich finden wir ähnliche Situationen.
Ein Verb (\textit{laufen}, \textit{philosophieren} usw.) hat in keiner seiner Formen ein grammatisches Geschlecht (das sog.\ \textit{Genus}, also \textit{Femininum}, \textit{Maskulinum} oder \textit{Neutrum}).
Substantive wie \textit{Sahne}, \textit{Kuchen} und \textit{Kompott} haben allerdings immer ein spezifisches Genus, wie wir unter anderem an dem wechselnden Artikel sehen können: \textit{die Sahne}, \textit{der Kuchen}, \textit{das Kompott}.
Sobald wir sagen, ein Wort sei ein Nomen oder ein Verb, wissen wir also, dass bestimmte Merkmale bei diesem Wort vorhanden sind, andere aber nicht.
Das wissen wir, auch ohne den konkreten Wert (hier also \textit{feminin}, \textit{maskulin} oder \textit{neutral}) zu kennen.
Daher müssen wir prinzipiell angeben:

\begin{enumerate}\Lf
  \item Welche Merkmale gibt es?
  \item Welche Werte können diese Merkmale haben?
  \item Welche Klassen von Einheiten (\zB Vokale, Konsonanten, Verben, Substantive) haben ein bestimmtes Merkmal?
  \item Was sind die Werte dieser Merkmale bei jeder konkreten Einheit (beim Vokal \textit{a}, beim Konsonant \textit{t}, beim Verb \textit{laufen}, beim Substantiv \textit{Sahne})?
\end{enumerate}


\Definition{Merkmal und Wert}{\label{def:merkmal}%
Ein \textit{Merkmal} ist die Kodierung einer Eigenschaft einer grammatischen Einheit.
Zu jedem Merkmal gibt es eine Menge von \textit{Werten}, die es annehmen kann.
Grammatische Eigenschaften von Einheiten können vollständig durch Mengen von Merkmal-Wert-Paaren beschrieben werden.
\index{Merkmal}
}

Formal schreiben wir Merkmale und Werte folgendermaßen auf:

\begin{exe}
  \ex{\textbf{Merkmalsdefinition}\\
  \textsc{Merkmal}: \textit{wert}, \textit{wert},\ldots}
  \ex{\textbf{Merkmal-Wert-Kodierung}\\
  Einheit = [\textsc{Merkmal}: \textit{wert}, \textsc{Merkmal}: \textit{wert}, \ldots ]}
\end{exe}

Beispiele hierfür wären:

\begin{exe}
  \ex{\textsc{Genus}: \textit{feminin}, \textit{maskulin}, \textit{neutral}}
  \ex
  \begin{xlist}
    \ex{Sahne = [\textsc{Genus}: \textit{feminin}, \ldots]}
    \ex{Kuchen = [\textsc{Genus}: \textit{maskulin}, \ldots]}
    \ex{Kompott = [\textsc{Genus}: \textit{neutral}, \ldots]}
  \end{xlist}
\end{exe}

Das Gleichheitszeichen zwischen der Einheit und der gegebenen Merkmalsmenge steht für eine sehr starke Annahme.
Die grammatischen Merkmale, die wir den Einheiten zuweisen, sind alles, was uns an der Einheit interessiert, da wir uns hier nur mit Grammatik beschäftigen.
Die Angabe der Merkmalsmenge ist also die vollständige Definition der sprachlichen Einheit aus Sicht der Grammatik.

\Zusammenfassung{
Grammatische Einheiten (Laute, Wörter, Sätze usw.) können -- genau wie andere Betrachtungsgegenstände -- über Merkmale und ihre Werte beschrieben werden.
Die Unterscheidung von Merkmalen (wie Kasus oder Numerus) und ihren Werten erlaubt es, sprachliche Einheiten allein schon durch das Vorhandensein bzw.\ die Abwesenheit von Merkmalen in Klassen einzuteilen.
}

\section{Relationen}

\label{sec:relationengrundbegriffe}

In diesem Abschnitt werden weitere Begriffe eingeführt, die ähnlich zentral für die Grammatik sind wie der Begriff des Merkmals.
Alle diese Begriffe haben mit \textit{Relationen} (Beziehungen) zwischen sprachlichen Einheiten zu tun, allerdings auf verschiedene Art und Weise und auf verschiedenen Ebenen.

\subsection{Kategorien}

\label{sec:kategorien}

\index{Kategorie}

Schon im Zusammenhang mit Merkmalen haben wir von \textit{Kategorien} gesprochen.
Am Beispiel der \textit{lexikalischen Kategorien} wird jetzt genauer definiert, was Kategorien sind.
Es wird erst der Begriff des \textit{Lexikons} definiert, wobei wir auf den Merkmalsbegriff zurückgreifen.

\Definition{Lexikon}{\label{def:lexikon1}%
Das \textit{Lexikon} ist die Menge aller Wörter einer Sprache, definiert durch die vollständige Angabe ihrer Merkmale und deren Werte.
\index{Lexikon}
}

Die Definition hat einen vorläufigen Charakter vor allem deshalb, weil wir noch nicht definiert haben, was überhaupt ein Wort ist, aber die Definition des Lexikons darauf zurückgreift (vgl.\ Kapitel~\ref{sec:wortklassen}, Definition~\ref{def:wort}, S.~\pageref{def:wort}).
Man kann aber vorerst einen nicht definierten, intuitiven Wortbegriff zugrundelegen.%
\footnote{Wie schon beim Begriff der Grammatik (Abschnitt~\ref{sec:grammatikbegriff}) ist hier mit Lexikon also nicht ein Nachschlagewerk in Buchform gemeint.}

\index{Wort}

Den Begriff der \textit{Kategorie} kann man nun anhand der lexikalischen Kategorie gut einführen.
Wir haben bereits festgestellt, dass Wörter Gruppen bilden, je nachdem, welche Merkmale sie haben oder nicht haben.
Das heißt aber gleichzeitig, dass das Lexikon eigentlich nicht bloß eine ungeordnete Menge von Wörtern ist.
Die Elemente (die Wörter) in dieser Menge (dem Lexikon) sind allein dadurch geordnet, dass sie in unterschiedlichem Maße identische Merkmale und Werte für diese Merkmale haben.
Beispielhaft wurde gesagt, dass sich Verben und Substantive durch die Abwesenheit bzw.\ Anwesenheit des Merkmals \textsc{Genus} unterscheiden.\index{Genus}
Wir können also das Lexikon zumindest in Kategorien von Wörtern aufteilen, die \textsc{Genus} haben oder nicht (vgl.\ Abbildung~\ref{fig:lexkat01}).

\begin{figure}[!htbp]
  \centering
  \Tree[1]{
  & \K{Wort}\B{dl}\B{dr}\\
  \K{hat \textsc{Genus}} && \K{hat kein \textsc{Genus}}
  }
\caption{Vorschlag lexikalischer Kategorien}
\label{fig:lexkat01}
\end{figure}

Eventuell sind dies nicht die einzigen Kategorien, in die man das Lexikon aufteilen sollte.
Wenn wir Beispielwörter hinzufügen, wird dies wahrscheinlich sofort deutlich (vgl.\ Abbildung~\ref{fig:lexkat1woerter}).

\begin{figure}[!htbp]
  \centering
  \Tree[1]{
    &&& \K{Wort}\B{dll}\B{drr}\\
    & \K{hat \textsc{Genus}}\B{dl}\B{d}\B{dr} &&&& \K{hat kein \textsc{Genus}}\B{dll}\B{dl}\B{d}\B{dr}\B{drr} \\
    \K{\textit{Birne}} & \K{\textit{Apfel}} & \K{\textit{Kompott}} & \K{\textit{laufen}} & \K{\textit{essen}} & \K{\textit{bald}} & \K{\textit{unter}} & \K{\textit{dass}}
  }
  \caption{Einige Wörter in lexikalischen Kategorien}
  \label{fig:lexkat1woerter}
\end{figure}

\index{Wort!flektierbar}

Bei den Wörtern ohne \textsc{Genus} finden wir nicht nur Verben wie \textit{laufen}, sondern auch Adverben wie \textit{bald}, Präpositionen wie \textit{unter} oder Komplementierer wie \textit{dass}.
Es wird sich in Kapitel~\ref{sec:wortklassen} als günstiger erweisen, zuerst nach dem Vorhandensein des Merkmals \textsc{Numerus} zu kategorisieren.\index{Numerus}
Substantive und Verben (ggf.\ auch Adjektive) haben alle ein solches Merkmal, bilden also einen \textit{Singular} (Einzahl) und \textit{Plural} (Mehrzahl): \textit{Mann}, \textit{Männer} bzw.\ \textit{laufe}, \textit{laufen}.
Wörter wie \textit{bald}, \textit{unter} oder \textit{dass} haben dieses Merkmal nicht.
Man könnte also die Kategorisierung revidieren und den Baum in Abbildung~\ref{fig:lexkat2} als Analyse vorschlagen, der natürlich auch noch nicht die endgültige Fassung darstellt (vgl.\ Kapitel~\ref{sec:phrasen} und \ref{sec:saetze}).
Die Unterscheidung in Wörter mit und ohne \textsc{Numerus} entspricht der traditionellen Unterscheidung zwischen \textit{flektierbaren} (formveränderlichen oder auch \textit{beugbaren}) und \textit{nicht flektierbaren} Wörtern.
\index{Wort!flektierbar}

\begin{figure}[!htbp]
  \centering
  \Tree{
    &&&&&& \K{Wort}\B{dlll}\B{drrr}\\
    &&& \K{hat \textsc{Numerus}}\B{dll}\B{drr} &&&&&& \K{hat kein \textsc{Numerus}}\B{ddl}\B{dd}\B{ddr}\\
    & \K{hat \textsc{Genus}}\B{dl}\B{d}\B{dr} &&&& \K{hat kein \textsc{Genus}}\B{dl}\B{dr}\\
    \K{\textit{Birne}} & \K{\textit{Apfel}} & \K{\ \ \textit{Kompott}} && \K{\textit{laufen}} && \K{\textit{essen}} && \K{\textit{bald}} & \K{\textit{unter}} & \K{\textit{dass}}
  }
  \caption{Genauerer Vorschlag lexikalischer Kategorien}
  \label{fig:lexkat2}
\end{figure}


\index{Wortklasse}

Wenn man das Lexikon genauer auf diese Weise untersucht, ergibt sich eine vollständige Hierarchie, die die traditionell als \textit{Wortarten} oder \textit{Wortklassen} bezeichneten Kategorien abbildet.
Allerdings wird die Unterscheidung wesentlich feiner als die traditionellen Wortarten, denn jeder Unterschied in der Merkmalsausstattung erzeugt neue Unterkategorien.
Im Kapitel zu den Wortklassen (Kapitel~\ref{sec:wortklassen}) legen wir daher einen absichtlich groben Maßstab an, um die traditionellen Wortarten als ungefähre Orientierungshilfe in der Struktur des Lexikons zu rekonstruieren.
Die Definition der Kategorie ist jetzt relativ leicht zu geben.

\Definition{Kategorie}{\label{def:kategorie}%
Eine \textit{Kategorie} ist eine Menge sprachlicher Einheiten, die alle ein bestimmtes Merkmal haben oder bei denen der Wert eines bestimmten Merkmals gleich gesetzt ist.
\index{Kategorie}
}

Kategorisierungen anhand von Merkmalsausstattungen werden konkret gleich im nächsten Kapitel (Phonetik) eingeführt, zum Beispiel wenn Vokale und Konsonanten unterschieden werden.
Aber auch syntaktische Einheiten wie die sogenannten \textit{Satzglieder} (\zB Objekte oder adverbiale Bestimmungen) können so definiert werden, dass sich die wesentlichen Unterschiede im grammatischen Verhalten (und damit die Wortart) aus ihren Merkmalsausstattungen ergeben.
Kategorien sind Einordnungen von Einheiten in bestimmte Gruppen.
Die Relationen, die durch die lexikalischen Kategorien definiert werden, bestehen zwischen Wörtern im Lexikon.
Zwischen je zwei Wörtern besteht also die Relation \textit{Ist-in-derselben-Klasse}, oder die Relation \textit{Ist-nicht-in-derselben-Klasse}.
Im nächsten Abschnitt geht es um eine ganz andere Art der Relation zwischen Einheiten, nämlich um ihr konkretes Vorkommen in größeren Zusammenhängen oder Strukturen.

\subsection{Paradigma und Syntagma}

\label{sec:paradigmasyntagma}

Der Begriff des \textit{Paradigmas} hat viel mit unserer Definition der Kategorie zu tun.
Es folgt zunächst eine Zusammenstellung von Formen.


\begin{exe}
  \ex\label{ex:gb8991}
  \begin{xlist}
    \ex{(die) Tochter}
    \ex{(die) Töchter}
  \end{xlist}
  \ex\label{ex:gb8992}
  \begin{xlist}
    \ex{(der) Saum}
    \ex{(die) Säume}
  \end{xlist}
  \ex\label{ex:gb8993}
  \begin{xlist}
    \ex{(der) Mensch}
    \ex{(die) Menschen}
  \end{xlist}
  \ex\label{ex:gb8994}
  \begin{xlist}
    \ex{(sie) läuft}
    \ex{(sie) lief}
  \end{xlist}
  \ex\label{ex:gb8995}
  \begin{xlist}
    \ex{(sie) kauft}
    \ex{(sie) kaufte}
  \end{xlist}
\end{exe}

Es handelt sich bei (\ref{ex:gb8991})--(\ref{ex:gb8993}) um den Singular und den Plural der jeweiligen Wörter.
Im Falle von \textit{laufen} und \textit{kaufen} wurden die Formen der dritten Person des Singulars im Präsens und Präteritum angegeben.
Die Formen (\textit{die}) \textit{Tochter} und (\textit{die}) \textit{Töchter}, die Formen (\textit{der}) \textit{Saum} und (\textit{die}) \textit{Säume} sowie die Formen (\textit{der}) \textit{Mensch} und (\textit{die}) \textit{Menschen} stehen offensichtlich in einer besonderen Beziehung, und diese Beziehung ist systematisch an den Formen aller Substantive beobachtbar, auch wenn die Art der Formenbildung jeweils stark unterschiedlich ist bzw.\ manchmal gar kein Unterschied in der Form auftritt.
Vereinfacht könnte man auch sagen, dass alle Substantive einen (Nominativ) Singular und Plural haben.%
\footnote{Es gibt Ausnahmen, die aber vielmehr in der Bedeutung begründet sind als in der Grammatik.
Dies sind \zB Substantive, die nur im Plural auftreten, wie \textit{Ferien}.}
Genauso könnte man aus den Kasusformen der Substantive eine entsprechende Reihe für jedes Substantiv bilden.
Ähnliches gilt für die Verben.
Offensichtlich bilden \textit{laufen} und \textit{kaufen} ihre Formen unterschiedlich, auch wenn sie im Infinitiv sehr ähnlich aussehen.
Jedes Verb hat aber trotz dieser Unterschiede Formen für Präsens und Präteritum, und zwischen diesen Formen besteht jeweils dieselbe Beziehung, nämlich die des Tempusunterschieds (Unterschied der Zeitform).
Die damit demonstrierten Beziehungen sind \textit{paradigmatisch}.


\Definition{Paradigma}{\label{def:paradigma}%
Ein \textit{Paradigma} ist eine Reihe von Formen, in der die Einheiten einer bestimmten Kategorie (\zB einer Wortart) auftreten.
Die Formen sind dadurch verbunden, dass bei allen bestimmte Merkmale und Werte identisch sind (\zB Genus bei den Formen eines bestimmten Substantivs).
An jeder Position der Reihe muss aber eine Änderung eines Werts eines Merkmals auftreten (\zB Numerus, Tempus), die durch eine Formänderung angezeigt werden kann.
\index{Paradigma}
}

Diese Definition versteht das Paradigma als das Formenraster, in das sich bestimmte Einheiten einreihen.
Wir sprechen von \textit{Einheiten} statt von \textit{Wörtern}, weil nicht nur einzelne Wörter, sondern auch kleinere oder größere Einheiten prinzipiell Paradigmen bilden, auch wenn das morphologische Paradigma (die Formen eines Wortes) den prototypischen Fall eines Paradigmas darstellen.
Ein Beispiel für eine paradigmatische Beziehung zwischen größeren Einheiten wird in (\ref{ex:gb8282}) illustriert.

\begin{exe}
  \ex\label{ex:gb8282}
  \begin{xlist}
    \ex{\label{ex:gb8282a} Die Experten glauben, dass sie den Koffer wiedererkennen.}
    \ex{\label{ex:gb8282b} Die Experten glauben, den Koffer wiederzuerkennen.}
  \end{xlist}
\end{exe}

In (\ref{ex:gb8282a}) liegt ein Nebensatz mit \textit{dass} vor (vgl.\ Abschnitt~\ref{sec:komplementsaetze}), in (\ref{ex:gb8282b}) eine Infinitivkonstruktion (vgl.\ Abschnitt~\ref{sec:kontrollinfinitive}).\index{Nebensatz}\index{Infinitiv}
Beides sind nebensatzartige Strukturen, die hier auch genau dieselbe Position in einer größeren Struktur einnehmen.
Sie unterscheiden sich allerdings in ihrer Form und auch in weiteren Merkmalen, wie \zB dem Vorhandensein (\ref{ex:gb8282a}) oder Nichtvorhandensein (\ref{ex:gb8282b}) eines Subjektes bzw.\ einer Nominalphrase im Nominativ.

Das Beispiel (\ref{ex:gb8282}) leitet damit auch über zum Begriff des \textit{Syntagmas}, der den des Paradigmas ergänzt.
In (\ref{ex:gb8282}) nehmen nämlich, wie schon gesagt, die Formen des Paradigmas (\textit{dass}-Satz und \textit{zu}-Infinitiv) dieselbe Position in einer größeren Struktur ein.
Dies ist nicht immer so, wie (\ref{ex:gb8283}) zeigt.

\begin{exe}
  \ex\label{ex:gb8283} 
  \begin{xlist}
    \ex[]{\label{ex:gb8283a} Die Experten vermuten, dass sie ein schlechter Scherz sind.}
    \ex[*]{\label{ex:gb8283b} Die Experten vermuten, ein schlechter Scherz zu sein.}
  \end{xlist}
\end{exe}

Wenn als Verb im Hauptsatz \textit{vermuten} statt \textit{glauben} steht, kann der \textit{zu}-Infinitiv nicht stehen.
Es gibt also Kontexte, in denen Formen eines Paradigmas eingesetzt werden können, und Kontexte, in denen dies nicht geht.
Auch für die Formen des Singulars und Plurals kann man das zeigen.

\begin{exe}
  \ex\label{ex:gb8284}
  \begin{xlist}
    \ex[]{\label{ex:gb8284a} Ihre Tochter spielt heute in der A-Mannschaft.}
    \ex[*]{\label{ex:gb8284b} Ihre Töchter spielt heute in der A-Mannschaft.}
  \end{xlist}
\end{exe}

Der Singular in (\ref{ex:gb8284a}) ist völlig unauffällig, der Plural in derselben Umgebung in (\ref{ex:gb8284b}) führt zu einem ungrammatischen Satz.
Die Umgebungen, in denen Formen vorkommen, bestimmen also in der Regel, welche Form das Wort haben muss, also welche Form aus dem Paradigma gefordert wird.
Diese Umgebungen oder Kontexte machen die syntagmatische Ebene aus.

\Definition{Syntagma}{\label{def:syntagma}%
Das \textit{Syntagma} ist der aus anderen Einheiten bestehende Kontext, in dem eine bestimmte Einheit steht.
In bestimmten Positionen des Syntagmas werden dabei typischerweise spezifische Formen aus dem Paradigma der betreffenden Einheit gefordert.
\index{Syntagma}
}

Der Begriff des Syntagmas wird im Abschnitt zu den syntaktischen Relationen (Abschnitt~\ref{sec:rektionkongruenz}) wiederkehren, und er ist der zentrale theoretische Begriff, mittels dessen man die Bildung größerer Strukturen aus kleineren Einheiten verstehen kann.
Auch wenn nicht permanent auf die Begriffe des Paradigmas und Syntagmas zurückverwiesen wird, sind dies dennoch die zentralen Begriffe der Grammatik.

\index{Merkmal}

In der Definition des Paradigmas kommt der Begriff der Kategorie bereits vor, womit die Beziehung von Kategorie und Paradigma prinzipiell geklärt ist.
Sehr instruktiv ist es aber auch, der Erwähnung des Merkmalsbegriffs (Definition~\ref{def:merkmal}) in der Definition des Paradigmas nachzugehen.
Zunächst ergibt sich, dass ein Paradigma, wenn es spezifisch für Einheiten einer bestimmten Kategorie ist, automatisch auch verlangt, dass diese Einheiten bestimmte Merkmale gemein haben.
Dies ergibt sich vor allem deshalb, weil unsere Definition der Kategorie genau solche Übereinstimmung von Merkmalen voraussetzt.
Das heißt konkret, dass \zB \textit{Baum}, \textit{Wolke} und \textit{Gerät} eben genau deshalb zur Kategorie der Substantive (und nicht zu der der Verben, Adverben usw.) gehören, weil sie Merkmale wie \textsc{Genus}, \textsc{Kasus}, \textsc{Numerus} usw. haben.
Genau diese Merkmale ermöglichen es aber diesen Wörtern damit erst, im Paradigma der Kasus-Numerus-Formen eines Substantivs zu stehen.

Sehen wir uns die (natürlich nicht vollständigen) Merkmalsmengen der Wörter \textit{Gerät} und \textit{Gerätes} in (\ref{ex:gb8734}) an.

\begin{exe}
  \ex\label{ex:gb8734}
  \begin{xlist}
    \ex{Gerät = [\textsc{Genus}: \textit{neutral}, \textsc{Numerus}: \textit{singular}, \textsc{Kasus}: \textit{nominativ}, \ldots]}
    \ex{Gerätes = [\textsc{Genus}: \textit{neutral}, \textsc{Numerus}: \textit{singular}, \textsc{Kasus}: \textit{genitiv}, \ldots]}
  \end{xlist}
\end{exe}

Das Vorhandensein von Merkmalen wie \textsc{Genus} und \textsc{Numerus} weist die Wörter als Einheiten der Kategorie Substantiv aus.\index{Substantiv}
Das Kasus-Paradigma hat zusätzlich den Effekt, dass in den verschiedenen Formen bezüglich mindestens eines Merkmals (nämlich \textsc{Kasus}) jeweils ein anderer Wert gesetzt wird.
Neben der möglichen aber nicht notwendigen Veränderung einer Form ist also vor allem die spezifische Setzung eines Werts für bestimmte Merkmale in den einzelnen Formen eines Paradigmas typisch.
Weil es sogar relativ oft vorkommt, dass Formen in einem Paradigma äußerlich übereinstimmen (vgl.\ S.~\pageref{def:synkretismus}), ist es wesentlich günstiger, sich auf die verschiedenen Werte der Merkmale als auf die Form zu beziehen.
Man könnte also das Kasus-Paradigma von Substantiven auch wie in (\ref{ex:gb7127}) charakterisieren.

\index{Paradigma!Genus--}
\index{Paradigma!Numerus--}

\begin{exe}
  \ex{\label{ex:gb7127} \textbf{Kasus-Paradigma der Substantive}}
  \begin{xlist}
    \ex{[\textsc{Genus}, \textsc{Numerus}, \textsc{Kasus}: \textit{nominativ}]}
    \ex{[\textsc{Genus}, \textsc{Numerus}, \textsc{Kasus}: \textit{akkusativ}]}
    \ex{[\textsc{Genus}, \textsc{Numerus}, \textsc{Kasus}: \textit{genitiv}]}
    \ex{[\textsc{Genus}, \textsc{Numerus}, \textsc{Kasus}: \textit{dativ}]}
  \end{xlist}
\end{exe}

\textsc{Genus} und \textsc{Numerus} müssen zwar vorhanden sein, aber es werden keine bestimmten Werte verlangt.
Das Merkmal, das sich im Paradigma systematisch ändert, ist \textsc{Kasus}.

\begin{Vertiefung}{Was sind Merkmale?}
  \label{vert:merkmale}

\noindent Oben wurde von Merkmalen wie \textsc{Genus} oder \textsc{Kasus} gesprochen, als wären sie quasi natürlich gegeben.
Eine solche Auffassung von Merkmalen ist allerdings nicht angemessen.
Während Numerus noch ein einigermaßen motiviert erscheinendes Merkmal ist, weil mit ihm eine semantische Kategorie (die Anzahl der bezeichneten Objekte) zumindest in vielen Fällen korreliert, ist ein Merkmal wie \textsc{Kasus} rein strukturell.
Es gibt keinen präzise benennbaren Bedeutungsunterschied zwischen Nominativ und Akkusativ.
Während in dem einen Syntagma der Nominativ stehen muss, muss in einem anderen der Akkusativ stehen.

In den Kapiteln~\ref{sec:nominalflexion} und \ref{sec:verben} wird jeweils ausführlicher auf Motiviertheit oder Unmotiviertheit grammatischer Merkmale eingegangen.
Innerhalb der Grammatik spielen sie aber nur eine Rolle, um den Aufbau von Strukturen zu regeln und dürfen niemals mit Elementen der Bedeutung verwechselt werden.
Natürlich sind insofern auch die Namen der Merkmale und ihrer Werte beliebig und nicht zwangsläufig.
Wir verwenden hier im Normalfall die üblichen Namen aus Gründen der Textverständlichkeit.

\index{Merkmal!Motivation}

\end{Vertiefung}

Schließlich wird noch darauf hingewiesen, dass verschiedene Formen in einem Paradigma die gleiche Lautgestalt haben können, \zB \textit{Birne} in allen Kasus im Singular.
Wenn wir aber davon ausgehen, dass sprachliche Einheiten durch Mengen von Merkmalen und Werten definiert werden, dann sind die Wörter \textit{Birne} und \textit{Birne} in (\ref{ex:gb2347289}) trotz der Gleichheit ihrer Form nicht identisch.
(Es werden beispielhaft nur der Nominativ und der Akkusativ gezeigt.)
Nehmen wir die Lautfolge, die das Wort ausmacht, als Merkmal hinzu (hier der Einfachheit halber die Standardorthographie), wird das sofort deutlich.
In diesen Fällen spricht man von \textit{Synkretismus}, vgl.\ Definition~\ref{def:synkretismus}.

\begin{exe}
  \ex\label{ex:gb2347289}
  \begin{xlist}
    \ex{Birne = [\textsc{Laute}: \textit{birne}, \textsc{Genus}: \textit{feminin},\\ \textsc{Numerus}: \textit{singular}, \textsc{Kasus}: \textit{nominativ}]}
    \ex{Birne = [\textsc{Laute}: \textit{birne}, \textsc{Genus}: \textit{feminin},\\ \textsc{Numerus}: \textit{singular}, \textsc{Kasus}: \textit{akkusativ}]}
  \end{xlist}
\end{exe}

\Definition{Synkretismus}{\label{def:synkretismus}%
Sind zwei oder mehr Formen in einem Paradigma formal (lautlich) identisch, haben aber nicht die gleichen Werte ihrer Merkmale, spricht man von \textit{Synkretismus}.
\index{Synkretismus}
}

Über das Syntagma, also ganz allgemein den strukturellen Kontext einer Einheit, wurde noch nicht sehr viel gesagt.
In allen Teilgebieten der Grammatik ist aber der Aufbau größerer Strukturen aus kleineren Einheiten eins der wichtigsten Phänomene.
Daher widmet sich Abschnitt~\ref{sec:strukturbildung} den Grundlagen grammatischer Strukturbildung.

\subsection{Strukturbildung}

\label{sec:strukturbildung}

Die Einteilung der Ebenen des Sprachsystems und damit der Grammatik (vgl.\ Abschnitt~\ref{sec:ebenendergrammatik}) deutet darauf hin, dass es sich in der Grammatik als sinnvoll erwiesen hat, sprachliche Strukturen als zusammengesetzt aus jeweils kleineren Strukturen anzusehen.
Sätze bestehen aus Satzteilen (Syntax), Satzteile aus Wörtern, Wörter aus Wortbestandteilen (Morphologie) und Wortbestandteile aus Lauten (Phonetik\slash Phonologie).
Die Analyse eines Satzes kann also so vonstatten gehen, dass wir ihn in immer kleinere Teile aufteilen und uns dabei von oben nach unten durch die Ebenen arbeiten.
Ein informell analysiertes Beispiel ist (\ref{ex:gb58365}), wo jeweils Einheiten einer Ebene in eckige Klammern gesetzt sind.

\begin{exe}
  \ex\label{ex:gb58365}
  \begin{xlist}
    \ex \textbf{Satz} \\
    {[Alexandra schießt den Ball ins gegnerische Tor.]}
    \ex \textbf{Satzteile} \\
    {[Alexandra] [schießt] [den Ball] [ins gegnerische Tor]}
    \ex \textbf{Wörter} \\
    {[Alexandra] [schießt] [den] [Ball] [ins] [gegnerische] [Tor]}
    \ex \textbf{Wortteile} \\
    {[Alexandra] [schieß][t] [den] [Ball] [ins] [gegner][isch][e] [Tor]}
    \ex \textbf{Laute} \\
    {[A][l][e][k][s][a][n][d][r][a] \ldots \\}
  \end{xlist}
\end{exe}

In den Kapiteln zur Syntax (vor allem in Kapitel~\ref{sec:phrasen}) wird im Einzelnen argumentiert, warum die Einteilung der Satzteile in dieser konkreten Form sinnvoll ist.
Auch wenn man sich im Einzelfall vielleicht um die genaue Analyse streiten kann, so wird doch klar, dass sprachliche Struktur dadurch zustande kommt, dass Einheiten zu größeren Einheiten zusammengesetzt werden.


\Definition{Struktur}{\label{def:struktur}%
Eine \textit{Struktur} ist definiert durch (1) die kleineren Einheiten, die als ihre Bestandteile fungieren, und (2) die Reihenfolge, in der diese Bestandteile zusammengesetzt sind.
Durch Wiederholung von einfachen strukturbildenden Prozessen ergibt sich eine hierarchische Makrostruktur.
Auf jeder linguistischen Ebene werden durch spezifische Regeln charakteristische Strukturen aufgebaut.
\index{Struktur}
}

Oft stellt man Strukturbildung mit Hilfe sogenannter \textit{Baumdiagramme} dar, wobei die oben gegebenen Kategorienbäume (\zB Abbildung~\ref{fig:lexkat2} auf S.~\pageref{fig:lexkat2}) natürlich grundsätzlich davon verschieden sind.\index{Baumdiagramm}
Die Kategorienbäume bilden die Einordnung von Einheiten in bestimmte Klassen (Kategorien) ab.
Strukturbäume zeigen, wie Einheiten zu konkreten hierarchischen Strukturen zusammengefügt sind.
Einige Ausschnitte von Bäumen zu Beispiel (\ref{ex:gb58365}) finden sich in den Abbildungen~\ref{fig:gb5836a} und \ref{fig:gb5836b}.

\begin{figure}[!htbp]
  \centering
  \Tree[1]{
    &&& \K{Alexandra schießt den Ball ins gegnerische Tor}\B{dlll}\B{dll}\B{d}\B{drrr}\\
    \K{Alexandra}\B{d} & \K{schießt}\B{d} && \K{den Ball}\B{dl}\B{dr} &&& \K{ins gegnerische Tor}\B{dl}\B{d}\B{dr}\\
    \K{Alexandra} & \K{schießt} & \K{den} && \K{Ball} & \K{ins} & \K{gegnerische} & \K{Tor}
  }
  \caption{Strukturen auf Satzebene (Syntax)}
  \label{fig:gb5836a}
\end{figure}

\begin{figure}[!htbp]
  \centering
  \Tree{
    && \K{gegnerische}\B{dl}\B{dr} \\
    & \K{gegnerisch}\B{dl}\B{dr} && \K{e} \\
    \K{gegner} && \K{isch} \\
  }
  \caption{Strukturen auf Wortebene (Morphologie)}
  \label{fig:gb5836b}
\end{figure}

Gerade an dem Baum in \ref{fig:gb5836a}, der eine syntaktische Struktur wiedergibt, zeigt sich der hierarchische Aufbau sehr gut.
Die einzelnen Einheiten in einer Struktur werden \textit{Konstituenten} (Bauteile) der gesamten Struktur genannt.

\Definition{Konstituente}{\label{def:konstituente}%
\textit{Konstituenten} einer Einheit sind die (meistens kleineren und maximal genauso großen) Einheiten, aus denen eine Struktur besteht.
\index{Konstituente}
}

Man würde sagen, \textit{gegner} und \textit{isch} sind Konstituenten von \textit{gegnerisch}.
Innerhalb einer Struktur haben Konstituenten dadurch auch eine Beziehung zu anderen Konstituenten, sie sind \textit{Ko-Konstituenten}.
Im Falle der Struktur aus Wortteilen wären \textit{gegner}, \textit{isch} und \textit{e} also zueinander Ko-Konstituenten in der Struktur \textit{gegnerische}.

\index{Konstituente!mittelbar}
\index{Konstituente!unmittelbar}

Eine wichtige Unterscheidung ist die zwischen \textit{unmittelbarer Konstituente} und \textit{mittelbarer Konstituente}.
Wenn wir uns das Diagramm in Abbildung~\ref{fig:gb5836a} ansehen, dann ist \textit{den} durchaus eine Konstituente des gesamten Satzes.
Allerdings ist es nur auf eine mittelbare Weise eine Konstituente des Satzes, denn es ist zunächst eine Konstituente der Gruppe \textit{den Ball}.
Man kann also sagen, dass \textit{den} und \textit{Ball} zwar unmittelbare Konstituenten von \textit{den Ball} sind, dass sie aber nur mittelbare Konstituenten von \textit{Alexandra schießt den Ball ins gegnerische Tor} sind.
Neben den besprochenen strukturbezogenen Eigenschaften gibt es weitere Arten von Beziehungen zwischen Einheiten in Strukturen, um die es im Folgenden gehen wird.

\subsection{Rektion und Kongruenz}

\label{sec:rektionkongruenz}

Was wir in Abschnitt~\ref{sec:strukturbildung} zur Struktur gesagt haben, ist relativ eindimensional.
Es besagt nur, dass Einheiten zu größeren Einheiten zusammengesetzt werden, und dass sich daraus schließlich eine lineare Anordnung (eine Aneinanderreihung) von Einheiten ergibt.
Viele erfolgreiche Erklärungsansätze in der Grammatik leben aber davon, dass man die Konstituenten von Strukturen als in Beziehungen stehend analysiert.
Diese Relationen werden \textit{syntaktische Relationen} genannt.\index{Relation!syntaktisch}
Man kann diese Relationen auch \textit{syntagmatische Relationen} nennen, weil sie auf allen Ebenen der Strukturbildung (nicht nur in der Syntax) angenommen werden.
Da sie aber in der Syntax die wichtigste Rolle spielen, sprechen wir hier nur von syntaktischen Relationen im engeren Sinn.

\Definition{Syntaktische Relation}{\label{def:relation}%
Eine \textit{syntaktische Relation} besteht zwischen zwei Einheiten in einer Struktur.
Sie stellt eine Beziehung zwischen diesen Einheiten (bzw.\ den Werten ihrer Merkmale) dar, die sich nicht allein aus der Struktur (also der linearen bzw.\ hierarchischen Anordnung der Konstituenten) ergibt.
\index{Relation}
}

Es folgen einige Beispiele, um Definition~\ref{def:relation} mit Leben zu füllen.
Einige Relationen, die allgemein geläufig sind, sind \textit{Subjekt}, \textit{Objekt} oder \textit{adverbiale Bestimmung}.
Die Sätze in (\ref{ex:gb7304}) illustrieren diese Relationen.

\begin{exe}
  \ex\label{ex:gb7304}
  \begin{xlist}
    \ex{\label{ex:gb-7304a}[Dzsenifer] [schießt] [ein Tor].}
    \ex{\label{ex:gb-7304b}[Kim] [läuft] [schnell].}
  \end{xlist}
\end{exe}

Im ersten Satz besteht die Relation \textit{Objekt} zwischen \textit{ein Tor} und \textit{schießt}.
Traditionell gesprochen ist in (\ref{ex:gb-7304a}) \textit{ein Tor} das Objekt von \textit{schießt}, aber da es kein Objekt ohne ein Verb gibt, ist der Begriff \textit{Objekt} an sich relational.
Im zweiten Satz liegt die Relation \textit{adverbiale Bestimmung} zwischen \textit{schnell} und \textit{läuft} vor.
In beiden Sätzen ist die Struktur im Stile von (\ref{ex:gb58365}) mit eckigen Klammern markiert.
Man sieht, dass die genannten Beziehungen aus der linearen Struktur alleine nicht ablesbar sind.
Im einen Fall steht nach dem Verb das Objekt, im anderen die adverbiale Bestimmung.
Dass wir Objekte und adverbiale Bestimmungen in Relation zum Verb unterschiedlich klassifizieren können, muss eine Ursache jenseits der reinen Struktur haben.
Obwohl unterschiedliche Relationen nicht unbedingt unterschiedliche Strukturen voraussetzen, können die Relationen durchaus eine zentrale Rolle beim Aufbau der Strukturen spielen.
Zwei wichtige syntaktische Relationen, an denen dies deutlich wird, werden jetzt diskutiert.

Dass syntaktische Relationen unbedingt als Teil der Grammatik betrachtet werden müssen, wird deutlich, wenn aus ihnen konkrete Anforderungen an Formen und Merkmale von Einheiten resultieren.
\textit{Rektion} und \textit{Kongruenz} verlangen genau dies.
Der letzte Satz von Definition~\ref{def:rektion} ist vielleicht etwas verwirrend, wird aber im Zusammenhang mit der Definition von Kongruenz (Definition~\ref{def:kongruenz} auf S.~\pageref{def:kongruenz}) klarer.

\Definition{Rektion}{\label{def:rektion}%
In einer \textit{Rektionsrelation} werden durch die regierende Einheit (das \textit{Regens}) Werte für bestimmte Merkmale (und ggf.\ auch die Form) beim regierten Element (dem \textit{Rectum}) verlangt.
Die Werte stimmen im Fall der Rektion bei Regens und Rectum nicht (oder nur durch Zufall) überein.
\index{Rektion}\index{Regens}\index{Rectum}
}

Es folgt ein Beispiel für eine Rektionsrelation.
Verben wie \textit{gedenken} und \textit{besiegen} werden normalerweise mit genau zwei weiteren Einheiten (dem traditionellen \textit{Subjekt} und \textit{Objekt}) verwendet.

\begin{exe}
  \ex{Der Torwart gedenkt der Niederlage.}
  \ex{Der FCR Duisburg besiegt den FFC Frankfurt.}
\end{exe}

Eine der Einheiten (das traditionelle Subjekt) muss immer im Nominativ stehen, alle anderen Kasus sind ausgeschlossen.

\begin{exe}
  \ex
  \begin{xlist}
    \ex[*]{Den Torwart gedenkt der Niederlage.}
    \ex[*]{Dem Torwart gedenkt der Niederlage.}
    \ex[*]{Des Torwarts gedenkt der Niederlage.}
  \end{xlist}
\end{exe}

Gleichzeitig steht aber die zweite Einheit (das Objekt) je nach Verb in einem anderen Kasus, nämlich Genitiv bei \textit{gedenken} und Akkusativ bei \textit{besiegen}.
Jedes Verb verlangt also einerseits einen Nominativ bzw.\ ein Subjekt.%
\footnote{In Kapitel~\ref{sec:relationenpraedikate} werden wir diskutieren, ob wirklich jedes Verb ein Subjekt im Nominativ hat.
Auf die mit Abstand meisten Verben trifft es aber zu.}
Es verlangt aber auch, dass sein Objekt (falls es eines hat) einen bestimmten Kasus hat.
Dieser Fall passt genau zu der Definition von Rektion, denn offensichtlich regiert das Verb das Subjekt und die Objekte bezüglich ihrer Kasusmerkmale.
Was das Verb nicht regiert, sind zum Beispiel die Merkmale \textsc{Genus} oder \textsc{Numerus} seiner Objekte.
Letzteres ist ganz einfach nachzuvollziehen:
Das Objekt kann im Singular (\ref{ex:gb-4910a}) oder Plural (\ref{ex:gb-4910b}) stehen.
Das Verb hat hier nicht mitzureden.

\begin{exe}
  \ex
  \begin{xlist}
    \ex{\label{ex:gb-4910a} Der FCR besiegt die andere Mannschaft.}
    \ex{\label{ex:gb-4910b} Der FCR besiegt alle anderen Mannschaften.}
  \end{xlist}
\end{exe}

Der letzte Satz aus Definition~\ref{def:rektion} lässt sich auch leicht mit diesen Beispielen demonstrieren.
Der Wert, den das Verb verlangt, ist [\textsc{Kasus}: \textit{genitiv}] bzw.\ [\textsc{Kasus}: \textit{akkusativ}], aber das Verb selber hat das Merkmal \textsc{Kasus} nicht.

Es ist nur noch zu erklären, was mit der Formulierung in Definition~\ref{def:rektion} gemeint ist, dass die Merkmale bei Regens und Rectum nicht (oder nur durch Zufall) übereinstimmen.
Dazu kann man Beispiele wie in (\ref{ex:gb3632327}) heranziehen.

\begin{exe}
  \ex\label{ex:gb3632327} 
  \begin{xlist}
    \ex{\label{ex:gb3632327a} dank des Einsatzes meines Kollegen}
    \ex{\label{ex:gb3632327b} durch den Einsatz meines Kollegen}
  \end{xlist}
\end{exe}

In (\ref{ex:gb3632327a}) haben \textit{des Einsatzes} und \textit{meines Kollegen} denselben Kasus, nämlich Genitiv.
Wie in Abschnitt~\ref{sec:rektioninderngr} argumentiert wird, kann man davon ausgehen, dass der Genitiv von \textit{meines Kollegen} von \textit{Einsatz} regiert wird.
Ändert sich der Kasus zum Akkusativ in \textit{den Einsatz} (der von \textit{durch} regiert wird), bleibt in dieser Konstruktion trotzdem der Genitiv von \textit{meines Kollegen} erhalten, wie man in (\ref{ex:gb3632327b}) sieht.
Auch wenn \textit{des Einsatzes} und \textit{meines Kollegen} in (\ref{ex:gb3632327a}) beide im Genitiv stehen, ist das reiner Zufall, und es gibt für die beiden Genitive voneinander unabhängige Motivationen.
Bei der \textit{Kongruenz} handelt es sich hingegen immer um eine nicht-zufällige Übereinstimmung von Merkmalen.

\Definition{Kongruenz}{\label{def:kongruenz}%
In einer \textit{Kongruenzrelation} muss eine Übereinstimmung von Werten bestimmter Merkmale zwischen den kongruierenden Einheiten bestehen.
\index{Kongruenz}
}

In (\ref{ex:gb888818}) und (\ref{ex:gb888819}) werden Beispiele für Kongruenz gegeben.

\begin{exe}
  \ex{\label{ex:gb888818} Die Verteidigerinnen lassen keinen Ball durch.}
  \ex{\label{ex:gb888819} Die Defensive lässt keinen Ball durch.}
\end{exe}

Die beiden Sätze sind weitestgehend identisch, abgesehen davon, dass im ersten Satz das Subjekt \textit{die Verteidigerinnen} ein Plural ist, im zweiten Satz das Subjekt \textit{die Defensive} aber ein Singular.
Parallel dazu muss auch das Verb im Plural (\textit{lassen}) bzw.\ im Singular (\textit{lässt}) stehen.
Das entspricht genau der Definition der Kongruenz, weil beide Kategorien (Substantive und Verben) ein Merkmal \textsc{Numerus} haben, und zwischen Subjekt und Verb diese Merkmale immer übereinstimmen müssen.
Zwischen dem Verb und dem Objekt besteht diese Kongruenzrelation nicht.\\
Damit sind zwei wichtige syntaktische Relationen definiert, und der nächste Abschnitt führt unter Verwendung des Rektionsbegriffs den Valenzbegriff ein.

\Zusammenfassung{
Neben der Beschreibung sprachlicher Einheiten geht es in der Grammatik vor allem um Relationen zwischen diesen Einheiten.
Diese können kontextunabhängig sein, wie \zB die Relationen zwischen Wörtern im Lexikon.
Darüberhinaus bilden Einheiten aber auch Kontexte für andere Einheiten, und nicht jede Art von Einheit kann in jedem Kontext (Syntagma) stehen.
Neben der reinen Oberflächenstruktur -- also der linearen Zusammensetzung von Lauten, Wörtern usw. zu längeren Einheiten -- gibt es weitere Relationen, die die Grammatikalität von größeren Einheiten steuern.
Bei der Kongruenz müssen Merkmalswerte von Einheiten übereinstimmen, bei der Rektion verlangen Einheiten bestimmte Werte bei anderen Einheiten.
}

\section{Valenz}

\label{sec:valenz}

\index{Valenz}

Mit \textit{Valenz} bezeichnet man die Eigenschaft von Verben und anderen Wörtern wie Adjektiven (s.\ Abschnitt~\ref{sec:adjektivklassifikation}) und Präpositionen (s.\ Abschnitt~\ref{sec:prpgr}) -- ganz vage gesagt -- das Vorhandensein einer oder mehrerer anderer Konstituenten in einer Struktur zu steuern.
Obwohl Valenz auch außerhalb des verbalen Bereichs eine wichtige Rolle spielt, dreht sich die Diskussion zentral immer wieder um Verben, weswegen wir hier fürs Erste so tun, als beträfe das Phänomen nur die Verben.
Im Kern geht es darum, den unterschiedlichen Status der sogenannten \textit{Ergänzungen} wie \textit{ein Bild} in (\ref{ex:gb82882}) und der sogenannten \textit{Angaben} wie \textit{gerne} in (\ref{ex:gb82883}) zu definieren.%
\footnote{In anderen Terminologien heißen \textit{Ergänzungen} auch \textit{Komplemente}, wobei von diesen manchmal die Subjekte ausgenommen werden.
Die \textit{Angaben} heißen dann meist \textit{Adjunkte}.
Man findet für \textit{Ergänzung} auch den Ausdruck \textit{Argument}.
}

\begin{exe}
  \ex\label{ex:gb82882u3}
  \begin{xlist}
    \ex{\label{ex:gb82882} Gabriele malt ein Bild.}
    \ex{\label{ex:gb82883} Gabriele malt gerne.}
  \end{xlist}
\end{exe}

An den Beispielen sieht man sofort, dass Objekte wie das Akkusativobjekt \textit{ein Bild} in (\ref{ex:gb82882}), die man prinzipiell zu den Ergänzungen zählt, auf keinen Fall bei bestimmten Verben immer stehen müssen, denn in (\ref{ex:gb82883}) gibt es keinen Akkusativ, und der Satz ist immer noch grammatisch.
Trotzdem gehört die adverbiale Bestimmung (und damit Angabe) \textit{gerne} in (\ref{ex:gb82883}) nach der allgemeinen Einschätzung deutlich weniger eng zu \textit{malen} als das Akkusativobjekt.
Es kommt erschwerend hinzu, dass man nicht einfach bestimmte Kasus wie den Akkusativ oder den Dativ als eindeutiges Kennzeichen von Ergänzungen nehmen kann, denn den Akkusativ in \textit{einen ganzen Tag} in (\ref{ex:gb82884a}) und den Dativ \textit{ihrem Mann} in (\ref{ex:gb82884b}) will man normalerweise zu den Angaben zählen.
Für viele Linguisten ist außerdem \textit{im Russenhaus} in (\ref{ex:gb82885}) zwar ein Adverbial, aber dabei eben doch eine Ergänzung (vgl.\ dazu aber Vertiefung \ref{vert:nregerg} auf S.~\pageref{vert:nregerg}).

\begin{exe}
  \ex\label{ex:gb82884} 
  \begin{xlist}
    \ex{\label{ex:gb82884a} Gabriele malt einen ganzen Tag.} 
    \ex{\label{ex:gb82884b} Gabriele malt ihrem Mann zu figürlich.}
  \end{xlist}
  \ex{\label{ex:gb82885} Gabriele wohnt im Russenhaus.}
\end{exe}

Das Problem mit dem Valenzbegriff ist, dass sich Linguisten meist schnell über die typischen Fälle von Valenz einig sind, aber dass die von ihnen jeweils angenommenen Definition(en) entweder ungenau sind oder die weniger typischen und eindeutigen Fälle nicht einheitlich entweder den Ergänzungen oder den Angaben zuordnen.
Damit uns im Rahmen der Sprachbeschreibung die Unterscheidung überhaupt irgendetwas bringt, müssen wir jetzt genau angeben, warum das eine eine Angabe und das andere eine Ergänzung sein soll, und welche spezifische Definition des Unterschieds unseren Ansprüchen genügt.

Wie schon zu (\ref{ex:gb82882u3}) angedeutet wurde, ist es nun offensichtlich so, dass die Ergänzungen ganz unabhängig von der Bedeutung eines Verbs bei diesem in unterschiedlichem Maß stehen müssen, können oder gar nicht stehen dürfen.
Die Sätze (\ref{ex:gb292116})--(\ref{ex:gb292118}) illustrieren dies anhand der Verben \textit{verschlingen}, \textit{essen} und \textit{speisen}, die im Prinzip alle eine sehr ähnliche Bedeutung haben.
Der Akkusativ bei \textit{verschlingen} muss stehen, bei \textit{essen} darf er stehen, bei \textit{speisen} darf er auf keinen Fall stehen.
Man spricht auch davon, dass Ergänzungen entweder \textit{obligatorisch} sind, wenn sie stehen müssen, oder \textit{fakultativ}, wenn sie weglassbar sind.\index{Ergänzung!fakultativ und obligatorisch}

\begin{exe}
  \ex\label{ex:gb292116}
  \begin{xlist}
    \ex[]{\label{ex:gb292116a} Wir verschlingen den Salat.}
    \ex[*]{\label{ex:gb292116b} Wir verschlingen.}
  \end{xlist}
  \ex\label{ex:gb292117}
  \begin{xlist}
    \ex[]{\label{ex:gb292117a} Wir essen den Salat.}
    \ex[]{\label{ex:gb292117b} Wir essen.}
  \end{xlist}
  \ex\label{ex:gb292118}
  \begin{xlist}
    \ex[*]{\label{ex:gb292118a} Wir speisen den Salat.}
    \ex[]{\label{ex:gb292118b} Wir speisen.}
  \end{xlist}
\end{exe}

Kritisch sind die fakultativen Fälle wie in (\ref{ex:gb292117}), denn sie sind dafür verantwortlich, dass man nicht (wie früher öfters getan) sagen kann, dass Ergänzungen die Satzteile seien, die bei einem Verb auf jeden Fall stehen \textit{müssen}.
Bezüglich der Weglassbarkeit sind die fakultativen Ergänzungen nämlich nicht von Angaben wie \textit{gerne} zu unterscheiden.
Wir gehen daher hier genau den anderen argumentativen Weg und bauen die Definition darauf auf, dass Ergänzungen solche Einheiten sind, die in manchen Fällen aber nicht stehen \textit{dürfen}.

Auch wenn zu einem bestimmten Verb eine Ergänzung fakultativ ist, so ist sie doch üblicherweise vom Verb regiert, muss also in einem bestimmten Kasus bzw.\ in einer bestimmten Form stehen.
Die Sätze (\ref{ex:gb1358}) enthalten Verben, die einen Akkusativ regieren (\textit{anschieben} und \textit{essen}) sowie ein Verb, das ein Präpositionalobjekt -- nämlich eine Ergänzung mit der Präposition \textit{an} -- regiert (\textit{glauben}).
Diese Ergänzungen sind jeweils fakultativ und hier auch tatsächlich weggelassen.

\begin{exe}
  \ex\label{ex:gb1358}
  \begin{xlist}
    \ex{\label{ex:gb1358a} Der Motor springt nicht an.
      Wir müssen anschieben.}
    \ex{\label{ex:gb1358b} Wir sitzen in der Mensa und essen.}
    \ex{\label{ex:gb1358c} Ein Atheist glaubt nicht.}
  \end{xlist}
\end{exe}

Wenn die Ergänzungen hier realisiert werden, aber in einer unangemessenen Form stehen (falscher Kasus oder falsche Präposition), sind die Sätze ungrammatisch, wie in (\ref{ex:gb1359}) demonstriert wird.
Es muss sich also um Fälle von Rektion durch das Verb handeln.

\begin{exe}
  \ex\label{ex:gb1359}
  \begin{xlist}
    \ex[*]{\label{ex:gb1359a} Wir müssen des Wagens anschieben.}
    \ex[*]{\label{ex:gb1359b} Wir essen einem Salat.}
    \ex[*]{\label{ex:gb1359c} Ein Atheist glaubt nicht bei einem Gott.}
  \end{xlist}
\end{exe}

Man kann nun diese Sätze so betrachten, dass jeweils in (\ref{ex:gb1359a}) kein Genitiv, in (\ref{ex:gb1359b}) kein Dativ und in (\ref{ex:gb1359c}) keine Präposition \textit{bei} stehen dürfen.
Wir sagen also, dass bestimmte Verben es überhaupt erst zulassen, dass ein bestimmter Kasus oder eine bestimmte Präposition bei einem Verb stehen darf.
Man kann auch davon sprechen, dass ein Verb (oder ganz allgemein eine Einheit) eine andere Einheit lizenziert bzw.\ eben nicht lizenziert.

\Definition{Lizenzierung}{\label{def:lizenzierung}%
Eine Einheit A \textit{lizenziert} eine andere Einheit B genau dann, wenn Einheit B mit einer bestimmten Merkmal-Wert-Konfiguration ohne Einheit A nicht uneingeschränkt in einer größeren Einheit (Struktur) auftreten kann.
\index{Lizenzierung}
}

Die Struktur, auf die sich die Definition bezieht, ist für unsere Zwecke zunächst ein Satz.
Konkret heißt das \zB für Ergänzungen von Verben, dass es erst eines bestimmten Verbs im Satz bedarf, damit überhaupt ein Akkusativ usw. auftreten kann.
Traditionell gesprochen kann ein Akkusativobjekt (in erster Annäherung) nur in einem Satz vorkommen, der ein sogenanntes \textit{transitives Verb} enthält, usw.
Ein regierendes Element lizenziert ein anderes Element dabei aber nur genau einmal.
Ein Satz mit einem Verb, das einen Akkusativ lizenziert, kann nicht ohne weiteres mehrere Akkusative enthalten, wie (\ref{ex:gb01275}) zeigt.%
\footnote{Die Akkusative wurden hier bewusst nicht hintereinander hingeschrieben, um eine Lesart als Aufzählung (wie mit \textit{und} bzw.\ Komma) zu blockieren.}

\begin{exe}
  \ex\label{ex:gb01275}
  \begin{xlist}
    \ex[*]{Den Wagen müssen wir den Transporter anschieben.}
    \ex[*]{Einen Salat essen wir einen Tofu-Burger.}
  \end{xlist}
\end{exe}

Die sogenannten Angaben können nun aber ebenfalls bei den genannten Verben stehen, in (\ref{ex:gb1360}) sind dies \textit{jetzt} und \textit{schnell}.

\begin{exe}
  \ex\label{ex:gb1360}
  \begin{xlist}
    \ex{\label{ex:gb1360a} Wir müssen jetzt anschieben.}
    \ex{\label{ex:gb1360b} Wir essen schnell.}
  \end{xlist}
\end{exe}

Diese Angaben müssen scheinbar nicht explizit durch ein Verb lizenziert werden, sondern können als von jedem Verb lizenziert angesehen werden.
Beispiele dafür mit \textit{jetzt} finden sich in (\ref{ex:gb1362}).

\begin{exe}
  \ex\label{ex:gb1362}
  \begin{xlist}
    \ex{\label{ex:gb1362a} Die Sonne scheint jetzt.}
    \ex{\label{ex:gb1362b} Sie liest jetzt das Buch.}
    \ex{\label{ex:gb1362c} Andromeda bewegt sich jetzt auf die Milchstraße zu.}
    \ex{\label{ex:gb1362d} Sie geben mir jetzt das Buch!}
  \end{xlist}
\end{exe}

Die Angaben stehen dabei nicht in irgendeiner Art von Konkurrenz zu den Ergänzungen.
In (\ref{ex:gb1361}) tauchen Ergänzungen und Angaben nebeneinander auf, und die Sätze sind zweifellos grammatisch, anders als in (\ref{ex:gb01275}).

\begin{exe}
  \ex\label{ex:gb1361}
  \begin{xlist}
    \ex{\label{ex:gb1361a} Den Wagen müssen wir jetzt anschieben.}
    \ex{\label{ex:gb1361b} Einen Salat essen wir schnell.}
  \end{xlist}
\end{exe}

Außerdem sieht man leicht, dass die Angaben im Gegensatz zu den Ergänzungen prinzipiell beliebig oft lizenziert sind.
Man sagt auch, Angaben seien \textit{iterierbar} (wiederholbar).
\index{Iterierbarkeit}%
In (\ref{ex:gb01274}) sind jeweils mehrere Angaben enthalten, und die Sätze werden offensichtlich nicht ungrammatisch.
Auch dies ist ein Gegensatz zu (\ref{ex:gb01275}).

\begin{exe}
  \ex\label{ex:gb01274}
  \begin{xlist}
    \ex{Wir müssen den Wagen jetzt mit aller Kraft vorsichtig anschieben.}
    \ex{Wir essen schnell mit Appetit an einem Tisch mit der Gabel einen Salat.}
  \end{xlist}
\end{exe}

Damit haben wir schon im Grunde alle Argumente gesammelt, um den Unterschied zwischen Ergänzungen und Angaben -- und in deren Folge den Valenzbegriff -- zu definieren.
Interessant ist lediglich noch, dass ein Kasus wie der Akkusativ an sich keine eigenständige Bedeutung hat (vgl.\ auch Abschnitt~\ref{sec:kasus}).\index{Kasus!Bedeutung}
In den Beispielen in (\ref{ex:gb77774}) ist es unmöglich, ganz allgemein (also für alle drei Fälle einheitlich) zu sagen, was den an den Ereignissen beteiligten Objekten, die jeweils durch einen Akkusativ bezeichnet werden, gemein sein soll.
Wenn der Akkusativ an sich eine Bedeutung hätte, müsste dies aber möglich sein.

\begin{exe}
  \ex\label{ex:gb77774}
  \begin{xlist}
    \ex{\label{ex:gb77774a}Ich lösche die Datei.}
    \ex{\label{ex:gb77774b}Ich mähe den Rasen.}
    \ex{\label{ex:gb77774c}Ich fürchte den Sturm.}
  \end{xlist}
\end{exe}

Der Akkusativ hilft uns hier lediglich, zu erkennen, dass es eine bestimmte grammatische Beziehung zwischen \textit{die Datei} usw.\ und dem jeweiligen Verb gibt.
Es muss dann ein semantisches Wissen über das Verb geben, das den Sprachbenutzern sagt, dass die Dinge, die im Akkusativ bei einem konkreten Verb bezeichnet werden, eine bestimmte Rolle in dem vom Verb bezeichneten Ereignis spielen.\index{Rolle}
Dass in (\ref{ex:gb77774a}) die Datei der gelöschte (und damit vernichtete) Gegenstand ist, dass in (\ref{ex:gb77774b}) der Rasen der (über seine Fläche) verkürzte Gegenstand ist, und dass in (\ref{ex:gb77774c}) der Sturm etwas dem Sprecher Angst Verursachendes ist, wird also nicht durch den Akkusativ kodiert, sondern ist in der Bedeutung der Verben verankert.
Der Akkusativ ist ein Vermittler zwischen dem grammatischen Aufbau des Satzes und der Bedeutung des Verbs.
Man kann sich dies auch so vergegenwärtigen, dass wir, wenn wir isoliert \textit{den Rasen} lesen oder hören, durch den eindeutig erkennbaren Akkusativ noch keinerlei Begriff davon haben, an welcher Art von Ereignis der Rasen beteiligt ist (und folglich auch nicht, auf welche Weise er dies ist).
Das ist bei den Angaben tendentiell anders.
Angaben wie \textit{jetzt}, \textit{schnell}, \textit{im Universum}, \textit{mit einer Harke}, \textit{ohne mit der Wimper zu zucken} usw.\ sind auch für sich genommen semantisch relativ spezifisch.
Wir können die Unterschiede zwischen Ergänzungen und Angaben wie in Tabelle~\ref{tab:ergang} zusammenfassen.

\begin{table}[!htbp]
  \centering
  \begin{tabular}{lll}
    \lsptoprule
    & \textbf{Ergänzung} & \textbf{Angabe} \\
    \midrule
    \textbf{fakultativ} & manchmal & ja \\
    \textbf{regiert} & ja & nein \\
    \textbf{lizenziert} & (verb)spezifisch & allgemein \\
    \textbf{iterierbar} & nein & ja \\
    \textbf{interpretierbar} & (verb)gebunden & eigenständig \\
    \lspbottomrule
  \end{tabular}
  \caption{Eigenschaften von Ergänzungen und Angaben beim Verb}
  \label{tab:ergang}
\end{table}

Von den Eigenschaften in Tabelle~\ref{tab:ergang} müssen wir jetzt definitorisch hinreichende auswählen.
Fakultativität kommt nicht infrage, da sie bei der Unterscheidung zwischen Angaben und fakultativen Ergänzungen versagt.
Regiertheit ist besser für eine Definition tauglich, weil sie klar zwischen Ergänzungen und Angaben trennt.
Die unterschiedliche Art der Lizenzierung ist das solideste Kriterium.
Die Iterierbarkeit folgt quasi der Lizenzierung.
Die Interpretierbarkeit ist ein semantisches Kriterium, das konzeptuell sehr wichtig ist, aber das man oft nicht gut am gegebenen Material testen kann.
Auch wenn dieses Kriterium später (vor allem in Abschnitt~\ref{sec:ppergang}) noch zu Hilfe genommen wird, soll es hier zunächst nicht in der Definition verwendet werden.

\Definition{Ergänzung und Angabe}{\label{def:ergang}%
\textit{Angaben} sind uneingeschränkt (und iterierbar) von Einheiten einer Klasse (\zB von Verben) grammatisch lizenziert.
\textit{Ergänzungen} sind jeweils nur von einem Teil der Einheiten einer Klasse grammatisch lizenziert.
Einschränkungen der Lizenzierung von Angaben sind immer semantisch motiviert.
\index{Angabe}
\index{Ergänzung}
}

Man kann die Essenz dieser Definition zusammenfassen, indem man sagt, dass Ergänzungen \textit{subklassenspezifisch lizenziert} sind.
Akkusative und Dative sind \zB nur bei entsprechenden Verben (\zB \textit{essen}, \textit{geben}) lizenziert, die damit Subklassen (Teilklassen) der Klasse der Verben darstellen.
Ganz ohne Probleme ist die Definition nicht, denn der letzte Satz bringt eine gewisse Unschärfe mit sich.
In (\ref{ex:gb945801}) finden sich einige Sätze mit Angaben, die offensichtlich nicht von allen Verben lizenziert werden.

\begin{exe}
  \ex\label{ex:gb945801}
  \begin{xlist}
    \ex[?]{\label{ex:gb945801a} Der Ballon flog freiwillig.}
    \ex[?]{\label{ex:gb945801b} Die Kinder rennen dialektisch.}
    \ex[?]{\label{ex:gb945801c} Ich denke unter den Tisch.}
    \ex[?]{\label{ex:gb945801d} Der Ballon platzt seit drei Minuten.}
  \end{xlist}
\end{exe}

In allen Sätzen in (\ref{ex:gb945801}) führt die Angabe dazu, dass der Satz so gut wie nicht äußerbar ist.
Dies liegt aber nicht wie in (\ref{ex:gb1359}) daran, dass die Verben an sich die Art der Angabe (\zB ein Adverb wie \textit{freiwillig} oder ein Zeit-Adverbial wie \textit{seit drei Minuten}) grammatisch nicht lizenzieren, sondern ausschließlich an semantischen Inkompatibilitäten.
Man kann dies oft zeigen, indem man Kontexte bzw.\ Geschichten erfindet, in denen diese Angaben dann doch nicht zur Ungrammatikalität führen.
In (\ref{ex:gb55352}) wird das für zwei Fälle demonstriert.

\begin{exe}
  \ex\label{ex:gb55352}
  \begin{xlist}
    \ex{\label{ex:gb55352a} Die Besatzungen eines Heißluftballons und eines Sportflugzeuges wurden vor die Wahl gestellt, sofort freiwillig loszufliegen oder die Entscheidung der Behörde abzuwarten.
    Der Ballon flog freiwillig.}
    \ex{\label{ex:gb55352b} Im Labor wird die Hochgeschwindigkeitsaufnahme eines platzenden Ballons gezeigt.
    Der Laborleiter kommt verspätet, und der Assistent sagt: \textit{Der Ballon platzt seit drei Minuten.}}
  \end{xlist}
\end{exe}

Auch wenn diese Geschichten stark forciert klingen, wird eine vergleichbare Rettung von Sätzen mit einem Genitiv bei \textit{anschieben} wie in (\ref{ex:gb1359a}) usw.\ nicht gelingen.
Insbesondere ist interessant, dass es sich bei (\ref{ex:gb945801a}) und dann (\ref{ex:gb55352a}) gar nicht um eine Inkompatibilität mit dem Verb handelt, sondern um eine, die erst in Verbindung mit dem speziellen Subjekt entsteht.
Sobald \textit{der Ballon} wie in (\ref{ex:gb55352a}) als \textit{die Besatzung des Ballons} gelesen wird, kann die Angabe stehen, was ein eindeutiger Hinweis auf eine semantische Bedingung ist.
Wenn also Angaben nicht völlig frei kombinierbar sind, so hat dies immer rein semantische Gründe.
Unschärfen werden realistisch gesehen auch bei dieser Definition trotzdem bleiben.

\begin{Vertiefung}{Scheinbare nicht regierte Ergänzungen}
  \label{vert:nregerg}
\noindent Es gibt prominente Beispiele von scheinbaren nicht regierten Ergänzungen.
Das Verb \textit{wohnen} ist ein typisches Beispiel, s.\ (\ref{ex:gb1946}).

\begin{exe}
  \ex\label{ex:gb1946}
  \begin{xlist}
    \ex[]{\label{ex:gb1946a}Wir wohnen in Bochum.}
    \ex[]{\label{ex:gb1946b}Wir wohnen neben dem Bahnhof.}
    \ex[]{\label{ex:gb1946c}Wir wohnen ziemlich ruhig.}
    \ex[*]{\label{ex:gb1946d}Wir wohnen.}
    \ex[*]{\label{ex:gb1946e}Wir wohnen seit gestern.}
  \end{xlist}
\end{exe}

Bei diesen Verben wird irgendeine Art von Adverb oder Adverbial des Ortes oder der Art und Weise gefordert.
Das völlige Fehlen eines Adverbials (\ref{ex:gb1946d}) führt genauso zu Ungrammatikalität wie das Vorhandensein eines nicht kompatiblen Adverbials wie \textit{seit gestern} (\ref{ex:gb1946e}).

Das Problem ist nun, dass sich die scheinbare Rektionsanforderung grammatisch nicht vernünftig einschränken lässt, sondern dass es sich offensichtlich um semantische Bedingungen handelt, die die kompatiblen Adverbiale erfüllen müssen.
Es kann sich also nicht um grammatische Rektion und damit nicht um Valenz in unserem Sinn handeln.
Es lägen hier also nicht-regierte obligatorische Ergänzungen vor.
Dass in sprachspielerischem Gebrauch vereinzelt \textit{wohnen} ohne Adverbial auftritt (\ref{ex:gb10001a}), ist nur ein schwaches Argument, denn im Ernstfall kann man so auch andere Verben mit obligatorischen Ergänzungen dehnen (\ref{ex:gb10001b}).


\begin{exe}
  \ex\label{ex:gb10001}
  \begin{xlist}
    \ex{\label{ex:gb10001a} Wohnst du noch, oder lebst du schon?\\
      (Ikea-Slogan aus dem Jahr 2002)}
    \ex{\label{ex:gb10001b} Die essen nicht, die verschlingen nur noch!}
  \end{xlist}
\end{exe}

Wir gehen hier daher davon aus, dass es schlicht eine Üblichkeit des Sprachgebrauchs (der Pragmatik) ist, \textit{wohnen} nicht ohne Adverbial zu benutzen.
Das Verb \textit{wohnen} ist nach dieser Auffassung von seiner kommunikativen Funktion her so angelegt, dass es den Hörer eben gerade über die Begleitumstände des Wohnens informieren soll.
Es ohne eine Angabe der Begleitumstände zu verwenden, ist also kommunikativ nicht zielführend.
Mit Grammatik im engeren Sinne (hier Rektion und Valenz) hat das dann nichts zu tun.

\end{Vertiefung}

Damit haben wir uns der Unterscheidung von Ergänzungen und Angaben vergleichsweise gut angenähert.
\textit{Valenz} ist nun ganz einfach mittels des Ergänzungsstatus zu definieren.

\Definition{Valenz}{\label{def:valenz}%
Die \textit{Valenz} einer Einheit ist die Liste seiner Ergänzungen.
\index{Valenz}
}

Die Valenz (wörtlich \textit{Wertigkeit}) eines Verbs ist also nichts weiter als die Spezifikation der Anzahl und Art seiner Ergänzungen.
Entsprechend spricht man bei den Verben auch von den einwertigen (\textit{schnarchen}), zweiwertigen (\textit{lieben}) und dreiwertigen (\textit{geben}).
Diesen Begriffen entsprechen die eher traditionellen Begriffe von (in derselben Reihenfolge) \textit{intransitiv}, \textit{transitiv} und \textit{ditransitiv}.
\index{Verb!transitiv}
\index{Verb!ditransitiv}
\index{Verb!intransitiv}
Bei Verben wie \textit{glauben} (\textit{an}), die ein Präpositionalobjekt als Ergänzung lizenzieren, spricht man analog auch von \textit{präpositional zwei- und dreiwertigen} Verben.\label{abs:praepditrans}
Verb\-valenzen werden vertiefend vor allem in den Abschnitten~\ref{sec:kasus}, \ref{sec:passiv} und \ref{sec:objekte} behandelt.
Die Valenzen anderer Wortklassen wie Adjektive, Substantive und Präpositionen werden an vielen weiteren Orten in diesem Buch behandelt.

\index{Merkmal!Listen--}
\index{Valenz!als Liste}

Eine wichtige Erkenntnis zur Valenz als Merkmal im formalen Sinn ist, dass anders als bei Merkmalen wie \textsc{Kasus} oder \textsc{Numerus} eine Liste als Wert angesetzt werden muss.
Verben können \zB eine, zwei oder mehr Valenzstellen haben, und es muss für jede Stelle Information gespeichert werden, welche Eigenschaften die Valenznehmer haben sollen.
Daher deklarieren wir \textsc{Valenz} wie in (\ref{ex:gb3543}).

\begin{exe}
  \ex{\label{ex:gb3543} \textsc{Valenz}: \textit{liste}}
\end{exe}

Wenn wir die Elemente einer Liste in $\langle\ \rangle$ setzen, ergeben sich Spezifikationen von Valenzlisten wie in (\ref{ex:gb3544}).

\begin{exe}
  \ex\label{ex:gb3544} 
  \begin{xlist}
    \ex{gehen = [\textsc{Temp}, \textsc{Mod}, \textsc{Valenz}:$\langle$ [\textsc{Kas}: \textit{nom}] $\rangle$]}
    \ex{sehen = [\textsc{Temp}, \textsc{Mod}, \textsc{Valenz}:$\langle$ [\textsc{Kas}: \textit{nom}], [\textsc{Kas}: \textit{akk}] $\rangle$]}
    \ex{geben = [\textsc{Temp}, \textsc{Mod}, \textsc{Valenz}:$\langle$ [\textsc{Kas}: \textit{nom}], [\textsc{Kas}: \textit{dat}], [\textsc{Kas}: \textit{akk}] $\rangle$]}
  \end{xlist}
\end{exe}

Verben wie \textit{gehen} spezifizieren auf ihrer Valenzliste, dass sie eine Einheit fordern, die [\textsc{Kas}: \textit{nom}] ist, bei \textit{sehen} wird zusätzlich eine Einheit [\textsc{Kas}: \textit{akk}] verlangt usw.
Es sollte klar werden, warum die Valenz aufgelistet werden muss, und dass man dafür eine besondere Art von Merkmal -- nämlich eine Liste -- benötigt.

\Zusammenfassung{
Valenz steuert das gemeinsame Auftreten von Einheiten in einem Satz.
Die Valenz eines Wortes legt fest, welche anderen Einheiten mit diesem Wort im Satz vorkommen dürfen (Lizenzierung), und welche mit ihm vorkommen müssen.
Die von einem Valenzgeber lizenzierten Einheiten sind seine Ergänzungen.
Sie können typischerweise nicht mit allen Arten von Valenzgebern kombiniert werden.
Einheiten, die uneingeschränkt kombinierbar sind, sind Angaben.
}

\WeitereLiteratur

\begin{sloppypar}

\paragraph*{Grammatik und Linguistik}

Einführungen in die Linguistik wie \citet{EgL07} liefern Diskussionen vieler grundlegender Begriffe.
Neben Kapitel~1 aus \citet{Eisenberg1} ist \citet{Engel09} besonders einschlägig, der auch zur Valenz eine kurze, aber aufschlussreiche Diskussion liefert \citep[70--73]{Engel09}.
Einen Einblick in die Sprachtheorie mit sehr genauen Definitionen gibt \citet{Mueller10}, ebenfalls mit ausführlicher Diskussion des Valenzbegriffs.%
\footnote{\url{https://hpsg.fu-berlin.de/~stefan/Pub/grammatiktheorie.html}}
Eine einführende Darstellung, die das Potential von Merkmal-Wert-Kodierungen formal ausschöpft, ist \citet{Mueller08}.%
\footnote{\url{https://hpsg.fu-berlin.de/~stefan/Pub/hpsg-lehrbuch.html}}

\paragraph*{Sprachnorm und Sprachkritik}

Eine Auseinandersetzung mit populärer Sprachkritik aus linguistischer Sicht ist \citet{Meinunger08}.
Etwas anspruchsvoller ist \citet{Eisenberg08}, ebenso wie Kapitel~1 aus \citet{Eisenberg1} und Kapitel~2 aus \citet{Eisenberg2}.
Die Betrachtung der Sprachgeschichte hilft, zu verstehen, dass synchrone Sprachnormen niemals Absolutheitsanspruch haben können, und \citet{NueblingEa2010} wird als Einstieg in die relevante Literatur empfohlen.

\paragraph*{Empirie}

Einen kompakten Überblick über empirische Verfahren in der Linguistik bietet \citet{Albert07}.
Die Korpuslinguistik wird in \citet{Perkuhn-ea2012} einführend dargestellt.

\paragraph*{Valenz}

Alles Wesentliche zur klassischen Valenztheorie und verschiedenen Weiterentwicklungen kann der Einleitung von \citet{HS91} entnommen werden.
In diesem Buch findet sich auch ein Verzeichnis der Valenzmuster von deutschen Verben.
Ein weiteres Valenzlexikon der deutschen Sprache ist VALBU \citep{Valbu}, inklusive einer Online-Fassung.%
\footnote{\url{http://www.ids-mannheim.de/gra/valbu.html}}

\end{sloppypar}
