\chapter{Sätze}

\label{sec:saetze}

\section{Hauptsatz und Matrixsatz}

\label{sec:hauptsatzmatrixsatz}

Außer dem durch Komplementierer eingeleiteten Nebensatz -- also der Komplementiererphrase (KP) mit eingebetteter VP -- gibt es weitere wichtige Satztypen im Deutschen, die sich jeweils durch eine besondere Satzgliedstellung auszeichnen.
Dies sind die Satzgliedstellung des \textit{unabhängigen Aussagesatzes} (\ref{ex:saetze9001a}) und die des \textit{Fragesatzes mit Fragepronomen} (\textit{w}"=Fragesatz) (\ref{ex:saetze9001c}), die des \textit{Entscheidungsfragesatzes} (\textit{Ja\slash Nein-Frage}) wie in (\ref{ex:saetze9001b}) und die des \textit{Relativsatzes} wie in (\ref{ex:saetze9001d}) sowie des sehr ähnlichen \textit{eingebetteten \textit{w}"=Fragesatzes} wie in (\ref{ex:saetze9001e}).

\begin{exe}
  \ex\label{ex:saetze9001}
  \begin{xlist}
    \ex{\label{ex:saetze9001a} Wahrscheinlich hat der Arzt das Bild gekauft.}
    \ex{\label{ex:saetze9001c} Was hat der Arzt gekauft?}
    \ex{\label{ex:saetze9001b} Hat der Arzt das Bild gekauft?}
    \ex{\label{ex:saetze9001d} (Das ist das Bild,) das der Arzt gekauft hat.}
    \ex{\label{ex:saetze9001e} (Ischariot fragt sich,) was der Arzt gekauft hat.}
  \end{xlist}
\end{exe}

Während schon definiert wurde, was wir unter einem Nebensatz verstehen, soll jetzt noch gesagt werden, was wir unter einem \textit{unabhängigen Satz} -- landläufig \textit{Hauptsatz} -- verstehen wollen.

\Definition{Unabhängiger Satz\slash Hauptsatz}{
\label{def:satz}
Ein unabhängiger Satz (den man auch einen \textit{Hauptsatz} oder abgekürzt einfach einen \textit{Satz} nennt) ist eine Struktur, in der alle Konstituenten mittelbar oder unmittelbar von einem nicht regierten finiten Verb abhängen, in dem alle Valenzanforderungen erfüllt sind.
\index{Satz}
}

Folglich ist (\ref{ex:satz0002a}) ein Satz, weil \textit{ist} ein finites Verb ist, das nicht regiert wird.
Außerdem sind offensichtlich alle Valenzanforderungen erfüllt.
Hingegen kann (\ref{ex:satz0002b}) kein Hauptsatz sein, weil zwar genau ein finites Verb vorkommt, dieses aber von \textit{dass} regiert wird.
Konstruktionen wie (\ref{ex:satz0002c}) und (\ref{ex:satz0002d}) können zwar als Äußerungen verwendet werden, aber in dem hier vertretenen Verständnis sind sie keine Sätze.
Aus Sicht der Grammatik ist diese Auffassung durchaus zielführend, weil beide Äußerungen deutlich anders strukturiert sind als (\ref{ex:satz0002a}).

\begin{exe}
  \ex\label{ex:satz0002} 
  \begin{xlist}
    \ex{\label{ex:satz0002a} Die Post ist da.}
    \ex{\label{ex:satz0002b} dass die Post da ist}
    \ex{\label{ex:satz0002c} Hurra!}
    \ex{\label{ex:satz0002d} Nieder mit dem König!}
  \end{xlist}
\end{exe}

Wichtig ist weiterhin der Begriff des \textit{Matrixsatzes}, der in der Beschreibung von Satz-Einbettungen gebräuchlich ist.

\Definition{Matrixsatz}{
\label{def:matrixsatz}
Der Matrixsatz eines Nebensatzes ist der Satz, in den er unmittelbar eingebettet ist.
\index{Matrixsatz}
\index{Matrixsatz}
} 

In Abschnitt~\ref{sec:felder} wird zunächst dafür argumentiert, dass man diese verschiedenen Satzgliedstellungen mittels \textit{Bewegung} von Konstituenten aus der in Abschnitt~\ref{sec:vgr} beschriebenen VP ableiten kann.
Außerdem wird das sogenannte \textit{Feldermodell} eingeführt, dass diese Satzgliedstellungen deskriptiv klassifiziert.
In Abschnitt~\ref{sec:satzschemata} werden dann Phrasenschemata für Sätze angegeben, die alle wichtigen Satzgliedstellungsvarianten beschreiben.
Schließlich wird in Abschnitt~\ref{sec:nebensaetze} auf Besonderheiten verschiedener Typen von Nebensätzen eingegangen.


\Zusammenfassung{
In einen unabhängigen Satz können satzartige Strukturen (\textit{Nebensätze}) eingebettet sein.
Der einbettende Satz ist der \textit{Matrixsatz} des eingebetteten Satzes.
}


\section{Satzgliedstellung und Feldermodell}

\label{sec:felder}

\subsection{Satzgliedstellung in unabhängigen Sätzen}

\index{Verbphrase}
\index{Konstituente}
Die in Abschnitt~\ref{sec:vgr} besprochene VP definiert die Abfolge der Konstituenten innerhalb der VP untereinander nicht exakt.
Dass man die Abfolge auch nicht spezifizieren muss, zeigen die Beispiele (\ref{ex:vgrorder}) aus Kapitel~\ref{sec:phrasen}, hier als (\ref{ex:vgrorder-rep}) wiederholt.

\begin{exe}
  \ex\label{ex:vgrorder-rep}
  \begin{xlist}
    \ex{\ThePhrasenExOne}
    \ex{\ThePhrasenExTwo}
  \end{xlist}
\end{exe}

\index{Verbkomplex}
Die Stellung des Verbkomplexes als Ganzes am rechten Ende der VP ist allerdings festgelegt.
Im unabhängigen Aussagesatz gilt dies nicht in der gleichen Form.
In (\ref{ex:v2fx}) sieht man an den beispielhaften Umformungen eines eingeleiteten Nebensatzes (\ref{ex:v2fxa}) in verschiedene uneingeleitete Sätze (also Hauptsätze), welche zahlreichen Umstellungen möglich sind, nämlich (\ref{ex:v2fxb})--(\ref{ex:v2fxf}).

\begin{exe}
  \ex\label{ex:v2fx}
  \begin{xlist}
    \ex{\label{ex:v2fxa} dass Ischariot wahrscheinlich dem Arzt das Bild verkauft hat}
    \ex{\label{ex:v2fxb} Ischariot hat wahrscheinlich dem Arzt das Bild verkauft.}
    \ex{\label{ex:v2fxc} Wahrscheinlich hat Ischariot dem Arzt das Bild verkauft.}
    \ex{\label{ex:v2fxd} Dem Arzt hat Ischariot wahrscheinlich das Bild verkauft.}
    \ex{\label{ex:v2fxe} Das Bild hat Ischariot wahrscheinlich dem Arzt verkauft.}
    \ex{\label{ex:v2fxf} Verkauft hat Ischariot wahrscheinlich dem Arzt das Bild.}
  \end{xlist}
\end{exe}

Wie bereits mehrfach angemerkt, steht hier das finite Verb immer an zweiter Stelle, und davor steht irgendein anderes Satzglied.
Die Optionen der Voranstellung aus (\ref{ex:v2fx}) werden durch die Voranstellung von komplexeren Satzteilen erweitert, von denen einige in (\ref{ex:v2fx2}) gezeigt werden.
Die vor das finite Verb gestellte Konstituente ist eingeklammert.

\begin{exe}
  \ex\label{ex:v2fx2}
  \begin{xlist}
    \ex{\label{ex:v2fxf2a} [Das Bild verkauft] hat Ischariot wahrscheinlich dem Arzt.}
    \ex{\label{ex:v2fxf2b} [Dem Arzt das Bild verkauft] hat Ischariot wahrscheinlich gestern.}
  \end{xlist}
\end{exe}

Im Vergleich zur VP ergeben sich mindestens zwei Unterschiede.
Einerseits wird das finite Verb alleine (auch wenn es aus einem Verbkomplex mit mehreren Verbformen kommt) nach links gestellt.
Sowohl die infiniten Verbformen als auch eventuelle Verbpartikeln (nicht aber Verbpräfixe) bleiben als Rest eines Verbkomplexes ohne finite Form am rechten Rand zurück.
Außerdem wird eine andere (in erster Näherung beliebige) Konstituente davor gestellt.
Beispiel (\ref{ex:vkv2inf}) zeigt das Zurückbleiben eines infiniten Verbs, und Beispiel (\ref{ex:vkv2prt}) zeigt das Zurückbleiben einer Verbpartikel.

\begin{exe}
  \ex\label{ex:vkv2inf}
  \begin{xlist}
    \ex{dass Ischariot dem Arzt das Bild verkauft hat}
    \ex{Ischariot hat dem Arzt das Bild verkauft.}
    \ex{Das Bild hat Ischariot dem Arzt verkauft.}
  \end{xlist}
  \ex\label{ex:vkv2prt}
  \begin{xlist}
    \ex{dass der Arzt Ischariot das Bild gerne abkauft}
    \ex{Der Arzt kauft Ischariot das Bild gerne ab.}
    \ex{Gerne kauft der Arzt Ischariot das Bild ab.}
  \end{xlist}
\end{exe}

\index{Bewegung}

Innerhalb der Nebensatz-VP ist die Reihenfolge der Teilkonstituenten nicht eindeutig festgelegt.
Immerhin ist aber klar, dass alle durch Statusrektion verbundenen Verben einschließlich des finiten Verbs rechts stehen, und dass sämtliche anderen Konstituenten in einer Kette links davon positioniert werden.  
Im unabhängigen Satz kommen nun die Schwierigkeiten hinzu, dass das finite Verb zwar eine festgelegte Stellung hat, dafür aber der Verbkomplex auseinandergerissen wird, und dass eine beliebige Konstituente außerhalb des VP-Zusammenhangs positioniert wird.
Trotz der Flexibilität der Konstituentenstellung ist die Struktur des eingeleiteten Nebensatzes (VP innerhalb einer KP) also einfacher systematisch zu beschreiben als die Struktur des Hauptsatzes.
Deswegen beschreiben wir die Konstituentenstellung des Hauptsatzes als Abweichung von der des Nebensatzes, und zwar der Anschaulichkeit halber als Umstellungen bzw.\ \textit{Bewegungen}.

Nehmen wir also an, wir hätten eine VP wie in Abbildung~\ref{fig:vgreinstelligwh} und sollten angeben, was sich im Vergleich zu dieser im unabhängigen Aussagesatz ändert.
Wir sprechen wie soeben erwähnt davon, dass Konstituenten \textit{bewegt} werden.
Einige Theorien wie \zB die \textit{Government and Binding Theory} (GB) oder das \textit{Minimalist Program} (MP) nehmen tatsächlich Bewegung im Sinne eines mehrstufigen Umbaus von Strukturen an.
Andere Theorien wie die \textit{Head-Driven Phrase Structure Grammar} (HPSG) modellieren dieselben Phänomene ohne solche Umbauoperationen, erzielen aber denselben Effekt.
Aus unserer deskriptiven Sicht ist der Begriff der \textit{Bewegung} in jedem Fall als Hilfsvorstellung zu betrachten, und wir benutzen ihn ohne theoretisch Partei nehmen zu wollen.
Es geht uns im Prinzip nur darum, die Strukturen im Haupt- und Nebensatz zueinander in Beziehung zu setzen.

\begin{figure}[!htbp]
  \resizebox{\textwidth}{!}{
    \Tree{
      &&&&&&&&&&&&& \K{VP}\B{ddllllllllllll}\B{ddllllllllll}\B{ddllllllll}\B{ddllllll}\B{ddllll}\B{ddll}\B{dd} \\
      \\
      & \K{NP}\TRi[-3] && \K{AdvP}\TRi[-2] && \K{NP}\TRi[-3] && \K{AdvP}\TRi[-4] && \K{NP}\TRi[-4] && \K{AdvP}\TRi[-4] && \K{\textbf{V}}\B{d} \\
      & \K{\textit{Ischariot}} && \K{\textit{wahrscheinlich}} && \K{\textit{dem Arzt}} && \K{\textit{heimlich}} && \K{\textit{das Bild}} && \K{\textit{schnell}} && \K{\textit{verkauft}} & \\
    }
  }
  \caption{VP mit dreistelliger Valenz und Adverbialen}
  \label{fig:vgreinstelligwh}
\end{figure}

Wir wissen, dass das finite Verb im Ergebnis an der zweiten Position im Satz stehen soll.
Statt in einem fertigen Satz oder einer fertigen VP die zweite Position zu suchen, gibt es eine einfachere Art, automatisch sicherzustellen, dass das finite Verb am Ende aller Umstellungen an zweiter Position steht und irgendein anderes Satzglied davor positioniert wird.
Man führt in dieser Reihenfolge die folgenden beiden Bewegungsoperationen an einer normalen VP durch:

\begin{enumerate}\Lf
  \item Stelle das finite Verb vor die VP.
  \item Stelle dann eine andere Konstituente vor das finite Verb.
\end{enumerate}

Die gegenüber Abbildung~\ref{fig:vgreinstelligwh} etwas vereinfachte VP \textit{Ischariot wahrscheinlich das Bild verkauft hat}, aus der gemäß diesen Anweisungen Konstituenten herausbewegt wurden, sieht aus wie in Abbildung~\ref{fig:movev2}.
Es ergibt sich ein unabhängiger Aussagesatz allein dadurch, dass erst das finite Verb \textit{hat} und dann die Konstituente \textit{das Bild} nach links gestellt wurde.

\begin{figure}[!htbp]
    \Tree{
      &&&&&&&&&& \K{VP}\B{ddlllll}\B{ddlll}\B{ddll}\B{d} \\
      &&&&&&&&&& \K{\textbf{V}}\B{dl}\B{d} \\
      & \K{NP\ORii}\TRi[-7] && \K{\textbf{V\ORi}}\B{d} && \K{NP}\TRi[-6] && \K{AdvP}\TRi[-2] & \K{\Tii}
      \POS[]-(0,4)\ar@{-->}@/^{4pc}/[lllllll]-(0,4) & \K{\textbf{V}}\B{d} & \K{\Ti}\POS[]-(0,4)\ar@{-->}@/^{4pc}/[lllllll]-(0,4) \\
      & \K{\textit{das Bild}} && \K{\textit{hat}} && \K{\textit{Ischariot}} && \K{\textit{wahrscheinlich}} && \K{\textit{verkauft}} & \\
    }
  \vspace{0.3cm}
  \caption{VP mit hinausbewegten Konstituenten}
  \label{fig:movev2}
\end{figure}

\index{Spur}

Wir verstehen die Stellung im Verb"=Zweit"=Satz (V2) also als das Ergebnis zweier Umstellungsoperationen bzw.\ Bewegungen.
Es bleibt eine VP mit zwei Lücken zurück, wobei diese Lücken in vielen Theorien als \textit{Trace} (engl.\ für \textit{Spur}) bezeichnet werden und daher meist als \textit{t} symbolisiert werden.
Wenn man die Lücken bzw.\ Spuren notiert und mit den dazugehörigen bewegten Konstituenten durchnumeriert, sind die Bewegungsoperationen auch am fertigen Umstellungsprodukt eindeutig nachvollziehbar.
Die gepunkteten Pfeile, die die Bewegung andeuten, sind dann im Prinzip nicht nötig und dienen hier nur der Verdeutlichung.

\subsection{Das Feldermodell}

\label{sec:feldermodell}

Unser Ziel ist es nun, diese Strukturen möglichst auch mit Phrasenschemata zu beschreiben, denn in Abbildung~\ref{fig:movev2} stehen die bewegten Konstituenten V\ORi\ und NP\ORii\ im syntaktischen Nichts.
Diese Beschreibung ist insofern problematisch, weil sie keine Baumstruktur ist (vgl.\ Abschnitt~\ref{sec:baumterminologie}) und wir sie in unserem Strukturformat daher nicht formulieren können.
In Abschnitt~\ref{sec:satzschemata} werden Satzschemata vorgestellt.

Vorher wird jetzt aber ein anderes Beschreibungsmodell eingeführt, das dabei hilft, die Regularitäten des Satzbaus im Deutschen zu verdeutlichen, bevor die phrasenstrukturelle Modellierung erfolgt.
Das sogenannte \textit{Feldermodell} liefert eine einfache Terminologie zur Beschreibung der sich durch den Bau der VP und die gerade besprochenen Umsortierungen der Konstituenten im unabhängigen Aussagesatz ergebenden Satzgliedstellungsvarianten.
Das Modell bezieht sich dabei nicht auf Konstituentenstrukturen, sondern nur auf die lineare Abfolge der Satzteile.

\Enl[2]

\Satz{Feldermodell}{
\label{satz:felder}
Das Feldermodell ist ein Beschreibungsmodell, das ohne Bezug auf die Phrasenstruktur die lineare Abfolge von Satzteilen beschreibt.
\index{Feldermodell}
}

\index{Satzklammer}
\index{Vorfeld}
\index{Mittelfeld}
\index{Komplementierer}

Die erste wichtige Idee des Feldermodells ist es, dass der Verbkomplex in allen Arten von Sätzen wegen seiner Stellung am rechten Rand (der VP) eine gut erkennbare rechte Grenze, die \textit{rechte Satzklammer} (RSK), bildet.
Zusätzlich gibt es in allen Arten von (abhängigen und unabhängigen) Sätzen eine gut erkennbare linke Begrenzung.
Im eingeleiteten Nebensatz (den wir als KP analysieren) steht der Komplementierer ganz links, und kein Satzglied des Nebensatzes darf links davon stehen.
Im unabhängigen Aussagesatz (ohne Komplementierer) steht das finite Verb links an zweiter Stelle (in unserer Terminologie links von der VP).
Wegen ihrer markanten Position im linken Satzbereich werden der Komplementierer und das links stehende finite Verb im unabhängigen Aussagesatz in der Terminologie des Feldermodells die sogenannte \textit{linke Satzklammer} (LSK) genannt.

Anhand der beiden Satzklammern kann man dann den Rest des Satzes stellungsmäßig aufteilen:
Das \textit{Vorfeld} (Vf) ist der Bereich links von der linken Satzklammer.
Das \textit{Mittelfeld} (Mf) ist der Bereich zwischen den Satzklammern.
Für den durch einen Komplementierer eingeleiteten Nebensatz und den unabhängigen Aussagesatz ergeben sich also die Einteilungen in Felder wie in Abbildung~\ref{fig:felder1}.

\begin{figure}[!htbp]
  \resizebox{\textwidth}{!}{
    \begin{tabular}{lp{0.1em}cp{0.1em}cp{0.1em}cp{0.1em}c}
      \textbf{Satztyp} && \textbf{Vf} && \textbf{LSK} && \textbf{Mf} && \textbf{RSK} \\
      \cmidrule{1-1}\cmidrule{3-3}\cmidrule{5-5}\cmidrule{7-7}\cmidrule{9-9}
      \textbf{unabh.\ Aussagesatz} && \textit{das Bild} && \textit{hat} && \textit{Ischariot} \textit{wahrscheinlich} && \textit{verkauft} \\
      \textbf{eingel.\ Nebensatz} &&&& \textit{dass} && \textit{Ischariot das Bild wahrscheinlich} && \textit{verkauft hat} \\
    \end{tabular}
  }
  \caption{Felder im unabhängigen Aussagesatz und im Nebensatz}
  \label{fig:felder1}
\end{figure}

\index{w-Frage}
\index{Echofrage}
\index{Fragesatz}
\index{w-Satz}

Das Feldermodell kann auch auf andere Satztypen angewandt werden.
Besonders sind hier der \textit{w}"=Fragesatz (\ref{ex:saetze2374a}), der Entscheidungsfragesatz (\ref{ex:saetze2374b}) und der Relativsatz (\ref{ex:saetze2374c}) bzw.\ der eingebettete \textit{w}"=Fragesatz (\ref{ex:saetze2374d}) zu behandeln.

\begin{exe}
  \ex\label{ex:saetze2374} 
  \begin{xlist}
    \ex{\label{ex:saetze2374a} Wem hat Ischariot das Bild verkauft?}
    \ex{\label{ex:saetze2374b} Hat Ischariot das Bild verkauft?}
    \ex{\label{ex:saetze2374c} Das ist der Arzt, dem Ischariot das Bild verkauft hat.}
    \ex{\label{ex:saetze2374d} Ischariot weiß, wer die guten Bilder verkauft.}
  \end{xlist}
\end{exe}

\label{abs:923478} Der \textit{w}"=Fragesatz stellt sich im Grunde wie ein unabhängiger Aussagesatz dar, wobei aber das Fragepronomen (\textit{w}"=Pronomen) oder eine größere Frage"=Konstituente (wie \textit{welchem dubiosen Arzt}) und nicht irgendeine frei wählbare Konstituente obligatorisch im Vorfeld steht.
Wenn die Frage"=Konstituente nicht im Vorfeld steht, ergibt sich eine sogenannte \textit{In-Situ-Frage} oder auch \textit{Echofrage} wie in (\ref{ex:saetze2375a}).
Der zugehörige Aussagesatz wird in (\ref{ex:saetze2375b}) zum Vergleich angegeben.

\begin{exe}
  \ex\label{ex:saetze2375}
  \begin{xlist}
    \ex{\label{ex:saetze2375a} Ischariot hat wem das Bild verkauft?}
    \ex{\label{ex:saetze2375b} Ischariot hat dem dubiosen Arzt das Bild verkauft.}
  \end{xlist}
\end{exe}

Bei einer solchen Frage bleibt das \textit{w}"=Pronomen an der Stelle, an der die korrespondierende Phrase im zugehörigen Aussagesatz stehen würde.
Ins Vorfeld wird dann in der In-Situ-Frage eine andere Konstituente gestellt (hier \zB \textit{Ischariot}).
Echofragen sind typisch in Kontexten, in denen der Fragende eine Verständnisfrage stellt, weil er das betreffende Satzglied \zB akustisch nicht verstanden hat.

Falls mehrere \textit{w}"=Pronomina (oder komplexe Frage"=Konstituenten) im \textit{w}"=Fragesatz vorkommen, muss (In-Situ-Fragen ausgenommen) eines von diesen in das Vorfeld gestellt werden, die anderen verbleiben in der VP.
Dies ist in (\ref{ex:saetze2377}) dargestellt.

\begin{exe}
  \ex\label{ex:saetze2377}
  \begin{xlist}
    \ex{Wem hat Ischariot was wie verkauft?} 
    \ex{Wie hat Ischariot wem was verkauft?} 
    \ex{Was hat Ischariot wem wie verkauft?} 
  \end{xlist}
\end{exe}

Die \textit{Entscheidungsfrage} in (\ref{ex:saetze2374b}) ist nur teilweise dem unabhängigen Aussagesatz ähnlich.
Das finite Verb wird ebenfalls nach links bewegt, allerdings entfällt die Besetzung des Vorfelds, und die linke Satzklammer (in Form des finiten Verbs) bildet die linke Grenze des Satzes.

\index{Relativsatz}
\index{Fragesatz!eingebettet}

Der Relativsatz und der eingebettete \textit{w}"=Fragesatz werden hier gemeinsam behandelt.
Dabei wird der Relativsatz exemplarisch besprochen, und der eingebettete \textit{w}"=Fragesatz ist strukturell völlig identisch.
Zur Verwendung des eingebetteten \textit{w}"=Fragesatzes s.\ Abschnitt~\ref{sec:komplementsaetze}.
Ein Relativsatz wie in (\ref{ex:saetze2374c}) ähnelt dem durch einen Komplementierer eingeleiteten Fragesatz insofern, als der Verbkomplex am rechten Rand intakt bleibt und das finite Verb nicht nach links bewegt wird.
Dafür wird das Relativpronomen (hier \textit{wem}) obligatorisch nach links bewegt und steht im Vorfeld.

Man kann nun die Satztypen wie in den Abbildungen~\ref{fig:feldertypen1} bis~\ref{fig:feldertypen4} zusammenfassen.

\begin{figure}[!htbp]
  \resizebox{\textwidth}{!}{
    \begin{tabular}{cp{0.1em}cp{0.1em}cp{0.1em}c}
      \textbf{Vf} && \textbf{LSK} && \textbf{Mf} && \textbf{RSK} \\
      \cmidrule{1-1}\cmidrule{3-3}\cmidrule{5-5}\cmidrule{7-7}
	eine Konstituente && finites Verb && (Rest) && infinite Verben \\
	\textit{das Bild} && \textit{hat} && \textit{Ischariot wahrscheinlich} && \textit{verkauft} \\
    \end{tabular}
  }
  \caption{Feldermodell: unabhängiger Aussagesatz (V2)}
  \label{fig:feldertypen1}
\end{figure}

\begin{figure}[!htbp]
  \resizebox{\textwidth}{!}{
    \begin{tabular}{cp{0.1em}cp{0.1em}cp{0.1em}c}
      \textbf{Vf} && \textbf{LSK} && \textbf{Mf} && \textbf{RSK} \\
      \cmidrule{1-1}\cmidrule{3-3}\cmidrule{5-5}\cmidrule{7-7}
	(leer) && Komplementierer && (Rest) && Verbkomplex \\
	&& \textit{dass} && \textit{Ischariot das Bild wahrscheinlich} && \textit{verkauft hat} \\
    \end{tabular}
  }
  \caption{Feldermodell: Nebensatz mit Komplementierer (VL)}
  \label{fig:feldertypen2}
\end{figure}

\begin{figure}[!htbp]
    \begin{tabular}{cp{0.1em}cp{0.1em}cp{0.1em}c}
      \textbf{Vf} && \textbf{LSK} && \textbf{Mf} && \textbf{RSK} \\
      \cmidrule{1-1}\cmidrule{3-3}\cmidrule{5-5}\cmidrule{7-7}
	(leer) && finites Verb && (Rest) && infinite Verben \\
	&& \textit{hat} && \textit{Ischariot das Bild} && \textit{verkauft} \\
    \end{tabular}
  \caption{Feldermodell: Entscheidungsfragesatz (V1)}
  \label{fig:feldertypen3}
\end{figure}

\begin{figure}[!htbp]
  \resizebox{\textwidth}{!}{
    \begin{tabular}{cp{0.1em}cp{0.1em}cp{0.1em}c}
      \textbf{Vf} && \textbf{LSK} && \textbf{Mf} && \textbf{RSK} \\
      \cmidrule{1-1}\cmidrule{3-3}\cmidrule{5-5}\cmidrule{7-7}
	Relativpronomen && (leer) && (Rest) && Verbkomplex \\
	\textit{dem} &&&& \textit{Ischariot das Bild wahrscheinlich} && \textit{verkauft hat} \\ 
    \end{tabular}
  }
  \caption{Feldermodell: Relativsatz (VL)}
  \label{fig:feldertypen4}
\end{figure}

\index{Verb-Erst-Satz}
\index{Verb-Letzt-Satz}
\index{Verb-Zweit-Satz}

Für die grundlegenden Satztypen gibt es konkurrierende Bezeichnungen.
Meistens werden sie nach der Stellung des finiten Verbs kategorisiert.
Man spricht dann vom \textit{Verb"=Erst"=Satz} oder \textit{V1-Satz} (Entscheidungsfragesatz), vom \textit{Verb"=Zweit"=Satz} oder \textit{V2-Satz} (unabhängiger Aussagesatz und \textit{w}"=Fragesatz) und vom \textit{Verb"=Letzt"=Satz} oder \textit{VL-Satz} (eingeleiteter Nebensatz und Relativsatz).
All diese Bezeichnungen kategorisieren die Sätze nach der Art, wie die vier primären Positionen Vorfeld, linke Satzklammer, Mittelfeld und rechte Satzklammer gefüllt werden.
Neben diesen vier werden noch mindestens zwei weitere Felder angenommen.
Zunächst betrachten wir Sätze wie die in (\ref{ex:saetze2381}).

\begin{exe}
  \ex\label{ex:saetze2381}
  \begin{xlist}
    \ex{\label{ex:saetze2381a} Ischariot hat dem Arzt das Bild verkauft, das er selber gemalt hatte.}
    \ex{\label{ex:saetze2381b} Der Arzt hat Ischariot nicht geglaubt, dass das Bild echt war.}
  \end{xlist}
\end{exe}

\index{Relativsatz}
\index{Komplementsatz}

In diesen Sätzen stehen einmal ein Relativsatz (\ref{ex:saetze2381a}) und einmal ein Komplementsatz (\ref{ex:saetze2381b}) nach dem infiniten Verb.
Im Fall des Relativsatzes kann man besonders gut erkennen, dass dieser nach rechts bewegt wurde, denn die NP, zu der er strukturell gehört (\textit{das Bild}), befindet sich im Mittelfeld, und dadurch sind die NP und der zugehörige Relativsatz durch die rechte Satzklammer (\textit{verkauft}) voneinander getrennt.
Man geht im Falle solcher rechts von der rechten Satzklammer positionierten Konstituenten davon aus, dass sie wegen ihrer Länge aus dem Mittelfeld herausbewegt (\textit{rechtsversetzt}) werden.
Im Rahmen des Feldermodells nennt man die entsprechende Position das \textit{Nachfeld} (Nf).
Eine Analyse wird in Abbildung~\ref{fig:nachfeld} gegeben.

\index{Nachfeld}

\begin{figure}[!htbp]
  \centering
  \resizebox{\textwidth}{!}{
    \begin{tabular}{cp{0.1em}cp{0.1em}cp{0.1em}cp{0.1em}c}
      \textbf{Vf} && \textbf{LSK} && \textbf{Mf} && \textbf{RSK} && \textbf{Nf} \\
      \cmidrule{1-1}\cmidrule{3-3}\cmidrule{5-5}\cmidrule{7-7}\cmidrule{9-9}
      \textit{Ischariot} && \textit{hat} && \textit{dem Arzt das Bild} && \textit{verkauft} && \textit{das er selber gemalt hatte} \\
    \end{tabular}
  }
  \caption{Felderanalyse mit Nachfeld}
  \label{fig:nachfeld}
\end{figure}

Außerdem gibt es vermeintliche Komplementierer wie \textit{denn}, die sich aber anders als echte Komplementierer verhalten, vgl.\ (\ref{ex:saetze1924}).

\begin{exe}
  \ex\label{ex:saetze1924}
  \begin{xlist}
    \ex{\label{ex:saetze1924a} Der Arzt ist froh, weil Ischariot ihm das Bild verkauft hat.}
    \ex{\label{ex:saetze1924b} Der Arzt ist froh, denn Ischariot hat ihm das Bild verkauft.}
  \end{xlist}
\end{exe}

\index{Konnektor}
\index{Konnektorfeld}

Das Wort \textit{denn} muss gemäß unserer Wortklassifikation als Partikel (nicht etwa als Komplementierer) klassifiziert werden, denn es bettet keinen Nebensatz mit Verb-Letzt-Stellung ein.
Nach \textit{denn} steht ein Satz, der wie ein unabhängiger Aussagesatz (V2) strukturiert ist.
Solche Partikeln nennt man auch \textit{Konnektoren}, und man kann innerhalb des Feldermodells für sie ein \textit{Konnektorfeld} (Kf) oder \textit{Vor-Vorfeld} ansetzen, das noch vor dem Vorfeld positioniert ist.
Eine solche Analyse ist in Abbildung~\ref{fig:konnektorfeld} angegeben.

\begin{figure}[!htbp]
  \centering
  \begin{tabular}{cp{0.1em}cp{0.1em}cp{0.1em}cp{0.1em}c}
    \textbf{Kf} && \textbf{Vf} && \textbf{LSK} && \textbf{Mf} && \textbf{RSK} \\
    \cmidrule{1-1}\cmidrule{3-3}\cmidrule{5-5}\cmidrule{7-7}\cmidrule{9-9}
    \textit{denn} && \textit{Ischariot} && \textit{hat} && \textit{ihm das Bild} && \textit{verkauft} \\
  \end{tabular}
  \caption{Felderanalyse mit Konnektorfeld}
  \label{fig:konnektorfeld}
\end{figure}

Zwei typische Strukturen werden abschließend in Abbildung~\ref{fig:leerefelder} und Abbildung~\ref{fig:verbpartikelalleinzuhaus} gezeigt.
Abbildung~\ref{fig:leerefelder} stellt einen kurzen Hauptsatz dar, in dem die meisten Felder leer bleiben.
Abbildung~\ref{fig:verbpartikelalleinzuhaus} illustriert eine abgetrennte Verbpartikel, die in der rechten Satzklammer steht, während das finite Verb in der linken Satzklammer steht.

\begin{figure}[!htbp]
  \centering
  \begin{tabular}{cp{0.1em}cp{0.1em}cp{0.1em}cp{0.1em}cp{0.1em}c}
    \textbf{Kf} && \textbf{Vf} && \textbf{LSK} && \textbf{Mf} && \textbf{RSK} && \textbf{Nf} \\
    \cmidrule{1-1}\cmidrule{3-3}\cmidrule{5-5}\cmidrule{7-7}\cmidrule{9-9}\cmidrule{11-11}
    && \textit{Ischariot} && \textit{malt} &&&&&& \\
  \end{tabular}
  \caption{Felderanalyse eines V2-Satzes mit leeren Feldern}
  \label{fig:leerefelder}
\end{figure}

\begin{figure}[!htbp]
  \centering
  \begin{tabular}{cp{0.1em}cp{0.1em}cp{0.1em}cp{0.1em}cp{0.1em}c}
    \textbf{Kf} && \textbf{Vf} && \textbf{LSK} && \textbf{Mf} && \textbf{RSK} && \textbf{Nf} \\
    \cmidrule{1-1}\cmidrule{3-3}\cmidrule{5-5}\cmidrule{7-7}\cmidrule{9-9}\cmidrule{11-11}
    && \textit{Ischariot} && \textit{fährt} && \textit{den Pfosten} && \textit{um=} & \\
  \end{tabular}
  \caption{Felderanalyse eines V2-Satzes mit Verbpartikel}
  \label{fig:verbpartikelalleinzuhaus}
\end{figure}

\subsection{LSK-Test und Nebensätze}

\label{sec:lsktest}

In Abschnitt~\ref{sec:konstituententestsimeinzelnen} wurde im Zusammenhang mit dem Vorfeldtest darauf verwiesen, dass es nicht immer trivial ist, die linke Satzklammer (und damit das Vorfeld) zu identifizieren.
Das Problem rührt daher, dass je nach Satzstruktur das erste finite Verb auch das Verb eines eingebetteten Nebensatzes sein kann, wenn dieser Nebensatz \zB im Vorfeld eines anderen Satzes steht.
Die Beispiele in (\ref{ex:syn3334w}) illustrieren das Problem.
In (\ref{ex:syn3334wa}) sind sowohl \textit{glaubt} als auch \textit{haben} finit, in (\ref{ex:syn3334wb}) kommt \textit{irrt} hinzu.

\begin{exe}
  \ex\label{ex:syn3334w}
  \begin{xlist}
    \ex{\label{ex:syn3334wa} [Wer] glaubt, dass Tiere im Tierheim ein schönes Leben haben?}
    \ex{\label{ex:syn3334wb} [Wer glaubt, dass Tiere im Tierheim ein schönes Leben haben], irrt.}
  \end{xlist}
\end{exe}

\index{w-Satz}
Das Feldermodell ermöglicht nun sowohl Analysen der Struktur des Matrixsatzes als auch der eingebetteten Nebensätze.
Um systematisch die Analyse des Matrixsatzes und der Nebensätze durchzuführen, muss zunächst die linke Satzklammer des Matrixsatzes (also der äußeren Struktur) gefunden werden.
Eine Testprozedur zur Ermittlung der linken Satzklammer des Matrixsatzes macht sich zunutze, dass in der Entscheidungsfrage immer das finite Verb des Matrixsatzes an erster Stelle steht.
Entweder ist ein Satz also bereits eine Entscheidungsfrage (oder \textit{w}"=Frage, s.\,u.) und wir müssen ihn nur noch als solche erkennen.
Andernfalls muss der Satz in eine Entscheidungsfrage umformuliert werden.
In beiden Fällen steht dann das finite Verb des Matrixsatzes an erster Stelle.
Nehmen wir zunächst einen einfacheren Satz wie (\ref{ex:saetze9292}).

\begin{exe}
  \ex{\label{ex:saetze9292} Der Maler hat dem Arzt ein Bild geschenkt, das jetzt in der Praxis hängt.}
\end{exe}

Entweder \textit{hat} oder \textit{hängt} muss das finite Verb des Matrixsatzes sein.
Da jede Satzstruktur (ob unabhängig oder abhängig) genau ein finites Verb enthält, zu dem alle weiteren Konstituenten dependent sind, muss das andere zu einem tiefer eingebetteten Nebensatz gehören.
Formuliert man (\ref{ex:saetze9292}) in eine Entscheidungsfrage um, erkennt man, dass \textit{hat} das finite Verb des unabhängigen Satzes sein muss, s.\ (\ref{ex:satze9293}).

\begin{exe}
  \ex{\label{ex:satze9293} Hat der Maler dem Arzt ein Bild geschenkt, das jetzt in der Praxis hängt?}
\end{exe}

In der Umformung steht \textit{hat} am Satzanfang, und es kann daher geschlossen werden, dass es in (\ref{ex:saetze9292}) die linke Satzklammer besetzt.
Dass man mit dem Test die richtige Frage produziert hat, erkennt man daran, dass der ursprüngliche Satz mit vorangestelltem \textit{Ja} eine adäquate (wenn auch umständliche) positive Antwort wäre, hier also (\ref{ex:saetze9292ant}).

\begin{exe}
  \ex{\label{ex:saetze9292ant} Ja, der Maler hat dem Arzt ein Bild geschenkt, das jetzt in der Praxis hängt.}
\end{exe}

Wenn wir nun auf (\ref{ex:syn3334w}) zurückkommen, gilt es zunächst zu beachten, dass (\ref{ex:syn3334wa}) bereits eine \textit{w}"=Frage ist.
Insofern ist \textit{glaubt} prinzipiell ohne Umstellung als finites Verb (LSK) zu identifizieren, denn im \textit{w}"=Fragesatz ist das Vorfeld immer mit dem \textit{w}"=Pro\-no\-men (hier \textit{wer}) besetzt.
Die Umformung in eine Entscheidungsfrage ergibt das gleiche Ergebnis, wobei allerdings ein Pronomen ausgetauscht werden muss, nämlich hier \textit{wer} zu einem Pronomen wie \textit{irgendjemand}, s.\ (\ref{ex:saetze9294}).

\begin{exe}
  \ex{\label{ex:saetze9294} Glaubt irgendjemand, dass Tiere im Tierheim ein schönes Leben haben?}
\end{exe}

Sätze wie der in (\ref{ex:syn3334wb}) sind insofern schwierig, als im Vorfeld hier ein sogenannter freier Relativsatz [\textit{wer \ldots\ haben}] (vgl.\ Abschnitt~\ref{sec:relativsaetze}) steht, der wiederum einen Komplementsatz [\textit{dass \ldots\ haben}] (vgl.\ Abschnitt~\ref{sec:komplementsaetze}) enthält.
Das finite Verb des Matrixsatzes ist dadurch das insgesamt dritte, nämlich \textit{irrt}.
Bei der Umformung in eine Entscheidungsfrage müssen nun Pronomina ausgetauscht und hinzugefügt werden, um den Satz völlig akzeptabel zu machen.
Die einfache Umstellung (ohne Austausch und Ergänzung von Pronomina), die bezüglich ihrer Grammatikalität etwas fragwürdig ist, findet sich in (\ref{ex:saetze1426a}), die völlig akzeptable Version (mit Austausch\slash Ergänzung von Pronomina) in (\ref{ex:saetze1426b}).
Mit dieser Umformung in eine Entscheidungsfrage ist also auch hier das richtige finite Verb zu identifizieren.

\begin{exe}
  \ex\label{ex:saetze1426}
  \begin{xlist}
    \ex{\label{ex:saetze1426a} Irrt, wer glaubt, dass Tiere im Tierheim ein schönes Leben haben?}
    \ex{\label{ex:saetze1426b} Irrt derjenige, der glaubt, dass Tiere im Tierheim ein schönes Leben haben?}
  \end{xlist}
\end{exe}

Wie oben angedeutet, kann für den unabhängigen Satz (Matrixsatz) und die abhängigen Sätze (Nebensätze) je eine Felderanalyse durchgeführt werden.
Im Fall von Nebensätzen ist sozusagen eine vollständige Felderstruktur in eine andere eingebettet.
Für (\ref{ex:syn3334wa}) sieht das aus wie in den Abbildung~\ref{fig:syn3334wa1} (Matrixsatz) und Abbildung~\ref{fig:syn3334wa2} (Nebensatz).
Für (\ref{ex:syn3334wb}) erhalten wir \ref{fig:syn3334wb1} (Matrixsatz) und \ref{fig:syn3334wb2} (Nebensatz).

\begin{figure}[!htbp]
  \centering
  \resizebox{\textwidth}{!}{
  \begin{tabular}{cp{0.1em}cp{0.1em}cp{0.1em}cp{0.1em}c}
    \textbf{Vf} && \textbf{LSK} && \textbf{Mf} && \textbf{RSK} && \textbf{Nf} \\
    \cmidrule{1-1}\cmidrule{3-3}\cmidrule{5-5}\cmidrule{7-7}\cmidrule{9-9}
    \textit{Wer} && \textit{glaubt} && && && \textit{dass Tiere im Tierheim ein schönes Leben haben} \\
  \end{tabular}
  }
  \caption{Felderanalyse eines V2-Satzes mit Nebensatz}
  \label{fig:syn3334wa1}
\end{figure}

\begin{figure}[!htbp]
  \centering
  \begin{tabular}{cp{0.1em}cp{0.1em}cp{0.1em}c}
    \textbf{Vf} && \textbf{LSK} && \textbf{Mf} && \textbf{RSK} \\
    \cmidrule{1-1}\cmidrule{3-3}\cmidrule{5-5}\cmidrule{7-7}
    && \textit{dass} && \textit{Tiere im Tierheim ein schönes Leben} && \textit{haben} \\
  \end{tabular}
  \caption{Felderanalyse für den Nebensatz aus Abbildung~\ref{fig:syn3334wa1}}
  \label{fig:syn3334wa2}
\end{figure}

\begin{figure}[!htbp]
  \centering
  \resizebox{\textwidth}{!}{
    \begin{tabular}{cp{0.1em}cp{0.1em}cp{0.1em}c}
      \textbf{Vf} && \textbf{LSK} && \textbf{Mf} && \textbf{RSK} \\
      \cmidrule{1-1}\cmidrule{3-3}\cmidrule{5-5}\cmidrule{7-7}
      \textit{Wer glaubt, dass Tiere im Tierheim ein schönes Leben haben} && \textit{irrt} &&&& \\
    \end{tabular}
  }
  \caption{Felderanalyse eines V2-Satzes mit komplexem Vorfeld}
  \label{fig:syn3334wb1}
\end{figure}

\begin{figure}[!htbp]
  \centering
  \resizebox{\textwidth}{!}{
    \begin{tabular}{cp{0.1em}cp{0.1em}cp{0.1em}cp{0.1em}c}
      \textbf{Vf} && \textbf{LSK} && \textbf{Mf} && \textbf{RSK} && \textbf{Nf} \\
      \cmidrule{1-1}\cmidrule{3-3}\cmidrule{5-5}\cmidrule{7-7}\cmidrule{9-9}
      \textit{wer} &&&&&& \textit{glaubt} && \textit{dass Tiere im Tierheim ein schönes Leben haben} \\
    \end{tabular}
  }
  \caption{Felderanalyse des komplexen Vorfelds aus Abbildung~\ref{fig:syn3334wb1}}
  \label{fig:syn3334wb2}
\end{figure}


\Zusammenfassung{
Im \textit{unabhängigen Aussagesatz} steht das finite Verb nicht im Verbalkomplex, sondern an zweiter Stelle nach einer (fast) beliebig wählbaren anderen Konstituente (\textit{V2-Satz}).
Ein unabhängiger Aussagesatz kann als VP betrachtet werden, aus dem zuerst das finite Verb und dann eine andere Konstituente \textit{herausbewegt} wurde.
Das \textit{Feldermodell} bietet für diese und andere Satzstrukturen eine Oberflächenbeschreibung an, die mit unserer phrasenstrukturellen Darstellung aber nicht direkt kompatibel ist.
In \textit{Entscheidungsfragesätzen} steht das finite Verb an erster Stelle (\textit{V1-Satz}).
}

 
\section{Schemata für Sätze}

\label{sec:satzschemata}
\label{sec:verbzweitsatz}

\subsection{Verb-Zweit-Sätze}

\label{sec:konstituentenstrukturinv2}

Der Bau der Phrasen (Kapitel~\ref{sec:phrasen}) ist geprägt von einer reichen internen Struktur und von Valenz- und Rektionsbeziehungen.
Das Feldermodell hingegen ist ein von diesem Phrasenbau unabhängiges reines Linearisierungsmodell, also eine Beschreibung der Abfolge von Satzteilen, ohne dass deren Struktur weiter betrachtet wird.
Das ist der Grund, warum das Feldermodell die üblicherweise angenommene Konstituentenstruktur nicht direkt nachbilden kann.
Die Beziehung zwischen Feldermodell und Konstituentenstruktur wird daher jetzt verdeutlicht.
Dabei muss berücksichtigt werden, dass die beiden Beschreibungsmodelle (Feldermodell und Phrasenstruktur) zwar denselben Gegenstand beschreiben (deutsche Syntax), aber eigentlich konkurrierende und nicht vereinbare Modelle darstellen.
Beide sind ausgesprochen populär, und man kann sie (wie es jetzt hier geschehen wird) miteinander vergleichen, aber in einem Phrasenstrukturbaum haben Felderbezeichnungen nichts verloren, genauso wie in einer Felderanalyse Phrasenbezeichnungen nichts verloren haben.

Beginnen wir damit, unter einer Konstituentenanalyse eines V2-Satzes (inklusive Bewegung) zu notieren, welche Felder den Konstituenten entsprechen.
In Abbildung~\ref{fig:movev2konstituenten} geschieht dies durch die Boxen mit den Namen der Felder unter dem Baum mit den herausbewegten Konstituenten (vgl.\ schon Abbildung~\ref{fig:movev2}).
Offensichtlich können bestimmte Knoten im Strukturbaum der VP und die herausgestellten Konstituenten bestimmten Feldern des Feldermodells zugeordnet werden.
Das Vorfeld und die linke Satzklammer entsprechen den herausbewegten Konstituenten, das Mittelfeld entspricht der VP ohne Verbkomplex, und die rechte Satzklammer entspricht dem Verbkomplex.
Weil der Rest-Verbkomplex aber eben eine Teilkonstituente der VP ist, können wir das Feldermodell phrasenstrukturell nicht genau nachbilden.
Sobald wir sagen, \textit{die VP entspreche dem Mittelfeld}, machen wir den Verbkomplex zum Teil des Mittelfelds, obwohl er eigentlich ein eigenes Feld bildet.
Eine hierarchische Phrasenstruktur und das Feldermodell passen nicht wirklich zueinander, und wir versuchen daher jetzt ein rein phrasenstrukturelles Modell des unabhängigen Aussagesatzes zu erarbeiten, das die Grammatik aus Kapitel~\ref{sec:konstituentenstruktur} und Kapitel~\ref{sec:phrasen} vervollständigt.

\begin{figure}[!htbp]
  \resizebox{\textwidth}{!}{
    \Tree{
      &&&&&&&&&&& \K{VP}\B{ddllllll}\B{ddllll}\B{ddlll}\B{d} \\
      &&&&&&&&&&& \K{\textbf{V}}\B{dll}\B{d} \\
      & \K{NP\ORii}\TRi[-7] && \K{\textbf{V\ORi}}\B{d} && \K{NP}\TRi[-6] && \K{AdvP}\TRi[-2] & \K{\Tii} & \K{\textbf{V}}\B{d} && \K{\Ti} \\
      & \K{\textit{das Bild}} && \K{\textit{hat}} && \K{\textit{Ischariot}} && \K{\textit{wahrscheinlich}} && \K{\textit{verkauft}} && \\
      & \K{Vf} && \K{LSK} &&& \K{Mf} &&&& \K{RSK} & \\
      \QS{5,2}{5,2}
      \QS{5,4}{5,4}
      \QS{5,6}{5,9}
      \QS{5,10}{5,12}
    }
  }
  \caption{Zuordnung der Felder zu Konstituenten (V2)}
  \label{fig:movev2konstituenten}
\end{figure}

\begin{figure}[!htbp]
  \resizebox{\textwidth}{!}{
    \Tree{
      &&& \K{S}\B{dddll}\B{ddd}\B{drrrrrrr} \\
      &&&&&&&&&& \K{VP}\B{ddlllll}\B{ddlll}\B{ddll}\B{d} \\
      &&&&&&&&&& \K{\textbf{V}}\B{dl}\B{d} \\
      & \K{NP\ORii}\TRi[-7] && \K{\textbf{V\ORi}}\B{d} && \K{NP}\TRi[-6] && \K{AdvP}\TRi[-2] & \K{\Tii}\POS[]-(0,4)\ar@{-->}@/^{4pc}/[lllllll]-(0,4) & \K{\textbf{V}}\B{d} & \K{\Ti}\POS[]-(0,4)\ar@{-->}@/^{4pc}/[lllllll]-(0,4) \\
      & \K{\textit{das Bild}} && \K{\textit{hat}} && \K{\textit{Ischariot}} && \K{\textit{wahrscheinlich}} && \K{\textit{verkauft}} & \\
    }
  }
  \vspace{0.3cm}
  \caption{V2-Satz}
  \label{fig:v2satz}
\end{figure}

\index{Verb-Zweit-Satz}
\index{Bewegung}
\index{Spur}

Die angestrebte Konstituentenstrukturanalyse eines V2-Satzes sieht aus wie in Abbildung~\ref{fig:v2satz}.
Ein unabhängiger Aussagesatz (Symbol S) wird hier als eine zusammenhängende Konstituente analysiert.
Die Konstituenten, die man im Feldermodell dem Mittelfeld und der rechten Satzklammer zuordnet, entsprechen den Resten der VP und des Verbkomplexes.
Die Bewegung des finiten Verbs in die zweite Position in S entspricht der Besetzung der linken Satzklammer.
Die Bewegung einer beliebigen Phrase (wobei für \textit{beliebige Phrase} üblicherweise XP geschrieben wird) in die linke Position von S entspricht der Besetzung des Vorfelds.
Das Schema, das diese Konstituentenstruktur erzeugt, muss die Anforderungen kodieren, dass eine VP mit zwei Spuren (der Spur des finiten Verbs und der des Vorfeldbesetzers) sich mit den Konstituenten verbindet, die in einer Nebensatz-VP an der Stelle der Spuren stünden.

\Phrasenschema{V2-Satz}{\label{str:v2}
  \begin{tabular}{c|c|c|c|}
    \cline{2-4}
    S = & XP\ORii & [\textsc{Tempus}]\ORi & VP[\ldots \Tii \ldots \Ti] \\
    \cline{2-4}
  \end{tabular}
}

Im Schema~\ref{str:v2} wird die Notation VP[\ldots\Tii\ldots\Ti] verwendet, um anzuzeigen, dass eine VP mit zwei Spuren eingesetzt werden muss, egal was die VP sonst noch enthält.%
\footnote{Die Abkürzung \ldots\ zeigt an, dass an ihrer Stelle beliebiges Material stehen kann.}
Die aus den Positionen der Spuren herausbewegten Konstituenten werden vorne in die S-Struktur eingefügt.
Über die Konstituente, die zu Spur \Ti\ gehört, wird gesagt, dass sie für [\textsc{Tempus}] spezifiziert sein soll, also gemäß Definition~\ref{def:finitheit} von S.~\pageref{def:finitheit} und Filter \ref{wfilt:verbennomina} ein finites Verb sein muss.
Das Feldermodell kann also vollständig durch eine sehr einfache phrasenstrukturelle Analyse ersetzt werden.

Abschließend sei angemerkt, dass nicht immer davon ausgegangen wird, dass alle Vorfeldbesetzer aus dem Mittelfeld herausbewegt werden.
Adverbiale wie \textit{erfreulicherweise} \zB könnten auch ohne Weiteres direkt in S eingefügt werden, sofern die VP nur die Spur \Ti\ mit dem finiten Verb enthält.
Das sähe dann so aus wie in Abbildung~\ref{fig:vorfelddirektbesetzung}.%
\footnote{Das Schema für S müsste angepasst werden, um auch diesen Fall zu beschreiben.}
Wir gehen hier in den besprochenen Sätzen immer von Bewegung aus, nicht ohne darauf hinzuweisen, dass dies eine Übersimplifizierung sein könnte.

\begin{figure}[!htbp]
  \centering
  \resizebox{\textwidth}{!}{
    \Tree{
      &&& \K{S}\B{dddll}\B{ddd}\B{drrrrrrrr} \\
      &&&&&&&&&&& \K{VP}\B{ddllllll}\B{ddllll}\B{d} \\
      &&&&&&&&&&& \K{\textbf{V}}\B{dll}\B{d} \\
      & \K{AdvP}\TRi[2] && \K{\textbf{V\ORi}}\B{d} && \K{NP}\TRi[-4] && \K{NP}\TRi[-4] && \K{\textbf{V}}\B{d} && \K{\Ti}\POS[]-(0,4)\ar@{-->}@/^{4pc}/[llllllll]-(0,4) \\
      & \K{\textit{Erfreulicherweise}} && \K{\textit{hat}} && \K{\textit{Ischariot}} && \K{\textit{das Bild}} && \K{\textit{verkauft}} && \K{} \\
    }
  }
  \vspace{0.3cm}
  \caption{Konstituentenanalyse bei direkter Vorfeldbesetzung}
  \label{fig:vorfelddirektbesetzung}
\end{figure}

Damit haben wir ein Modell der Satzgliedstellung im eingeleiteten Nebensatz (normale KP, vgl.\ Abschnitt~\ref{sec:subjgr}) und im V2-Satz (unabhängiger Aussagesatz, Schema~\ref{str:v2}).
Der \textit{w}"=Fragesatz benötigt kein eigenes Schema, denn er ist lediglich eine Variante des V2-Aussagesatzes.
Die Spur \Tii\ muss dabei immer eine \textit{w}"=Pronomen-Spur sein, wie die Analyse in \ref{fig:v2fragesatz} zeigt.
Im nächsten Abschnitt wird ein Schema für den V1-Entscheidungsfragesatz eingeführt.

\begin{figure}[!htbp]
  \centering
  \Tree{
    &&& \K{S}\B{dddll}\B{ddd}\B{drrrrrrr} \\
    &&&&&&&&&& \K{VP}\B{ddlllll}\B{ddlll}\B{ddll}\B{d} \\
    &&&&&&&&&& \K{\textbf{V}}\B{dl}\B{d} \\
    & \K{NP\ORii}\TRi[-7] && \K{\textbf{V\ORi}}\B{d} && \K{NP}\TRi[-6] && \K{AdvP}\TRi[-2] & \K{\Tii}\POS[]-(0,4)\ar@{-->}@/^{4pc}/[lllllll]-(0,4) & \K{\textbf{V}}\B{d} & \K{\Ti}\POS[]-(0,4)\ar@{-->}@/^{4pc}/[lllllll]-(0,4) \\
    & \K{\textit{Was}} && \K{\textit{hat}} && \K{\textit{Ischariot}} && \K{\textit{wahrscheinlich}} && \K{\textit{verkauft}} & \\
  }
  \vspace{0.3cm}
  \caption{V2-\textit{w}"=Fragesatz}
  \label{fig:v2fragesatz}
\end{figure}


\subsection{Verb-Erst-Sätze}

\label{sec:verberstsatz}

\index{Verb-Erst-Satz}
\index{Fragesatz!Entscheidungs--}

\Phrasenschema{V1-Satz}{\label{str:v1}
  \begin{tabular}{c|c|c|}
    \cline{2-3}
    FS = & [\textsc{Tempus}]\ORi & VP[\ldots \Ti] \\
    \cline{2-3}
  \end{tabular}
}

\begin{sloppypar}
V1-Fragesätze (FS) sind denkbar einfach zu beschreiben, nachdem wir bereits V2-Sätze analysiert haben.
Einen Satz wie (\ref{ex:saetze6661}) erklärt Schema~\ref{str:v1}, s.\ Abbildung~\ref{fig:v1satz}.
\end{sloppypar}

\begin{exe}
  \ex{\label{ex:saetze6661} Hat Ischariot tatsächlich das Bild verkauft?}
\end{exe}

\begin{figure}[!htbp]
  \centering
  \Tree{
    \K{FS}\B{ddd}\B{drrrrrrrrr} \\
    &&&&&&&&& \K{VP}\B{ddlllllll}\B{ddlllll}\B{ddlll}\B{d} \\
    &&&&&&&&& \K{\textbf{V}}\B{dl}\B{d} \\
    \K{\textbf{V\ORi}}\B{d} && \K{NP}\TRi[-4] && \K{AdvP}\TRi[-3] && \K{NP}\TRi[-6] && \K{\textbf{\textbf{V}}}\B{d} & \K{\Ti}\POS[]-(0,4)\ar@{-->}@/^{4pc}/[lllllllll]-(0,4) \\
    \K{\textit{Hat}} && \K{\textit{Ischariot}} && \K{\textit{tatsächlich}} && \K{\textit{das Bild}} && \K{\textit{verkauft}} & \\
  } 
  \vspace{0.3cm}
  \caption{Entscheidungsfragesatz}
  \label{fig:v1satz}
\end{figure}

Es entfällt bei dem V1-Satz lediglich die Bewegung der zweiten Konstituente nach der Bewegung des finiten Verbs.
Nur das finite Verb muss nach links gestellt werden, und das Schema ist damit einfacher als das V2-Schema.
Es bleibt anzumerken, dass wir hier die Bezeichnung FS mehr oder weniger informell benutzen.
Mit der Beschriftung FS wird die Information kodiert, dass es sich um einen Fragesatz handelt.

\index{Imperativ!Satz}
Imperative wie in (\ref{ex:saetze1557}) sind im Prinzip wie V1-Sätze strukturiert.

\begin{exe}
  \ex{\label{ex:saetze1557} Verkaufe das Bild.}
\end{exe}

Es sei allerdings darauf hingewiesen, dass wir in Abschnitt~\ref{sec:impflex} morphologisch argumentiert haben, dass imperativische Verbformen nicht finit sind.
Wenn man dies annimmt, wird in Imperativsätzen eine infinite Verbform herausbewegt.
Das müsste in einem angepassten Schema reflektiert werden, was hier aus Platzgründen nicht ausbuchstabiert wird.

\index{Verb!Partikel--}

Damit sind jetzt alle Stellungstypen prinzipiell erklärt.
Zu Nebensätzen kann und sollte man allerdings mehr sagen, als einfach ihre Konstituentenstruktur anzugeben.
Über Verwendung, Anschluss und Stellung der drei wichtigen Nebensatztypen folgen (nach Bemerkungen zu Partikelverben in Abschnitt~\ref{sec:partikelverben} und zu Kopulasätzen in Abschnitt~\ref{sec:kopulakonstruktionen}) weitere Überlegungen in Abschnitt~\ref{sec:nebensaetze}.

\subsection{Syntax der Partikelverben}

\label{sec:partikelverben}

Durch die Bewegung von finiten Verben ergibt sich ein Problem, wenn wir Partikelverben als eine Wortform analysieren.
In einer V2-Struktur bleibt die Partikel zurück, die Bewegung müsste aus einer Wortform heraus geschehen, vgl. (\ref{ex:saetze7778}).

\begin{exe}
  \ex{\label{ex:saetze7778} Sarah isst den Kuchen alleine auf=.}
\end{exe}

\index{Verbkomplex}

Das syntaktische Herausbewegen aus einer Wortform ist problematisch, denn Wortformen sollen auf der Ebene der Syntax als atomare Konstituenten gelten.
Die Lösung besteht darin, Kombinationen aus Partikel und Verb als syntaktische Struktur zu analysieren, wie in Abbildung~\ref{fig:v2mitpartikel}.
Damit ist es möglich, die Bewegung des finiten Verbs durchzuführen.
Eigentlich müsste das Phrasenschema für den Verbkomplex für diesen Zweck erweitert werden, was als Übungsaufgabe von den Lesern durchgeführt werden kann.
Außerdem würden sich evtl.\ Änderungen an den Wortklassen bzw.\ den Aussagen zur Verbalmorphologie (\zB Bildung der Partizipien) ergeben.

\begin{figure}[!htbp]
  \centering
  \Tree{
    &&& \K{S}\B{dddll}\B{ddd}\B{drrrrrrr} \\
    &&&&&&&&&& \K{VP}\B{ddllllll}\B{ddlllll}\B{ddlll}\B{d} \\
    &&&&&&&&&& \K{\textbf{V}}\B{dl}\B{d} \\
    & \K{NP\ORii}\TRi[-8] && \K{\textbf{V\ORi}}\B{d} & \K{\Tii}\POS[]-(0,4)\ar@{-->}@/^{4pc}/[lll]-(0,4) & \K{NP}\TRi[-4] && \K{AdvP}\TRi[-6] && \K{Ptkl}\B{d} & \K{\Ti}\POS[]-(0,4)\ar@{-->}@/^{4pc}/[lllllll]-(0,4) \\
    & \K{\textit{Sarah}} && \K{\textit{isst}} && \K{\textit{den Kuchen}} && \K{\textit{alleine}} && \K{\textit{auf=}} \\
  }
  \vspace{0.3cm}
  \caption{V2-Satz mit Partikelverb}
  \label{fig:v2mitpartikel}
\end{figure}

\subsection{Kopulasätze}

\label{sec:kopulakonstruktionen}

\index{Kopula}
\index{Kopulasatz}

Für die Beschreibung von Kopulasätzen wie (\ref{ex:satz82821a}) müssen keine besonderen Satzstrukturen eingeführt werden.
Wir können sie als Ergebnis der üblichen Bewegungsoperationen betrachten und Strukturen wie (\ref{ex:satz82821b}) zugrundelegen.

\begin{exe}
  \ex\label{ex:satz82821} 
  \begin{xlist}
    \ex{\label{ex:satz82821a} Die Frau ist stolz auf ihre Tochter.}
    \ex{\label{ex:satz82821b} dass die Frau auf ihre Tochter stolz ist}
  \end{xlist}
\end{exe}

Die AP ist hier strukturell etwas anders gebaut als eine attributive AP innerhalb einer NP.
Die in der NP prototypische Abfolge [[\textit{auf ihre Tochter}] \textit{stolze}] wird (zumindest optional) umgekehrt zu [\textit{stolz} [\textit{auf ihre Tochter}]].
Außerdem besteht keine Kongruenz des Adjektivs zu irgendeinem Bezugsnomen, und das Adjektiv steht in der unflektierten Kurzform (s.\ Abschnitt~\ref{sec:adjektivklassifikation}).
Ansonsten fällt auf, dass die Nominativ-NP \textit{die Frau} mit der Kopula in Person und Numerus kongruiert und frei im Satz bewegt werden kann.
Eine besondere morphosyntaktische Beziehung zum Adjektiv hat das Subjekt nicht.
Die Konstituente [\textit{stolz auf ihre Tochter}] kann außerdem auch frei bewegt werden wie in (\ref{ex:satz82822}).

\begin{exe}
  \ex{\label{ex:satz82822} [Stolz auf ihre Tochter] ist die Frau.}
\end{exe}

Die Analyse in Abbildung~\ref{fig:kopulav22} bietet sich daher an.
Dabei regiert die Kopula eine AP, die zwar eine andere Abfolge ihrer Konstituenten realisiert als die AP innerhalb einer NP, die aber aus denselben Konstituenten besteht.
Die Kopula regiert außerdem eine NP im Nominativ.

\begin{figure}[!htbp]
  \Tree{
    &&& \K{S}\B{ddddll}\B{dddd}\B{drrrrr} \\
    &&&&&&&& \K{VP}\B{ddllll}\B{ddd}\B{ddlll} \\
    \\
    &&&& \K{\Tii} & \K{AP}\B{d}\B{drr} \\
    & \K{NP\ORii}\TRi[-8] && \K{\textbf{V\ORi}}\B{d} && \K{\textbf{A}}\B{d} && \K{PP}\TRi[-8] & \K{\Ti}\\
    & \K{\textit{Die Frau}} && \K{\textit{ist}} && \K{\textit{stolz}} && \K{\textit{auf ihre}}\Below{\textit{Tochter}} & \\
  }
  \caption{Analyse eines Kopulasatzes mit AP}
  \label{fig:kopulav22}
\end{figure}


\Zusammenfassung{
Wir stellen Sätze durch Phrasenschemata dar.
Sätze haben in der hier vertretenen Analyse aber keinen Kopf.
}


\section{Nebensätze}

\label{sec:nebensaetze}

In diesem Abschnitt werden die verschiedenen Typen von Nebensätzen und ihre Besonderheiten im internen Aufbau und in ihrem externen syntaktischen Verhalten besprochen.
Die Definition des Nebensatzes aus Kapitel~\ref{sec:wortklassen} (Definition~\ref{def:nebensatz} auf S.~\pageref{def:nebensatz}) kann unverändert zugrundegelegt werden.
Es handelt sich also um eine Konstituente, die ein finites Verb enthält, in der alle Valenzen gesättigt sind, und die nicht alleine stehen kann.

\Enl

Fälle wie (\ref{ex:saetze1450a}), in denen ein Nebensatz scheinbar alleine steht, analysieren wir als Ellipsen, also Strukturen, in denen eine hauptsatzartige Struktur getilgt wurde, s.\ (\ref{ex:saetze1450b}).
Andere Analysen sind in einem größeren theoretischen Rahmen natürlich möglich und vielleicht erwünscht.

\Np

\begin{exe}
  \ex\label{ex:saetze1450} 
  \begin{xlist}
    \ex{\label{ex:saetze1450a} Ob das wohl stimmt!}
    \ex{\label{ex:saetze1450b} Ich frage mich\slash Ich bin nicht sicher\slash\ldots, ob das wohl stimmt!}
  \end{xlist}
\end{exe}

\subsection{Relativsätze}

\label{sec:relativsaetze}

Ein Relativsatz (RS) wie in (\ref{ex:saetze9228}) ist im prototypischen Fall ein Attribut zu einem nominalen Kopf, dem Bezugsnomen (vgl.\ Abschnitt~\ref{sec:ngr}).
Der Sonderfall des \textit{freien Relativsatzes} wird weiter unten besprochen.

\begin{exe}
  \ex{\label{ex:saetze9228} Einen Tofu, der mir nicht geschmeckt hat, habe ich noch nie gegessen.}
\end{exe}

Wie schon in Abschnitt~\ref{sec:felder} angedeutet, ist der Relativsatz unter den satzförmigen Strukturen ein Sonderfall bezüglich seiner internen Satzgliedstellung.
Das Verb bleibt im Verbkomplex stehen (VL-Satz), und das Relativpronomen -- genauer die Relativphrase -- wird nach links (in das Vorfeld) bewegt.
Man kann sich die Struktur eines RS wie (\ref{ex:saetze6419a}) verdeutlichen, indem man aus dem Relativsatz und seinem Bezugsnomen einen unabhängigen Satz baut.
Man ersetzt das Relativpronomen durch das Bezugsnomen und setzt dabei das Bezugsnomen in den Kasus, den das Relativpronomen hatte, s.\ (\ref{ex:saetze6419b}). 
Dann stellt man durch Umstellung des finiten Verbs eine V2-Stellung her, vgl.\ (\ref{ex:saetze6419c}).

\begin{exe}
  \ex\label{ex:saetze6419}
  \begin{xlist}
    \ex{\label{ex:saetze6419a} einen Tofu, der mir nicht geschmeckt hat}
    \ex{\label{ex:saetze6419b} ein Tofu mir nicht geschmeckt hat}
    \ex{\label{ex:saetze6419c} Ein Tofu hat mir nicht geschmeckt.}
  \end{xlist}
\end{exe}

Der Relativsatz wird durch Schema~\ref{str:rs} beschrieben, eine Analyse liefert Abbildung~\ref{fig:saetze9228}.

\index{Relativsatz}

\Phrasenschema{Relativsatz}{\label{str:rs}
  \begin{tabular}{c|c|c|}
    \cline{2-3}
    RS = & [\textsc{Rel}:$+$]\ORi & VP[\ldots \Ti \ldots] \\
    \cline{2-3}
  \end{tabular}
}

\begin{figure}[!htbp]
  \centering
  \Tree[-0.3]{
    && \K{NP}\B{ddddll}\B{dddd}\B{drr} \\
    &&&& \K{RS} \B{ddd}\B{drrrrrrrr} \\
    &&&&&&&&&&&& \K{VP}\B{ddllllll}\B{ddlllll}\B{ddllll}\B{d} \\
    &&&&&&&&&&&& \K{\textbf{V}}\B{dll}\B{d} \\
    \K{Art}\B{d} && \K{\textbf{N}}\B{d} && \K{NP\ORi}\TRi[-8] && \K{NP}\TRi[-8] & \K{\Ti}\POS[]-(0,4)\ar@{-->}@/^{4pc}/[lll]-(0,4) & \K{Ptkl}\B{d} && \K{\textbf{V}}\B{d} && \K{\textbf{V}}\B{d} \\
    \K{\textit{einen}} && \K{\textit{Tofu}} && \K{\textit{der}} && \K{\textit{mir}} && \K{\textit{nicht}} && \K{\textit{geschmeckt}} && \K{\textit{hat}} \\
  }
  \vspace{0.3cm}
  \caption{NP mit Relativsatz}
  \label{fig:saetze9228}
\end{figure}

\index{Relativphrase}
\index{Relativsatz!Einleitung}

Wir müssen uns nun fragen, welche Form bzw.\ welche Merkmale die Relativphrase in allen möglichen Arten von Relativsätzen genau hat.
In Satz (\ref{ex:saetze9228}) ist sie eine NP aus einem einfachen maskulinen Relativpronomen im Nominativ Singular.
Ein Relativpronomen muss, damit das Schema anwendbar ist, im Lexikon bereits mit dem Merkmal [\textsc{Rel}: $+$] ausgestattet sein, um auf die hier gezeigte Weise bewegt werden zu können.
Das Schema spezifiziert ausdrücklich, dass das vorangestellte Element das Merkmal [\textsc{Rel}: $+$] haben muss.
Es gelten zwei weitere wichtige Beschränkungen.
Die Relativphrase kongruiert mit dem Bezugsnomen in \textsc{Genus} und \textsc{Numerus}, und sie erhält ihren Wert für \textsc{Kasus} innerhalb des Relativsatzes.
In Satz (\ref{ex:saetze9228}) hat \textit{der} das Merkmal [\textsc{Kasus}: \textit{nom}] dank der Rektion durch \textit{geschmeckt}.
Dass \textit{der} aber [\textsc{Genus}: \textit{mask}, \textsc{Numerus}: \textit{sg}] ist, kommt durch Kongruenz mit \textit{Tofu} zustande.

Die zweite Bedingung ist etwas zu eng gefasst, weil die Relativphrase nicht unbedingt eine einfache NP sein muss, deren Kasus vom Verb des Relativsatzes regiert wird.
Es gibt auch Relativsätze wie in (\ref{ex:saetze5552a}), in denen die Relativphrase komplexer ist.
Wenn wir diesen Relativsatz genauso wie in (\ref{ex:saetze6419}) in einen unabhängigen Satz umwandeln, erhalten wir über (\ref{ex:saetze5552b}) schließlich (\ref{ex:saetze5552c}).
Die Relativphrase (im Beispiel eingeklammert) hat die Form einer PP mit einem Relativpronomen als Kopf der von der Präposition regierten NP.
Die Präposition bleibt bei der Umwandlung erhalten, die Relativphrase (die PP mit dem eingebetteten Relativpronomen) wird also nur teilweise ersetzt.

\begin{exe}
  \ex\label{ex:saetze5552} 
    \begin{xlist}
      \ex{\label{ex:saetze5552a} der Tofu, [auf den] ich mich freue}
      \ex{\label{ex:saetze5552b} [auf den Tofu] ich mich freue}
      \ex{\label{ex:saetze5552c} [Auf den Tofu] freue ich mich.}
    \end{xlist} 
\end{exe}

Da \textit{freue} eine PP mit \textit{auf} regiert, wird der Relativphrase hier offensichtlich nicht einfach ein Kasus per Rektion innerhalb des Relativsatzes zugewiesen, sondern eine präpositionale Form.
Eine Relativ-PP muss nicht einmal regiert sein.
Die PP [\textit{auf der Straße}] (bzw.\ die Relativphrase [\textit{auf der}]) ist  in (\ref{ex:saetze3139a}) keine Ergänzung von \textit{laufen}, sondern eine Angabe.

\begin{exe}
  \ex\label{ex:saetze3139}
  \begin{xlist}
    \ex{\label{ex:saetze3139a} Die Straße, [auf der] wir den Marathon laufen, ist eine Autobahn.}
    \ex{\label{ex:saetze3139b} [Auf der Straße] laufen wir den Marathon.}
  \end{xlist}
\end{exe}

In (\ref{ex:saetze5553a}) haben wir es mit der vielleicht kompliziertesten Art von Relativphrase zu tun.
Das Pronomen \textit{dessen} ist ein pränominaler Genitiv innerhalb einer NP [\textit{dessen Geschmack}].
Die Relativphrase ist hier die gesamte NP, innerhalb derer das Pronomen den Kasus (Genitiv) erhält, den auch eine korrespondierende Genitiv-NP in einer unabhängigen NP erhalten würde, vgl.\ (\ref{ex:saetze5553b}).
Es ist nicht so, dass die gesamte Relativphrase (die NP im Akkusativ) in Genus und Numerus mit dem Bezugsnomen kongruiert, sondern nur die pränominale Genitiv-NP.

\begin{exe}
  \ex\label{ex:saetze5553} 
    \begin{xlist}
      \ex{\label{ex:saetze5553a} Der Tofu, [dessen Geschmack] ich mag, ist ausverkauft.}
      \ex{\label{ex:saetze5553b} [Den Geschmack [des Tofus]] mag ich.}
    \end{xlist}
\end{exe}

\index{Genitiv!pränominal}

In Abbildung~\ref{fig:saetze5553} wird die Struktur dieser Konstruktion abgebildet.
Die Rektionsanforderung des Verbs \textit{mag} (Akkusativ) geht wie zu erwarten an die NP, deren Kopf \textit{Geschmack} ist.
Der Kasus von \textit{dessen} ist ein Attributsgenitiv und abhängig von \textit{Geschmack}.
Das Relativpronomen \textit{dessen} kongruiert mit \textit{Tofu} in Genus und Numerus.%
\footnote{Der Bewegungspfeil wird der Übersicht wegen weggelassen.}

\begin{figure}[!htbp]
  \centering
  \Tree[-0.3]{
    && \K{NP}\B{dddll}\B{ddd}\B{drrrr} \\
    &&&&&& \K{RS}\B{d}\B{drrrr} \\
    &&&&&& \K{NP\ORi}\B{dll}\B{d} &&&& \K{VP}\B{dll}\B{dl}\B{d} \\
    \K{Art}\B{d} && \K{\textbf{N}}\B{d} && \K{NP}\TRi[-4] && \K{\textbf{N}}\B{d} && \K{NP}\TRi[-7] & \K{\Ti} & \K{\textbf{V}}\B{d} \\
    \K{\textit{der}} && \K{\textit{Tofu}}
    \POS[]-(0,4)\ar@{-->}@/_{0.5pc}/[rr]-(0,4)_{\sc\mathrm{Gen,Num}} && \K{\textit{dessen}} && \K{\textit{Geschmack}}\ar@{-->}@/^{1pc}/[l]-(8,4)^{\sc\mathrm{Kas}} && \K{\textit{ich}} && \K{\textit{mag}}\POS[]-(0,4)\ar@{-->}@/^{1pc}/[llll]-(0,4)^{\sc\mathrm{Kas}} & \\
  }
  \caption{NP mit Relativsatz mit genitivischer Relativphrase}
  \label{fig:saetze5553}
\end{figure}

\index{Relativadverb}

Neben den normalen Relativpronomina gibt es noch eine Reihe von sogenannten \textit{Relativadverben} wie \textit{womit}, \textit{worin}, \textit{worauf} usw., die für sich alleine eine Relativphrase bilden.
Wie geben hier nur ein Beispiel in (\ref{ex:saetze2998}).

\begin{exe}
  \ex{\label{ex:saetze2998} Alles, womit man rechnet, tritt auch ein.}
\end{exe}

Eine Sonderklasse von Relativsätzen sind die sogenannten \textit{freien Relativsätze}.
Freie Relativsätze sind intern wie jeder andere Relativsatz aufgebaut, beziehen sich aber nicht auf ein Bezugsnomen, sondern nehmen für sich allein den Platz einer NP ein.

\begin{exe}
  \ex\label{ex:saetze1991}
  \begin{xlist}
    \ex{\label{ex:saetze1991a} Wer Klaviermusik mag, mag Chopin.}
    \ex{\label{ex:saetze1991b} Wen man mag, beschenkt man.}
    \ex{\label{ex:saetze1991c} Wir glauben, wem wir Vertrauen schenken.}
  \end{xlist}
\end{exe}

\index{Relativsatz!frei}

Im Normalfall muss die Relativphrase den Kasus haben, den auch eine NP an der Position des RS im einbettenden Satz hätte.
Dies hat zur Folge, dass der Kasus der Relativphrase im Relativsatz gleich dem externen Kasus sein muss.
Abbildung~\ref{fig:saetze1991b} zeigt die Kasusanforderungen.%
\footnote{Es ist eine theoretisch problematische Annahme, dass eine NP (hier \textit{wen}) von zwei verschiedenen Verben regiert wird.
Insofern sind die Valenzpfeile als Veranschaulichung zu verstehen, nicht als theoretische Position.}

\begin{figure}[!htbp]
  \centering
  \resizebox{\textwidth}{!}{
    \Tree{
      &&&&&&& \K{S}\B{dllllll}\B{ddd}\B{ddrrrr} \\
      & \K{RS\ORiii}\B{dd}\B{drrrr} \\
      &&&&& \K{VP}\B{dll}\B{dl}\B{d} &&&&&& \K{VP}\B{dll}\B{dl}\B{d} \\
      & \K{NP\ORi}\TRi[-7] && \K{NP}\TRi[-7] & \K{\Ti} & \K{\textbf{V}}\B{d} && \K{\textbf{V\ORii}}\B{d} && \K{NP}\TRi[-7] & \K{\Tiii} & \K{\Tii} \\
      & \K{\textit{wen}} && \K{\textit{man}} && \K{\textit{mag}}\POS[]-(0,4)\ar@{-->}@/^{1pc}/[llll]-(0,4)^{\sc\mathrm{Kas}} && \K{\textit{beschenkt}}\POS[]-(0,4)\ar@{--}@/^{3pc}/[llllll]-(0,4)^{\sc\mathrm{Kas}} && \K{\textit{man}} && \\
    }
  }
  \caption{Satz mit freiem Relativsatz}
  \label{fig:saetze1991b}
\end{figure}

Die Ungrammatikalität von (\ref{ex:saetze1992a}) rührt aus einer Verletzung dieser speziellen Kasusanforderung her.

\begin{exe}
  \ex[*]{\label{ex:saetze1992a} Wer Klaviermusik mag, beschenkt man mit Konzertkarten.}
\end{exe}

Um einen Satz wie (\ref{ex:saetze1992a}) zu reparieren, muss der Relativsatz an einen pronominalen Kopf als Bezugsnomen angeschlossen werden, der die Kasusanforderung des einbettenden Satzes erfüllen kann.
In (\ref{ex:saetze1992b}) ist ein solches Pronomen in Form von \textit{denjenigen} eingesetzt.
Es erfüllt als Akkusativ die Rektionsanforderung von \textit{beschenkt}, während die Relativphrase \textit{der} die Rektionsanforderung (Nominativ) von \textit{mag} innerhalb des Relativsatzes erfüllt.

\begin{exe}
  \ex{\label{ex:saetze1992b} Denjenigen, der Klaviermusik mag, beschenkt man mit Konzertkarten.}
\end{exe}

Eine andere Möglichkeit ist es, den freien Relativsatz mit \textit{w}"=Pronomen vor das Vorfeld zu stellen und ein im Kasus angepasstes korrelierendes Pronomen ins Vorfeld zu stellen, wie in (\ref{ex:saetze1998000}).

\begin{exe}
  \ex{\label{ex:saetze1998000} Wer Klaviermusik mag, den beschenkt man mit Konzertkarten.}
\end{exe}

Manche Sprecher akzeptieren es allerdings auch, wenn der Kasus der Relativphrase obliker ist, als per Rektion im Matrixsatz gefordert, vgl.\ (\ref{ex:saetze8282821a}).
Wenn die Relativphrase in einem weniger obliken Kasus steht, funktioniert das allerdings nicht, wie in (\ref{ex:saetze8282821b}).

\begin{exe}
  \ex\label{ex:saetze8282821} 
  \begin{xlist}
    \ex[?]{\label{ex:saetze8282821a} Wen es stört, kann gehen.}
    \ex[*]{\label{ex:saetze8282821b} Wer hier stört, beschenkt man.}
  \end{xlist}
\end{exe}

Abschließend müssen einige Stellungsbesonderheiten der Relativsätze diskutiert werden. 
Bezüglich der Stellung der Relativsätze im einbettenden Satz müssen zwei Fälle unterschieden werden.
Die Fälle sind in (\ref{ex:saetze3338}) und (\ref{ex:saetze3339}) illustriert.

\begin{exe}
  \ex\label{ex:saetze3338}
  \begin{xlist}
    \ex{\label{ex:saetze3338a} Die Gavotte, die ich am liebsten mag, hat Tanja gespielt.}
    \ex{\label{ex:saetze3338b} Tanja hat die Gavotte, die ich am liebsten mag, gespielt.}
  \end{xlist}
  \ex\label{ex:saetze3339}
  \begin{xlist}
    \ex{\label{ex:saetze3339a} Tanja hat die Gavotte gespielt, die ich am liebsten mag.}
    \ex{\label{ex:saetze3339b} Ich glaube, dass Tanja die Gavotte gespielt hat, die ich am liebsten mag.}
  \end{xlist}
\end{exe}
\index{Nachfeld}

In (\ref{ex:saetze3338}) ist der Relativsatz innerhalb der NP rechts vom Kopf positioniert, also genau dort, wo er gemäß Schema~\ref{str:ngr} stehen soll.
Dabei ist es egal, ob die NP mit dem Relativsatz ins Vorfeld bewegt wird wie in (\ref{ex:saetze3338a}), oder ob die NP im Mittelfeld verbleibt wie in (\ref{ex:saetze3338b}).
Bereits in Abschnitt~\ref{sec:feldermodell} (s.\ vor allem Abbildung~\ref{fig:nachfeld}) wurden aber Sätze wie die in (\ref{ex:saetze3339}) gezeigt.
Hier wird der Relativsatz nach rechts ins Nachfeld herausgestellt und damit von der NP getrennt.
Dies kann sowohl aus unabhängigen Sätzen wie in (\ref{ex:saetze3339a}) als auch aus eingebetteten Sätzen wie in (\ref{ex:saetze3339b}) geschehen.
In (\ref{ex:saetze3339b}) wurde der Relativsatz aus dem \textit{dass}-Satz über die rechte Satzklammer seines Matrixsatzes nach rechts ins Nachfeld verschoben.%
\footnote{Wir geben hier keine Strukturen dafür an.
Übung \ref{u124} auf S.~\pageref{u124} beschäftigt sich mit der Frage von Konstituentenstrukturen bei Bewegung ins Nachfeld.}

\subsection{Komplementsätze}

\label{sec:komplementsaetze}

\textit{Komplementsätze} oder \textit{Ergänzungssätze} sind Sätze, die als Ergänzung zu Verben fungieren, die also eine Valenzanforderung saturieren.%
\footnote{Die Begriffe \textit{Komplement} und \textit{Ergänzung} sind weitgehend synonym, vgl.\ Abschnitt~\ref{sec:valenz}.}
Dabei unterscheidet man \textit{Subjektsätze} und \textit{Objektsätze}.

\Definition{Komplementsatz}{
\label{def:komplementsatz}
Ein Komplementsatz ist eine Ergänzung in Form eines Nebensatzes.
Der Untertyp des Subjektsatzes nimmt die Stelle ein, die auch von einer NP im Nominativ eingenommen werden könnte.
Alle anderen Komplementsätze sind Objektsätze.
\index{Komplementsatz}
\index{Subjektsatz}
\index{Objektsatz}
}

Wenden wir uns zunächst den Objektsätzen zu, müssen formal zwei Typen unterschieden werden.
Das Verb \textit{wissen} in (\ref{ex:saetze8881}) regiert einen \textit{dass}-Satz.
In (\ref{ex:saetze8882a}) regiert es einen \textit{w}"=Fragesatz und in (\ref{ex:saetze8882b}) einen Fragesatz mit dem Frage"=Komplementierer \textit{ob}.

\begin{exe}
  \ex{\label{ex:saetze8881} Michelle weiß, [dass die Corvette nicht anspringen wird].}
  \ex\label{ex:saetze8882}
  \begin{xlist}
    \ex{\label{ex:saetze8882a} Michelle will wissen, [wer die Corvette gewartet hat].}
    \ex{\label{ex:saetze8882b} Michelle will wissen, [ob die Corvette gewartet wurde].}
  \end{xlist}
\end{exe}

Verben, die Objektsätze fordern, folgen drei Mustern, je nachdem, mit welchen Arten von Objektsätzen sie stehen können.
Entweder stehen sie nur mit \textit{dass}-Sätzen wie \textit{beklagen} in (\ref{ex:saetze727211}), nur mit Fragesätzen wie \textit{untersuchen} in (\ref{ex:saetze727212}) oder eben mit beidem wie \textit{hören} in (\ref{ex:saetze727213}) oder \textit{wissen} in (\ref{ex:saetze8881}) und (\ref{ex:saetze8882}).

\begin{exe}
  \ex\label{ex:saetze727211} 
  \begin{xlist}
    \ex[]{\label{ex:saetze727211a} Michelle beklagt, dass die Corvette nicht anspringt.}
    \ex[*]{\label{ex:saetze727211b} Michelle beklagt, wie\slash ob die Corvette nicht anspringt.}
  \end{xlist}
  \ex\label{ex:saetze727212} 
  \begin{xlist}
    \ex[*]{\label{ex:saetze727212a} Michelle untersucht, dass der Vergaser funktioniert.}
    \ex[]{\label{ex:saetze727212b} Michelle untersucht, wie\slash ob der Vergaser funktioniert.}
  \end{xlist}
  \ex\label{ex:saetze727213} 
  \begin{xlist}
    \ex[]{\label{ex:saetze727213a} Michelle hört, dass die Nockenwelle läuft.}
    \ex[]{\label{ex:saetze727213b} Michelle hört, wie\slash ob die Nockenwelle läuft.}
  \end{xlist}
\end{exe}

Bei \textit{dass}-Sätzen gibt es Alternationen mit Infinitivkonstruktionen mit \textit{zu} (selbständigen infiniten VP) wie in (\ref{ex:saetze8111}).
Diese Infinitive werden in Abschnitt~\ref{sec:infkonstr} genauer besprochen.

\begin{exe}
  \ex\label{ex:saetze8111}
  \begin{xlist}
    \ex{\label{ex:saetze8111a} Michelle glaubt, [dass sie das Geräusch erkennt].}
    \ex{\label{ex:saetze8111b} Michelle glaubt, [das Geräusch zu erkennen].}
  \end{xlist}
\end{exe}

Nach Definition~\ref{def:komplementsatz} nehmen Subjektsätze die Position der NP im Nominativ ein.
Ein Subjektsatz ist in (\ref{ex:saetze7771a}) illustriert.
In (\ref{ex:saetze7771b}) ersetzt ein Nominativ den Subjektsatz.
Eine solche Ersetzung funktioniert nicht immer, und der Status eines Nebensatzes als Subjektsatz ist nicht an die Möglichkeit der Ersetzung durch eine NP gebunden.

\begin{exe}
  \ex\label{ex:saetze7771} 
  \begin{xlist}
    \ex{\label{ex:saetze7771a} [Dass die Sonne scheint], freut uns.}
    \ex{\label{ex:saetze7771b} [Der Sonnenschein] freut uns.}
  \end{xlist}
\end{exe}

\index{Mittelfeld}
\index{Korrelat}
\index{Nebensatz}
Die bisher besprochenen Komplementsätze standen alle entweder im Vorfeld oder im Nachfeld.
Tatsächlich ist es ungewöhnlich (wenn auch je nach Sprecher und Gestalt des Satzes nicht ganz ausgeschlossen), dass Komplementsätze im Mittelfeld stehen, wo sie als Ergänzungen des Verbs eigentlich zu erwarten wären.
Die (c)-Sätze in (\ref{ex:saetze7171})--(\ref{ex:saetze7173}) illustrieren dies.

\begin{exe}
  \ex\label{ex:saetze7171}
  \begin{xlist}
    \ex[]{\label{ex:saetze7171a} [Dass sie unseren Kuchen mag], hat Sarah uns nun doch eröffnet.}
    \ex[]{\label{ex:saetze7171b} Sarah hat uns nun doch eröffnet, [dass sie unseren Kuchen mag].}
    \ex[?]{\label{ex:saetze7171c} Sarah hat uns, [dass sie unseren Kuchen mag], nun doch eröffnet.}
  \end{xlist}

  \Np

  \ex\label{ex:saetze7172}
  \begin{xlist}
    \ex[]{\label{ex:saetze7172a} [Ob Pavel unseren Kuchen mag], haben wir uns oft gefragt.}
    \ex[]{\label{ex:saetze7172b} Wir haben uns oft gefragt, [ob Pavel unseren Kuchen mag].}
    \ex[?]{\label{ex:saetze7172c} Wir haben uns, [ob Pavel unseren Kuchen mag], oft gefragt.}
  \end{xlist}
  \ex\label{ex:saetze7173}
  \begin{xlist}
    \ex[]{\label{ex:saetze7173a} [Wer die Rosinen geklaut hat], wollen wir endlich wissen.}
    \ex[]{\label{ex:saetze7173b} Wir wollen endlich wissen, [wer die Rosinen geklaut hat].}
    \ex[?]{\label{ex:saetze7173c} Wir wollen, [wer die Rosinen geklaut hat], endlich wissen.}
  \end{xlist}
\end{exe}

Die Komplementsätze werden also überwiegend aus dem Mittelfeld herausbewegt.
Die (a)-Sätze in (\ref{ex:saetze7171})--(\ref{ex:saetze7173}) sind mit entsprechender Betonung einwandfrei.

Wenn ein Objektsatz ins Nachfeld gestellt wird, dann können (wie eine sichtbare Spur) sogenannte \textit{Korrelate} im Mittelfeld stehen.\index{Spur}
Die (b)-Sätze aus (\ref{ex:saetze7171})--(\ref{ex:saetze7173}) werden in (\ref{ex:saetze7175}) mit dem Korrelat \textit{es} wiederholt.

\begin{exe}
  \ex\label{ex:saetze7175}
  \begin{xlist}
    \ex{\label{ex:saetze7175a} Sarah hat es uns eröffnet, [dass sie unseren Kuchen mag].}
    \ex{\label{ex:saetze7175b} Wir haben es uns gefragt, [ob Pavel unseren Kuchen mag].}
    \ex{\label{ex:saetze7175c} Wir wollen es wissen, [wer die Rosinen aus dem Kuchen geklaut hat].}
  \end{xlist}
\end{exe}

Das Korrelat \textit{es} ist hier optional, muss also nicht stehen.
Wenn der Komplementsatz ein Präpositionalobjekt vertritt, wird das Korrelat bei vielen Verben wie \textit{hinweisen} obligatorisch, s.\ (\ref{ex:saetze7177}).
Das Verb \textit{hinweisen} fordert eine NP im Nominativ und eine PP mit \textit{auf}, vgl.\ (\ref{ex:saetze7177a}).
Wenn ein Komplementsatz vorliegt, wird die Komplementstelle formal durch \textit{darauf} im Mittelfeld gesättigt, das in (\ref{ex:saetze7177b}) als Korrelat zum Komplementsatz fungiert.
Satz (\ref{ex:saetze7177c}) zeigt, dass das Korrelat nicht fehlen darf.

\begin{exe}
  \ex\label{ex:saetze7177}
  \begin{xlist}
    \ex[]{\label{ex:saetze7177a} Ich weise [auf den leckeren Kuchen] hin.}
    \ex[]{\label{ex:saetze7177b} Ich weise darauf hin, [dass der Kuchen lecker ist].}
    \ex[*]{\label{ex:saetze7177c} Ich weise hin, [dass der Kuchen lecker ist].}
  \end{xlist}
\end{exe}

Auch Subjektsätze können in Konstruktionen mit Korrelaten stehen wie in (\ref{ex:saetze7178}).
Das Korrelat muss dabei immer vor dem Nebensatz stehen.

\begin{exe}
  \ex\label{ex:saetze7178}
  \begin{xlist}
    \ex[ ]{\label{ex:saetze7178a} Es hat uns gefreut, [dass Sarah unseren Kuchen mochte].}
    \ex[ ]{\label{ex:saetze7178b} Uns hat es gefreut, [dass Sarah unseren Kuchen mochte].}
    \ex[ ]{\label{ex:saetze7178c} Uns hat gefreut, [dass Sarah unseren Kuchen mochte].}
    \ex[*]{\label{ex:saetze7178d} [Dass Sarah unseren Kuchen mochte], hat es uns gefreut.}
  \end{xlist}
\end{exe}

Damit endet die sehr knappe Darstellung der Komplementsätze.
Den Komplementsätzen verwandt sind die Adverbialsätze, die sich im Wesentlichen dadurch von den Komplementsätzen unterscheiden, dass sie keine Valenzstelle saturieren.

\subsection{Adverbialsätze}

\label{sec:adverbialsaetze}
\index{Komplementierer}

Adverbialsätze sind VL-Sätze, die von einem Komplementierer eingeleitet werden.
Sie werden normalerweise nach der semantischen Funktion ihrer Komplementierer unterklassifiziert.%
\footnote{Wir gehen auf die Unterklassifikation nicht besonders ein.
Für die Betrachtung der Syntax sind die Unterklassen wie \textit{Finalsatz}, \textit{Konsekutivsatz} oder \textit{Konzessivsatz} in erster Näherung irrelevant.
Je nach Komplementierer ergeben sich aber durchaus Besonderheiten, die in vollständigeren Grammatiken beschrieben werden.}

\Definition{Adverbialsatz}{
\label{def:adverbialsatz}
Ein Adverbialsatz ist ein mit Komplementierer eingeleiteter VL-Nebensatz, der keine Valenzstelle im Matrixsatz saturiert.
\index{Adverbialsatz}
}

Beispiele werden in (\ref{ex:saetze1117}) gegeben, und zwar in dieser Reihenfolge ein \textit{Kausalsatz}, ein \textit{Temporalsatz} und ein \textit{Konzessivsatz}.

\begin{exe}
  \ex\label{ex:saetze1117}
  \begin{xlist}
    \ex{\label{ex:saetze1117a} [Weil es regnet], bleibe ich lieber zuhause.}
    \ex{\label{ex:saetze1117b} Wir haben Kaffee getrunken, [nachdem der Kuchen aufgegessen war].}
    \ex{\label{ex:saetze1117c} [Obwohl das Buch interessant ist], ignorieren wir es.}
  \end{xlist}
\end{exe}

Adverbialsätze lassen sich oft als ein nicht-satzförmiges Adverbial umformulieren, \zB als PP.
Parallel zu (\ref{ex:saetze1117}) sind in (\ref{ex:saetze1118}) solche PPs realisiert.

\begin{exe}
  \ex\label{ex:saetze1118}
  \begin{xlist}
    \ex{\label{ex:saetze1118a} [Wegen des Regens] bleibe ich lieber zuhause.}
    \ex{\label{ex:saetze1118b} Wir haben [nach dem Kuchenessen] Kaffee getrunken.}
    \ex{\label{ex:saetze1118c} [Trotz unseres Interesses an dem Buch] ignorieren wir es.}
  \end{xlist}
\end{exe}

\index{Nachfeld}
\index{Mittelfeld}

Wie die Beispiele in (\ref{ex:saetze1117}) zeigen, stehen Adverbialsätze genauso wie Komplementsätze gerne im Vorfeld oder Nachfeld.
Ob sie aus dem Mittelfeld herausbewegt werden, oder ob sie direkt in diese Positionen gestellt werden, kann und muss hier nicht entschieden werden (vgl.\ zu dieser Frage auch schon Abschnitt~\ref{sec:konstituentenstrukturinv2}, besonders Abbildung~\ref{fig:vorfelddirektbesetzung}).

\Np

\Satz{Eigenschaften von Adverbialsätzen}{
\label{satz:eigenschaftadverbialsatz}
Adverbialsätze lassen sich (im Gegensatz zu Komplementsätzen) oft unter Beibehaltung der Bedeutung in nicht-satzförmige Adverbiale (\zB PPs) umformen.
Sie stehen i.\,d.\,R.\ im Vorfeld oder Nachfeld.
\index{Adverbialsatz}
}

\index{Konditionalsatz}
\index{Verb-Erst-Satz}

Einen Sonderfall bilden die Konditionalsätze, die normalerweise mit Komplementierern wie \textit{wenn}, \textit{falls}, \textit{sofern} eingeleitet werden, s.\ (\ref{ex:saetze0001a}).
Der Komplementierer kann entfallen.
Der Konditionalsatz wird dann als V1-Satz realisiert, der im Vorfeld seines Matrixsatzes stehen muss, vgl.\ (\ref{ex:saetze0001b}).

\begin{exe}
  \ex\label{ex:saetze0001}
  \begin{xlist}
    \ex{\label{ex:saetze0001a} [Wenn der Kuchen aufgegessen ist], stürzen wir uns auf die Kekse.}
    \ex{\label{ex:saetze0001b} [Ist der Kuchen aufgegessen], stürzen wir uns auf die Kekse.}
  \end{xlist}
\end{exe}

\Zusammenfassung{
In einem \textit{Relativsatz} bezieht sich das Pronomen als Teil der \textit{Relativphrase} auf das \textit{Bezugsnomen} und kongruiert mit ihr in Numerus und Genus.
Ihren Kasus bzw.\ seine Form (\zB als PP) erhält die Relativphrase innerhalb des Relativsatzes per Rektion oder durch ihren Status als Angabe.
Relativsätze können auch ohne Bezugsnomen als \textit{freie Relativsätze} auftreten und verhalten sich dann wie eine NP.
\textit{Komplementsätze} sind Sätze, die eine Valenzstelle (Subjekt oder Objekt) des Matrixverbs füllen (\zB mit \textit{dass}).
\textit{Adverbialsätze} (\zB mit \textit{während}, \textit{damit} oder \textit{weil}) sind Angaben zum Matrixverb.
Nebensätze können nach rechts aus dem Satz versetzt werden, was bei Komplementsätzen nahezu immer der Fall ist (ggf.\ unter Einsetzung eines \textit{Korrelats}).
Alternativ können Adverbialsätze und Komplementsätze (stets ohne Korrelat) im Vorfeld steht.
}


\Uebungen

\Uebung \label{u121} Analysieren Sie die eingeklammerten Strukturen im Rahmen des Feldermodells nach dem Muster des ersten Beispiels.
Bei den Sätzen \ref{it:7493} und \ref{it:7494} handelt es sich um Transferaufgaben.

\begin{enumerate}\Lf
  \item{[Sarah isst den Kuchen alleine auf.]}
    \begin{itemize}\Lf
      \item Kf: ---
      \item Vf: Sarah
      \item LSK: isst
      \item Mf: den Kuchen alleine
      \item RSK: auf
      \item Nf: ---
    \end{itemize}
  \item{[Man sollte den Tag genießen.]}
  \item{[Kann mal jemand das Fenster aufmachen?]}
  \item\label{it:934254} Das ist das Eis, [das wir selber gemacht haben].
  \item{[Was hat Ischariot gemalt?]}
  \item{[Gehst du?]}
  \item{\label{it:7493} [Geh!]}
  \item\label{it:7494} Es ist eine tolle Sommernacht, [denn der Mond scheint hell].
  \item{[Den leckeren Kuchen auf dem Tisch hatte Rigmor sofort entdeckt.]}
  \item{[Obwohl Liv einkaufen wollte], ist nichts im Haus.}
  \item Kann man feststellen, [wer den Kuchen gegessen hat]?
\end{enumerate}

\Uebung \label{u122} Analysieren Sie die folgenden komplexen Sätzen im Rahmen des Feldermodells nach dem Muster des ersten Beispiels.
Dabei sind von eingebetteten Nebensätzen keine Analysen durchzuführen.

\begin{enumerate}\Lf
  \item Dass der Kuchen gegessen wurde, bedauern alle sehr, die es erfahren haben.
    \begin{itemize}\Lf
      \item Kf: ---
      \item Vf: Dass der Kuchen gegessen wurde
      \item LSK: bedauern
      \item Mf: alle sehr
      \item RSK: ---
      \item Nf: die es erfahren haben
    \end{itemize}
  \item Wohin man auch blickt, kann man die Bäume kaum erkennen, denn der Schnee bedeckt alles.
  \item Geht derjenige, der kommt, auch wieder?
  \item Die Kollegen, denen wir nichts vom Kuchen gegeben haben, schimpfen.
  \item Denn ob es Eis gibt, kann nur einer wissen, der Zugang zur Eismaschine hat.
  \item Liv will, dass Rigmor ihr von dem Eis abgibt.
\end{enumerate}

\Uebung \label{u123} Führen Sie Konstituentenanalysen der folgenden Auswahl einfacher Sätze aus Übung \ref{u121} durch (ohne Bewegungspfeile).
Für ein Beispiel (erster Satz) vgl.\ Abbildung~\ref{fig:v2mitpartikel}.

\begin{enumerate}\Lf
  \item{Sarah isst den Kuchen alleine auf.}
  \item{Man sollte den Tag genießen.}
  \item{Kann mal jemand das Fenster aufmachen?}
  \item{Was hat Ischariot gemalt?}
  \item{Gehst du?}
  \item{Den leckeren Kuchen auf dem Tisch hatte Rigmor sofort entdeckt.}
\end{enumerate}

\Uebung[\tristar] \label{u124} Führen Sie Konstituentenanalysen für die folgende Auswahl aus den komplexen Sätzen aus Übung \ref{u122} durch.
Es handelt sich überwiegend um eine Transferaufgabe:
Überlegen Sie, wie das Nachfeld in den Konstituentenstrukturen abgebildet werden kann.

\begin{enumerate}\Lf
  \item Dass der Kuchen gegessen wurde, bedauern alle sehr, die es erfahren haben.
  \item Die Kollegen, denen wir nichts vom Kuchen gegeben haben, schimpfen.
  \item Liv will, dass Rigmor ihr von dem Eis abgibt.
\end{enumerate}

\Uebung[\tristar] \label{u125} Analysieren Sie die folgenden NPs mit Relativsatz nach dem Muster von Abbildung~\ref{fig:saetze9228} (s.\ S.~\pageref{fig:saetze9228}), aber ohne Kongruenz- und Rektionspfeile.

\begin{enumerate}\Lf
  \item{[Das Buch, das ich lese], gehört nicht mir.}
  \item Wir mögen [Menschen, auf die wir vertrauen können].
  \item Wir treffen [die Kommilitoninnen, deren Kuchen wir gegessen haben].
\end{enumerate}

\Uebung[\tristar] \label{u126} Dialektal gibt es Relativsätze bzw.\ eingebettete \textit{w}"=Sätze wie in (\ref{ex:ex234}).

\begin{exe}
  \ex{\label{ex:ex234} Ich weiß, [wer dass kommt].}
\end{exe}

Überlegen Sie, was hier anders ist als im Standard und geben Sie eine Felderanalyse und eine Konstituentenstruktur an.

\Uebung[\tristar] \label{u127} Die deutsche Orthographie zeigt viele interessante grammatische Beziehungen auf.
Überlegen Sie, warum die Form des Verbs \textit{zurückbleiben} in (\ref{ex:ex23456348a}) zusammengeschrieben, aber in (\ref{ex:ex23456348b}) auseinandergeschrieben wird.

\begin{exe}
  \ex\label{ex:ex23456348}
  \begin{xlist}
    \ex{\label{ex:ex23456348a} Es ist in Ordnung, wenn der große Schreibtisch erst einmal zurückbleibt.}
    \ex{\label{ex:ex23456348b} Zurück bleibt der Schreibtisch nur, wenn der LKW randvoll ist.}
  \end{xlist}
\end{exe}

