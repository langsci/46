\chapter{Sätze}

\label{sec:saetze}

\section{Überblick}

Dieses Kapitel beantwortet die Frage nach der Satzgliedstellung in Sätzen, in denen die Konstituentenstellung deutlich von der in einer Verbphrase (VP), wie sie in Abschnitt~\ref{sec:vgr} beschrieben wurde, abweicht.
Außer dem durch Komplementierer eingeleiteten Nebensatz, also einer Komplementiererphrase (KP) mit eingebetteter VP, gibt es weitere wichtige andere Satztypen im Deutschen, die sich jeweils durch eine besondere Satzgliedstellung auszeichnen.
Dies sind die Satzgliedstellung des unabhängigen Aussagesatzes (\ref{ex:saetze9001a}) und die des Fragesatzes mit Fragepronomen (\textit{w}"=Fragesatz) (\ref{ex:saetze9001c}), die des Entscheidungsfragesatzes (Ja\slash Nein-Frage) wie in (\ref{ex:saetze9001b}) und die des Relativsatzes wie in (\ref{ex:saetze9001d}) und des sehr ähnlichen eingebetteten \textit{w}"=Fragesatzes wie in (\ref{ex:saetze9001e}).

\begin{exe}
  \ex\label{ex:saetze9001}
  \begin{xlist}
    \ex{\label{ex:saetze9001a} Wahrscheinlich hat der Arzt das Bild gekauft.}
    \ex{\label{ex:saetze9001c} Was hat der Arzt gekauft?}
    \ex{\label{ex:saetze9001b} Hat der Arzt das Bild gekauft?}
    \ex{\label{ex:saetze9001d} (Das ist das Bild,) das der Arzt gekauft hat.}
    \ex{\label{ex:saetze9001e} (Ischariot fragt sich,) was der Arzt gekauft hat.}
  \end{xlist}
\end{exe}

Während schon definiert wurde, was wir unter einem Nebensatz verstehen, soll jetzt noch gesagt werden, was wir unter einem Satz verstehen wollen.

\Definition{(Unabhängiger) Satz\slash Hauptsatz}{
\label{def:satz}
Ein unabhängiger Satz (den man auch einen Hauptsatz oder abgekürzt einfach einen Satz nennen kann) ist eine Struktur, in der alle Konstituenten mittelbar oder unmittelbar von einem nicht regierten finiten Verb abhängen, in dem alle Valenzanforderungen erfüllt sind, und der nicht von einer weiteren Konstituente syntaktisch abhängig ist.
\index{Satz}
}

Folglich ist (\ref{ex:satz0001a}) ein Satz, weil \textit{ist} ein finites Verb ist, das nicht regiert wird.
Außerdem sind offensichtlich alle Valenzanforderungen erfüllt.
Hingegen kann (\ref{ex:satz0001b}) kein Hauptsatz sein, weil zwar genau ein finites Verb vorkommt, dieses aber von \textit{dass} regiert wird.
Dinge wie (\ref{ex:satz0001c}) und (\ref{ex:satz0001d}) sind nun zwar Äußerungen, aber in dem hier vertretenen Verständnis keine Sätze.
Aus Sicht der Grammatik ist dies durchaus zielführend, weil beide Äußerungen deutlich anders strukturiert sind als (\ref{ex:satz0001a}).

\begin{exe}
  \ex\label{ex:satz0001} 
  \begin{xlist}
    \ex{\label{ex:satz0001a} Die Post ist da.}
    \ex{\label{ex:satz0001b} dass die Post da ist}
    \ex{\label{ex:satz0001c} Hurra!}
    \ex{\label{ex:satz0001d} Nieder mit dem König!}
  \end{xlist}
\end{exe}

Da jetzt alle Satztypen definiert worden sind, kann noch der Begriff des Matrixsatzes eingeführt werden.
Es handelt sich um einen Hilfsbegriff, der zur Beschreibung von Satz-Einbettungen sehr nützlich ist.

\Definition{Matrixsatz}{
\label{def:matrixsatz}
Der Matrixsatz eines Nebensatzes ist der Satz, in den er unmittelbar eingebettet ist.
\index{Matrixsatz}
\index{Matrixsatz}
} 

In Abschnitt~\ref{sec:felder} wird zunächst dafür argumentiert, dass man diese verschiedenen Satzgliedstellungen mittels Bewegung von Konstituenten aus der in Abschnitt~\ref{sec:vgr} beschriebenen VP ableiten kann.
Außerdem wird das sogenannte Feldermodell erläutert, dass diese Satzgliedstellungen deskriptiv klassifiziert.
In Abschnitt~\ref{sec:satzschemata} werden dann Phrasenschemata für Sätze angegeben, die alle wichtigen Satzgliedstellungsvarianten beschreiben.
Schließlich wird in Abschnitt~\ref{sec:nebensaetze} auf Besonderheiten verschiedener Typen sogenannter Nebensätze eingegangen.

\section{Satzgliedstellung und Feldermodell}

\label{sec:felder}

\subsection{Satzgliedstellung in unabhängigen Sätzen}

\index{Verbphrase}
\index{Konstituente}
Die in Abschnitt~\ref{sec:vgr} besprochene VP definiert die Abfolge der Konstituenten innerhalb der VP untereinander nicht.
Dass man die Abfolge nicht spezifizieren muss, zeigen die Beispiele (\ref{ex:vgrorder}) aus Kapitel~\ref{sec:phrasen}, hier als (\ref{ex:vgrorder-rep}) wiederholt.

\begin{exe}
  \ex\label{ex:vgrorder-rep}
  \begin{xlist}
    \ex{Ich glaube, dass dem Jungen seine Mutter ein Eis geschenkt hat.}
    \ex{Ich glaube, dass einen Tofu-Burger der Mann seiner Tochter geschenkt hat.}
  \end{xlist}
\end{exe}

\index{Verbalkomplex}
Was allerdings festgelegt ist, ist, dass der Verbalkomplex ganz am rechten Ende der VP steht.
Im unabhängigen Aussagesatz ist dies nun teilweise anders.
In (\ref{ex:v2fx}) sieht man an den Umformungen eines eingeleiteten Nebensatzes (\ref{ex:v2fxa}) in uneingeleitete Sätze (also sogenannte Hauptsätze), welche zahlreichen Umstellungen möglich sind, nämlich \zB (\ref{ex:v2fxb})--(\ref{ex:v2fxf}).

\begin{exe}
  \ex\label{ex:v2fx}
  \begin{xlist}
    \ex{\label{ex:v2fxa} dass Ischariot wahrscheinlich dem Arzt das Bild verkauft hat}
    \ex{\label{ex:v2fxb} Ischariot hat wahrscheinlich dem Arzt das Bild verkauft.}
    \ex{\label{ex:v2fxc} Wahrscheinlich hat Ischariot dem Arzt das Bild verkauft.}
    \ex{\label{ex:v2fxd} Dem Arzt hat Ischariot wahrscheinlich das Bild verkauft.}
    \ex{\label{ex:v2fxe} Das Bild hat Ischariot wahrscheinlich dem Arzt verkauft.}
    \ex{\label{ex:v2fxf} Verkauft hat Ischariot wahrscheinlich dem Arzt das Bild.}
  \end{xlist}
\end{exe}

Wie bereits mehrfach angemerkt, steht hier das finite Verb immer an zweiter Stelle, und davor steht irgendein anderes Satzglied.
Die Optionen der Voranstellung aus (\ref{ex:v2fx}) werden durch die Voranstellung von komplexeren Satzteilen erweitert, von denen einige in (\ref{ex:v2fx2}) gezeigt werden.

\begin{exe}
  \ex\label{ex:v2fx2}
  \begin{xlist}
    \ex{\label{ex:v2fxf2a} Das Bild verkauft hat Ischariot wahrscheinlich dem Arzt.}
    \ex{\label{ex:v2fxf2b} Dem Arzt das Bild verkauft hat Ischariot wahrscheinlich gestern.}
  \end{xlist}
\end{exe}

Im Vergleich zur VP ergeben sich also mindestens zwei Unterschiede.
Einerseits wird das finite Verb alleine (auch wenn es aus einem Verbalkomplex mit mehreren Verbformen kommt) nach links gestellt.
Sowohl die infiniten Verbformen als auch eventuelle Verbalpartikeln (nicht aber Verbpräfixe) bleiben als Rest eines Verbalkomplexes ohne finite Form am rechten Rand zurück.
Außerdem wird eine andere (scheinbar beliebige) Konstituente davor gestellt.
Die Besonderheiten des Verbalkomplexes bei den Umstellungen werden verdeutlicht in den Beispielen (\ref{ex:vkv2inf}) und (\ref{ex:vkv2prt}).

\begin{exe}
  \ex\label{ex:vkv2inf}
  \begin{xlist}
    \ex{dass Ischariot dem Arzt das Bild hat verkaufen wollen}
    \ex{Ischariot hat dem Arzt das Bild verkaufen wollen.}
    \ex{Das Bild hat Ischariot dem Arzt verkaufen wollen.}
  \end{xlist}
  \ex\label{ex:vkv2prt}
  \begin{xlist}
    \ex{dass der Arzt Ischariot das Bild gerne abkauft}
    \ex{Der Arzt kauft Ischariot das Bild gerne ab.}
    \ex{Gerne kauft der Arzt Ischariot das Bild ab.}
  \end{xlist}
\end{exe}

\index{Bewegung}

Während also innerhalb der VP zwar die Reihenfolge der Teilkonstituenten nicht ganz eindeutig festgelegt ist, aber die VP wenigstens immer eine zusammenhängende Kette von Wörtern bildet, kommt im unabhängigen Aussagesatz die Schwierigkeit hinzu, dass das finite Verb zwar eine festgelegte Stellung hat, dafür aber der Verbalkomplex auseinandergerissen wird und eine beliebige Konstituente aus dem VP-Zusammenhang herausbewegt wird.
Weil aber eben die Stellung im eingeleiteten Nebensatz (VP innerhalb einer KP) wesentlich besser systematisch zu beschreiben ist, ist es sinnvoll, die syntaktische Analyse von Sätzen mit der VP zu beginnen (so wie im letzten Kapitel), und alle anderen Satzgliedstellungstypen als Umstellungen dieser Grundstellung zu beschreiben.

Nehmen wir also an, wir hätten eine VP wie in Abbildung~\ref{fig:vgreinstelligwh} und sollten angeben, was sich im Vergleich zu dieser im unabhängigen Aussagesatz ändert.
Wir sprechen im Folgenden davon, dass Konstituenten bewegt werden.
Einige Theorien wie \zB die Government and Binding Theory (GB) oder das Minimalist Program (MP) nehmen tatsächlich Bewegung im Sinne eines mehrstufigen Umbaus von Strukturen an.
Andere Theorien wie die Head-Driven Phrase Structure Grammar (HPSG) modellieren dieselben Phänomene ohne solche Umbauoperationen, modellieren aber denselben Effekt.
Aus unserer deskriptiven Sicht ist der Begriff der Bewegung in jedem Fall als Hilfsvorstellung akzeptabel, und wir benutzen ihn ohne theoretisch Partei nehmen zu wollen.

\begin{figure}
  \resizebox{\textwidth}{!}{
    \Tree{
      &&&&&&&&&&&&& \K{VP}\B{ddllllllllllll}\B{ddllllllllll}\B{ddllllllll}\B{ddllllll}\B{ddllll}\B{ddll}\B{dd} \\
      \\
      & \K{NP}\TRi[-3] && \K{AdvP}\TRi[-2] && \K{NP}\TRi[-3] && \K{AdvP}\TRi[-4] && \K{NP}\TRi[-4] && \K{AdvP}\TRi[-4] && \K{\textbf{V}}\B{d} \\
      & \K{\textit{Ischariot}} && \K{\textit{wahrscheinlich}} && \K{\textit{dem Arzt}} && \K{\textit{heimlich}} && \K{\textit{das Bild}} && \K{\textit{schnell}} && \K{\textit{verkauft}} & \\
    }
  }
  \caption{VP mit dreistelliger Valenz und Adverbialen}
  \label{fig:vgreinstelligwh}
\end{figure}

Wir wissen, dass das finite Verb im Ergebnis an der zweiten Position im Satz stehen soll.
Es stellt sich die Frage, wie man die Umstellungsoperationen am besten formulieren kann, so dass das finite Verb die zweite Position findet.
Innerhalb der VP (\zB in Abbildung~\ref{fig:vgreinstelligwh}) die zweite Position zu suchen und das finite Verb dort einzuschieben, hätte aus diversen technischen und konzeptuellen Gründen wenig Sinn.
Statt in einer fertigen VP die zweite Position zu suchen, gibt es eine einfachere Art, automatisch sicherzustellen, dass das finite Verb am Ende an zweiter Stelle steht und irgendein anderes Satzglied davor positioniert wird.
Man führt in dieser Reihenfolge die folgenden beiden Operationen an einer normalen VP durch:

\begin{enumerate}\Lf
  \item Stelle das finite Verb vor die VP.
  \item Stelle dann eine andere Konstituente vor das finite Verb.
\end{enumerate}

Die etwas einfachere VP (\textit{dass}) \textit{Ischariot wahrscheinlich das Bild verkauft hat}, aus der gemäß dieser Anweisungen Konstituenten herausbewegt wurden, sieht aus wie in Abbildung~\ref{fig:movev2}.
Es ergibt sich ein unabhängiger Aussagesatz allein dadurch, dass erst das finite Verb \textit{hat} und dann die Konstituente \textit{das Bild} nach links gestellt wurde.

\begin{figure}
    \Tree{
      &&&&&&&&&& \K{VP}\B{ddlllll}\B{ddlll}\B{ddll}\B{d} \\
      &&&&&&&&&& \K{\textbf{V}}\B{dl}\B{d} \\
      & \K{NP\ORii}\TRi[-7] && \K{\textbf{V\ORi}}\B{d} && \K{NP}\TRi[-6] && \K{AdvP}\TRi[-2] & \K{\Tii}
      \POS[]-(0,4)\ar@{-->}@/^{4pc}/[lllllll]-(0,4) & \K{\textbf{V}}\B{d} & \K{\Ti}\POS[]-(0,4)\ar@{-->}@/^{4pc}/[lllllll]-(0,4) \\
      & \K{\textit{das Bild}} && \K{\textit{hat}} && \K{\textit{Ischariot}} && \K{\textit{wahrscheinlich}} && \K{\textit{verkauft}} & \\
    }
  \vspace{0.3cm}
  \caption{VP mit hinausbewegten Konstituenten}
  \label{fig:movev2}
\end{figure}

\index{Spur}

Wir verstehen die Stellung im Verb"=Zweit"=Satz (V2) als das Ergebnis von zwei Umstellungsoperationen bzw.\ Bewegungen.
Es bleibt eine VP mit zwei Lücken zurück, wobei diese Lücken in vielen Theorien als \textit{Trace} (engl.\ für \textit{Spur}) bezeichnet werden und daher meist als \textit{t} symbolisiert werden.
Wenn man die Lücken bzw.\ Spuren und die dazugehörigen bewegten Konstituenten durchnumeriert, sind die Bewegungsoperationen eindeutig nachvollziehbar.
Die gepunkteten Pfeile, die die Bewegung andeuten, sind dann im Prinzip nicht nötig und dienen hier nur der Verdeutlichung.

\subsection{Das Feldermodell}

Unser Ziel ist es nun, diese Strukturen möglichst auch mit Phrasenschemata zu beschreiben, denn in Abbildung~\ref{fig:movev2} stehen die bewegten Konstituenten V\ORi\ und NP\ORii\ im syntaktischen Nichts, sie sind in keine Struktur eingebunden.
Diese Beschreibung ist insofern problematisch, weil wir sie in unserem Strukturformat (Konstituentenbäume) nicht richtig ausformulieren können.
In Abschnitt~\ref{sec:satzschemata} wird ein Vorschlag gemacht, wie Bewegung relativ einfach phrasenstrukturell dargestellt werden kann.

Vorher wird jetzt aber ein anderes sehr populäres Beschreibungsmodell eingeführt, das dabei helfen soll, die Regularitäten des Satzbaus im Deutschen nochmals zu verdeutlichen, bevor die phrasenstrukturelle Modellierung in Angriff genommen wird.
Dieses sogenannte Feldermodell liefert eine einfache Terminologie zur Beschreibung der sich durch den Bau der VP und die gerade besprochenen Umsortierungen der Konstituenten im unabhängigen Aussagesatz ergebenden Satzgliedstellungsvarianten.
Das Modell bezieht sich dabei nicht auf Konstituentenstrukturen, sondern nur auf die lineare Abfolge der Satzteile.

\Satz{Feldermodell}{
\label{satz:felder}
Das Feldermodell ist ein deskriptives Modell, das ohne Bezug auf die Phrasenstruktur die lineare Abfolge von Satzteilen im Deutschen beschreibt.
\index{Feldermodell}
}

\index{Satzklammer}
\index{Vorfeld}
\index{Mittelfeld}
\index{Komplementierer}

Die erste wichtige Idee des Feldermodells ist es, dass der Verbalkomplex in allen Arten von Sätzen wegen seiner Stellung am rechten Rand (der VP) eine gut erkennbare rechte Grenze, die rechte Satzklammer (RSK), bildet.
Zusätzlich gibt es in allen Arten von (abhängigen und unabhängigen) Sätzen eine gut erkennbare linke Begrenzung:
Im eingeleiteten Nebensatz (den wir als KP analysieren) steht der Komplementierer ganz links, und kein Satzglied des Nebensatzes darf links davon stehen.
Im unabhängigen Aussagesatz (ohne Komplementierer) steht das finite Verb links an zweiter Stelle (in unserer Terminologie links von der VP).
Wegen ihrer markanten Position im linken Satzbereich werden der Komplementierer und das links stehende finite Verb im unabhängigen Aussagesatz in der Terminologie des Feldermodells die sogenannte linke Satzklammer (LSK) genannt.

Anhand der beiden Satzklammern kann man dann den Rest des Satzes stellungsmäßig aufteilen:
Das Vorfeld (Vf) ist der Bereich links von der LSK.
Das Mittelfeld (MF) ist der Bereich zwischen den Satzklammern.
Für den durch einen Komplementierer eingeleiteten Nebensatz und den unabhängigen Aussagesatz ergeben sich also die Einteilungen in Felder wie in Abbildung~\ref{fig:felder1}.

\begin{figure}
  \resizebox{\textwidth}{!}{
    \begin{tabular}{lp{0.1em}cp{0.1em}cp{0.1em}cp{0.1em}c}
      \textbf{Satztyp} && \textbf{Vf} && \textbf{LSK} && \textbf{Mf} && \textbf{RS} \\
      \cmidrule{1-1}\cmidrule{3-3}\cmidrule{5-5}\cmidrule{7-7}\cmidrule{9-9}
      \textbf{unabh.\ Aussagesatz} && \textit{das Bild} && \textit{hat} && \textit{Ischariot} \textit{wahrscheinlich} && \textit{verkauft} \\
      \textbf{eingel.\ Nebensatz} &&&& \textit{dass} && \textit{Ischariot das Bild wahrscheinlich} && \textit{verkauft hat} \\
    \end{tabular}
  }
  \caption{Felder im unabhängigen Aussagesatz und im Nebensatz}
  \label{fig:felder1}
\end{figure}

\index{w-Frage}
\index{Echofrage}
\index{Fragesatz}
\index{w-Satz}

Jetzt soll gezeigt werden, wie auch in einigen anderen Satztypen das Feldermodell eine adäquate Beschreibung der linearen Satzgliedfolge liefert.
Besonders sind hier der \textit{w}"=Fragesatz (\ref{ex:saetze2374a}), der Entscheidungsfragesatz (\ref{ex:saetze2374b}) und der Relativsatz (\ref{ex:saetze2374c}) bzw.\ der eingebettete \textit{w}"=Fragesatz (\ref{ex:saetze2374d}) zu behandeln.

\begin{exe}
  \ex\label{ex:saetze2374} 
  \begin{xlist}
    \ex{\label{ex:saetze2374a} Wem hat Ischariot das Bild verkauft?}
    \ex{\label{ex:saetze2374b} Hat Ischariot das Bild verkauft?}
    \ex{\label{ex:saetze2374c} Das ist der Mann, dem Ischariot das Bild verkauft hat.}
    \ex{\label{ex:saetze2374d} Ischariot weiß, wer die guten Bilder verkauft.}
  \end{xlist}
\end{exe}

\label{abs:923478} Der \textit{w}"=Fragesatz stellt sich im Grunde wie ein unabhängiger Aussagesatz dar, wobei aber das Fragepronomen (\textit{w}"=Pronomen) und nicht irgendeine frei wählbare Konstituente obligatorisch im Vf stehen muss.
Wenn dies nicht der Fall ist, erhält man eine sogenannte In-Situ-Frage oder auch Echofrage wie in (\ref{ex:saetze2375a}), zu der der Aussagesatz (\ref{ex:saetze2375b}) zum Vergleich angegeben ist.

\begin{exe}
  \ex\label{ex:saetze2375}
  \begin{xlist}
    \ex{\label{ex:saetze2375a} Ischariot hat wem das Bild verkauft?}
    \ex{\label{ex:saetze2375b} Ischariot hat dem Mann das Bild verkauft.}
  \end{xlist}
\end{exe}

Bei einer solchen Frage bleibt das \textit{w}"=Pronomen an der Stelle, an der die korrespondierende Phrase im zugehörigen Aussagesatz stehen würde.
Ins Vorfeld wird dann in der In-Situ-Frage eine andere Konstituente gestellt (hier \zB \textit{Ischariot}).
Echofragen sind typisch in Kontexten, in denen der Fragende eine Verständnisfrage stellt, weil er das betreffende Satzglied \zB akustisch nicht verstanden hat, vgl.\ den kurzen Dialog in (\ref{ex:saetze2376}).

\begin{exe}
  \ex\label{ex:saetze2376} A: Ischariot hat dem Mann das Bild verkauft.\\
  B: Ischariot hat wem das Bild verkauft?
\end{exe}

Falls mehrere \textit{w}"=Pronomina (oder komplexe \textit{w}"=Ausdrücke) im \textit{w}"=Fragesatz vorkommen, muss (In-Situ-Fragen ausgenommen) eines von diesen in das Vf gestellt werden, die anderen verbleiben in der VP.
Dies ist in (\ref{ex:saetze2377}) dargestellt.

\begin{exe}
  \ex\label{ex:saetze2377}
  \begin{xlist}
    \ex{Wem hat Ischariot was wie verkauft?} 
    \ex{Wie hat Ischariot wem was verkauft?} 
    \ex{Was hat Ischariot wem wie verkauft?} 
  \end{xlist}
\end{exe}

Der Entscheidungsfragesatz in (\ref{ex:saetze2374b}) ist ebenfalls dem unabhängigen Aussagesatz ähnlich, weil das finite Verb nach links bewegt wird.
Allerdings entfällt die Besetzung des Vf, die LSK (in Form des finiten Verbs) bildet also den absolut linken Abschluss des Satzes.

\index{Relativsatz}
\index{Fragesatz!eingebettet}

Relativsatz und eingebetteter \textit{w}"=Fragesatz werden hier gemeinsam behandelt.
Dabei wird der Relativsatz exemplarisch besprochen, der eingebettete \textit{w}"=Fragesatz ist strukturell völlig identisch.
Zur Verwendung des eingebetteten \textit{w}"=Fragesatzes s.\ Abschnitt~\ref{sec:komplementsaetze}.
Ein Relativsatz wie in (\ref{ex:saetze2374c}) ähnelt dem durch einen Komplementierer eingeleiteten Fragesatz insofern, als der Verbalkomplex am rechten Rand intakt bleibt und das finite Verb nicht nach links bewegt wird.
Dafür wird das Relativpronomen (hier \textit{wem}) obligatorisch nach links bewegt und steht im Vf.

Man kann nun die Satztypen wie in den Abbildungen~\ref{fig:feldertypen1} bis~\ref{fig:feldertypen4} zusammenfassen.

\begin{figure}
  \resizebox{\textwidth}{!}{
    \begin{tabular}{cp{0.1em}cp{0.1em}cp{0.1em}c}
      \textbf{Vf} && \textbf{LSK} && \textbf{Mf} && \textbf{RSK} \\
      \cmidrule{1-1}\cmidrule{3-3}\cmidrule{5-5}\cmidrule{7-7}
	eine Konstituente && finites Verb && (Rest) && infinite Verben \\
	\textit{das Bild} && \textit{hat} && \textit{Ischariot wahrscheinlich} && \textit{verkauft} \\
    \end{tabular}
  }
  \caption{Feldermodell: unabhängiger Aussagesatz (V2)}
  \label{fig:feldertypen1}
\end{figure}

\begin{figure}
  \resizebox{\textwidth}{!}{
    \begin{tabular}{cp{0.1em}cp{0.1em}cp{0.1em}c}
      \textbf{Vf} && \textbf{LSK} && \textbf{Mf} && \textbf{RSK} \\
      \cmidrule{1-1}\cmidrule{3-3}\cmidrule{5-5}\cmidrule{7-7}
	(leer) && Komplementierer && (Rest) && Verbalkomplex \\
	&& \textit{dass} && \textit{Ischariot das Bild wahrscheinlich} && \textit{verkauft hat} \\
    \end{tabular}
  }
  \caption{Feldermodell: Nebensatz mit Komplementierer (VL)}
  \label{fig:feldertypen2}
\end{figure}

\begin{figure}
    \begin{tabular}{cp{0.1em}cp{0.1em}cp{0.1em}c}
      \textbf{Vf} && \textbf{LSK} && \textbf{Mf} && \textbf{RSK} \\
      \cmidrule{1-1}\cmidrule{3-3}\cmidrule{5-5}\cmidrule{7-7}
	(leer) && finites Verb && (Rest) && infinite Verben \\
	&& \textit{hat} && \textit{Ischariot das Bild} && \textit{verkauft} \\
    \end{tabular}
  \caption{Feldermodell: Entscheidungsfragesatz (V1)}
  \label{fig:feldertypen3}
\end{figure}

\begin{figure}
  \resizebox{\textwidth}{!}{
    \begin{tabular}{cp{0.1em}cp{0.1em}cp{0.1em}c}
      \textbf{Vf} && \textbf{LSK} && \textbf{Mf} && \textbf{RSK} \\
      \cmidrule{1-1}\cmidrule{3-3}\cmidrule{5-5}\cmidrule{7-7}
	Relativpronomen && (leer) && (Rest) && Verbalkomplex \\
	\textit{dem} &&&& \textit{Ischariot das Bild wahrscheinlich} && \textit{verkauft hat} \\ 
    \end{tabular}
  }
  \caption{Feldermodell: Relativsatz (VL)}
  \label{fig:feldertypen4}
\end{figure}

\index{Verb-Erst-Satz}
\index{Verb-Letzt-Satz}
\index{Verb-Zweit-Satz}

\begin{sloppypar}
Für die grundlegenden Satztypen gibt es konkurrierende Bezeichnungen.
Meistens werden sie nach der Stellung des finiten Verbs kategorisiert.
Man spricht dann vom Verb"=Erst"=Satz oder V1-Satz (Entscheidungsfragesatz), vom Verb"=Zweit"=Satz oder V2-Satz (unabhängiger Aussagesatz und \textit{w}"=Fragesatz) und vom Verb"=Letzt"=Satz oder VL-Satz (eingeleiteter Nebensatz und Relativsatz).
\end{sloppypar}

All diese Bezeichnungen kategorisieren die Sätze nach der Art, wie die vier primären Felder Vf, LSK, Mf und RSK gefüllt werden.
Für weitere Stellungsvarianten werden zusätzliche Felder angenommen, um die es in Abschnitt~\ref{sec:nebenfelder} geht.
Zuvor soll ein einfacher Test besprochen werden, mit dem das Vf in komplizierteren Sätzen ermittelt werden kann.

Neben den primären Feldern, für die genau angegeben werden kann, wie sie in den verschiedenen Satztypen zu füllen sind, werden noch mindestens zwei weitere Felder angenommen.
Zunächst betrachten wir Sätze wie die in (\ref{ex:saetze2381}).

\begin{exe}
  \ex\label{ex:saetze2381}
  \begin{xlist}
    \ex{\label{ex:saetze2381a} Ischariot hat dem Arzt das Bild verkauft, das er selber gemalt hatte.}
    \ex{\label{ex:saetze2381b} Der Arzt hat Ischariot nicht geglaubt, dass das Bild echt war.}
  \end{xlist}
\end{exe}

\index{Relativsatz}
\index{Komplementsatz}

In diesen Sätzen stehen einmal ein Relativsatz (\ref{ex:saetze2381a}) und einmal ein Komplementsatz (\ref{ex:saetze2381b}) nach dem infiniten Verb.
Im Fall des Relativsatzes kann man besonders gut erkennen, dass dieser nach rechts bewegt wurde, denn die NP, zu der er strukturell gehört (\textit{das Bild}), befindet sich im Mf, und NP und Relativsatz sind durch die RSK (\textit{verkauft}) voneinander getrennt.
Man geht im Falle solcher rechts von der RSK positionierten Konstituenten davon aus, dass sie wegen ihrer Länge aus dem Mf herausbewegt (rechtsversetzt) werden.
Im Rahmen des Feldermodells nennt man die entsprechende Position das Nachfeld (Nf).
Eine Analyse wird in Abbildung~\ref{fig:nachfeld} gegeben.

\index{Nachfeld}

\begin{figure}
  \centering
  \resizebox{\textwidth}{!}{
    \begin{tabular}{cp{0.1em}cp{0.1em}cp{0.1em}cp{0.1em}c}
      \textbf{Vf} && \textbf{LSK} && \textbf{Mf} && \textbf{RSK} && \textbf{Nf} \\
      \cmidrule{1-1}\cmidrule{3-3}\cmidrule{5-5}\cmidrule{7-7}\cmidrule{9-9}
      \textit{Ischariot} && \textit{hat} && \textit{dem Arzt das Bild} && \textit{verkauft} && \textit{das er selber gemalt hatte} \\
    \end{tabular}
  }
  \caption{Felderanalyse mit Nf}
  \label{fig:nachfeld}
\end{figure}

Außerdem gibt es vermeintliche Komplementierer wie \textit{denn}, die sich aber anders als echte Komplementierer verhalten, vgl.\ (\ref{ex:saetze1924}).

\begin{exe}
  \ex\label{ex:saetze1924}
  \begin{xlist}
    \ex{\label{ex:saetze1924a} Der Arzt ist froh, weil Ischariot ihm das Bild verkauft hat.}
    \ex{\label{ex:saetze1924b} Der Arzt ist froh, denn Ischariot hat ihm das Bild verkauft.}
  \end{xlist}
\end{exe}

\index{Konnektor}
\index{Konnektorfeld}

Das Wort \textit{denn} muss gemäß unserer Wortklassifikation als Partikel (nicht etwa als Komplementierer) klassifiziert werden, denn es bettet keinen Nebensatz mit Verb-Letzt-Stellung ein, sondern einen Satz, der wie ein unabhängiger Aussagesatz strukturiert ist.
Solche Partikeln nennt man auch Konnektoren, und man kann innerhalb des Feldermodells für sie ein Konnektorfeld (Kf) oder Vor-Vorfeld ansetzen, das noch vor dem Vf positioniert ist.
Eine solche Analyse ist in Abbildung~\ref{fig:konnektorfeld} angegeben.

\begin{figure}
  \centering
  \begin{tabular}{cp{0.1em}cp{0.1em}cp{0.1em}cp{0.1em}c}
    \textbf{Kf} && \textbf{Vf} && \textbf{LSK} && \textbf{Mf} && \textbf{RSK} \\
    \cmidrule{1-1}\cmidrule{3-3}\cmidrule{5-5}\cmidrule{7-7}\cmidrule{9-9}
    \textit{denn} && \textit{Ischariot} && \textit{hat} && \textit{ihm das Bild} && \textit{verkauft} \\
  \end{tabular}
  \caption{Felderanalyse mit Kf}
  \label{fig:konnektorfeld}
\end{figure}

Abschließend sei angemerkt, dass viele Felder natürlich leer bleiben können, s.\ Abbildung~\ref{fig:leerefelder}.
Außerdem kann beobachtet werden, wie Verbpartikel in der RSK zurückbleiben, wenn das finite Verb in die LSK gestellt wird, vgl.\ Abbildung~\ref{fig:verbalpartikelalleinzuhaus}.
Die abgekürzte Felderanalyse aus Abbildung~\ref{fig:syn3334wa1} (und entsprechend \ref{fig:syn3334wb2}) kann jetzt auch mit Bezugnahme auf das Nf ergänzt werden, s.\ Abbildung~\ref{fig:syn3334wa1w}.

\begin{figure}
  \centering
  \begin{tabular}{cp{0.1em}cp{0.1em}cp{0.1em}cp{0.1em}cp{0.1em}c}
    \textbf{Kf} && \textbf{Vf} && \textbf{LSK} && \textbf{Mf} && \textbf{RSK} && \textbf{Nf} \\
    \cmidrule{1-1}\cmidrule{3-3}\cmidrule{5-5}\cmidrule{7-7}\cmidrule{9-9}
    && \textit{Ischariot} && \textit{malt} &&&&&& \\
  \end{tabular}
  \caption{Felderanalyse eines V2-Satzes mit leeren Feldern}
  \label{fig:leerefelder}
\end{figure}

\begin{figure}
  \centering
  \begin{tabular}{cp{0.1em}cp{0.1em}cp{0.1em}cp{0.1em}cp{0.1em}c}
    \textbf{Kf} && \textbf{Vf} && \textbf{LSK} && \textbf{Mf} && \textbf{RSK} && \textbf{Nf} \\
    \cmidrule{1-1}\cmidrule{3-3}\cmidrule{5-5}\cmidrule{7-7}\cmidrule{9-9}\cmidrule{11-11}
    && \textit{Ischariot} && \textit{fährt} && \textit{den Pfosten} && \textit{um=} & \\
  \end{tabular}
  \caption{Felderanalyse eines V2-Satzes mit Verbalpartikel}
  \label{fig:verbalpartikelalleinzuhaus}
\end{figure}

\begin{figure}
  \centering
  \resizebox{\textwidth}{!}{
    \begin{tabular}{cp{0.1em}cp{0.1em}cp{0.1em}cp{0.1em}c}
      \textbf{Vf} && \textbf{LSK} && \textbf{Mf} && \textbf{RSK} && \textbf{Nf} \\
      \cmidrule{1-1}\cmidrule{3-3}\cmidrule{5-5}\cmidrule{7-7}\cmidrule{9-9}
      \textit{wer} &&&&&& \textit{glaubt} && \textit{dass Tiere im Tierheim ein schönes Leben haben} \\
    \end{tabular}
  }
  \caption{Felderanalyse mit Komplementsatz im Relativsatz}
  \label{fig:syn3334wa1w}
\end{figure}

In den Übungen (inkl.\ Musterlösungen) werden noch diverse Felderanalysen konkreter Sätze gegeben.
Hier belassen wir es bei diesem Überblick und überlegen im weiteren Verlauf dieses Kapitels, wie man die Beschreibung der verschiedenen Satztypen in Konstituentenstrukturen ausdrücken kann, also im selben Strukturformat wie beim Gruppenbau.
Dies ist das Thema des nachfolgenden Abschnitts.


\subsection{LSK-Test und Nebensätze}

\label{sec:lsktest}

In Abschnitt~\ref{sec:vorfeldtest} wurde im Rahmen der Besprechung des Vorfeldtests darauf verwiesen, dass es nicht immer trivial ist, die LSK (und damit das Vf) zu identifizieren.
Das Problem rührt daher, dass je nach Satzstruktur das erste finite Verb auch das Verb eines eingebetteten Nebensatzes sein kann, wenn dieser Nebensatz \zB im Vorfeld eines anderen Satzes steht.
Die Beispiele (\ref{ex:syn3334}) aus Kapitel~\ref{sec:phrasen} werden hier zur Illustration als (\ref{ex:syn3334w}) wiederholt.
In (\ref{ex:syn3334wa}) sind sowohl \textit{glaubt} als auch \textit{haben} finit, in (\ref{ex:syn3334wb}) kommt \textit{irrt} hinzu.

\begin{exe}
  \ex\label{ex:syn3334w}
  \begin{xlist}
    \ex{\label{ex:syn3334wa} [Wer] glaubt, dass Tiere im Tierheim ein schönes Leben haben?}
    \ex{\label{ex:syn3334wb} [Wer glaubt, dass Tiere im Tierheim ein schönes Leben haben], irrt.}
  \end{xlist}
\end{exe}

\index{w-Satz}
Das Feldermodell ermöglicht nun sowohl Analysen der Struktur des Matrixsatzes als auch der eingebetteten Nebensätze.
Um systematisch die Analyse des Matrixsatzes und der Nebensätze anzugehen, muss zunächst die LSK des Matrixsatzes (also der äußeren Struktur) gefunden werden.
Eine Testprozedur zur Ermittlung der LSK des Matrixsatzes besteht darin, den Satz als Entscheidungs- oder \textit{w}"=Frage zu erkennen und daraus die richtigen Schlüsse zu ziehen, oder den Satz in eine Entscheidungsfrage umzuformen.
Im Entscheidungsfragesatz steht das finite Verb immer am Anfang, und es ist daher eindeutig zu erkennen.
Nehmen wir zunächst einen einfacheren Satz wie (\ref{ex:saetze9292}).

\begin{exe}
  \ex{\label{ex:saetze9292} Der Maler hat dem Arzt ein Bild geschenkt, das jetzt in der Praxis hängt.}
\end{exe}

Hier bieten sich \textit{hat} und \textit{hängt} als finite Verben des Matrixsatzes an.
Das andere finite Verb muss das finite Verb eines eingebetteten Nebensatzes sein, da jede Satzstruktur (ob unabhängig oder abhängig) nur maximal ein finites Verb enthält.
Formuliert man (\ref{ex:saetze9292}) nun in eine Entscheidungsfrage um, erkennt man sofort, dass \textit{hat} das finite Verb des unabhängigen Satzes sein muss, s.\ (\ref{ex:satze9293}).

\begin{exe}
  \ex{\label{ex:satze9293} Hat der Maler dem Arzt ein Bild geschenkt, das jetzt in der Praxis hängt?}
\end{exe}

In der Umformung steht \textit{hat} am Satzanfang, und es kann daher geschlossen werden, dass es in (\ref{ex:saetze9292}) die LSK besetzt.
Dass man mit dem Test die richtige Frage produziert hat, erkennt man daran, dass der ursprüngliche Satz mit vorangestelltem \textit{Ja} eine adäquate (wenn auch umständliche) positive Antwort wäre, hier also (\ref{ex:saetze9292ant}).

\begin{exe}
  \ex{\label{ex:saetze9292ant} Ja, der Maler hat dem Arzt ein Bild geschenkt, das jetzt in der Praxis hängt.}
\end{exe}

Wenn wir nun auf (\ref{ex:syn3334w}) zurückkommen, gilt es zunächst zu beachten, dass (\ref{ex:syn3334wa}) bereits eine \textit{w}"=Frage ist.
Insofern ist \textit{glaubt} prinzipiell ohne Umstellung als finites Verb (LSK) zu identifizieren, denn im \textit{w}"=Fragesatz ist das Vf immer mit dem \textit{w}"=Pro\-no\-men (hier \textit{wer}) besetzt.
Die Umformung in eine Entscheidungsfrage ergibt das gleiche Ergebnis, wobei allerdings ein Pronomen ausgetauscht werden muss, nämlich hier \textit{wer} zu einem Pronomen wie \textit{irgendjemand}, s.\ (\ref{ex:saetze9294}).

\begin{exe}
  \ex{\label{ex:saetze9294} Glaubt irgendjemand, dass Tiere im Tierheim ein schönes Leben haben?}
\end{exe}

Sätze wie der in (\ref{ex:syn3334wb}) sind insofern schwierig, als im Vf hier ein sogenannter freier Relativsatz [\textit{wer \ldots haben}] (vgl.\ Abschnitt~\ref{sec:relativsaetze}) steht, der wiederum einen Komplementsatz [\textit{dass \ldots haben}] (vgl.\ Abschnitt~\ref{sec:komplementsaetze}) enthält.
Das finite Verb des Matrixsatzes ist dadurch das insgesamt dritte, nämlich \textit{irrt}.
Bei der Umformung in eine Entscheidungsfrage müssen nun Pronomina ausgetauscht und hinzugefügt werden, um den Satz völlig akzeptabel zu machen.
Die einfache Umstellung (ohne Austausch und Ergänzung von Pronomina), die bezüglich ihrer Grammatikalität etwas fragwürdig ist, findet sich in (\ref{ex:saetze1426a}), die völlig akzeptable Version (mit Austausch\slash Ergänzung von Pronomina) in (\ref{ex:saetze1426b}).
Mit dieser Umformung in eine Entscheidungsfrage ist also auch hier das richtige finite Verb zu identifizieren.

\begin{exe}
  \ex\label{ex:saetze1426}
  \begin{xlist}
    \ex{\label{ex:saetze1426a} Irrt, wer glaubt, dass Tiere im Tierheim ein schönes Leben haben?}
    \ex{\label{ex:saetze1426b} Irrt derjenige, der glaubt, dass Tiere im Tierheim ein schönes Leben haben?}
  \end{xlist}
\end{exe}

Wie oben angedeutet, muss natürlich für den unabhängigen Satz (Matrixsatz) und die abhängigen Sätze (Nebensätze) je eine Felderanalyse durchgeführt werden.
Im Fall von Nebensätzen ist sozusagen eine vollständige Felderstruktur in eine andere eingebettet.
Für die Sätze (\ref{ex:syn3334w}) sieht das aus wie in den Abbildungen~\ref{fig:syn3334wa1}--\ref{fig:syn3334wb2}.

\begin{figure}
  \centering
  \begin{tabular}{cp{0.1em}cp{0.1em}c}
    \textbf{Vf} && \textbf{LSK} && \textbf{weiteres Feld} \\
    \cmidrule{1-1}\cmidrule{3-3}\cmidrule{5-5}
    \textit{Wer} && \textit{glaubt} && \textit{dass Tiere im Tierheim ein schönes Leben haben} \\
  \end{tabular}
  \caption{Felderanalyse eines V2-Satzes mit Nebensatz}
  \label{fig:syn3334wa1}
\end{figure}

\begin{figure}
  \centering
  \begin{tabular}{cp{0.1em}cp{0.1em}cp{0.1em}c}
    \textbf{Vf} && \textbf{LSK} && \textbf{Mf} && \textbf{RSK} \\
    \cmidrule{1-1}\cmidrule{3-3}\cmidrule{5-5}\cmidrule{7-7}
    && \textit{dass} && \textit{Tiere im Tierheim ein schönes Leben} && \textit{haben} \\
  \end{tabular}
  \caption{Felderanalyse für den Nebensatz aus Abbildung~\ref{fig:syn3334wa1}}
  \label{fig:syn3334wa2}
\end{figure}

\begin{figure}
  \centering
  \resizebox{\textwidth}{!}{
    \begin{tabular}{cp{0.1em}cp{0.1em}cp{0.1em}c}
      \textbf{Vf} && \textbf{LSK} && \textbf{Mf} && \textbf{RSK} \\
      \cmidrule{1-1}\cmidrule{3-3}\cmidrule{5-5}\cmidrule{7-7}
      \textit{Wer glaubt, dass Tiere im Tierheim ein schönes Leben haben} && \textit{irrt} &&&& \\
    \end{tabular}
  }
  \caption{Felderanalyse eines V2-Satzes mit komplexem Vf}
  \label{fig:syn3334wb1}
\end{figure}

\begin{figure}
  \centering
  \resizebox{\textwidth}{!}{
    \begin{tabular}{cp{0.1em}cp{0.1em}cp{0.1em}cp{0.1em}c}
      \textbf{Vf} && \textbf{LSK} && \textbf{Mf} && \textbf{RSK} && \textbf{weiteres Feld} \\
      \cmidrule{1-1}\cmidrule{3-3}\cmidrule{5-5}\cmidrule{7-7}\cmidrule{9-9}
      \textit{wer} &&&&&& \textit{glaubt} && \textit{dass Tiere im Tierheim ein schönes Leben haben} \\
    \end{tabular}
  }
  \caption{Felderanalyse eines VL-Satzes mit Komplementsatz}
  \label{fig:syn3334wb2}
\end{figure}

In Abbildung~\ref{fig:syn3334wa1} wurde der Bereich nach der LSK nicht weiter analysiert, und in Abbildung~\ref{fig:syn3334wb2} wurde ein Bereich nach der RSK eingeführt, aber nicht benannt.
Die Nebensätze stehen in diesen Fällen im Nachfeld, einem weiteren Feld, das in Abschnitt~\ref{sec:nachfeld} eingeführt wird.



\section{Schemata für Sätze}

\label{sec:satzschemata}
\label{sec:verbzweitsatz}

\subsection{Konstituentenstruktur und V2-Sätze}

\label{sec:konstituentenstrukturinv2}

Der Bau der Gruppen ist geprägt von einer reichen internen Struktur und von Valenz- und Rektions-Beziehungen.
Das Feldermodell hingegen ist ein von diesem Gruppenbau unabhängiges reines Linearisierungsmodell, also eine Beschreibung der Abfolge von Satzteilen, ohne dass deren Struktur weiter betrachtet wird.
Das ist der Grund, warum das Feldermodell die üblicherweise angenommene Konstituentenstruktur nicht direkt nachbilden kann.
Die Beziehung zwischen Feldermodell und Konstituentenstruktur wird daher jetzt verdeutlicht, aber es soll dabei immer klar sein, dass die beiden Beschreibungsmodelle (Feldermodell und Phrasenstruktur) nichts direkt miteinander zu tun haben, außer dass sie beide den Satzbau des Deutschen beschreiben.
Beide sind ausgesprochen populär, und man kann sie (wie es jetzt hier geschehen wird) miteinander vergleichen, aber in einem Phrasenstrukturbaum haben Felderbezeichnungen nichts verloren, genauso wie in einer Felderanalyse Phrasenbezeichnungen nichts verloren haben.

Beginnen wir damit, parallel zu einer Konstituentenanalyse eines V2-Satzes (inkl.\ Bewegung) die Felder zu markieren.
In Abbildung~\ref{fig:movev2konstituenten} geschieht dies durch die Felder-Boxen unter dem Baum mit den herausbewegten Konstituenten.
Offensichtlich können bestimmte Knoten im Strukturbaum der VP und die herausgestellten Konstituenten bestimmten Feldern des Feldermodells zugeordnet werden.
Das Vf und die LSK entsprechen den herausbewegten Konstituenten, das Mf entspricht der VP (ohne Verbalkomplex), und die RSK entspricht dem Verbalkomplex.
Weil der (Rest-)Verbalkomplex aber eben eine Teilkonstituente der VP ist, können wir das Feldermodell phrasenstrukturell nicht genau nachbilden.
Sobald wir sagen, die VP entspricht dem Mittelfeld, machen wir den Verbalkomplex zum Teil des Mittelfelds, obwohl er eigentlich ein eigenes Feld bildet.
Die hierarchische Struktur und das Feldermodell passen also nicht wirklich zueinander, und wir versuchen daher jetzt ein rein phrasenstrukturelles Modell des unabhängigen Aussagesatzes zu erarbeiten.

\begin{figure}
  \resizebox{\textwidth}{!}{
    \Tree{
      &&&&&&&&&&& \K{VP}\B{ddllllll}\B{ddllll}\B{ddlll}\B{d} \\
      &&&&&&&&&&& \K{\textbf{V}}\B{dll}\B{d} \\
      & \K{NP\ORii}\TRi[-7] && \K{\textbf{V\ORi}}\B{d} && \K{NP}\TRi[-6] && \K{AdvP}\TRi[-2] & \K{\Tii} & \K{\textbf{V}}\B{d} && \K{\Ti} \\
      & \K{\textit{das Bild}} && \K{\textit{hat}} && \K{\textit{Ischariot}} && \K{\textit{wahrscheinlich}} && \K{\textit{verkauft}} && \\
      & \K{Vf} && \K{LSK} &&& \K{Mf} &&&& \K{RSK} & \\
      \QS{5,2}{5,2}
      \QS{5,4}{5,4}
      \QS{5,6}{5,9}
      \QS{5,10}{5,12}
    }
  }
  \caption{Zuordnung der Felder zu Konstituenten (V2)}
  \label{fig:movev2konstituenten}
\end{figure}

\begin{figure}
  \resizebox{\textwidth}{!}{
    \Tree{
      &&& \K{S}\B{dddll}\B{ddd}\B{drrrrrrr} \\
      &&&&&&&&&& \K{VP}\B{ddlllll}\B{ddlll}\B{ddll}\B{d} \\
      &&&&&&&&&& \K{\textbf{V}}\B{dl}\B{d} \\
      & \K{NP\ORii}\TRi[-7] && \K{\textbf{V\ORi}}\B{d} && \K{NP}\TRi[-6] && \K{AdvP}\TRi[-2] & \K{\Tii}\POS[]-(0,4)\ar@{-->}@/^{4pc}/[lllllll]-(0,4) & \K{\textbf{V}}\B{d} & \K{\Ti}\POS[]-(0,4)\ar@{-->}@/^{4pc}/[lllllll]-(0,4) \\
      & \K{\textit{das Bild}} && \K{\textit{hat}} && \K{\textit{Ischariot}} && \K{\textit{wahrscheinlich}} && \K{\textit{verkauft}} & \\
    }
  }
  \vspace{0.3cm}
  \caption{V2-Satz}
  \label{fig:v2satz}
\end{figure}

\index{Verb-Zweit-Satz}
\index{Bewegung}
\index{Spur}

Die angestrebte Konstituentenstrukturanalyse eines V2-Satzes sieht aus wie in Abbildung~\ref{fig:v2satz}.
Ein unabhängiger Aussagesatz (Symbol S) wird hier als eine zusammenhängende Konstituente analysiert.
Das Mf und die RSK ergeben sich automatisch durch die Struktur der Reste der VP und des Verbalkomplexes.
Die erste Bewegung des finiten Verbs in die zweite Position in S entspricht der Besetzung der LSK.
Die zweite Bewegung einer beliebigen Phrase (wobei für eine beliebige Phrase üblicherweise XP geschrieben wird) in die linke Position von S entspricht der Besetzung des Vorfelds.
Das Schema, das diese Konstituentenstruktur erzeugen soll, muss nun einfach die Anforderungen kodieren, dass eine VP mit zwei Spuren (der Spur des finiten Verbs und der des Vorfeldbesetzers) sich mit den Konstituenten verbindet, die diese Lücken füllen können.

\Phrasenschema{V2-Satz}{\label{str:v2}
  \begin{tabular}{c|c|c|c|}
    \cline{2-4}
    S = & XP\ORii & [\textsc{Tempus}]\ORi & VP[\ldots \Tii \ldots \Ti] \\
    \cline{2-4}
  \end{tabular}
}

Im Schema~\ref{str:v2} wird die Notation VP[\ldots\Ti\ldots\Tii] verwendet, um anzuzeigen, dass eine VP mit zwei Spuren eingesetzt werden muss, egal was die VP sonst noch enthält.
Die Füller der Lücken werden vorne in die S-Struktur eingefügt.
Über den Füller zu Spur \Ti\ wird außerdem gesagt, dass er für [\textsc{Tempus}] spezifiziert sein soll, also gemäß Definition~\ref{def:finitheit} von S.~\pageref{def:finitheit} und Filter \ref{wfilt:verbennomina} ein finites Verb sein muss.
Das Feldermodell kann also vollständig durch eine sehr einfache phrasenstrukturelle Analyse ersetzt werden.

In Theorien, die tatsächlich Bewegungsoperationen als Teil der Grammatik annehmen, heißen Lücken wie bereits angedeutet meist Spuren.
Spuren werden dort wie tatsächliche syntaktische Einheiten behandelt, die nur unsichtbar bzw.\ unhörbar sind.
Andere Theorien verfolgen ein Konzept von echten Lücken, also einer Auslassung bestimmter Teile von Strukturen, wobei je nach Theorievariante auch Spuren angenommen werden.
Wenn an einer Stelle eine Lücke entsteht, muss an anderer Stelle in der Struktur dann ein passender Lückenfüller stehen.

Abschließend sei angemerkt, dass nicht immer davon ausgegangen wird, dass alle Vorfeldbesetzer aus dem Mf herausbewegt werden.
Adverbiale wie \textit{erfreulicherweise} \zB könnten auch ohne Weiteres direkt in S eingefügt werden, sofern die VP nur die Spur \Ti\ mit dem finiten Verb enthält.
Das sähe dann so aus wie in Abbildung~\ref{fig:vorfelddirektbesetzung}.%
\footnote{Das Schema für S müsste natürlich etwas angepasst werden, um auch diesen Fall zu beschreiben.
Wir gehen hier in den besprochenen Sätzen immer von Bewegung aus, nicht ohne darauf hinzuweisen, dass dies eine Übersimplifizierung sein könnte.}

\begin{figure}
  \centering
  \resizebox{\textwidth}{!}{
    \Tree{
      &&& \K{S}\B{dddll}\B{ddd}\B{drrrrrrrr} \\
      &&&&&&&&&&& \K{VP}\B{ddllllll}\B{ddllll}\B{d} \\
      &&&&&&&&&&& \K{\textbf{V}}\B{dll}\B{d} \\
      & \K{AdvP}\TRi[2] && \K{\textbf{V\ORi}}\B{d} && \K{NP}\TRi[-4] && \K{NP}\TRi[-4] && \K{\textbf{V}}\B{d} && \K{\Ti}\POS[]-(0,4)\ar@{-->}@/^{4pc}/[llllllll]-(0,4) \\
      & \K{\textit{Erfreulicherweise}} && \K{\textit{hat}} && \K{\textit{Ischariot}} && \K{\textit{das Bild}} && \K{\textit{verkauft}} && \K{} \\
    }
  }
  \vspace{0.3cm}
  \caption{Konstituentenanalyse bei direkter Vorfeldbesetzung}
  \label{fig:vorfelddirektbesetzung}
\end{figure}

Damit haben wir eine Erklärung der Satzgliedstellung im eingeleiteten Nebensatz (normale KP, vgl.\ Abschnitt~\ref{sec:subjgr}) und des V2-Satzes (unabhängiger Aussagesatz, Schema für S).
Der \textit{w}"=Fragesatz benötigt kein eigenes Schema, denn er ist lediglich eine Variante des V2-Aussagesatzes.
Die Spur \Tii\ muss dabei immer eine \textit{w}"=Pronomen-Spur sein, wie die Analyse in \ref{fig:v2fragesatz} zeigt.

\begin{figure}
  \centering
  \Tree{
    &&& \K{S}\B{dddll}\B{ddd}\B{drrrrrrr} \\
    &&&&&&&&&& \K{VP}\B{ddlllll}\B{ddlll}\B{ddll}\B{d} \\
    &&&&&&&&&& \K{\textbf{V}}\B{dl}\B{d} \\
    & \K{NP\ORii}\TRi[-7] && \K{\textbf{V\ORi}}\B{d} && \K{NP}\TRi[-6] && \K{AdvP}\TRi[-2] & \K{\Tii}\POS[]-(0,4)\ar@{-->}@/^{4pc}/[lllllll]-(0,4) & \K{\textbf{V}}\B{d} & \K{\Ti}\POS[]-(0,4)\ar@{-->}@/^{4pc}/[lllllll]-(0,4) \\
    & \K{\textit{Was}} && \K{\textit{hat}} && \K{\textit{Ischariot}} && \K{\textit{wahrscheinlich}} && \K{\textit{verkauft}} & \\
  }
  \vspace{0.3cm}
  \caption{V2-\textit{w}"=Fragesatz}
  \label{fig:v2fragesatz}
\end{figure}

Im nächsten Abschnitt wird ein Schema für den V1-Entscheidungsfragesatz eingeführt.

\subsection{Verb-Erst-Sätze}

\label{sec:verberstsatz}

\index{Verb-Erst-Satz}
\index{Fragesatz!Entscheidungs--}

\Phrasenschema{V1-Satz}{\label{str:v1}
  \begin{tabular}{c|c|c|}
    \cline{2-3}
    FS = & [\textsc{Tempus}]\ORi & VP[\ldots \Ti] \\
    \cline{2-3}
  \end{tabular}
}

\begin{sloppypar}
V1-Fragesätze (FS) sind denkbar einfach zu beschreiben, nachdem wir bereits V2-Sätze analysiert haben.
Einen Satz wie (\ref{ex:saetze6661}) erklärt Schema~\ref{str:v1}, s.\ Abbildung~\ref{fig:v1satz}.
\end{sloppypar}

\begin{exe}
  \ex{\label{ex:saetze6661} Hat Ischariot tatsächlich das Bild verkauft?}
\end{exe}

\begin{figure}
  \centering
  \Tree{
    \K{FS}\B{ddd}\B{drrrrrrrrr} \\
    &&&&&&&&& \K{VP}\B{ddlllllll}\B{ddlllll}\B{ddlll}\B{d} \\
    &&&&&&&&& \K{\textbf{V}}\B{dl}\B{d} \\
    \K{\textbf{V\ORi}}\B{d} && \K{NP}\TRi[-4] && \K{AdvP}\TRi[-3] && \K{NP}\TRi[-6] && \K{\textbf{\textbf{V}}}\B{d} & \K{\Ti}\POS[]-(0,4)\ar@{-->}@/^{4pc}/[lllllllll]-(0,4) \\
    \K{\textit{Hat}} && \K{\textit{Ischariot}} && \K{\textit{tatsächlich}} && \K{\textit{das Bild}} && \K{\textit{verkauft}} & \\
  } 
  \vspace{0.3cm}
  \caption{Entscheidungsfragesatz}
  \label{fig:v1satz}
\end{figure}

Es entfällt bei dem V1-Satz lediglich die Bewegung der zweiten Konstituente nach der Bewegung des finiten Verbs.
Nur das finite Verb muss nach links gestellt werden, und das Schema ist damit einfacher als das V2-Schema.
Es bleibt anzumerken, dass wir hier die Bezeichnung FS mehr oder weniger informell benutzen.
Mit der Beschriftung FS wird die Information kodiert, dass es sich um einen Fragesatz handelt.

\index{Imperativ!Satz}
Imperative wie in (\ref{ex:saetze1557}) sind im Prinzip wie V1-Sätze strukturiert.

\begin{exe}
  \ex{\label{ex:saetze1557} Verkaufe das Bild.}
\end{exe}

Man würde sie auch normalerweise genauso wie andere V1-Sätze analysieren.
Es sei nur darauf verwiesen, dass wir in Abschnitt~\ref{sec:impflex} morphologisch argumentiert haben, dass imperativische Verbformen nicht finit sind.
Wenn man dies annimmt, wird in Imperativsätzen eine infinite Verbform herausbewegt.
Das müsste in einem angepassten Schema reflektiert werden, was hier aus Platzgründen nicht ausbuchstabiert wird.

\index{Verb!Partikel--}

\begin{sloppypar}
Damit sind jetzt alle Stellungstypen prinzipiell erklärt.
Zu Nebensätzen kann und sollte man allerdings mehr sagen, als einfach ihre Konstituentenstruktur anzugeben.
Über Verwendung, Anschluss und Stellung von den drei wichtigen Nebensatztypen folgen (nach einer Bemerkung zu Partikelverben in Abschnitt~\ref{sec:partikelverben}) in Abschnitt~\ref{sec:nebensaetze} weitere Überlegungen.
\end{sloppypar}

\subsection{Syntax der Partikelverben}

\label{sec:partikelverben}

Durch die Bewegung von finiten Verben ergibt sich ein Problem, wenn wir Partikelverben als eine Wortform analysieren.
In einer V2-Struktur bleibt die Partikel zurück, die Bewegung müsste aus einer Wortform heraus geschehen, vgl. (\ref{ex:saetze7778}).

\begin{exe}
  \ex{\label{ex:saetze7778} Sarah isst den Kuchen alleine auf=.}
\end{exe}

\index{Verbalkomplex}

Das syntaktische Herausbewegen aus einer Wortform ist problematisch, denn Wortformen sollen auf der Ebene der Syntax als atomare Konstituenten gelten.
Die Lösung besteht darin, Kombinationen aus Partikel und Verb als syntaktische Struktur zu analysieren, wie in Abbildung~\ref{fig:v2mitpartikel}.
Damit ist es möglich, die Bewegung des finiten Verbs durchzuführen.
Eigentlich müsste das Phrasenschema für den Verbalkomplex für diesen Zweck erweitert werden, was als Übungsaufgabe von den Lesern durchgeführt werden kann.
Außerdem würden sich evtl.\ Änderungen an den Wortklassen bzw.\ den Aussagen zur Verbalmorphologie (\zB Bildung der Partizipien) ergeben.

\begin{figure}
  \centering
  \Tree{
    &&& \K{S}\B{dddll}\B{ddd}\B{drrrrrrr} \\
    &&&&&&&&&& \K{VP}\B{ddllllll}\B{ddlllll}\B{ddlll}\B{d} \\
    &&&&&&&&&& \K{\textbf{V}}\B{dl}\B{d} \\
    & \K{NP\ORii}\TRi[-8] && \K{\textbf{V\ORi}}\B{d} & \K{\Tii}\POS[]-(0,4)\ar@{-->}@/^{4pc}/[lll]-(0,4) & \K{NP}\TRi[-4] && \K{AdvP}\TRi[-6] && \K{Ptkl}\B{d} & \K{\Ti}\POS[]-(0,4)\ar@{-->}@/^{4pc}/[lllllll]-(0,4) \\
    & \K{\textit{Sarah}} && \K{\textit{isst}} && \K{\textit{den Kuchen}} && \K{\textit{alleine}} && \K{\textit{auf}} \\
  }
  \vspace{0.3cm}
  \caption{V2-Satz mit Partikelverb}
  \label{fig:v2mitpartikel}
\end{figure}

\subsection{Kopulasätze}

\label{sec:kopulakonstruktionen}

\index{Kopula}
\index{Kopulasatz}

Für die Beschreibung von Kopulasätzen wie (\ref{ex:satz82821a}) müssen keine besonderen Satzstrukturen eingeführt werden.
Wir können sie als Ergebnis der üblichen Bewegungsoperationen betrachten und Strukturen wie (\ref{ex:satz82821b}) zugrundelegen.

\begin{exe}
  \ex\label{ex:satz82821} 
  \begin{xlist}
    \ex{\label{ex:satz82821a} Die Frau ist stolz auf ihre Tochter.}
    \ex{\label{ex:satz82821b} dass die Frau auf ihre Tochter stolz ist}
  \end{xlist}
\end{exe}

Die AP ist hier strukturell etwas anders gebaut als eine attributive AP innerhalb einer NP.
Die in der NP prototypische Abfolge [[\textit{auf ihre Tochter}] \textit{stolze}] wird (zumindest optional) umgekehrt zu [\textit{stolz} [\textit{auf ihre Tochter}]].
Außerdem besteht keine Kongruenz des Adjektivs zu irgendeinem Bezugsnomen, und das Adjektiv steht in der unflektierten Kurzform (s.\ Abschnitt~\ref{sec:adjektivklassifikation}).
Ansonsten fällt auf, dass die Nominativ-NP \textit{die Frau} mit der Kopula in Person und Numerus kongruiert und frei im Satz bewegt werden kann.
Eine besondere morpho-syntaktische Beziehung zum Adjektiv hat das Subjekt aber offensichtlich nicht.
Die Konstituente [\textit{stolz auf ihre Tochter}] kann außerdem auch frei bewegt werden wie in (\ref{ex:satz82822}).

\begin{exe}
  \ex{\label{ex:satz82822} [Stolz auf ihre Tochter] ist die Frau.}
\end{exe}

Die Analyse in Abbildung~\ref{fig:kopulav22} bietet sich daher an.
Dabei regiert die Kopula eine AP, die zwar eine andere Abfolge ihrer Konstituenten realisiert als die AP innerhalb einer NP, die aber aus denselben Konstituenten besteht.
Die Kopula regiert außerdem eine NP im Nominativ, das gewöhnliche Subjekt.

\begin{figure}
  \Tree{
    &&& \K{S}\B{ddddll}\B{dddd}\B{drrrrr} \\
    &&&&&&&& \K{VP}\B{ddllll}\B{ddd}\B{ddlll} \\
    \\
    &&&& \K{\Tii} & \K{AP}\B{d}\B{drr} \\
    & \K{NP\ORii}\TRi[-8] && \K{\textbf{V\ORi}}\B{d} && \K{\textbf{A}}\B{d} && \K{PP}\TRi[-8] & \K{\Ti}\\
    & \K{\textit{Die Frau}} && \K{\textit{ist}} && \K{\textit{stolz}} && \K{\textit{auf ihre}}\Below{\textit{Tochter}} & \\
  }
  \caption{Analyse eines Kopulasatzes mit AP}
  \label{fig:kopulav22}
\end{figure}

\section{Nebensätze}

\label{sec:nebensaetze}

In diesem Abschnitt werden die verschiedenen Typen von Nebensätzen und ihre Besonderheiten im internen Aufbau und in ihrem externen syntaktischen Verhalten besprochen.
Die Definition des Nebensatzes aus Kapitel~\ref{sec:wortklassen} (Definition~\ref{def:nebensatz} auf S.~\pageref{def:nebensatz}) kann unverändert zugrundegelegt werden.
Es handelt sich also um eine Konstituente, die ein finites Verb enthält, in der alle Valenzen gesättigt sind und die nicht alleine stehen kann.

Fälle wie (\ref{ex:saetze1450a}), in denen ein Nebensatz scheinbar alleine steht, analysieren wir als Ellipsen, also Strukturen, in denen eine hauptsatzartige Struktur getilgt wurde, s.\ (\ref{ex:saetze1450b}).
Andere Analysen sind in einem größeren theoretischen Rahmen natürlich möglich und vielleicht erwünscht.

\begin{exe}
  \ex\label{ex:saetze1450} 
  \begin{xlist}
    \ex{\label{ex:saetze1450a} Ob das wohl stimmt!}
    \ex{\label{ex:saetze1450b} Ich frage mich\slash Ich bin nicht sicher\slash\ldots, ob das wohl stimmt!}
  \end{xlist}
\end{exe}

\subsection{Relativsätze}

\label{sec:relativsaetze}

Ein Relativsatz (RS) wie in (\ref{ex:saetze9228}) ist im prototypischen Fall ein Attribut zu einem nominalen Kopf, dem Bezugsnomen (vgl.\ Abschnitt~\ref{sec:ngr}, für einen Sonderfall s.\ Abschnitt~\ref{sec:freierelativsaetze}).

\begin{exe}
  \ex{\label{ex:saetze9228} [Einen Seidentofu, [den ich nicht gemocht habe]], habe ich noch nie gegessen.}
\end{exe}

Wie schon in Abschnitt~\ref{sec:felder} angedeutet, ist der Relativsatz unter den satzförmigen Strukturen ein Sonderfall bezüglich seiner internen Satzgliedstellung.
Das Verb bleibt im Verbalkomplex stehen (VL-Satz), und das Relativpronomen -- genauer die Relativphrase -- wird nach links (in das Vf) bewegt.
Man kann sich die Struktur eines RS wie (\ref{ex:saetze6419a}) verdeutlichen, indem man aus dem Relativsatz und seinem Bezugsnomen wieder einen unabhängigen Satz baut:
Man ersetzt das Relativpronomen durch das Bezugsnomen (\ref{ex:saetze6419b}) und stellt dann durch Umstellung des finiten Verbs eine V2-Stellung her (\ref{ex:saetze6419c}).

\begin{exe}
  \ex\label{ex:saetze6419}
  \begin{xlist}
    \ex{\label{ex:saetze6419a} einen Seidentofu, [den ich nicht gemocht habe]}
    \ex{\label{ex:saetze6419b} einen Seidentofu ich nicht gemocht habe}
    \ex{\label{ex:saetze6419c} Einen Seidentofu habe ich nicht gemocht.}
  \end{xlist}
\end{exe}

Das Schema~\ref{str:rs} spiegelt diesen Sachverhalt wieder, eine Analyse liefert Abbildung~\ref{fig:saetze9228}.

\index{Relativsatz}

\Phrasenschema{Relativsatz}{\label{str:rs}
  \begin{tabular}{c|c|c|}
    \cline{2-3}
    RS = & [\textsc{Rel}:$+$]\ORi & VP[\ldots \Ti \ldots] \\
    \cline{2-3}
  \end{tabular}
}

Wie schon auf S.~\pageref{abs:923478} besprochen, ist der eingeleitete \textit{w}"=Fragesatz strukturell identisch zum RS und wird daher hier nicht weiter analysiert.
Der einzige Unterschied ist, dass es sich bei dem bewegten Element nicht um eine Relativphrase, sondern um eine \textit{w}"=Konstituente [\textsc{W}: $+$] handeln muss.

\index{Fragesatz!w-Frage}

\Phrasenschema{\textit{w}"=Satz}{\label{str:ws}
  \begin{tabular}{c|c|c|}
    \cline{2-3}
    WS = & [\textsc{W}:$+$]\ORi & VP[\ldots \Ti \ldots] \\
    \cline{2-3}
  \end{tabular}
}

\begin{figure}
  \centering
  \Tree[-0.3]{
    && \K{NP}\B{ddddll}\B{dddd}\B{drr} \\
    &&&& \K{RS} \B{ddd}\B{drrrrrrrr} \\
    &&&&&&&&&&&& \K{VP}\B{ddllllll}\B{ddlllll}\B{ddllll}\B{d} \\
    &&&&&&&&&&&& \K{\textbf{V}}\B{dll}\B{d} \\
    \K{Art}\B{d} && \K{\textbf{N}}\B{d} && \K{NP\ORi}\TRi[-8] && \K{NP}\TRi[-8] & \K{\Ti}\POS[]-(0,4)\ar@{-->}@/^{4pc}/[lll]-(0,4) & \K{Ptkl}\B{d} && \K{\textbf{V}}\B{d} && \K{\textbf{V}}\B{d} \\
    \K{\textit{einen}} && \K{\textit{Seidentofu}} && \K{\textit{den}} && \K{\textit{ich}} && \K{\textit{nicht}} && \K{\textit{gemocht}} && \K{\textit{habe}} \\
  }
  \vspace{0.3cm}
  \caption{NP mit Relativsatz}
  \label{fig:saetze9228}
\end{figure}

\index{Relativphrase}
\index{Relativsatz!Einleitung}

Wir müssen uns nun fragen, welche Form (und damit welche Merkmale) die Relativphrase in allen möglichen Arten von Relativsätzen genau hat.
Im gegebenen Satz (\ref{ex:saetze9228}) ist sie eine NP im Akkusativ des maskulinen Singulars, wie an der Form des Pronomens \textit{den} eindeutig identifizierbar.
Wir diskutieren jetzt, wie die Form der Relativphrase allgemein zu bestimmen ist.
Ein Relativpronomen muss, damit das Schema anwendbar ist, im Lexikon bereits mit dem Merkmal [\textsc{Rel}: $+$] ausgestattet sein, um auf die hier gezeigte Weise bewegt werden zu können.
Das Schema spezifiziert ausdrücklich, dass \Ti\ das Merkmal [\textsc{Rel}: $+$] haben muss.
Bezüglich der Form der nominalen Relativphrase gelten nun zwei Beschränkungen:

\begin{enumerate}\Lf
  \item Die Relativphrase kongruiert mit dem Bezugsnomen in \textsc{Genus} und \textsc{Numerus}.
  \item Die Relativphrase erhält ihren Wert für \textsc{Kasus} innerhalb der VP, aus der sie herausbewegt wird.
\end{enumerate}

\noindent In Satz (\ref{ex:saetze9228}) ist also \textit{den} [\textsc{Kasus}: \textit{akk}] dank der Rektion durch \textit{gemocht}.
Dass \textit{den} aber außerdem [\textsc{Genus}: \textit{mask}, \textsc{Numerus}: \textit{sg}] ist, kommt allerdings durch Kongruenz zu \textit{Seidentofu} zustande.

Die zweite Bedingung ist etwas zu eng gefasst, weil die Relativphrase nicht unbedingt eine einfache NP sein muss, deren Kasus vom Verb des Relativsatzes regiert wird.
Es gibt auch Relativsätze wie in (\ref{ex:saetze5552a}), in denen die Relativphrase komplexer als ein einfaches Pronomen ist.
Wenn wir diesen Relativsatz wie in (\ref{ex:saetze6419}) in einen unabhängigen Satz umwandeln, erhalten wir (\ref{ex:saetze5552}).
Die Präposition bleibt bei der Umwandlung erhalten, die Relativphrase (die PP mit dem eingebetteten Relativpronomen) wird also nur teilweise ersetzt.

\begin{exe}
  \ex\label{ex:saetze5552} 
    \begin{xlist}
      \ex{\label{ex:saetze5552a} der Tofu, [auf den ich mich freue]}
      \ex{\label{ex:saetze5552b} [auf den Tofu ich mich freue]}
      \ex{\label{ex:saetze5552c} [Auf den Tofu freue ich mich.]}
    \end{xlist} 
\end{exe}

Da \textit{freue} sowieso eine solche PP mit \textit{auf} regiert (vgl. (\ref{ex:saetze5552b})), erhält die Relativphrase hier nicht einen Kasus, sondern eine präpositionale Form innerhalb des Relativsatzes.
Die Form der PP muss nicht einmal regiert sein, es kann sich sogar um eine Angabe handeln, wie in (\ref{ex:saetze3139a}).
Die PP [\textit{auf der Straße}] (bzw.\ die Relativphrase [\textit{auf der}]) ist keine Ergänzung von \textit{laufen}, sondern eine Angabe.

\begin{exe}
  \ex\label{ex:saetze3139}
  \begin{xlist}
    \ex{\label{ex:saetze3139a} Die Straße, [auf der wir den Marathon laufen], ist eine Autobahn.}
    \ex{\label{ex:saetze3139b} Wir laufen den Marathon (auf der Straße).}
  \end{xlist}
\end{exe}

In (\ref{ex:saetze5553a}) liegt noch ein anderer Fall einer Relativphrase vor.

\begin{exe}
  \ex\label{ex:saetze5553} 
    \begin{xlist}
      \ex{\label{ex:saetze5553a} Der Tofu, [dessen Geschmack ich mag], ist ausverkauft.}
      \ex{\label{ex:saetze5553b} dass [ich [den Geschmack] [des Tofus] mag]}
    \end{xlist}
\end{exe}

\index{Genitiv!pränominal}

Das Pronomen \textit{dessen} ist ein pränominaler Genitiv innerhalb einer NP [\textit{dessen Geschmack}].
Die Relativphrase ist hier die gesamte NP, innerhalb derer das Pronomen den Kasus (Genitiv) erhält, den es auch in einer unabhängigen NP erhalten würde, vgl.\ (\ref{ex:saetze5553b}).
Es ist nicht so, dass die gesamte Relativphrase (die NP) in Genus und Numerus mit dem Bezugsnomen kongruiert, sondern nur der pränominale Genitiv.

In Abbildung~\ref{fig:saetze5553} wird die Struktur dieser Konstruktion abgebildet.
Die Kasusrektion des Verbs \textit{mag} geht wie zu erwarten an die NP, deren Kopf \textit{Geschmack} ist.
Der Kasus von \textit{dessen} ist nicht regiert, sondern ein freier Attributs-Genitiv (kein Pfeil).
Die Kongruenz des Relativpronomens \textit{dessen} wird durch einen gestrichelten Pfeil angezeigt.%
\footnote{Der Bewegungspfeil wird der Übersicht wegen weggelassen.}

\begin{figure}
  \centering
  \Tree[-0.3]{
    && \K{NP}\B{dddll}\B{ddd}\B{drrrr} \\
    &&&&&& \K{RS}\B{d}\B{drrrr} \\
    &&&&&& \K{NP\ORi}\B{dll}\B{d} &&&& \K{VP}\B{dll}\B{dl}\B{d} \\
    \K{Art}\B{d} && \K{\textbf{N}}\B{d} && \K{NP}\TRi[-4] && \K{\textbf{N}}\B{d} && \K{NP}\TRi[-7] & \K{\Ti} & \K{\textbf{V}}\B{d} \\
    \K{\textit{der}} && \K{\textit{Tofu}}
    \POS[]-(0,4)\ar@{-->}@/_{0.5pc}/[rr]-(0,4)_{\sc\mathrm{Gen,Num}} && \K{\textit{dessen}} && \K{\textit{Geschmack}} && \K{\textit{ich}} && \K{\textit{mag}}\POS[]-(0,4)\ar@{-->}@/^{1pc}/[llll]-(0,4)^{\sc\mathrm{Kas}} & \\
  }
  \caption{NP mit Relativsatz mit genitivischer Relativphrase}
  \label{fig:saetze5553}
\end{figure}

\index{Relativadverb}

Neben den normalen Relativpronomina gibt es noch eine Reihe von sogenannten Relativadverben wie \textit{womit}, \textit{worin}, \textit{worauf} usw., die für sich alleine eine Relativphrase bilden.
Wie geben hier nur ein Beispiel in (\ref{ex:saetze2998}).

\begin{exe}
  \ex{\label{ex:saetze2998} Alles, [womit man rechnet], tritt auch ein.}
\end{exe}

Eine Sonderklasse von Relativsätzen sind die sogenannten \textit{freien Relativsätze}.
Freie Relativsätze sind intern wie jeder andere Relativsatz aufgebaut, beziehen sich aber nicht auf ein Bezugsnomen, sondern nehmen für sich allein den Platz einer NP ein.

\begin{exe}
  \ex\label{ex:saetze1991}
  \begin{xlist}
    \ex{\label{ex:saetze1991a} [Wer Klaviermusik mag], mag Chopin.}
    \ex{\label{ex:saetze1991b} [Wen man mag], beschenkt man.}
    \ex{\label{ex:saetze1991c} Wir glauben, [wem wir Vertrauen schenken].}
  \end{xlist}
\end{exe}

\index{Relativsatz!frei}

Im Normalfall muss die Relativphrase den Kasus haben, den auch eine NP an der Position des RS im einbettenden Satz hätte.
Dies hat zur Folge, dass der Kasus der Relativphrase im Relativsatz gleich dem externen Kasus sein muss.
Abbildung~\ref{fig:saetze1991b} zeigt die Kasusanforderungen.%
\footnote{Es theoretisch eine problematische Annahme, dass eine Einheit (hier \textit{wen}) zwei unterschiedliche Valenzanforderungen erfüllt.
Insofern sind die Valenzpfeile als Veranschaulichung zu verstehen, nicht als theoretische Position.}

\begin{figure}
  \centering
  \resizebox{\textwidth}{!}{
    \Tree{
      &&&&&&& \K{S}\B{dllllll}\B{ddd}\B{ddrrrr} \\
      & \K{RS\ORiii}\B{dd}\B{drrrr} \\
      &&&&& \K{VP}\B{dll}\B{dl}\B{d} &&&&&& \K{VP}\B{dll}\B{dl}\B{d} \\
      & \K{NP\ORi}\TRi[-7] && \K{NP}\TRi[-7] & \K{\Ti} & \K{\textbf{V}}\B{d} && \K{\textbf{V\ORii}}\B{d} && \K{NP}\TRi[-7] & \K{\Tiii} & \K{\Tii} \\
      & \K{\textit{wen}} && \K{\textit{man}} && \K{\textit{mag}}\POS[]-(0,4)\ar@{-->}@/^{1pc}/[llll]-(0,4)^{\sc\mathrm{Kas}} && \K{\textit{beschenkt}}\POS[]-(0,4)\ar@{--}@/^{3pc}/[llllll]-(0,4)^{\sc\mathrm{Kas}} && \K{\textit{man}} && \\
    }
  }
  \caption{Satz mit freiem Relativsatz}
  \label{fig:saetze1991b}
\end{figure}

Die Ungrammatikalität von (\ref{ex:saetze1992a}) rührt aus einer Verletzung dieser speziellen Kasusanforderung her.

\begin{exe}
  \ex[*]{\label{ex:saetze1992a} [Wer Klaviermusik mag], beschenkt man mit Karten für den Kammermusiksaal.}
\end{exe}

Um einen Satz wie (\ref{ex:saetze1992a}) zu reparieren, muss der Relativsatz an einen pronominalen Kopf als Bezugsnomen angeschlossen werden, der die Kasusanforderung des einbettenden Satzes erfüllen kann.
In (\ref{ex:saetze1992b}) ist ein solches Pronomen in Form von \textit{denjenigen} eingesetzt.
Es erfüllt als Akkusativ die Rektionsanforderung von \textit{beschenkt}, während die Relativphrase \textit{der} die Rektionsanforderung (Nominativ) von \textit{mag} innerhalb des Relativsatzes erfüllt.

\begin{exe}
  \ex{\label{ex:saetze1992b} [Denjenigen, [der Klaviermusik mag]], beschenkt man mit Karten für den Kammermusiksaal.}
\end{exe}

Eine andere Möglichkeit ist es, den freien Relativsatz mit \textit{w}"=Pronomen vor das Vorfeld zu stellen und ein im Kasus angepasstes korrelierendes Pronomen ins Vorfeld zu stellen, wie in (\ref{ex:saetze1998000}).

\begin{exe}
  \ex{\label{ex:saetze1998000} Wer Klaviermusik mag, den beschenkt man mit Karten für den Kammermusiksaal}
\end{exe}

Manche Sprecher akzeptieren es allerdings auch, wenn der Kasus der Relativphrase obliker ist, als per Rektion im Matrixsatz gefordert, vgl.\ (\ref{ex:saetze8282821a}).
Wenn die Relativphrase in einem weniger obliken Kasus steht, funktioniert das allerdings nicht, wie in (\ref{ex:saetze8282821b}).

\begin{exe}
  \ex\label{ex:saetze8282821} 
  \begin{xlist}
    \ex[?]{\label{ex:saetze8282821a} Wen es stört, kann gehen.}
    \ex[*]{\label{ex:saetze8282821b} Wer hier stört, beschenkt man.}
  \end{xlist}
\end{exe}

Abschließend müssen einige Stellungsbesonderheiten der Relativsätze diskutiert werden. 
Außer freien Relativsätzen können Relativsätze zunächst einmal nicht im Vorfeld stehen.
Bezüglich der Stellung der Relativsätze im einbettenden Satz müssen zwei Fälle unterschieden werden.
Die Fälle sind in (\ref{ex:saetze3338}) und (\ref{ex:saetze3339}) illustriert.

\begin{exe}
  \ex\label{ex:saetze3338}
  \begin{xlist}
    \ex{\label{ex:saetze3338a} [Die Gavotte, [die ich am liebsten mag]], hat Tanja gespielt.}
    \ex{\label{ex:saetze3338b} Tanja hat [die Gavotte, [die ich am liebsten mag]], gespielt.}
  \end{xlist}
  \ex\label{ex:saetze3339}
  \begin{xlist}
    \ex{\label{ex:saetze3339a} Tanja hat [die Gavotte] gespielt, [die ich am liebsten mag].}
    \ex{\label{ex:saetze3339b} Ich glaube, dass Tanja [die Gavotte] gespielt hat, [die ich am liebsten mag].}
  \end{xlist}
\end{exe}
\index{Nachfeld}

In (\ref{ex:saetze3338}) ist der Relativsatz innerhalb der NP rechts vom Kopf positioniert, also genau dort, wo er gemäß Schema~\ref{str:ngr} stehen soll.
Dabei ist es egal, ob die NP nach links (ins Vf) bewegt wird wie in (\ref{ex:saetze3338a}), oder ob die NP in der VP (dem Mf) verbleibt wie in (\ref{ex:saetze3338b}).

Bereits in Abschnitt~\ref{sec:nebensaetze} (s.\ vor allem Abbildung~\ref{fig:nachfeld}) wurden aber Sätze wie die in (\ref{ex:saetze3339}) gezeigt.
Hier wird der Relativsatz nach rechts herausgestellt (Nf) und damit von der NP getrennt.
Dies kann sowohl aus unabhängigen Sätzen (S) geschehen wie in (\ref{ex:saetze3339a}), aber auch aus eingebetteten Sätzen (also einer VP), wie in (\ref{ex:saetze3339b}), wo der Relativsatz aus dem \textit{dass}-Satz nach rechts herausbewegt wurde.%
\footnote{Wir geben hier keine Strukturen dafür an.
Übung \ref{u124} auf S.~\pageref{u124} beschäftigt sich mit der Frage von Konstituentenstrukturen bei Bewegung ins Nachfeld.}

Die verbleibenden zwei Abschnitte widmen sich den Komplementsätzen (Abschnitt~\ref{sec:komplementsaetze}) und den Adverbialsätzen (Abschnitt~\ref{sec:adverbialsaetze}).
Diese unterscheiden sich vor allem bezüglich ihrer Funktion im Einbettungskontext.

\subsection{Komplementsätze}

\label{sec:komplementsaetze}


\textit{Komplementsätze} oder \textit{Ergänzungssätze} sind Sätze, die als Ergänzung zu Verben fungieren, die also eine Valenzanforderung saturieren.%
\footnote{Die Begriffe Komplement und Ergänzung sind weitgehend synonym, vgl.\ Abschnitt~\ref{sec:valenzundkoepfe}.}
Dabei unterscheidet man Subjektsätze und Objektsätze.

\Definition{Komplementsatz}{
\label{def:komplementsatz}
Ein Komplementsatz ist eine Ergänzung in Form eines Nebensatzes.
Der Untertyp des Subjektsatzes nimmt die Stelle ein, die auch von einer NP im Nominativ eingenommen werden könnte.
Alle anderen Komplementsätze sind Objektsätze.
\index{Komplementsatz}
\index{Subjektsatz}
\index{Objektsatz}
}

Wenden wir uns zunächst den Objektsätzen zu, müssen formal zwei Typen unterschieden werden, vgl.\ (\ref{ex:saetze8881}).

\begin{exe}
  \ex{\label{ex:saetze8881} Die Experten wissen, [dass der Koffer nicht explodieren wird].}
  \ex\label{ex:saetze8882}
  \begin{xlist}
    \ex{\label{ex:saetze8882a} Die Polizei will wissen, [wie der Passant auf diese Idee gekommen ist].}
    \ex{\label{ex:saetze8882b} Wir testen, [ob der Koffer bei Berührung explodiert].}
  \end{xlist}
\end{exe}

Das Verb \textit{wissen} in (\ref{ex:saetze8881}) und (\ref{ex:saetze8882a}) nimmt einmal einen \textit{dass}-Satz und einmal einen eingeleiteten Fragesatz (vgl.\ Schema~\ref{str:ws}).
In (\ref{ex:saetze8882b}) regiert \textit{testen} einen Fragesatz mit \textit{ob}, das ebenfalls als Fragepronomen behandelt werden kann.
Die \textit{dass}-Sätze sind normale KPs, also durch Komplementierer eingeleitete Nebensätze (VL-Sätze).
Strukturell davon unterschieden sind Fragesätze, die die Form eines Relativsatzes (VL-Satz) mit Fragepronomen (\textit{w}-Pronomen oder \textit{ob}) haben.

Verben, die Objektsätze fordern, folgen drei Mustern, je nachdem, mit welchen Arten von Objektsätzen sie stehen können.
Entweder stehen sie nur mit \textit{dass} wie in (\ref{ex:saetze727211}), nur mit Fragesätzen wie in (\ref{ex:saetze727212}) oder mit beidem wie in (\ref{ex:saetze727213}).

\begin{exe}
  \ex\label{ex:saetze727211} 
  \begin{xlist}
    \ex[]{\label{ex:saetze727211a} Michelle beklagt, dass die Corvette nicht anspringt.}
    \ex[*]{\label{ex:saetze727211b} Michelle beklagt, wie\slash ob die Corvette nicht anspringt.}
  \end{xlist}
  \ex\label{ex:saetze727212} 
  \begin{xlist}
    \ex[*]{\label{ex:saetze727212a} Michelle untersucht, dass der Vergaser funktioniert.}
    \ex[]{\label{ex:saetze727212b} Michelle untersucht, wie\slash ob der Vergaser funktioniert.}
  \end{xlist}
  \ex\label{ex:saetze727213} 
  \begin{xlist}
    \ex[]{\label{ex:saetze727213a} Michelle hört, dass die Nockenwelle läuft.}
    \ex[]{\label{ex:saetze727213b} Michelle hört, wie\slash ob die Nockenwelle läuft.}
  \end{xlist}
\end{exe}

Bei \textit{dass}-Sätzen gibt es Alternationen mit Infinitivkonstruktionen mit \textit{zu} (selbständigen infiniten VP) wie in (\ref{ex:saetze8111}).
Diese Infinitive werden in Abschnitt~\ref{sec:infkonstr} genauer besprochen.

\begin{exe}
  \ex\label{ex:saetze8111}
  \begin{xlist}
    \ex{\label{ex:saetze8111a} Die Experten glauben, [dass sie den Koffer wiedererkennen].}
    \ex{\label{ex:saetze8111b} Die Experten glauben, [den Koffer wiederzuerkennen].}
  \end{xlist}
\end{exe}

Nach Definition~\ref{def:komplementsatz} nehmen Subjektsätze die Position der NP im Nominativ ein.
Ein Subjektsatz ist in (\ref{ex:saetze7771a}) illustriert.
In (\ref{ex:saetze7771b}) ersetzt ein Nominativ den Subjektsatz.

\begin{exe}
  \ex\label{ex:saetze7771} 
  \begin{xlist}
    \ex{\label{ex:saetze7771a} [Dass die Sonne scheint], freut die Ausflügler.}
    \ex{\label{ex:saetze7771b} [Der Sonnenschein] freut die Ausflügler.}
  \end{xlist}
\end{exe}

\index{Mittelfeld}
\index{Korrelat}
\index{Nebensatz}
Die bisher besprochenen Komplementsätze standen alle entweder im Vf oder im Nf.
Tatsächlich ist es ungewöhnlich (wenn auch je nach Sprecher und Gestalt des Satzes nicht ganz ausgeschlossen), dass Komplementsätze im Mf stehen, wo sie als Ergänzungen des Verbs eigentlich zu erwarten wären.
Die Sätze mit Fragezeichen in (\ref{ex:saetze7171})--(\ref{ex:saetze7173}) illustrieren dies.

\begin{exe}
  \ex\label{ex:saetze7171}
  \begin{xlist}
    \ex[]{\label{ex:saetze7171a} [Dass sie unseren Kuchen mag], hat Sarah uns nun doch eröffnet.}
    \ex[]{\label{ex:saetze7171b} Sarah hat uns nun doch eröffnet, [dass sie unseren Kuchen mag].}
    \ex[?]{\label{ex:saetze7171c} Sarah hat uns, [dass sie unseren Kuchen mag], nun doch eröffnet.}
  \end{xlist}
  \ex\label{ex:saetze7172}
  \begin{xlist}
    \ex[]{\label{ex:saetze7172a} [Ob Pavel unseren Kuchen mag], haben wir uns oft gefragt.}
    \ex[]{\label{ex:saetze7172b} Wir haben uns oft gefragt, [ob Pavel unseren Kuchen mag].}
    \ex[?]{\label{ex:saetze7172c} Wir haben uns, [ob Pavel unseren Kuchen mag], oft gefragt.}
  \end{xlist}
  \ex\label{ex:saetze7173}
  \begin{xlist}
    \ex[]{\label{ex:saetze7173a} [Wer die Rosinen geklaut hat], wollen wir endlich wissen.}
    \ex[]{\label{ex:saetze7173b} Wir wollen endlich wissen, [wer die Rosinen geklaut hat].}
    \ex[?]{\label{ex:saetze7173c} Wir wollen, [wer die Rosinen geklaut hat], endlich wissen.}
  \end{xlist}
\end{exe}

Die Komplementsätze werden also überwiegend aus dem Mf herausbewegt.
Die Sätze (a) in (\ref{ex:saetze7171})--(\ref{ex:saetze7173}) sind mit entsprechender Betonung auf jeden Fall einwandfrei.
Wenn ein Objektsatz ins Nf gestellt wird, dann können (wie eine sichtbare Spur) sogenannte Korrelate im Mf stehen.\index{Spur}
Die Sätze (b) aus (\ref{ex:saetze7171})--(\ref{ex:saetze7173}) werden in (\ref{ex:saetze7175}) mit dem Korrelat \textit{es} wiederholt.

\begin{exe}
  \ex\label{ex:saetze7175}
  \begin{xlist}
    \ex{\label{ex:saetze7175a} Sarah hat es uns eröffnet, [dass sie unseren Kuchen mag].}
    \ex{\label{ex:saetze7175b} Wir haben es uns gefragt, [ob Pavel unseren Kuchen mag].}
    \ex{\label{ex:saetze7175c} Wir wollen es wissen, [wer die Rosinen aus dem Kuchen geklaut hat].}
  \end{xlist}
\end{exe}

Das Korrelat \textit{es} ist hier optional, muss also nicht stehen.
Wenn der Komplementsatz ein Präpositionalobjekt vertritt, wird das Korrelat bei vielen Verben wie \textit{hinweisen} obligatorisch, wie in (\ref{ex:saetze7177}) gezeigt wird.
Das Verb \textit{hinweisen} fordert eine NP im Nominativ und eine PP mit \textit{auf}, vgl.\ (\ref{ex:saetze7177a}).
Wenn ein Komplementsatz vorliegt, wird die Komplementstelle formal durch \textit{darauf} im Mf gesättigt, das als Korrelat zum Komplementsatz fungiert, wie in (\ref{ex:saetze7177b}).
Satz (\ref{ex:saetze7177c}) zeigt, dass das Korrelat nicht fehlen darf.

\begin{exe}
  \ex\label{ex:saetze7177}
  \begin{xlist}
    \ex[]{\label{ex:saetze7177a} Ich weise [auf den leckeren Kuchen] hin.}
    \ex[]{\label{ex:saetze7177b} Ich weise darauf hin, [dass der Kuchen lecker ist].}
    \ex[*]{\label{ex:saetze7177c} Ich weise hin, [dass der Kuchen lecker ist].}
  \end{xlist}
\end{exe}

Auch Subjektsätze können in Konstruktionen mit Korrelaten stehen wie in (\ref{ex:saetze7178}).

\begin{exe}
  \ex\label{ex:saetze7178}
  \begin{xlist}
    \ex[ ]{\label{ex:saetze7178a} Es hat uns gefreut, [dass Sarah unseren Kuchen mochte].}
    \ex[ ]{\label{ex:saetze7178b} Uns hat es gefreut, [dass Sarah unseren Kuchen mochte].}
    \ex[ ]{\label{ex:saetze7178c} Uns hat gefreut, [dass Sarah unseren Kuchen mochte].}
    \ex[*]{\label{ex:saetze7178d} [Dass Sarah unseren Kuchen mochte], hat es uns sehr gefreut.}
  \end{xlist}
\end{exe}

Damit endet die sehr knappe Darstellung der Komplementsätze.
Den Komplementsätzen verwandt sind die Adverbialsätze, die sich im Wesentlichen dadurch von den Komplementsätzen unterscheiden, dass sie keine Valenzstelle saturieren.

\subsection{Adverbialsätze}

\label{sec:adverbialsaetze}
\index{Komplementierer}

Bis auf eine Ausnahme sind alle Adverbialsätze VL-Sätze, die mit einem Komplementierer eingeleitet werden.
Sie werden normalerweise nach der semantischen Funktion ihrer Komplementierer unterklassifiziert.%
\footnote{Aus diesem Grund gehen wir hier auf die Unterklassifikation nicht besonders ein.
Für die Betrachtung der Syntax sind die Unterklassen wie Finalsatz, Konsekutivsatz oder Konzessivsatz weitgehend irrelevant.}

\Definition{Adverbialsatz}{
\label{def:adverbialsatz}
Ein Adverbialsatz ist ein mit Komplementierer eingeleiteter VL-Nebensatz, der keine Valenzstelle im Matrixsatz saturiert.
\index{Adverbialsatz}
}

Beispiele sind in (\ref{ex:saetze1117}) gegeben, und zwar in dieser Reihenfolge ein Kausalsatz, ein Temporalsatz und ein Konzessivsatz.

\begin{exe}
  \ex\label{ex:saetze1117}
  \begin{xlist}
    \ex{\label{ex:saetze1117a} [Weil es regnet], bleibe ich lieber zuhause.}
    \ex{\label{ex:saetze1117b} Wir haben Kaffee getrunken, [nachdem der Kuchen aufgegessen war].}
    \ex{\label{ex:saetze1117c} [Obwohl das Buch interessant ist], ignorieren wir es.}
  \end{xlist}
\end{exe}

Adverbialsätze lassen sich oft als ein nicht-satzförmiges Adverbial umformulieren, \zB als PP.
Parallel zu (\ref{ex:saetze1117}) sind in (\ref{ex:saetze1118}) solche adverbiellen PP realisiert.

\begin{exe}
  \ex\label{ex:saetze1118}
  \begin{xlist}
    \ex{\label{ex:saetze1118a} [Wegen des Regens] bleibe ich lieber zuhause.}
    \ex{\label{ex:saetze1118b} Wir haben [nach dem Kuchenessen] Kaffee getrunken.}
    \ex{\label{ex:saetze1118c} [Trotz unseres Interesses an dem Buch] ignorieren wir es.}
  \end{xlist}
\end{exe}

\index{Nachfeld}
\index{Mittelfeld}

Wie die Beispiele in (\ref{ex:saetze1117}) zeigen, stehen Adverbialsätze genauso wie Komplementsätze gerne im Vf oder Nf.
Ob sie aus dem Mf herausbewegt werden, oder ob sie direkt in diese Positionen gestellt werden, kann und muss hier nicht entschieden werden (vgl.\ auch schon Abschnitt~\ref{sec:konstituentenstrukturinv2}, besonders Abbildung~\ref{fig:vorfelddirektbesetzung}).

\Satz{Eigenschaften von Adverbialsätzen}{
\label{satz:eigenschaftadverbialsatz}
Adverbialsätze lassen sich (im Gegensatz zu Komplementsätzen) oft unter Beibehaltung der Bedeutung in nicht-satzförmige Adverbiale (\zB PPs) umformen.
Sie stehen i.\,d.\,R.\ im Vf oder Nf.
\index{Adverbialsatz}
}

\index{Konditionalsatz}
\index{Verb-Erst-Satz}

Einen Sonderfall bilden die Konditionalsätze, die normalerweise mit Komplementierern wie \textit{wenn}, \textit{falls}, \textit{sofern} eingeleitet werden, s.\ (\ref{ex:saetze0001a}).
Der Komplementierer kann aber entfallen.
Der Konditionalsatz wird dann als V1-Satz realisiert, wie in (\ref{ex:saetze0001b}) demonstriert wird.

\begin{exe}
  \ex\label{ex:saetze0001}
  \begin{xlist}
    \ex{\label{ex:saetze0001a} [Wenn der Kuchen aufgegessen ist], stürzen wir uns auf die Kekse.}
    \ex{\label{ex:saetze0001b} [Ist der Kuchen aufgegessen], stürzen wir uns auf die Kekse.}
  \end{xlist}
\end{exe}

Mit diesem kurzen Abriss der Adverbialsätze beenden wir auch die Darstellung der satzförmigen Strukturen.
In Kapitel~\ref{sec:relationen} wird diskutiert, welchen Stellenwert bestimmte Begriffe wie Subjekt und Objekt in dem hier vorgestellten grammatischen System haben.
Außerdem werden Konstruktionen mit besonderen Verben (bzw.\ Hilfsverben und ähnlichem), \zB Passiv oder Infinitivkonstruktionen behandelt.

\Zusammenfassung

\begin{enumerate}\Lf
  \item Im unabhängigen Aussagesatz steht das finite Verb nicht im Verbalkomplex, sondern an zweiter Stelle nach einer (fast) beliebig wählbaren anderen Konstituente (V2-Satz).
  \item Ein unabhängiger Aussagesatz kann als VP betrachtet werden, aus dem zuerst das finite Verb und dann eine andere Konstituente herausbewegt wurde.
  \item Das Feldermodell bietet für diese und andere Satzstrukturen eine Oberflächenbeschreibung an, die mit unserer phrasenstrukturellen Darstellung aber im Kern nichts zu tun hat.
  \item In Entscheidungsfragesätzen steht das finite Verb an erster Stelle (V1-Satz).
  \item In einem Relativsatz bezieht sich das Pronomen als Teil der Relativphrase auf das Bezugsnomen und kongruiert mit ihr in Numerus und Genus.
  \item Ihren Kasus bzw.\ seine Form (\zB als PP) erhält die Relativphrase innerhalb des Relativsatzes per Rektion oder durch ihren Status als Angabe.
  \item Relativsätze können auch ohne Bezugsnomen als freie Relativsätze auftreten und verhalten sich dann wie eine NP.
  \item Komplementsätze sind Sätze, die eine Valenzstelle (Subjekt oder Objekt) des Matrixverbs füllen (\zB mit \textit{dass}).
  \item Adverbialsätze (\zB mit \textit{während}, \textit{damit} oder \textit{weil}) sind Angaben zum Matrixverb.
  \item Nebensätze können nach rechts aus dem Satz versetzt werden, was bei Komplementsätzen nahezu immer der Fall ist (ggf.\ unter Einsetzung eines Korrelats).
    Alternativ können Adverbialsätze und Komplementsätze (stets ohne Korrelat) im Vorfeld steht.
\end{enumerate}

\Uebungen

\Uebung \label{u121} Analysieren Sie die eingeklammerten Strukturen im Rahmen des Feldermodells nach dem Muster des ersten Beispiels.
Bei den Sätzen \ref{it:7493} und \ref{it:7494} handelt es sich um Transferaufgaben.

\begin{enumerate}\Lf
  \item{[Sarah isst den Kuchen alleine auf.]}
    \begin{itemize}\Lf
      \item Kf: ---
      \item Vf: Sarah
      \item LSK: isst
      \item Mf: den Kuchen alleine
      \item RSK: auf
      \item Nf: ---
    \end{itemize}
  \item{[Man sollte den Tag genießen.]}
  \item{[Kann mal jemand das Fenster aufmachen?]}
  \item\label{it:934254} Das ist das Eis, [das wir selber gemacht haben].
  \item{[Was hat Ischariot gemalt?]}
  \item{[Gehst du?]}
  \item{\label{it:7493} [Geh!]}
  \item\label{it:7494} Es ist eine tolle Sommernacht, [denn der Mond scheint hell].
  \item{[Den leckeren Kuchen auf dem Tisch hatte Rigmor sofort entdeckt.]}
  \item{[Obwohl Liv einkaufen wollte], ist nichts im Haus.}
  \item Kann man feststellen, [wer den Kuchen gegessen hat]?
\end{enumerate}

\Uebung \label{u122} Analysieren Sie die folgenden komplexen Sätzen im Rahmen des Feldermodells nach dem Muster des ersten Beispiels.
Dabei sind von eingebetteten Nebensätzen keine Analysen durchzuführen.

\begin{enumerate}\Lf
  \item Dass der Kuchen gegessen wurde, bedauern alle sehr, die es erfahren haben.
    \begin{itemize}\Lf
      \item Kf: ---
      \item Vf: Dass der Kuchen gegessen wurde
      \item LSK: bedauern
      \item Mf: alle sehr
      \item RSK: ---
      \item Nf: die es erfahren haben
    \end{itemize}
  \item Wohin man auch blickt, kann man die Bäume kaum erkennen, denn der Schnee bedeckt alles.
  \item Geht derjenige, der kommt, auch wieder?
  \item Die Kollegen, denen wir nichts vom Kuchen gegeben haben, schimpfen.
  \item Denn ob es Eis gibt, kann nur einer wissen, der Zugang zur Eismaschine hat.
  \item Liv will, dass Rigmor ihr von dem Eis abgibt.
\end{enumerate}

\Uebung \label{u123} Führen Sie Konstituentenanalysen der folgenden Auswahl einfacher Sätze aus Übung \ref{u121} durch (ohne Bewegungspfeile).
Für ein Beispiel (erster Satz) vgl.\ Abbildung~\ref{fig:v2mitpartikel}.

\begin{enumerate}\Lf
  \item{Sarah isst den Kuchen alleine auf.}
  \item{Man sollte den Tag genießen.}
  \item{Kann mal jemand das Fenster aufmachen?}
  \item{Was hat Ischariot gemalt?}
  \item{Gehst du?}
  \item{Den leckeren Kuchen auf dem Tisch hatte Rigmor sofort entdeckt.}
\end{enumerate}

\Uebung[\tristar] \label{u124} Führen Sie Konstituentenanalysen für die folgende Auswahl aus den komplexen Sätzen aus Übung \ref{u122} durch.
Es handelt sich überwiegend um eine Transferaufgabe:
Überlegen Sie, wie das Nachfeld in den Konstituentenstrukturen abgebildet werden kann.

\begin{enumerate}\Lf
  \item Dass der Kuchen gegessen wurde, bedauern alle sehr, die es erfahren haben.
  \item Die Kollegen, denen wir nichts vom Kuchen gegeben haben, schimpfen.
  \item Liv will, dass Rigmor ihr von dem Eis abgibt.
\end{enumerate}

\Uebung[\tristar] \label{u125} Analysieren Sie die folgenden NPs mit Relativsatz nach dem Muster von Abbildung~\ref{fig:saetze9228} (s.\ S.~\pageref{fig:saetze9228}), aber ohne Kongruenz- und Rektionspfeile.

\begin{enumerate}\Lf
  \item{[Das Buch, das ich lese], gehört nicht mir.}
  \item Wir mögen [Menschen, auf die wir vertrauen können].
  \item Wir treffen [die Kommilitoninnen, deren Kuchen wir gegessen haben].
\end{enumerate}

\Uebung[\tristar] \label{u126} Dialektal gibt es Relativsätze bzw.\ eingebettete \textit{w}"=Sätze wie in (\ref{ex:ex234}).

\begin{exe}
  \ex{\label{ex:ex234} Ich weiß, [wer dass kommt].}
\end{exe}

Überlegen Sie, was hier anders ist als im Standard und geben Sie eine Felderanalyse und eine Konstituentenstruktur an.

\Uebung[\tristar] \label{u127} Die deutsche Orthographie zeigt viele interessante grammatische Beziehungen auf.
Überlegen Sie, warum die Form des Verbs \textit{zurückbleiben} in (\ref{ex:ex23456348a}) zusammengeschrieben, aber in (\ref{ex:ex23456348b}) auseinandergeschrieben wird.

\begin{exe}
  \ex\label{ex:ex23456348}
  \begin{xlist}
    \ex{\label{ex:ex23456348a} Es ist in Ordnung, wenn der große Schreibtisch erst einmal zurückbleibt.}
    \ex{\label{ex:ex23456348b} Zurück bleibt der Schreibtisch nur, wenn der LKW randvoll ist.}
  \end{xlist}
\end{exe}

