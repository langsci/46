\chapter{Verbalflexion}

\label{sec:verben}

Nach der Flexion der Nomina wird in diesem Kapitel nun die Flexion der zweiten Klasse der flektierbaren Wörter besprochen, nämlich die der Verben.%
\footnote{Traditionell bezeichnet man die Verbalflexion auch als \textit{Konjugation}.}
Die Verbalflexion ist insofern einfacher als die Nominalflexion, als die Verben weniger Flexionsklassen haben.
Im Wesentlichen muss man zwei große Flexionsklassen (\textit{starke} und \textit{schwache} Verben), eine Sonderklasse (sogenannte \textit{präteritalpräsentische} Verben) und einige mehr oder weniger unregelmäßige Verben unterscheiden.
Dieses Kapitel ist sehr einfach strukturiert:
Nach der Besprechung des Kategorieninventars der Verben (Abschnitt~\ref{sec:vkat}) folgt die Darstellung der Flexionsbesonderheiten (Abschnitt~\ref{sec:vvflex}).

\section{Kategorien}

\label{sec:vkat}

Aus der Tatsache, dass das Verb in bestimmten Merkmalen mit dem Subjekt -- also prototypisch mit einer NP -- kongruiert, folgt, dass es bestimmte Merkmale mit den Nomina gemein haben muss.
Dies sind Person und Numerus, die in Abschnitt~\ref{sec:vpersnum} nochmals kurz angesprochen werden.
Spezifisch verbale Merkmale sind \textit{Tempus} (Abschnitt~\ref{sec:tempus}), \textit{Modus} (Abschnitt~\ref{sec:modus}) und die \textit{Finitheit} bzw.\ die Art der \textit{Infinitheit} (Abschnitt~\ref{sec:finit}).
In Abschnitt~\ref{sec:genusverb} wird argumentiert, dass das sog.\ \textit{Genus verbi} (also die Unterscheidung nach \textit{Aktiv} und \textit{Passiv}) zwar eine verbale Kategorie ist, aber im Deutschen nicht als Flexionsmerkmal aufgefasst werden kann.

\subsection{Person und Numerus}

\label{sec:vpersnum}

\index{Numerus!Verb}
\index{Person!Verb}

In den Abschnitten~\ref{sec:person} und \ref{sec:numerus} wurde bereits mit Bezug auf die Nomina über die Merkmale \textsc{Person} und \textsc{Numerus} gesprochen.
Wir verstehen \textsc{Person} und \textsc{Numerus} als Merkmale, die im Bereich der Nomina motiviert sind und sehen sie bei den Verben als reine Kongruenzmerkmale an.
Wie mehrfach erwähnt, kongruiert die Nominativ-Ergänzung mit dem nach Tempus flektierten Verb in Person und Numerus.

Auf eine Besonderheit dieser Merkmale sei hier noch verwiesen.
Wie in Abschnitt~\ref{sec:komplementsaetze} ausführlich diskutiert wird, gibt es auch Subjekte, die nicht nominal, sondern (neben)satzförmig sind.
Einige Beispiele finden sich in (\ref{ex:vflex2340}) und (\ref{ex:vflex2341}) -- jeweils (a) und (b) --, wobei die Subjekte in [] gesetzt sind.

\begin{exe}
  \ex\label{ex:vflex2340}
  \begin{xlist}
    \ex[]{\label{ex:vflex2340a} [Dass es schneit] erfreut alle.}
    \ex[*]{\label{ex:vflex2340b} [Dass es schneit] erfreuen alle.}
    \ex[]{\label{ex:vflex2340c} [Das Schneien] erfreut alle.}
  \end{xlist}
  \ex\label{ex:vflex2341}
  \begin{xlist}
    \ex[]{\label{ex:vflex2341a} [Den Schnee zu schieben] macht ihnen Spaß.}
    \ex[*]{\label{ex:vflex2341b} [Den Schnee zu schieben] machen ihnen Spaß.}
    \ex[]{\label{ex:vflex2341c} [Das Schneeschieben] macht ihnen Spaß.}
  \end{xlist}
\end{exe}

In (\ref{ex:vflex2340}) ist das Subjekt ein Nebensatz, der mit \textit{dass} eingeleitet wird (ein Komplementsatz bzw.\ genauer ein Subjektsatz).
In (\ref{ex:vflex2341}) handelt es sich bei dem Subjekt um eine Infinitivkonstruktion.
In beiden Fällen können wir anstelle des satzförmigen Subjekts auch eine normale NP einsetzen, wie in (\ref{ex:vflex2340c}) und (\ref{ex:vflex2341c}) zu sehen ist.%
\footnote{Zu den syntaktischen Begriffen, die hier verwendet wurden vgl.\ genauer Kapitel~\ref{sec:saetze} und \ref{sec:relationenpraedikate}.}

Bezüglich der verbalen Kongruenzmerkmale ist nun festzustellen, dass sie bei solchen satzförmigen Subjekten immer mit [\textsc{Person}: \textit{3}, \textsc{Numerus}: \textit{sg}] kongruieren.
Mit [\textsc{Numerus}: \textit{pl}] werden die Sätze ungrammatisch wie in (\ref{ex:vflex2340b}) und (\ref{ex:vflex2341b}).
Diese Beobachtung bleibt hier rein deskriptiv stehen, da die Annahme, die Sätze trügen diese Kongruenzmerkmale selber, problematisch ist.
Dann ist allerdings zu erklären, wie sie in Merkmalen kongruieren können, die sie selber nicht haben.
Das Problem ist im gegebenen Rahmen nicht formal lösbar, und wir gehen stattdessen zu den verbalen Kategorien \textit{Tempus} (Abschnitt~\ref{sec:tempus}) und \textit{Modus} (Abschnitt~\ref{sec:modus}) über.

\subsection{Tempus}

\label{sec:tempus}

In diesem Abschnitt wird sehr kurz auf die semantische Funktion des Tempus eingegangen.
Im nächsten Abschnitt werden dann die unterschiedlichen Formen der Realisierung des Tempus im Deutschen dargestellt.

\index{Tempus!einfach}
Verben beschreiben alle Arten von Ereignissen oder Zuständen (\textit{fassen}, \textit{aufblitzen}, \textit{winken}, \textit{bauen}).
Einfach gesagt stellt ein spezifisches Tempus eine Beziehung zwischen der \textit{Sprechzeit} (S) und der Zeit des im gegebenen Satz beschriebenen Ereignisses -- der \textit{Ereigniszeit} (E) -- her.%
\footnote{Es ist zu beachten, dass die Abkürzungen S und E (und später R) die Zeitpunkte der Sprechhandlung, des Ereignisses usw. bezeichnen, nicht etwa die Sprechhandlung oder das Ereignis selber.}
Wenn man Beispiele des deutschen \textit{Präteritums} (einfache Vergangenheit) nimmt, ist dies eindeutig, \zB in (\ref{ex:vflex9132}).

\begin{exe}
  \ex\label{ex:vflex9132}
  \begin{xlist}
    \ex{Das Licht blitzte auf.}
    \ex{Kurt fasste den Mörder.}
  \end{xlist}
\end{exe}

In diesen Beispielen liegt das beschriebene Ereignis vom Äußerungsmoment aus betrachtet in der Vergangenheit.
Durch Hinzufügen von vergangenheitsbezogenen Adverbialen wie \textit{gestern}, \textit{letzte Woche} usw.\ kann der Zeitpunkt eindeutiger eingegrenzt werden.
Selbst wenn gegenwartsbezogene Adverbiale wie \textit{heute} hinzugefügt werden, geben diese Adverbiale zwar einen zeitlichen Rahmen vor, aber das Ereignis bleibt innerhalb des Rahmens in der Vergangenheit lokalisiert.
Das Adverb \textit{jetzt} ändert in (\ref{ex:vflex9132b}) seine Bedeutung und verweist nicht mehr auf den aktuellen Sprechmoment, sondern bekommt eine narrative Funktion im Sinne von \textit{in jenem Moment}, während der temporale Bezug auf die Vergangenheit erhalten bleibt.

\begin{exe}
  \ex\begin{xlist}
    \ex{Heute blitzte das Licht auf.}
    \ex{\label{ex:vflex9132b} Jetzt fasste Kurt den Mörder.}
  \end{xlist}
\end{exe}

\index{Sprechzeitpunkt}
\index{Ereigniszeitpunkt}

Wenn wir also den Sprechzeitpunkt mit S und den Ereigniszeitpunkt mit E bezeichnen und die Relation \textit{x liegt zeitlich vor y} mit $\ll$ angeben, lässt sich das Präteritum als E $\ll$ S, also als \textit{einfache Vergangenheit} darstellen.
Wir führen nun die Bezeichnung \textit{einfaches Tempus} ein.

\Definition{Tempus und einfaches Tempus}{
\label{def:einftemp}
Tempus ist eine grammatische Kategorie, die am Verb realisiert wird.
Sie spezifiziert eine zeitliche Relation zwischen dem Zeitpunkt des beschriebenen Ereignisses E und dem Sprechzeitpunkt S.
Ein einfaches Tempus ist eines, bei dem eine direkte Relation zwischen Sprechzeitpunkt S und Ereigniszeitpunkt E hergestellt wird.
\index{Tempus}
\index{Tempus!einfach}
}

Wie ist es nun mit dem sogenannten \textit{Präsens}?
Der lateinische Name und die landläufige Auffassung suggerieren, dass es sich um ein Tempus handelt, das Ereignisse beschreibt, die zum Sprechzeitpunkt (zum \textit{Jetzt}) geschehen.
In unserer Notation wäre also S=E.
Dass dies nicht so ist, lässt sich mit Beispielen wie denen in (\ref{ex:vflex8326}) zeigen, für die die Relation zwischen S und E angegeben wird.

\begin{exe}
  \ex\label{ex:vflex8326}
  \begin{xlist}
    \ex{\label{ex:vflex8326a} E $\ll$ S\\
    Im Jahr 1962 beginnt die DDR mit dem Bau der Mauer.}
    \ex{\label{ex:vflex8326b} S $\ll$ E\\
    Morgen esse ich Maronen.}
    \ex{\label{ex:vflex8326c} \ldots $\ll$ E\Sub{1} $\ll$ E\Sub{2} $\ll$ S $\ll$ E\Sub{3} $\ll$ E\Sub{4} $\ll$ \ldots\\
    Heute ist Mittwoch, und donnerstags kommt die Müllabfuhr.}
  \end{xlist}
\end{exe}

\index{Präsens!Bedeutung}

Das Präsens kann also sowohl für vergangene Ereignisse (\ref{ex:vflex8326a}), als auch für zukünftige oder geplante Ereignisse (\ref{ex:vflex8326b}) verwendet werden.
Außerdem gibt es Verwendungen wie in (\ref{ex:vflex8326c}), in denen das Präsens vielmehr anzeigt, dass ein Ereignis immer wieder eintritt und dass der Sprechzeitpunkt irgendwo zwischen einem dieser Wiederholungen des Ereignisses liegt.
Diese Interpretationen des Präsens kommen hier durch Adverben (\textit{gestern}, \textit{morgen} und \textit{donnerstags}) zustande, aber es könnte genausogut der Kontext oder die Situation sein, in denen ein Satz geäußert wird.
Auch wenn wir also ein Merkmal \textsc{Tempus} mit einem Wert \textit{präs}(\textit{ens}) annehmen, muss festgehalten werden, dass das Präsens im Grunde das Fehlen einer speziellen Zeitrelation markiert.
In der Notation führen wir für die Relation \textit{x steht in keiner spezifischen zeitlichen Folge zu y} das Symbol $\sim$ ein und stellen das Präsens damit als S $\sim$ E dar.

\index{Futur!Bedeutung}

Es bleibt von den einfachen Tempora noch das \textit{Futur}.
Das Futur scheint Ereignisse in der Zukunft zu beschreiben, diagrammatisch also S $\ll$ E.
An Sätzen wie (\ref{ex:vflex9173}) ist dies auch gut nachvollziehbar.

\begin{exe}
  \ex\label{ex:vflex9173}
  \begin{xlist}
    \ex{\label{ex:vflex9173a} Es wird regnen.}
    \ex{\label{ex:vflex9173b} Ich werde eine Allergikerkatze kaufen.}
  \end{xlist}
\end{exe}

Es wird hier an dieser Interpretation des Futurs festgehalten.
Allerdings ist eine prinzipielle Anmerkung zu machen.
Während beim Präteritum der Vergangenheitsbezug eindeutig ist, weil wir über die Ereignisse der Vergangenheit zumindest wissen \textit{können}, ob sie stattgefunden haben oder nicht, so liegt es beim Zukunftsbezug in der Natur der Dinge, dass das Eintreten des Ereignisses niemals garantiert werden kann.
In Satz (\ref{ex:vflex9173a}) handelt es sich vielmehr um den Ausdruck einer (informierten) Erwartung, dass es regnen wird.
In (\ref{ex:vflex9173b}) hingegen wird die Absicht kundgetan, eine Allergikerkatze zu kaufen.
Auch wenn aufgrund dieser Probleme mit der simplen Interpretation des Futurs teilweise versucht wird, das Futur nicht im Rahmen der Tempuskategorien zu behandeln, bleiben wir hier dabei.
Der Grund liegt vor allem darin, dass alle Absichtserklärungen, Vermutungen, Voraussagen usw., die im Futur formuliert werden, letztlich nur dadurch geeint werden, dass sie sich auf zukünftige Ereignisse beziehen.
Gerade weil diese verschiedenen Aussagen über die Zukunft unterschiedliche Motivationen haben, wäre es schwer, ein anderes gemeinsames Kriterium für die Definition des Futurs als den Zukunftsbezug zu finden.

Tabelle~\ref{tab:simptemp} fasst die Relationen zwischen Ereigniszeit und Sprechzeit für die einfachen Tempora zusammen.

\begin{table}[!htbp]
  \centering
  \begin{tabular}{llp{6cm}}
    \lsptoprule
    \textbf{Tempus} & \textbf{Relation} & \textbf{Beschreibung} \\
    \midrule
    Präsens & S $\sim$ E & Ereignis- und Sprechzeitpunkt stehen in keiner besonderen Ordnung. \\
    Präteritum & E $\ll$ S & Das Ereigniszeitpunkt liegt zeitlich vor dem Sprechzeitpunkt. \\
    Futur & S $\ll$ E & Der Sprechzeitpunkt liegt zeitlich vor dem Ereigniszeitpunkt. \\
    \lspbottomrule
  \end{tabular}
  \caption{Semantisch einfache Tempora}
  \label{tab:simptemp}
\end{table}

Die Rede von den \textit{einfachen Tempora} deutet an, dass es auch \textit{komplexe Tempora} gibt.
Der Terminus \textit{komplexes Tempus} bezieht sich hier nicht auf die Unterscheidung von \textit{synthetischen} (morphologischen) und \textit{analytischen} (syntagmatischen) Bildungen (vgl.\ Abschnitt~\ref{sec:deutemp}).
Vielmehr geht es um das Tempus in Sätzen wie denen in (\ref{ex:vflex9112}).

\begin{exe}
  \ex\label{ex:vflex9112}
  \begin{xlist}
    \ex{\label{ex:vflex9112a} Das Licht hatte bereits aufgeblitzt (als der Warnton ertönte).}
    \ex{\label{ex:vflex9112b} Kurt wird den Mörder (spätestens nächste Woche) gefasst haben.}
  \end{xlist}
\end{exe}

\index{Referenzzeitpunkt}

In diesen Sätzen liegen das sogenannte \textit{Plusquamperfekt} (\ref{ex:vflex9112a}) und das \textit{Futurperfekt} (auch \textit{Futur II} genannt) (\ref{ex:vflex9112b}) vor.
Das Besondere an diesen Tempora ist gegenüber dem Präsens, Präteritum und Futur, dass ein weiterer \textit{Referenzzeitpunkt} (R) eingeführt wird.
In (\ref{ex:vflex9112a}) wird von einem Ereignis in der Vergangenheit gesprochen, nämlich dem Ertönen des Warntons.
Weiterhin wird ausgesagt, dass zur Zeit dieses Ereignisses ein anderes Ereignis bereits eingetreten war.
Die tatsächliche temporale Beziehung von Sprechzeitpunkt und Ereigniszeitpunkt wird also über einen weiteren Referenzzeitpunkt hergestellt, der hier mit einem Temporalsatz (mit \textit{als}) eingeführt wird.\index{Futurperfekt!Bedeutung}
Formal muss man also zwei Bedingungen R $\ll$ S und E $\ll$ R formulieren.
Eine Analyse des Beispiels (\ref{ex:vflex9112a}) ist Abbildung~\ref{fig:reiplsq-bsp}.

\begin{figure}[!htbp]
  \centering
  \begin{tabular}{ccc}
    E: Aufblitzen des Lichts & $\ll$ & R: Ertönen des Warntons \\
    R: Ertönen des Warntons & $\ll$ & S: Sprechzeitpunkt \\
  \end{tabular}
  \caption{Analyse eines Satzes mit Plusquamperfekt}
  \label{fig:reiplsq-bsp}
\end{figure}

\index{Präteritumsperfekt!Bedeutung}
Beim Futurperfekt haben wir es mit einer ähnlichen Situation zu tun.
Über eine Referenzzeit in der Zukunft wird ein Ereignis in der Vergangenheit dieser Referenzzeit verortet.
Die Referenzzeit wird in (\ref{ex:vflex9112b}) mit der adverbialen Bestimmung \textit{spätestens nächste Woche} angegeben, das Ereignis ist in der Interpretation von (\ref{ex:vflex9112b}) das Fassen des Mörders.
Formal ist die Bedeutung des Futurperfekt also parallel zu der des Plusquamperfekts als E $\ll$ R und  S $\ll$ R darzustellen.
Die Analyse des Beispielsatzes (\ref{ex:vflex9112b}) findet sich in Abbildung~\ref{fig:reif2-bsp}.

\begin{figure}[!htbp]
  \centering
  \begin{tabular}{ccc}
    E: Fassen des Mörders & $\ll$ & R: nächste Woche \\
    S: Sprechzeitpunkt & $\ll$ & R: nächste Woche \\
  \end{tabular}
  \caption{Analyse eines Satzes mit Futurperfekt}
  \label{fig:reif2-bsp}
\end{figure}

Beim Futurperfekt ist also die Relation zwischen S und E nicht direkt spezifiziert, so dass E möglicherweise auch vor S liegen könnte.
Intuitiv ist die typische Deutung des Futurperfekt aber vielleicht eher S $\ll$ E $\ll$ R.
Man würde also erwarten, dass der Ereigniszeitpunkt E zwischen Sprechzeitpunkt S und Referenzzeitpunkt R liegt.
Angesichts von Sätzen wie (\ref{ex:vflex2885}) wäre dies aber eindeutig zu eng gefasst.

\begin{exe}
  \ex{\label{ex:vflex2885} Im Jahr 2100 wird Helmut Schmidt als Kanzler abgedankt haben.}
\end{exe}

Wenn wir diesen Satz zu einem Sprechzeitpunkt im Jahr 2010 auswerten, liegt das Ereignis (Abdanken Helmut Schmidts) in der Vergangenheit, nämlich im Jahr 1982.
Der Referenzzeitpunkt ist aber deutlich zukünftig, nämlich das Jahr 2100.
Im Gegensatz zu Satz (\ref{ex:vflex9112b}) haben wir hier ein Weltwissen, dass uns zu der Annahme zwingt, dass E $\ll$ S.
Die Analyse ist in Abbildung~\ref{fig:reif2-bsp2} angegeben, und sie ist völlig parallel zu Abbildung~\ref{fig:reif2-bsp}.
Bei genauem Hinsehen hat allerdings auch (\ref{ex:vflex9112b}) eine mögliche Interpretation, bei der zum Sprechzeitpunkt der Mörder bereits gefasst ist.
Es muss also immer zwischen dem tatsächlichen Bedeutungsbeitrag der sprachlichen Form und zusätzlichen Annahmen (\zB durch Weltwissen) getrennt werden.

\begin{figure}[!htbp]
  \centering
  \begin{tabular}{ccc}
    E: Abdanken Schmidts 1982 & $\ll$ & R: 2100 \\
    S: 2010 & $\ll$ & R: 2100 \\
  \end{tabular}
  \caption{Analyse eines Satzes mit Futurperfekt}
  \label{fig:reif2-bsp2}
\end{figure}

An Satz (\ref{ex:vflex2885}) fällt auf, dass nur der Referenzzeitpunkt R und nicht der Ereigniszeitpunkt E in der Zukunft liegt.
Damit verschwinden die Unsicherheiten (bzw.\ die semantischen Spezialisierungen) des Futurs als Bekundung von Absicht, Vermutung usw., denn der Satz ist zum jetzigen Zeitpunkt vollständig auswertbar.
Nur wenn S $\ll$ E vorliegt, ergeben sich die weiter oben beschriebenen Interpretationsprobleme des Futurs.

\index{Tempus!Folge}

Wie die Referenzzeit bestimmt wird, ist vielfältig.
In (\ref{ex:vflex9112}) liefern Adverbiale i.\,w.\,S.\ die Referenzzeitpunkte, es kann aber genausogut der Kontext (\ref{ex:vflex2887a}) oder die Äußerungssituation sein.
Es können auch Referenzzeitpunkte für Tempora in Nebensätzen aus den zugehörigen Hauptsätzen gewonnen werden (\ref{ex:vflex2887c}), wobei dann bestimmte Anforderungen an die Abfolge der Tempora eingehalten werden müssen (die sog.\ \textit{Tempusfolge} oder \textit{Consecutio temporum}).
Ein Temporalsatz im Plusquamperfekt, der von \textit{nachdem} eingeleitet wird, darf \zB nicht an einen Hauptsatz im Präsens angeschlossen werden wie in (\ref{ex:vflex2887d}), sehr wohl aber an einen im Präteritum (\ref{ex:vflex2887c}).

\begin{exe}
  \ex\label{ex:vflex2887b}
  \begin{xlist}
    \ex[]{\label{ex:vflex2887a} Frida nahm das Buch in die Hand.
      Sie hatte es bereits gelesen.}
    \ex[]{\label{ex:vflex2887c} Frida legte das Buch weg, nachdem sie es gelesen hatte.}
    \ex[*]{\label{ex:vflex2887d} Frida legt das Buch weg, nachdem sie es gelesen hatte.}
  \end{xlist}
\end{exe}

Wir können nun mit einer Definition des Begriffs des \textit{komplexen Tempus} schließen.
Tabelle~\ref{tab:komplextemp} fasst die komplexen Tempora zusammen.

\Definition{Komplexes Tempus}{
\label{def:komptemp}
Ein komplexes Tempus ist ein Tempus, bei dem keine direkte zeitliche Folgerelation zwischen Sprechzeit S und Ereigniszeit E besteht, sondern diese nur mittelbar über eine zusätzliche Referenzzeit R hergestellt wird.
\index{Tempus!komplex}
}

\begin{table}[!htbp]
  \centering
  \begin{tabular}{lll}
    \lsptoprule
    \textbf{Tempus} & \textbf{R-S-Bedingung} & \textbf{E-R-Bedingung} \\
    \midrule
	Plusquamperfekt & R $\ll$ S & E $\ll$ R \\
	Futurperfekt & S $\ll$ R & E $\ll$ R \\
    \lspbottomrule
  \end{tabular}
  \caption{Semantisch komplexe Tempora}
  \label{tab:komplextemp}
\end{table}

\subsection{Tempusformen}

\label{sec:deutemp}

Schulgrammatisch wird oft von den sechs Tempusformen des Deutschen als \textit{Konjugation} gesprochen, und man versteht darunter \idR die Formen in Tabelle~\ref{tab:sechstempora}.

\begin{table}[!htbp]
  \centering
  \begin{tabular}{ll}
    \lsptoprule
    \textbf{Tempus} & \textbf{Beispiel 3.~Person}\\
    \midrule
    Präsens & lacht \\
    Präteritum & lachte \\
    Perfekt & hat gelacht \\
    Plusquamperfekt & hatte gelacht \\
    Futur & wird lachen \\
    Futurperfekt & wird gelacht haben \\
    \lspbottomrule
  \end{tabular}
  \caption{Die sechs funktionalen Tempora des Deutschen}
  \label{tab:sechstempora}
\end{table}
\index{Präsens}\index{Präteritum}\index{Perfekt}\index{Präteritumsperfekt}\index{Futur}

Dass es sich bei diesen sechs Formen um Tempora handelt, soll natürlich nicht bestritten werden.
Da die Verbalflexion die Wortformenbildung des Verbs bezeichnet, müssen allerdings alle genannten Tempora bis auf Präsens und Präteritum davon ausgenommen werden.
Bei den beiden Perfekta und den beiden Futura handelt es sich offensichtlich um analytische Bildungen, also mehrere Wortformen von Hilfsverben und einem Vollverb, die zusammen einen bestimmten Tempus-Effekt haben.
\index{Perfekt}
Dass beim deutschen Perfekt, Plusquamperfekt usw.\ oft fälschlicherweise von Flexion gesprochen wird, liegt historisch an einer starken Anlehnung an die Lateingrammatik.
Im Lateinischen wird \zB das Perfekt als eine Form (von einem eigenen Stamm) gebildet und ist damit eine Flexionskategorie, vgl. (\ref{ex:vflex9291}).

\begin{exe}
  \ex\label{ex:vflex9291}\gll Et dixit illis angelus: Nolite timere!\\
  und {hat gesagt} ihnen {Engel}: {wollt nicht} fürchten\\
  \glt Und der Engel sagte zu ihnen: Fürchtet euch nicht. (Lukas 2, 10)
\end{exe}

\index{Tempus!synthetisch vs.\ analytisch}

Das deutsche Perfekt wird aus einer Präsensform des Hilfsverbs \textit{haben} oder \textit{sein} und dem \textit{Partizip} (in Tabelle~\ref{tab:sechstempora} \textit{gelacht}) -- also einer infiniten Verbform -- gebildet.%
\footnote{Finitheit wurde zuerst mit Definition~\ref{def:finitheit} auf S.~\pageref{def:finitheit} eingeführt.
Zur Formenbildung der infiniten Verben vgl.\ Abschnitt~\ref{sec:infinflex}.}
Sämtliche Flexionsmerkmale (Person, Numerus und morphologisches Tempus) werden in den einfachen Fällen wie hier am Hilfsverb markiert, das Partizip flektiert nicht weiter.
Das Plusquamperfekt ist dem Perfekt sehr ähnlich, es wird lediglich statt einer Präsensform des Hilfsverbs das Präteritum des Hilfsverbs (hier \textit{hatte}) verwendet.
Das einfache Futur (auch \textit{Futur I} genannt) wird aus dem Hilfsverb \textit{werden} und dem Infinitiv (hier \textit{lachen}) gebildet.
Das Futurperfekt kombiniert das Hilfsverb \textit{werden}, das Hilfsverb \textit{haben} im Infinitiv und das Vollverb im Partizip.

Wenn wir die Tempora aus Tabelle~\ref{tab:sechstempora} unter dem Gesichtspunkt der Wortformenanalyse betrachten, ergibt sich ein Bild wie in (\ref{ex:vsynana}), wo die Wortformen mit ihren typischen finiten Flexions-Merkmalen glossiert wurden.
Beim Infinitiv und Partizip sind ganz einfach gar keine Tempus- und Kongruenz-Merkmale vorhanden.
Wie die analytischen Bildungen genau zusammengefügt sind und was \zB das Perfekt im Unterschied zum Präteritum bedeutet, wird später noch in Abschnitt~\ref{sec:analytischetempora} besprochen.
Es sollte hier nur deutlich geworden sein, warum hier im Rahmen der Flexion lediglich die zwei synthetischen Tempora berücksichtigt werden.

\begin{exe}
  \ex\label{ex:vsynana}
  \begin{xlist}
    \ex\gll lacht \\
    {[\textsc{Temp}: \textit{präs}, \textsc{Per}:\textit{3}, \textsc{Num}:\textit{sg}]}\\
    \ex\gll lachte \\
    {[\textsc{Temp}: \textit{prät}, \textsc{Per}:\textit{3}, \textsc{Num}:\textit{sg}]}\\
    \ex\gll hat gelacht\\
    {[\textsc{Temp}: \textit{präs}, \textsc{Per}:\textit{3}, \textsc{Num}:\textit{sg}]} {[]}\\
    \ex\gll hatte gelacht\\
    {[\textsc{Temp}: \textit{prät}, \textsc{Per}:\textit{3}, \textsc{Num}:\textit{sg}]} {[]}\\
    \ex\gll wird lachen\\
    {[\textsc{Temp}: \textit{präs}, \textsc{Per}:\textit{3}, \textsc{Num}:\textit{sg}]} {[]}\\
    \ex\gll wird gelacht haben \\
    {[\textsc{Temp}: \textit{präs}, \textsc{Per}:\textit{3}, \textsc{Num}:\textit{sg}]} {[]} {[]}\\
  \end{xlist}
\end{exe}

\subsection{Modus}

\label{sec:modus}

Unter der Kategorie \textit{Modus} fasst man für das Deutsche mindestens den \textit{Indikativ} und den \textit{Konjunktiv}, gelegentlich auch den \textit{Imperativ}.
Den Imperativ behandeln wir aus Gründen, die in Abschnitt~\ref{sec:impflex} dargelegt werden, nicht als Modus.
Es folgt eine Diskussion des Unterschieds von Indikativ und Konjunktiv.
In (\ref{ex:vflex5550}) finden sich einige Beispiele für den sogenannten Konjunktiv I, in (\ref{ex:vflex5548})--(\ref{ex:vflex55491}) für den Konjunktiv II.%
\footnote{Manchmal bezeichnet man den Konjunktiv I auch als Konjunktiv Präsens und den Konjunktiv II als Konjunktiv Präteritum.
Die Gründe liegen ausschließlich in der Formenbildung, vgl.\ Abschnitt~\ref{sec:konjunktivflexion}.}
Die Konjunktive folgen jeweils den parallelen Beispielen im Indikativ.

\begin{exe}
  \ex \label{ex:vflex5550}
  \begin{xlist}
    \ex[]{\label{ex:vflex5550a} Sie sagte, der Kuchen schmeckt lecker. (Ind)}
    \ex[]{\label{ex:vflex5550b} Sie sagte, der Kuchen schmecke lecker. (Konj I)}
    \ex[]{\label{ex:vflex5550c} Sie sagte, dass der Kuchen lecker schmeckt. (Ind)}
    \ex[]{\label{ex:vflex5550d} Sie sagte, dass der Kuchen lecker schmecke. (Konj I)}
  \end{xlist}
  \ex \label{ex:vflex5548}
  \begin{xlist}
    \ex[]{\label{ex:vflex5548a} Wenn das geschieht, laufe ich weg. (Ind)}
    \ex[]{\label{ex:vflex5548b} Immer, wenn das geschieht, laufe ich weg. (Ind)}
    \ex[]{\label{ex:vflex5548c} Wenn das geschähe, liefe ich weg. (Konj II)}
    \ex[*]{\label{ex:vflex5548d} Immer, wenn das geschähe, liefe ich weg. (Konj II)}
  \end{xlist}
  \ex \label{ex:vflex5549}
  \begin{xlist}
    \ex[]{\label{ex:vflex5549a} Ohne Schnee sind die Ferien dieses Jahr nicht so schön. (Ind)}
    \ex[]{\label{ex:vflex5549b} Ohne Schnee wären die Ferien dieses Jahr nicht so schön. (Konj II)}
  \end{xlist}
  \ex \label{ex:vflex55491}
  \begin{xlist}
    \ex[]{\label{ex:vflex55491a} Im Urlaub hat kein Schnee gelegen. (Ind)}
    \ex[]{\label{ex:vflex55491b} Ach, hätte im Urlaub doch Schnee gelegen. (Konj II)}
  \end{xlist}
\end{exe}

Ohne im Einzelnen auf die Formenbildung einzugehen (dazu Abschnitt~\ref{sec:konjunktivflexion}), können wir uns überlegen, was der semantische Beitrag der Modusformen in diesen Sätzen ist.
(\ref{ex:vflex5550}) zeigt die typische Verwendung des Konjunktivs I.
Bei Verben des Sagens (\textit{sagen}, \textit{erzählen} usw.), der Einschätzung (\textit{denken}, \textit{glauben} usw.) oder des Fragens (\textit{fragen ob}) kann der Inhalt der Rede, der Einschätzung oder der Frage im Konjunktiv I formuliert werden, um das Gesagte als indirekt zu markieren.
Dies ist in (\ref{ex:vflex5550b}) und (\ref{ex:vflex5550d}) der Fall.
Mit dem Indikativ in (\ref{ex:vflex5550a}) und (\ref{ex:vflex5550c}) wird der Redeinhalt eher wie eine wörtliche Rede wiedergegeben und könnte im Falle von (\ref{ex:vflex5550a}) auch in Anführungsstrichen stehen.
Wir nennen den Konjunktiv I daher hier den \textit{quotativen Konjunktiv} (zitierenden Konjunktiv).

In (\ref{ex:vflex5548a}) transportiert der Indikativ eine vergleichsweise faktische Aussage über ein typisches oder gewöhnliches Verhalten des Sprechers.
Der Satz ist kompatibel mit \textit{immer} wie in (\ref{ex:vflex5548b}).
Die Aussage des Satzes in (\ref{ex:vflex5548c}) ist durch den Konjunktiv deutlich relativiert bzw.\ als hypothetisch gekennzeichnet.
Der Sprecher scheint nicht zu erwarten, dass das Ereignis unbedingt eintritt.
Ganz ähnlich ist es in (\ref{ex:vflex5549}):
Der Indikativ in (\ref{ex:vflex5549a}) ist nur in einer Situation angemessen, in der tatsächlich kein Schnee liegt und der Sprecher daher die Ferien faktisch als nur begrenzt schön einstuft.
Der Konjunktiv II in (\ref{ex:vflex5549b}) ist hingegen eine Mutmaßung darüber, wie die Ferien wären, wenn kein Schnee läge, obwohl in der Äußerungssituation Schnee liegt.
In (\ref{ex:vflex55491}) schließlich markiert der Konjunktiv den Wunsch, dass eine Sachlage anders hätte sein mögen, als sie es in der Wirklichkeit war.
Wir fassen alle Verwendungen des Konjunktivs II wegen ihres nicht-faktischen Charakters als \textit{irrealen Konjunktiv} zusammen.

Der Modus ist also in Teilen eine semantisch motivierte Kategorie (vor allem beim irrealen Konjunktiv).
Allerdings gibt es auch grammatisch motivierte Verwendungen des Konjunktivs (das Vorkommen bestimmter Verben des Zitierens).

\Definition{Modus}{
\label{def:modus}
Der Modus ist eine grammatische Kategorie, die am Verb realisiert wird.
Der Sprecher markiert durch die verschiedenen Modi unterschiedliche Grade der Faktizität, die er der Satzaussage zuschreibt.
}

\Satz{Indikativ und Konjunktiv}{
Der Indikativ markiert Satzinhalte als faktisch.
Der quotative Konjunktiv markiert Satzinhalte als indirektes Zitat und wird prototypisch (aber nicht nur) in der Umgebung von Verben des Sagens, der Einschätzung oder des Fragens verwendet.
Der irreale Konjunktiv markiert den Satzinhalt als nicht-faktisch bzw.\ hypothetisch.
}

Eine gewisse Austauschbarkeit von Indikativ wie in (\ref{ex:vflex2208a}), quotativem Konjunktiv wie in (\ref{ex:vflex2208b}) und irrealem Konjunktiv wie in (\ref{ex:vflex2208c}) ergibt sich in Kontexten, die eigentlich typisch für den quotativen Konjunktiv sind.

\begin{exe}
  \ex\label{ex:vflex2208}
  \begin{xlist}
    \ex{\label{ex:vflex2208a} Frida behauptet, dass der Mond am Himmel steht.}
    \ex{\label{ex:vflex2208b} Frida behauptet, dass der Mond am Himmel stehe.}
    \ex{\label{ex:vflex2208c} Frida behauptet, dass der Mond am Himmel stände.}
  \end{xlist}
\end{exe}

Die Effekte der verschiedenen Formen genau zu trennen, ist schwierig.
Es scheint, als mache sich der Sprecher des Satzes (also nicht Frida) die Aussage Fridas durch die verschiedenen Modi unterschiedlich stark zu eigen.
Vor allem mit dem irrealen Konjunktiv in (\ref{ex:vflex2208c}) kann der Sprecher andeuten, dass er sich die Äußerung Fridas ausdrücklich nicht zu eigen macht, ohne sie aber notwendigerweise zu verneinen.

\subsection{Finitheit und Infinitheit}

\label{sec:finit}

\index{Finitheit}
\index{Infinitheit}

Die \textit{infiniten} Verbformen stehen vollständig neben den für Person, Numerus, Tempus und Modus markierten finiten Verbformen. 
Die zwei einfachen infiniten Formen im Deutschen sind der \textit{Infinitiv} (\ref{ex:vflex7771}) und das \textit{Partizip} (\ref{ex:vflex7772}).%
\footnote{Unser \textit{Partizip} heißt in anderen Texten auch \textit{Partizip Perfekt} oder gar \textit{Partizip Perfekt Passiv}.
Es wird oft dem \textit{Partizip Präsens} (ggf.\ mit dem Zusatz \textit{Aktiv}) gegenübergestellt (\textit{lachend}, \textit{stehend} usw.), das wir nicht als Partizip behandeln.
Zu den Gründen dafür und der Bildung der infiniten Formen s.\ Abschnitt~\ref{sec:infinflex}.}

\begin{exe}
  \ex\label{ex:vflex7771}
  \begin{xlist}
    \ex{\label{ex:vflex7771a} Frida möchte Kuchen essen.}
    \ex{\label{ex:vflex7771b} Frida aß Kuchen, um satt zu werden.}
    \ex{\label{ex:vflex7771c} Maßvoll Kuchen zu essen macht glücklich.}
  \end{xlist}
  \ex\label{ex:vflex7772}
  \begin{xlist}
    \ex{\label{ex:vflex7772a} Der Kuchen wird gegessen.}
    \ex{\label{ex:vflex7772b} Frida hat den Kuchen gegessen.}
  \end{xlist}
\end{exe}

\index{Kongruenz!Subjekt--Verb--}
Finite Formen haben gemäß Definition~\ref{def:finitheit} auf S.~\pageref{def:finitheit} ein \textsc{Tempus}-Merkmal, infinite haben keins.
Zusätzlich sind die infiniten Formen von \textit{essen} in (\ref{ex:vflex7771a}), (\ref{ex:vflex7771c}) und (\ref{ex:vflex7772}) -- wie schon in Abschnitt~\ref{sec:deutemp} bei analytischen Tempora gezeigt wurde -- nicht von der Subjekt-Verb-Kongruenz betroffen und tragen deshalb keine Person\slash Numerus-Markierung.
Da sich Infinitive mit \textit{zu} deutlich anders verhalten als reine Infinitive, und weil die Partikel \textit{zu} nicht als eigenständige Wortform analysiert werden muss (weil sie syntaktisch untrennbar mit dem Infinitiv verbunden ist), wird der \textit{zu-Infinitiv} wie in (\ref{ex:vflex7771c}) als dritte infinite Form angenommen.

\begin{table}[!htbp]
  \centering
  \begin{tabular}{lll}
    \lsptoprule
    \textbf{Status} & \textbf{Name} & \textbf{Beispiel} \\
    \midrule
    1 & Infinitiv & essen \\
    2 & \textit{zu} + Infinitiv & zu essen \\
    3 & Partizip & gegessen \\
    \lspbottomrule
  \end{tabular}
  \caption{Die Status des infiniten Verbs}
  \label{tab:status}
\end{table}

\index{Status}
Da Finitheit gemäß der Definition schlicht die Anwesenheit eines \textsc{Tempus}-\-Merk\-mals bedeutet, benötigen wir zu ihrer grundsätzlichen Auszeichnung kein eigenes Merkmal.
Die drei verschiedenen infiniten Verbformen werden traditionell auch als die \textit{Status} des infiniten Verbs bezeichnet.
Dabei gilt Tabelle~\ref{tab:status}.
Um die drei Status unterscheiden zu können, muss die Merkmalsausstattung der Verben um ein Merkmal angereichert werden, dass die spezifische infinite Form kodiert, s.\ (\ref{ex:vflex1001}).

\begin{exe}
  \ex{\label{ex:vflex1001} \textsc{Status}: \textit{1}, \textit{2}, \textit{3}}
\end{exe}

Die Motivation der Unterscheidung von Finitheit und Infinitheit ist vor allem in der besonderen Art zu suchen, auf die das Deutsche bestimmte semantische Kategorien wie Modalität, Tempus (s.\ Abschnitt~\ref{sec:deutemp}) oder Genus verbi (s.\ Abschnitt~\ref{sec:genusverb}) realisiert.
Während in anderen Sprachen \zB der Ausdruck von \textit{Möglichkeit} oder \textit{Notwendigkeit} über spezielle morphologische Verbformen erfolgt, benutzt das Deutsche dafür ein Hilfsverb im weiteren Sinn (\textit{können}, \textit{dürfen}, \textit{müssen} usw.), von dem wiederum das lexikalische Verb (oder Vollverb) syntaktisch abhängt.%
\footnote{Zur Definition der Vollverben usw.\ s.\ Abschnitt~\ref{sec:vunterklass}.}
Es ergeben sich Gefüge wie \textit{laufen können} oder \textit{gehen dürfen}, in denen nur \textit{können} und \textit{dürfen} finit sein können.

Da Tempus- und Kongruenzmerkmale dabei nur einmal realisiert werden, entsteht ein Bedarf an Verbformen, die gerade keine Tempus- und Kongruenzmerkmale kodieren.
Dies sind die infiniten Formen.
Dass es davon drei gibt, die jeweils von unterschiedlichen Hilfsverben im weiteren Sinn regiert werden, ist mehr oder weniger Zufall.
Grammatisch gesehen handelt es sich bei dem Merkmal \textsc{Status} also um ein Rektionsmerkmal.

\label{abs:infinwortbild}Man kann sich nun fragen, ob der Übergang vom finiten zum infiniten Verb wirklich Flexion ist, oder ob im Sinne von Definition~\ref{def:wortbild} auf S.~\pageref{def:wortbild} (Wortbildung) nicht besser von Wortbildung zu sprechen wäre.
Im Grunde wird dies hier vertreten, denn es fallen Merkmale (\textsc{Tempus}, \textsc{Person} usw.) weg, und es kommt mindestens eins (\textsc{Status}) hinzu.
Die Bedingung für Wortbildung ist damit eigentlich schon erfüllt.
Außerdem ändert sich das syntaktische Verhalten vollständig, denn die Verben können nicht mehr ohne ein Hilfsverb satzbildend eingesetzt werden.
Zudem ändert sich in manchen Fällen die Valenz (s.\ Abschnitt~\ref{sec:passiv}), die man gewöhnlicherweise als statisch betrachten würde.%
\footnote{Es kommt das auf S.~\pageref{abs:praefixundflex} angeführte Argument bezüglich der Interaktion von Verbpräfixen und Verbpartikeln mit der Partizipbildung hinzu.}
Die an sich sinnvolle Auffassung der Bildung infiniter Formen als Wortbildung hat den Preis, dass die Infinitivformen einer Wortklasse angehören, in der sich sonst keine Wörter befinden.
Das ist aber kein starkes Gegenargument. 
Trotzdem verfolgen wir aus Gründen der einfacheren Darstellung im weiteren Verlauf diese Idee nicht und behandeln die infiniten Formen als Flexionsformen des Verbs.

\subsection{Genus verbi}

\label{sec:genusverb}

\index{Passiv}

Die \textit{Genera verbi} (oder \textit{Diathesen}), die man traditionell für das Deutsche annimmt, sind \textit{Aktiv} (\ref{ex:vflex8336}) und \textit{Passiv} (\ref{ex:vflex8337}).

\begin{exe}
  \ex{\label{ex:vflex8336} Frida isst den Kuchen.}
  \ex{\label{ex:vflex8337} Der Kuchen wird (von Frida) gegessen.}
\end{exe}

Das Passiv wird mit dem Hilfsverb \textit{werden} und dem Partizip des Verbs analytisch gebildet.
Wie schon bei den analytischen Tempora ist es nicht sinnvoll, bei der Bildung des Passivs von Flexion (oder Konjugation) zu sprechen.
Auch hier basiert der vor allem früher oft übliche inkorrekte Sprachgebrauch auf der Lateingrammatik, in der Passivformen tatsächlich synthetisch gebildet werden, vgl.\ (\ref{ex:vflex2349}).

\begin{exe}
  \ex\label{ex:vflex2349}\gll mittitur infestos alter speculator in hostes [\ldots]\\
  {wird geschickt} {feindliche} {der eine} Späher {in} {Feinde}\\
  \glt Der eine Späher wird mitten unter die (feindlichen) Feinde geschickt. (Ovid, Amores, 1.9, 17)
\end{exe}

Es ist charakteristisch, dass das Subjekt des Aktivs im entsprechenden Passiv ganz weggelassen wird oder mit der Präposition \textit{von} (\zB \textit{von Frida}) formuliert wird.
Das Akkusativobjekt des Aktivs (\textit{den Kuchen} in (\ref{ex:vflex8336})) wird zum nominativischen Subjekt (\textit{der Kuchen} in (\ref{ex:vflex8337})) des Passivs.
Da Passivbildungen im Deutschen nur analytisch sind, benötigen wir kein Merkmal \textsc{GenusVerbi} und verschieben die weitere Besprechung des Passivs in die Syntax (Abschnitt~\ref{sec:passiv}).

\subsection{Zusammenfassung}

Wir deklarieren jetzt abschließend die Merkmale (in verkürzter Schreibweise), die alle Verben haben, und fassen die wichtigen Ergebnisse zusammen.
Zunächst hier die Merkmale der finiten Verben.

\begin{exe}
  \ex{\label{ex:verb828291} \textsc{Num}: \textit{sg}, \textit{pl}}
  \ex{\label{ex:verb828292} \textsc{Per}: \textit{1}, \textit{2}, \textit{3}}
  \ex{\label{ex:verb828293} \textsc{Temp}: \textit{präs}, \textit{prät}}
  \ex{\label{ex:verb828294} \textsc{Mod}: \textit{ind}, \textit{konj}}
\end{exe}

\textsc{Numerus} und \textsc{Person} sind beim Nomen semantisch bzw.\ pragmatisch motiviert und beim Verb reine Kongruenzmerkmale, die im Rahmen der Subjekt-Verb-Kongruenz gesetzt werden.
\textsc{Tempus} und \textsc{Modus} sind semantisch motiviert und steuern Information über die Ereigniszeit und (im weitesten Sinn) das Maß an Hypothetizität bei, das der Sprecher dem Satz zuweist.
Bei den infiniten Verben entfallen sämtliche Merkmale aus (\ref{ex:verb828291})--(\ref{ex:verb828294}), und es tritt das Merkmal aus (\ref{ex:verb828295}) hinzu.

\begin{exe}
  \ex{\label{ex:verb828295} \textsc{Stat}: \textit{1}, \textit{2}, \textit{3}}
\end{exe}

\textsc{Status} ist rein strukturell, und die Status-Formen haben nur die Funktion, bestimmte Rektionsanforderungen anderer Verben (\zB Hilfsverben) zu erfüllen.
Genau deswegen ist es auch nicht zielführend, das Partizip \textit{Partizip Perfekt Passiv} o.\,ä.\ zu nennen.
Der Rest dieses Kapitels ist jetzt der Frage gewidmet, durch welche formalen Mittel diese Merkmale eindeutig oder nicht eindeutig an den Verben markiert werden.
Einerseits lassen sich dabei die Merkmale nicht so gut einzelnen Suffixen zuweisen wie beim Nomen, andererseits müssen zunächst die Unterklassen der Verben genauer definiert werden.


\Zusammenfassung{
  \textit{Person} und \textit{Numerus} sind beim Verb reine \textit{Kongruenzkategorien}.
  \textit{Einfache Tempora} kodieren eine Relation zwischen \textit{Ereigniszeit} und \textit{Sprechzeit}, \textit{komplexe Tempora} kodieren eine Relation zwischen diesen beiden und einem zusätzlichen \textit{Referenzzeitpunkt}.
  Das \textit{Präsens} im Deutschen hat keine spezifische Tempusbedeutung.
}


\section{Flexion}

\label{sec:vvflex}

In diesem Abschnitt wird zunächst der Unterschied zwischen \textit{Vollverben} und anderen Verben definiert, dann werden die Flexionsklassen der Verben eingeführt (Abschnitt~\ref{sec:vunterklass}).
Für die zwei wichtigen Flexionsklassen der schwachen und starken Verben wird dann zunächst die Tempus- und Person-Flexion im Indikativ besprochen (Abschnitt~\ref{sec:tempusnumerusperson}).
Davon ausgehend kann der Konjunktiv einheitlich für beide Flexionsklassen diskutiert werden (Abschnitt~\ref{sec:konjunktivflexion}).
Ebenso einheitlich werden dann die infiniten Formen (Abschnitt~\ref{sec:infinflex}) und der Imperativ (Abschnitt~\ref{sec:impflex}) behandelt.
Das Kapitel schließt mit einer Darstellung einiger kleiner Flexionsklassen und der wenigen echten unregelmäßigen Verben (Abschnitt~\ref{sec:kleineverbklassen}).

\subsection{Unterklassen}

\label{sec:vunterklass}

\index{Verb!Flexionsklassen}

Verglichen mit den Substantiven (Abschnitt~\ref{sec:subst}) sind die Verben flexionsseitig einfach untergliedert.
Man muss auf der Formseite nur zwischen \textit{starken} Verben wie \textit{laufen} und \textit{schwachen} Verben wie \textit{kaufen} sowie einigen kleinen Klassen wie den \textit{präteritalpräsentischen} Verben wie \textit{können} oder \textit{dürfen} (meistens mit den \textit{Modalverben} gleichgesetzt) und einigen \textit{unregelmäßigen} Verben wie \textit{sein} unterscheiden.
\index{Verb!Voll--}
Die Unterscheidung zwischen starken und schwachen Verben betrifft die sogenannten \textit{Vollverben}, weshalb zunächst der Unterschied zwischen Vollverben und anderen Klassen von Verben gemacht werden soll.
Dazu betrachten wir (\ref{ex:vflex2222}).
Die Klassenunterschiede sind teilweise funktional und semantisch.
Wir versuchen aber, die Klassen möglichst morphologisch bzw.\ formal zu erfassen.

\begin{exe}
  \ex\label{ex:vflex2222}
  \begin{xlist}
    \ex{\label{ex:vflex2222a} Frida isst den Marmorkuchen.}
    \ex{\label{ex:vflex2222b} Frida hat den Marmorkuchen gegessen.}
    \ex{\label{ex:vflex2222c} Der Marmorkuchen wird gegessen.}
    \ex{\label{ex:vflex2222d} Frida soll den Marmorkuchen essen.}
    \ex{\label{ex:vflex2222e} Dies hier ist der leckere Marmorkuchen.}
    \ex{\label{ex:vflex2222f} Der Marmorkuchen wird lecker.}
  \end{xlist}
\end{exe}

In (\ref{ex:vflex2222a}) liegt ein \textit{Vollverb} (\textit{isst}) vor.
Das Vollverb ist prototypisch dadurch ausgezeichnet, dass es eine nominale Valenz haben kann, dass seine Valenz also durch NPs gesättigt werden kann.
In (\ref{ex:vflex2222a}) sind die Valenznehmer \textit{Frida} und \textit{den Marmorkuchen}.
Es verlangt typischerweise nicht nach Ergänzungen in Form eines reinen Infinitiv oder eines Partizips.%
\footnote{In Kapitel~\ref{sec:relationenpraedikate} werden auch die Fälle besprochen, in denen Verben Ergänzungen in Form eines zu-Infinitivs nehmen.
Diese sind besser als Vollverben zu beschreiben.}
Außerdem ist die Klasse der Vollverben offen, es gibt also eine beliebig große Zahl von Vollverben, wobei jedes Verb eine eigene Semantik mitbringt.

\index{Hilfsverb}

Die Klasse der \textit{Nicht-Vollverben} ist hingegen geschlossen, es kommen also nicht ohne weiteres neue hinzu.
Die Nicht-Vollverben sind sämtlich mehr oder weniger als grammatische Hilfswörter bzw.\ Funktionswörter zu betrachten, die einen schwachen lexikalisch-semantischen Beitrag haben.
Man kann die Nicht-Vollverben weiter abgrenzen und unterklassifizieren.
Zunächst schauen wir auf (\ref{ex:vflex2222b}) und (\ref{ex:vflex2222c}).
In den Beispielen wird einerseits ein Perfekt (\textit{hat gegessen}), andererseits ein Passiv (\textit{wird gegessen}) analytisch gebildet (s.\ Abschnitte~\ref{sec:deutemp} und \ref{sec:genusverb}), wobei ein für diese Bildungen typisches Verb (\textit{sein}, \textit{haben}, \textit{werden}) benutzt wird.
Diese Verben, deren Funktion es ist, analytische Tempus- und Passivformen zu bilden, sind die klassischen \textit{Hilfsverben} (die auch \textit{Auxiliare} genannt werden), und sie regieren typischerweise den reinen Infinitiv (\textit{wird essen}, Futur) oder das Partizip (\textit{wird gegessen}, Passiv).

\index{Modalverb}

In (\ref{ex:vflex2222d}) ist ein sogenanntes \textit{Modalverb} bebeispielt.
Modalverben bilden eine geschlossene Gruppe (\textit{dürfen}, \textit{können}, \textit{mögen}, \textit{müssen}, \textit{sollen}, \textit{wollen}), und sind morphologisch alle \textit{Präteritalpräsentien}, die in Abschnitt~\ref{sec:kleineverbklassen} besprochen werden.
Morphosyntaktisch gesehen regieren sie immer einen reinen Infinitiv (ohne \textit{zu}) und verhalten sich auch syntaktisch besonders (vgl.\ Abschnitt~\ref{sec:modalverbkonstruktionen}).
Außerdem haben sie ähnliche semantische Eigenschaften, die ihnen auch die Benennung als Modalverben eingebracht haben, auf die hier aber aus Platzgründen nicht eingegangen werden kann.

\index{Kopula}

In (\ref{ex:vflex2222e}) und (\ref{ex:vflex2222f}) werden schließlich \textit{sein} und \textit{werden} als typische \textit{Kopulaverben} bebeispielt.
Das dritte eindeutige Kopulaverb ist \textit{bleiben}.
Kopulaverben verbinden sich im prototypischen Fall mit NPs oder Adjektiven (aber auch präpositionalen Gruppen, vgl.\ Abschnitt~\ref{sec:kopulakonstruktionen}), um mit diesen zusammen die Funktion im Satz einzunehmen, die sonst ein einfaches Vollverb einnimmt, und die man traditionell als \textit{Prädikat} bezeichnet.
Daher heißen die sich mit Kopulaverben verbindenden Einheiten auch \textit{Prädikatsnomen} oder \textit{Prädikatsadjektiv} usw.
Eine ausführlichere Diskussion der syntaktischen Konstruktionen mit vielen Nicht-Vollverben wird in Kapitel~\ref{sec:relationenpraedikate} geleistet.
Hier soll die Subklassifikation der Verben nur das Reden über die Flexionsbesonderheiten der verschiedenen Subklassen von Verben erleichtern.
Wir schließen mit Tabelle~\ref{tab:vklass}, die zur Orientierung die hier diskutierten traditionellen Klassen anhand ihrer morphosyntaktischen Merkmale zusammenfasst.
Einige Verben fallen dabei in mehrere Klassen (\zB \textit{sein} oder \textit{werden}), zeigen in den verschiedenen Klassen dann aber auch ein anderes grammatisches Verhalten.
Es gibt natürlich auch Verben, die sowohl als Voll- als auch als Hilfsverb fungieren (wie \textit{haben}).

\begin{table}[!htbp]
  \resizebox{\textwidth}{!}{
    \begin{tabular}{llll}
      \lsptoprule
      \textbf{Klasse} & \textbf{Morphologie} & \textbf{typische Valenz\slash Rektion} & \textbf{Beispiele} \\
      \midrule
      Voll-- & stark\slash schwach & NPs mit Kasus (oder \textit{zu}-Infinitiv) & \textit{laufen}, \textit{kaufen}\\
      Hilfs-- & unregelmäßig\slash stark & Verb im reinen Infinitiv\slash Partizip & \textit{haben}, \textit{werden}\\
      Modal-- & präteritalpräsentisch & Verb im reinen Infinitiv & \textit{können}, \textit{dürfen} \\
      Kopula-- & unregelmäßig\slash stark & NPs (Nom)\slash Präpositionen\slash Adjektive & \textit{sein}, \textit{bleiben} \\
      \lspbottomrule
    \end{tabular}
  }
  \caption{Traditionelle Verbklassen und ihre Eigenschaften}
  \label{tab:vklass}
\end{table}

Vor allem für die Vollverben gilt nun die Unterscheidung nach \textit{Stärke}.
Dabei handelt es sich schlicht um zwei Flexionsklassen ohne funktionale oder semantische Unterschiede.
In Tabelle~\ref{tab:stschwvbsp} sind die Formen der starken Verben \textit{heben}, \textit{springen}, \textit{brechen} und des schwachen Verbs \textit{lachen} aufgeführt, die den Unterschied illustrieren.

\begin{table}[!htbp]
  \resizebox{\textwidth}{!}{
    \begin{tabular}{llllll}
      \lsptoprule
       & \textbf{2-stufig} & \textbf{3-stufig} & \textbf{U3-stufig} & \textbf{4-stufig} & \textbf{schwach} \\
      \midrule
      \textbf{1 Pers Präs} & heb-e & spring-e & lauf-e & brech-e & lach-e \\
      \textbf{2 Pers Präs} & heb-st & spring-st & läuf-st & brich-st & lach-st \\
      \textbf{1 Pers Prät} & hob & sprang & lief & brach & lach-te \\
      \textbf{Partizip} & ge-hob-en & ge-sprung-en & ge-lauf-en & ge-broch-en & ge-lach-t \\
      \lspbottomrule
    \end{tabular}
  }
  \caption{Beispielformen starker und schwacher Verben}
  \label{tab:stschwvbsp}
\end{table}

\index{Ablaut}
Starke Verben haben das (schon in Abschnitt~\ref{sec:umablaut} besprochene) Merkmal des Ablauts.
Die sogenannten \textit{Ablautstufen} sind verschiedene Stämme des Verbs, die in bestimmten Formen des Paradigmas (vor allem im Präteritum und Partizip) verwendet werden.
Historisch gehen die Vokalveränderungen der starken Verben auf verschiedene Ursachen zurück.
Die Veränderung \textit{spreche} zu \textit{sprichst} würde man \zB in der historischen Linguistik nicht als \textit{Ablaut} bezeichnen, weil sie anderen Ursprungs ist als die in \textit{spreche} und \textit{sprach}.
Da synchron -- also für das gegenwärtige Deutsche -- diese Phänomene zusammengefasst werden können, wird hier von den \textit{Vokalstufen} der starken Verben gesprochen.

Minimal sind die starken Verben zweistufig, wobei dann entweder Präsens und Partizip oder Präteritum und Partizip dieselbe Stufe haben (\textit{rufe}--\textit{rief}--\textit{gerufen} oder \textit{hebe}--\textit{hob}--\textit{gehoben}), niemals aber Präsens und Präteritum.
Bei dreistufigen Verben sind Präsens, Präteritum und Partizip alle verschieden voneinander (\textit{springe}--\textit{sprang}--\textit{gesprungen}).
Bei vierstufigen Verben gibt es eine zusätzliche Vokalstufe in der zweiten und dritten Person Singular Präsens Indikativ (\textit{breche}--\textit{brichst}--\textit{brach}--\textit{gebrochen}).
Eine besondere Klasse ist die der dreistufigen Verben, bei denen die zweite Stufe (in der zweiten und dritten Person Singular Präsens) umgelautet ist (\textit{laufe}--\textit{läufst}--\textit{lief}--\textit{gelaufen}).
Diese Umlautstufe sollte wegen Besonderheiten der Bildung der Imperative gesondert behandelt werden (vgl.\ dazu Abschnitt~\ref{sec:impflex}), und wir nennen die entsprechenden Verben \textit{U3-stufig}.
Demgegenüber haben schwache Verben nur genau einen Stamm, an den lediglich Affixe angeschlossen werden.

\index{Ablaut!Stufen}
\index{Verb!stark}
\index{Stärke!Verb}
\index{Verb!schwach}

\Satz{Starke und schwache Verben}{
Starke Verben haben mindestens zwei und maximal vier verschiedene durch Vokalstufen unterschiedene Stämme (überwiegend identisch mit den \textit{Ablautstufen}).
Wenn es zwei sind, unterscheiden sich immer Präsens- und Präteritalstamm.
Schwache Verben haben im gesamten Paradigma nur genau einen Stamm.
}

Bezüglich des Sprachgebrauchs soll folgende Regelung gelten:
Starke Verben haben maximal vier verschiedene Vokalstufen, die sich wie in Tabelle~\ref{tab:ablstuf} verteilen.
Auch wenn ein starkes Verb zweistufig oder dreistufig ist, zählen wir terminologisch die Stufen von eins bis vier durch.
Dies erlaubt eine kürzere Sprechweise wie \textit{die zweite Vokalstufe} anstelle von \textit{die Vokalstufe der zweiten und dritten Person Präsens Indikativ}.

\begin{table}[!htbp]
  \resizebox{\textwidth}{!}{
    \begin{tabular}{lcccc}
      \lsptoprule
      & \textbf{Stufe 1} & \textbf{Stufe 2} & \textbf{Stufe 3} & \textbf{Stufe 4} \\
      & (Präsens, außer & (Präsens, nur & \multirow{2}{*}{(Präteritum)} & \multirow{2}{*}{(Partizip)} \\
      & 2/3 Sg Indikativ) & 2/3 Sg Indikativ) && \\
      \midrule
      \textbf{2-stufig} (\textit{heben}) & e & e & o & o \\
      \textbf{3-stufig} (\textit{springen}) & i & i & a & u \\
      \textbf{U3-stufig} (\textit{laufen}) & au & äu & ie & au \\
      \textbf{4-stufig} (\textit{brechen}) & e & i & a & o \\
      \lspbottomrule
    \end{tabular}
  }
  \caption{Vokalstufen an Beispielen}
  \label{tab:ablstuf}
\end{table}

Die Allerweltsverben sind die schwachen Verben.
Verben, die neu in das Lexikon aufgenommen werden, flektieren immer schwach (vgl.\ ältere oder rezente Entlehnungen wie \textit{rasieren}, \textit{goutieren}, \textit{freeclimben}, \textit{emailen}, \textit{twittern}).
Kinder generalisieren in Phasen des Spracherwerbs das produktive Bildungsmuster der schwachen Verben häufig auf alle Verben (*\textit{er gehte}, *\textit{du brechst}).
Während die Klasse der schwachen Verben also eine offene Klasse ist, ist die der starken Verben eine geschlossene Klasse, zu der nie oder fast nie neue Wörter hinzukommen.
Im Gegenteil wechseln Verben sogar historisch typischerweise von der starken zur schwachen Flexion (\textit{du bäckst} zu \textit{du backst} und \textit{ich buk} zu \textit{ich backte}).

\subsection{Tempus, Numerus und Person}

\label{sec:tempusnumerusperson}

\begin{table}[!htbp]
  \centering
  \begin{tabular}{llll}
    \lsptoprule
    \multicolumn{2}{c}{} & \textbf{Präsens} & \textbf{Präteritum} \\
    \midrule
    \multirow{3}{*}{\textbf{Singular}} & \textbf{1} & lach-(e) & lach-te \\
    & \textbf{2} & lach-st & lach-te-st \\
    &\textbf{3} & lach-t & lach-te \\
    \midrule
    \multirow{3}{*}{\textbf{Plural}} & \textbf{1} & lach-en & lach-te-n \\
    & \textbf{2} & lach-t & lach-te-t \\
    & \textbf{3} & lach-en & lach-te-n \\
    \lspbottomrule
  \end{tabular}
  \caption{Indikativ der schwachen Verben}
  \label{tab:schwvind}
\end{table}

\index{Verb!schwach!Flexion}
\index{Präsens}\index{Präteritum}\index{Indikativ}

Wir beginnen mit der Betrachtung der Formen der schwachen Verben im Präsens und Präteritum und schreiten von dort zu den starken Verben voran.
Das vollständige Formenraster des Indikativs der schwachen Verben findet sich in Tabelle~\ref{tab:schwvind}.
Die Markierungsfunktionen der Affixe sind erfrischend klar verteilt.
Das Präteritum der schwachen Verben wird einheitlich durch das Affix \textit{-te} markiert, das den Person\slash Numerus-Endungen vorangeht.%
\footnote{Überwiegend wird das Suffix als \textit{-t} analysiert und das \textit{-e} als Teil des Person\slash Numerus-Suffixes gesehen.
Außerdem wird \textit{-t} manchmal als \textit{Dentalsuffix} bezeichnet, obwohl strenggenommen /\textipa{t}/ ein alveolarer stimmloser Plosiv ist.}
Die die Wortform rechts abschließenden Suffixe im Präsens und Präteritum markieren spezifische \textit{Kombinationen} aus Person und Numerus.
Es gibt also keine getrennten Person- oder Pluralkennzeichen.

Die Endungssätze des Präsens und des Präteritums unterscheiden sich signifikant nur in der ersten und dritten Person Singular:
Die erste Person Singular Präsens ist optional durch Schwa markiert und die dritte Person hat im Präsens ein \textit{-t}.
Im Präteritum sind die erste und dritte Person Singular hingegen prinzipiell endungslos.
In beiden Tempora sind die erste und dritte Person Plural nie unterscheidbar, und im Präteritum sind wegen der Endungslosigkeit auch die erste und dritte Person Singular nicht unterscheidbar.

Im Vergleich zu den schwachen Verben ergeben sich kaum Unterschiede bei den Endungssätzen der starken Verben, s.\ Tabelle~\ref{tab:stvind}.

\begin{table}[!htbp]
  \centering
  \begin{tabular}{llll}
    \lsptoprule
    \multicolumn{2}{c}{} & \textbf{Präsens} & \textbf{Präteritum} \\
    \midrule
    \multirow{3}{*}{\textbf{Singular}} & \textbf{1} & brech-(e) & brach \\
    & \textbf{2} & brich-st & brach-st \\
    & \textbf{3} & brich-t & brach \\
    \midrule
    \multirow{3}{*}{\textbf{Plural}} & \textbf{1} & brech-en & brach-en \\
    & \textbf{2} & brech-t & brach-t \\
    & \textbf{3} & brech-en & brach-en \\
    \lspbottomrule
  \end{tabular}
  \caption{Indikativ der starken Verben}
  \label{tab:stvind}
\end{table}

\index{Verb!stark!Flexion}
\index{Präsens}\index{Präteritum}\index{Indikativ}

Das \textit{-te} als Präteritalmarkierung ist hier nicht vorhanden, und stattdessen ist die entsprechende Vokalstufe bei den starken Verben das Charakteristikum des Präteritums.
Die Person\slash Numerus-Suffixe unterscheiden sich nicht von denen der schwachen Verben, lediglich die erste und dritte Person Plural Präteritum haben \textit{-en} statt wie bei den schwachen Verben \textit{-n}.
Dieser Unterschied wird im Folgenden systematisch erklärt.

Man sieht sofort, dass sich die Suffixreihen stark gleichen.
Man kann die Darstellung weiter reduzieren, wenn man Tabelle~\ref{tab:indmarkerredux} annimmt.

\index{Numerus!Verb}
\index{Person!Verb}
\index{Verb!Person-Numerus-Suffixe}

\begin{table}[!htbp]
  \centering
  \begin{tabular}{llcc}
    \lsptoprule
    \multicolumn{2}{c}{} & \textbf{PN1} & \textbf{PN2} \\
    \midrule
    \multirow{3}{*}{\textbf{Singular}} & \textbf{1} & -(e) & \Dim \\
      & \textbf{2} & \multicolumn{2}{c}{-st} \\
      & \textbf{3} & -t & \Dim \\
    \midrule
    \multirow{2}{*}{\textbf{Plural}} & \textbf{1/3} & \multicolumn{2}{c}{-en} \\
      & \textbf{2} & \multicolumn{2}{c}{-t} \\
    \lspbottomrule
  \end{tabular}
  \caption{Reduzierte Person/Numerus-Suffixreihen}
  \label{tab:indmarkerredux}
\end{table}

Die verbalen Person\slash Numerus-Suffixe PN1 werden im Indikativ für das Präsens, die Suffixe PN2 für das Präteritum und -- wie sich zeigen wird -- alle anderen Formen verwendet.
Die unterschiedliche Schwa-Hal\-tig\-keit in der Endung bei \textit{lach-te-n} und \textit{brach-en} und ähnlichen Formen lässt sich phonotaktisch als Löschung aufeinanderfolgender Schwas erklären.
Wie schon in vielen Fällen in der Nominalflexion wird die Variante ohne Schwa, hier also \textit{-n}, gewählt, wenn ein Schwa vorausgeht.
Dies ist bei dem vorangehenden Präteritalsuffix \textit{-te} der Fall.%
\footnote{Es ist freilich sinnlos, zu fragen, welches der beiden Schwas in der zugrundeliegenden Form \textit{sie lach-te-en} getilgt wird.
Man könnte also \textit{lach-t-en} oder \textit{lach-te-n} analysieren.
Vgl.\ dazu weiter Abschnitt~\ref{sec:zusammenfassungverbflexion}.}

Diese reduzierte Darstellung ist ausgesprochen nützlich:
Erstens verdeutlicht sie den hohen Grad der Einheitlichkeit der Person\slash Numerus-Endungen zwischen starken und schwachen Verben auf der einen Seite und Präsens und Präteritum auf der anderen Seite.
Zweitens ist sie aber auch die ideale Basis zur Beschreibung der Konjunktivformen, die im nächsten Abschnitt folgt.

\subsection{Konjunktivflexion}

\label{sec:konjunktivflexion}

\index{Konjunktiv!Form vs.\ Funktion}

Bei der Analyse der Konjunktivformen stößt man auf Schwierigkeiten, die genauen Markierungsfunktionen der Affixe und Stammbildungen zu bestimmen.
Der Konjunktiv scheint formal eine Präsensform (quotativer Konjunktiv) und eine Präteritalform (irrealer Konjunktiv) zu haben.
Die Formen sind allerdings kaum temporal interpretierbar, sondern haben vielmehr die in Abschnitt~\ref{sec:modus} beschriebenen Funktionen von Quotativ und Irrealis.

Der fehlende Tempuseffekt beim Konjunktiv zeigt sich \zB daran, dass mit dem irrealen Konjunktiv (also formal dem Konjunktiv Präteritum) ein Bezug auf Zukünftiges ohne weiteres möglich ist (\ref{ex:vflex8881a}).
Mit dem Indikativ Präteritum geht das nicht (\ref{ex:vflex8881b}).
Genauso tritt der quotative Konjunktiv (also der formale Konjunktiv Präsens) in Kontexten auf, in denen ein klarer Vergangenheitsbezug vorliegt (\ref{ex:vflex8881c}), ohne dass etwa auf den irrealen Konjunktiv (also Konjunktiv Präteritum) ausgewichen würde.

\begin{exe}
  \ex\label{ex:vflex8881}
  \begin{xlist}
    \ex[]{\label{ex:vflex8881a} Falls Frida nächste Woche lachte, würde ich mich freuen.}
    \ex[*]{\label{ex:vflex8881b} Frida lachte nächste Woche.}
    \ex[]{\label{ex:vflex8881c} Letzte Woche dachte ich, der Ast breche unter der Schneelast ab.}
  \end{xlist}
\end{exe}

Wegen der noch genau zu beschreibenden Parallelen der Formenbildung zwischen Indikativ Präsens und quotativem Konjunktiv auf der einen Seite und Indikativ Präteritum und irrealem Konjunktiv auf der anderen Seite ist es trotzdem sinnvoll, den quotativen Konjunktiv formal als \textit{Konjunktiv Präsens} und den irrealen Konjunktiv formal als \textit{Konjunktiv Präteritum} zu analysieren.

\index{Konjunktiv!Flexion}

Für den Konjunktiv kommen durchweg die Endungen PN2 zum Einsatz.
Außerdem tritt zwischen den Stamm und die Endungen, die die Kongruenzmerkmale anzeigen, immer das Suffix \textit{-e}.%
\footnote{Das \textit{-e} wird nicht von allen Grammatikern als selbständiges Suffix analysiert.
Ein anderer Erklärungsansatz bezieht sich auf die Anforderung, dass Konjunktivformen immer zweisilbig sein müssen, wozu das Schwa dann quasi als Hilfsvokal herangezogen wird.}
Die wichtige Frage beim Konjunktiv ist, welcher Stamm verwendet wird, und welche Markierungen zwischen Stamm und \textit{-e} eingeschoben werden.
Wir beginnen mit den Formen des Konjunktivs der schwachen Verben, der in Tabelle~\ref{tab:schwkonj} bebeispielt ist.

\begin{table}[!htbp]
  \centering
  \begin{tabular}{llll}
    \lsptoprule
    \multicolumn{2}{c}{} & \textbf{Präsens} & \textbf{Präteritum} \\
    \midrule
    \multirow{3}{*}{\textbf{Singular}} & \textbf{1} & lach-e & lach-t-e \\
    & \textbf{2} & lach-e-st & lach-t-e-st \\
    & \textbf{3} & lach-e & lach-t-e \\
    \midrule
    \multirow{3}{*}{\textbf{Plural}} & \textbf{1} & lach-e-n & lach-t-e-n \\
    & \textbf{2} & lach-e-t & lach-t-e-t \\
    & \textbf{3} & lach-e-n & lach-t-e-n \\
    \lspbottomrule
  \end{tabular}
  \caption{Konjunktiv der schwachen Verben}
  \label{tab:schwkonj}
\end{table}

\index{Verb!schwach!Flexion}
\index{Präsens}\index{Präteritum}\index{Konjunktiv}

Der Konjunktiv Präsens basiert auf dem normalen (einzigen) Stamm, an den das Konjunktiv-Suffix \textit{-e} angefügt wird.
An dieses \textit{-e} treten die Endungen PN2, die erste und dritte Person Singular sind also endungslos.
Im Konjunktiv Präteritum wird das Präteritalsuffix \textit{-te} vor das Konjunktiv-Suffix \textit{-e} eingefügt, und es folgen wieder die Endungen PN2.
Das Schwa in \textit{-te} muss nun im Zuge der Schwa-Reduktion gelöscht werden.
Oberflächlich ist damit der Indikativ Präteritum nicht vom Konjunktiv Präteritum unterscheidbar.
Vermutlich deswegen setzt sich für die Formen des Konjunktiv Präteritums bei den schwachen Verben überwiegend die \textit{würde}-Paraphrase durch, vgl.\ (\ref{ex:vflex8833}).

\begin{exe}
  \ex\label{ex:vflex8833}
  \begin{xlist}
    \ex{\label{ex:vflex8833a} Wenn sie lachte, schenkte ich ihr auch ein Lachen.}
    \ex{\label{ex:vflex8833b} Wenn sie lachen würde, würde ich ihr auch ein Lachen schenken.}
  \end{xlist}
\end{exe}

Satz (\ref{ex:vflex8833a}) ist nicht klar als irrealer Konjunktiv erkennbar, sondern sieht vielmehr wie ein Indikativ Präteritum aus, was zu einer typischen Lesart im Sinne von \textit{immer wenn sie }(\textit{früher}) {lachte} führt.
Die irreale Lesart wird in (\ref{ex:vflex8833b}) mit \textit{würde} sichergestellt.
Es handelt sich bei der \textit{würde}-Paraphrase also in keiner Weise um schlechten Stil, sondern um eine wichtige Strategie, die eindeutige Markierung der Kategorie des Irrealis im Deutschen für die größte Klasse der Verben (die schwachen) zu erhalten.

Auch bei den starken Verben werden die Konjunktive nach dem Präsens- bzw.\ dem Präteritalmuster gebildet.
Hier wird der Präsensstamm (\textit{brech-}) für den Konjunktiv Präsens verwendet.
Der Konjunktiv Präteritum wird vom umgelauteten Präteritalstamm (\textit{bräch}, umgelautet von \textit{brach}) ausgehend gebildet.
An beide Stämme wird das \textit{-e} des Konjunktivs angehängt, auf das die Endungen PN2 folgen.
Die Übersicht ist in Tabelle~\ref{tab:stkonj} gegeben.

\begin{table}[!htbp]
  \centering
  \begin{tabular}{llll}
    \lsptoprule
    \multicolumn{2}{c}{} & \textbf{Präsens} & \textbf{Präteritum} \\
    \midrule
    \multirow{3}{*}{\textbf{Singular}} & \textbf{1} & brech-e & bräch-e \\
    & \textbf{2} & brech-e-st & bräch-e-st \\
    & \textbf{3} & brech-e & bräch-e \\
    \midrule
    \multirow{3}{*}{\textbf{Plural}} & \textbf{1} & brech-e-n & bräch-e-n \\
    & \textbf{2} & brech-e-t & bräch-e-t \\
    & \textbf{3} & brech-e-n & bräch-e-n \\
    \lspbottomrule
  \end{tabular}
  \caption{Konjunktiv der starken Verben}
  \label{tab:stkonj}
\end{table}

\index{Präsens}\index{Präteritum}\index{Konjunktiv}
\index{Verb!stark!Flexion}

\subsection{Zusammenfassung}

\label{sec:zusammenfassungverbflexion}

\index{Verb!Flexion!finit}

Wir haben uns für eine an der Form orientierte Analyse der Markierungsfunktionen entschieden, die den quotativen Konjunktiv als Konjunktiv Präsens und den irrealen Konjunktiv als Konjunktiv Präteritum beschreibt.
Beim Konjunktiv haben wir den Fall, dass die Markierungsfunktion teilweise über mehr als ein einzelnes Affix verteilt ist.
Das \textit{-e} ist zwar ein eindeutiges Konjunktivkennzeichen, aber es kommen fallweise zusätzliche Markierungen hinzu, wie \zB der Umlaut bei den starken Verben.
Hier ist oft erst die gesamte Form mit Stammbildung, Affixen und dem besonderen Endungssatz als quotativer oder irrealer Konjunktiv erkennbar.
Diese Schwierigkeiten bei der Funktionsbestimmung der Affixe gehören zu den Gründen, warum wir uns in Abschnitt~\ref{sec:morphemeallomorphe} gegen die klassische Morphem-Analyse entschieden haben.
Diese würde voraussetzen, dass man bestimmte Allomorphe eines Morphems identifizieren kann, und dass das Morphem eine identifizierbare Funktion hat.

\index{Schwa!Tilgung!Verb}

Eine letzte Bemerkung zu den Analysen, wie sie in den letzten Abschnitten vorkamen, schließt jetzt die Diskussion der finiten Flexion.
In diesen Analysen wurde meist so getan, als sei es völlig klar, welche Schwas getilgt werden, wenn mehrere Suffixe mit Schwa aufeinandertreffen.
Charakteristische Fälle beinhalten das Präteritalsuffix \textit{-te} mit dem Konjunktivzeichen \textit{-e} und folgender PN2-Endung \textit{-en}.
Es ergeben sich Analysen ohne Tilgung wie in (\ref{ex:schwaanalysenverb}).

\begin{exe}
  \ex\label{ex:schwaanalysenverb}
  \begin{xlist}
    \ex{\label{ex:schwaanalysenverba} \textit{lach-te-en} (1.\slash 3.\ Plural Präteritum Indikativ) $\Rightarrow$ \textit{lachten} }
    \ex{\label{ex:schwaanalysenverbb} \textit{lach-te-e} (1.\slash 3.\ Singular Präteritum Konjunktiv) $\Rightarrow$ \textit{lachte}}
    \ex{\label{ex:schwaanalysenverbc} \textit{lach-te-e-en} (1.\slash 3.\ Plural Präteritum Konjunktiv) $\Rightarrow$ \textit{lachten}}
  \end{xlist}
\end{exe}

Im Grunde haben wir es hier mit der Überführung von zugrundeliegenden Formen in Oberflächenformen zu tun, genauso wie es in der Phonologie in Abschnitt~\ref{sec:zugrundeliegendeformenstrukturbedingungen} eingeführt wurde.
Das System der Grammatik setzt eine Reihe von Morphen aneinander, in denen dann ggf.\ phonologische Reduktionsprozesse (also Schwa-Tilgung) stattfinden, um eine korrekte Oberflächenform zu erzeugen.%
\footnote{Ich weise nochmals darauf hin, dass die Darstellung hier gewählt wurde, weil sie eine hohe Beschreibungsökonomie im gegebenen Rahmen ermöglicht.
Es handelt sich nicht um mein theoretisches Glaubensbekenntnis.}
Welche Schwas es sind, die getilgt werden, ist im Prinzip egal.
Es können also Reduktionen für (\ref{ex:schwaanalysenverbc}) wie in (\ref{ex:schwaanalysenredux}) gleichberechtigt durchgeführt werden.

\begin{exe}
  \ex\label{ex:schwaanalysenredux} \textit{lach-te-e-en} $\Rightarrow$
  \begin{xlist}
    \ex{\textit{lach-t\cancel{e}-\cancel{e}-en}}
    \ex{\label{ex:schwaanalysenreduxb}\textit{lach-t\cancel{e}-e-\cancel{e}n}}
    \ex{\textit{lach-te-\cancel{e}-\cancel{e}n}}
  \end{xlist}
\end{exe}

Hier wurde konsequent Möglichkeit (\ref{ex:schwaanalysenreduxb}) gewählt, also \textit{lach-t\cancel{e}-e-\cancel{e}n} bzw.\ \textit{lach-t-e-n} als Analyse der 1.\ und 3.\ Person Präteritum Konjunktiv.
An dieser Analyse kann nämlich nachvollzogen werden, welche Affixe beteiligt sind.
Vor allem ist das \textit{-e} des Konjunktivs in der Analyse noch sichtbar, sonst wären das Präteritum Indikativ \textit{lach-te-n} und das Präteritum Konjunktiv \textit{lach-t-e-n} in der Analyse genausowenig unterscheidbar wie die Oberflächenformen.
Es wurde also aus Darstellungsgründen so getilgt, dass die Analysen möglichst eindeutig bleiben:
Zuerst im Person\slash Numerus-Suffix \textit{-en}, dann im Präteritalsuffix \textit{-te}, aber nie im Konjunktiv-Suffix \textit{-e}.
Falsch ist in dieser Hinsicht aber auch keine der anderen möglichen Analysen, es geht nur um eine transparentere Schreibweise.

\subsection{Infinite Formen}

\label{sec:infinflex}

\index{Verb!Flexion!infinit}

Komplett vom bisher besprochenen finiten Paradigma losgelöst sind die infiniten Formen der Verben.
Die infiniten Formen bilden ein eigenes Paradigma, in dessen Formen lediglich der Wert des Merkmals \textsc{Status} variiert, s.\ Abschnitt~\ref{sec:finit}.
Beim Verb müssen also mindestens zwei Paradigmen unterschieden werden: das finite und das infinite Paradigma.
Einige Beispiele sind in (\ref{ex:vflex5555}) zur Wiederholung angegeben.

\begin{exe}
  \ex\label{ex:vflex5555}
  \begin{xlist}
    \ex{Frida hat gelacht.}
    \ex{Frida wird lachen.}
    \ex{Frida wünscht zu lachen.}
  \end{xlist}
\end{exe}

Die Bildung der infiniten Formen ist gegenüber der Bildung der finiten Formen denkbar einfach.
Beispiele finden sich in Tabelle~\ref{tab:vinf-bsp}.

\index{Infinitiv}
\index{Partizip}
\index{Status}

\begin{table}[!htbp]
  \centering
  \begin{tabular}{lll}
    \lsptoprule
    & \textbf{Infinitiv} & \textbf{Partizip} \\
    \midrule
    \textbf{schwach} & lach-en & ge-lach-t\\
    \textbf{stark} & brech-en & ge-broch-en\\
    \lspbottomrule
  \end{tabular}
  \caption{Beispiele für die Bildung der infiniten Verbformen}
  \label{tab:vinf-bsp}
\end{table}

Der Infinitiv ist durch \textit{-en} am Präsensstamm gekennzeichnet, das Partizip durch das Zirkumfix \textit{ge-~-t} (schwach) bzw.\ \textit{ge-~-en} (stark).
Die starken Verben haben entweder eine eigene Vokalstufe für den Partizipstamm (\textit{ge-broch-en}) oder der Partizipstamm ist identisch mit dem Präsensstamm (\textit{ich geb-e}, \textit{ge-geb-en}) oder er ist identisch mit dem Präteritalstamm (\textit{soff}, \textit{ge-soff-en}).
Als Besonderheit kommt hinzu, dass Präfixverben und Partikelverben bei der Bildung der Partizipien unterschiedlich behandelt werden, vgl.\ Tabelle~\ref{tab:ppvpart}.

\index{Verb!Präfix-- vs.\ Partikel--}

\begin{table}[!htbp]
  \centering
  \begin{tabular}{lll}
    \lsptoprule
    & \textbf{Präfixverb} & \textbf{Partikelverb} \\
    \midrule
    \textbf{schwach} & ver:lach-t & aus=ge-lach-t \\
    \textbf{stark} & unter:broch-en & ab=ge-broch-en\\
    \lspbottomrule
  \end{tabular}
  \caption{Infinite Verbformen von Präfix- und Partikelverben}
  \label{tab:ppvpart}
\end{table}

Bei den Partikelverben wird die Partikel vor das mit \textit{ge-~-t} bzw.\ \textit{ge-~-en} gebildete Partizip gestellt, bei Präfixverben wird das \textit{ge-} des Zirkumfixes unterdrückt.
\textit{ge-} wird in der Zusammenfassung der Bildungsregularitäten in Tabelle~\ref{tab:vinf} daher eingeklammert.

\begin{table}[!htbp]
  \centering
  \begin{tabular}{lll}
    \lsptoprule
    & \textbf{Infinitiv} & \textbf{Partizip} \\
    \midrule
    \textbf{schwach} & Stamm-\textit{en} & (\textit{ge})-Stamm-\textit{t} \\
    \textbf{stark} & Präsensstamm-\textit{en} & (\textit{ge})-Partizipstamm-\textit{en} \\
    \lspbottomrule
  \end{tabular}
  \caption{Bildung der infiniten Verbformen}
  \label{tab:vinf}
\end{table}

Das sogenannte \textit{Partizip Präsens}, das mit dem (Präsens-)Stamm und \textit{-end} gebildet wird (\textit{lauf-end}, \textit{brech-end}), wird ausschließlich wie gewöhnliche Adjektive verwendet und wird hier daher nicht zum infiniten Paradigma des Verbs gezählt.
Es handelt sich in unserer Auffassung um ein adjektivisches Wortbildungssuffix.

\subsection{Formen des Imperativs}

\label{sec:impflex}

\index{Imperativ}

Der \textit{Imperativ} (also die \textit{Aufforderungsform}) bildet strenggenommen das dritte Paradigma des Verbs nach dem finiten und dem infiniten Paradigma.%
\footnote{Manche Grammatiken behandeln ihn auch als dritten Modus, was allerdings das ohnehin schwierig zu beschreibende Modussystem noch komplizierter macht.}
Ein eigenes Paradigma muss dem Imperativ vor allem deshalb zugesprochen werden, weil er nicht nach Tempus, Modus und Person, aber auch nicht nach Status flektiert.
Er ist eine reine Aufforderungsform, und auch eine vielleicht zunächst plausibel scheinende Analyse des Imperativs als statisch [\textsc{Person}: \textit{2}] ist schwierig, weil keine Subjektkongruenz besteht.
Es gibt beim typischen Imperativ eben gerade kein grammatisches Subjekt bzw.\ keine Nominativ-Ergänzung, mit dem er kongruieren könnte.
Wenn aber nach unserer Auffassung \textsc{Person} ein beim Nomen motiviertes Merkmal ist, das beim Verb als reines Kongruenzmerkmal auftaucht, kann eine prinzipiell subjektlose Verbform nicht nach \textsc{Person} spezifiziert sein.
Ein Subjekt wird konsequenterweise beim Imperativ nicht realisiert (\ref{ex:vflex7770a}), und in Beispielen wie (\ref{ex:vflex7770b}), in denen ein scheinbares Subjekt (\textit{du}) auftaucht, kann es als Anredeform (Vokativ) betrachtet werden.
Es sind beim Imperativ nur die Singular- und die Pluralform zu unterscheiden, vgl.\ Tabelle~\ref{tab:imp}.

\begin{exe}
  \ex\label{ex:vflex7770}
  \begin{xlist}
    \ex{\label{ex:vflex7770a} Geh da weg!}
    \ex{\label{ex:vflex7770b} Komm du mir nur nach Hause!}
  \end{xlist}
\end{exe}

\index{Verb!Flexion!Imperativ}

\begin{table}[!htbp]
  \centering
  \begin{tabular}{lll}
    \lsptoprule
    & \textbf{Singular} & \textbf{Plural} \\
    \midrule
    \textbf{schwach} & Stamm & Stamm \textit{-t} \\
    \textbf{stark} & 2.~Vokalstufe & 1.~Vokalstufe \textit{-t} \\
    \lspbottomrule
  \end{tabular}
  \caption{Bildung der Imperativformen}
  \label{tab:imp}
\end{table}

Die schwachen Verben sind wie immer völlig regelmäßig in der Bildung: \textit{lach} und \textit{lach-t}.
Falls ein starkes Verb eine von der ersten unterschiedene zweite Vokalstufe hat, die sonst in der zweiten und dritten Person Singular verwendet wird (\textit{geb-e} aber \textit{du gib-st}), wird im Imperativ diese zweite Vokalstufe für die Bildung des Imperativs genommen (\textit{gib}).%
\footnote{Manche Sprecher benutzen hier auch die erste Vokalstufe in Form von \textit{geb} oder \textit{geb-e}.}
Wenn es sich allerdings nur um eine Umlautstufe handelt (wie in \textit{du läufst}), wird diese im Imperativ nicht verwendet (also \textit{lauf} statt \textit{\Ast läuf}).

Streng vom eigentlichen Imperativ zu trennen sind andere Formen, die im gegebenen Kontext kommunikativ als Aufforderung verwendet werden können.
Dies sind \zB Konstruktionen mit Modalverben (\ref{ex:vflex9990a}), Partizipien (\ref{ex:vflex9990b}), Infinitiven (\ref{ex:vflex9990c}), Konjunktiven (\ref{ex:vflex9990d}), oder gar Fragekonstruktionen im Konjunktiv (\ref{ex:vflex9990e}) oder Indikativ (\ref{ex:vflex9990f}).
Sie alle haben nichts mit der morphologischen Kategorie des Imperativs zu tun.
Am ehesten sieht noch (\ref{ex:vflex9990d}) wie ein potentieller Imperativ der 3.~Person aus.
Satz (\ref{ex:vflex9990e}) legt aber nahe, dass generell die Konjunktive als Höflichkeitsmarker in Formen der Aufforderung verwendet werden, und es sich damit in (\ref{ex:vflex9990d}) wahrscheinlich um einen Konjunktiv Präsens in einer speziellen Aufforderungsform handelt.
Ein eindeutiger Test, der (\ref{ex:vflex9990a}) und (\ref{ex:vflex9990d})--(\ref{ex:vflex9990f}) als echte Imperativformen ausschließt, ist das Weglassen des Subjekts.
Da beim echten Imperativ nur ein Vokativ als Pseudo-Subjekt stehen kann, muss es immer weglassbar sein, was in diesen Fällen eben nicht geht, vgl.\ (\ref{ex:vflex99901}). 

\begin{exe}
  \ex\label{ex:vflex9990}
    \begin{xlist}
      \ex{\label{ex:vflex9990a} Du mögest kommen.}
      \ex{\label{ex:vflex9990b} Hiergeblieben!}
      \ex{\label{ex:vflex9990c} Den Eischnee langsam unterheben.}
      \ex{\label{ex:vflex9990d} Seien Sie so nett und schreiben das an die Tafel.}
      \ex{\label{ex:vflex9990e} Wären Sie so nett, das an die Tafel zu schreiben?}
      \ex{\label{ex:vflex9990f} Sind Sie so nett, das an die Tafel zu schreiben?}
    \end{xlist}
  \ex\label{ex:vflex99901}
    \begin{xlist}
      \ex[*]{\label{ex:vflex99901a} Mögest kommen.}
      \ex[*]{\label{ex:vflex99901d} Seien so nett und schreiben das an die Tafel.}
      \ex[*]{\label{ex:vflex99901e} Wären so nett, das an die Tafel zu schreiben?}
      \ex[*]{\label{ex:vflex99901f} Sind so nett, das an die Tafel zu schreiben?}
    \end{xlist}
\end{exe}

\subsection{Kleine Verbklassen}

\label{sec:kleineverbklassen}

\index{Modalverb!Flexion}

Zu den \textit{Modalverben} gehören \textit{dürfen}, \textit{können}, \textit{mögen}, \textit{müssen}, \textit{sollen} und \textit{wollen}.
Ihre Präsensbildungen sind in Tabelle~\ref{tab:modpraes} aufgeführt.

\index{Präteritalpräsens}

\begin{table}[!htbp]
  \centering
  \begin{tabular}{llllllll}
    \lsptoprule
    \multirow{2}{*}{\textbf{Sg}} & \textbf{1/3} & darf & kann & mag & muss & soll & will \\
    & \textbf{2} & darf-st & kann-st & mag-st & muss-t & soll-st & will-st \\
    \midrule
    \multirow{2}{*}{Pl} & \textbf{1/3} & dürf-en & könn-en & mög-en & müss-en & soll-en & woll-en \\
    & \textbf{2} & dürf-t & könn-t & mög-t & müss-t & soll-t & woll-t \\
    \lspbottomrule
  \end{tabular}
  \caption{Präsens der Modalverben}
  \label{tab:modpraes}
\end{table}

Diese Verben haben ein sonst im Präsens ungewöhnliches Muster von Vokalstufen, bei dem der Singular und der Plural durch je eine Stufe unterschieden werden.
Der Plural hat oft eine eigene Vokalstufe (außer bei \textit{sollen} und \textit{müssen}), und er wird zusätzlich umgelautet (außer bei \textit{sollen} und \textit{wollen}).
Im Althochdeutschen hatte allerdings das Präteritum eine Vokalstufe für den Singular und eine für den Plural, so dass hier ein historischer Rest dieses ehemals produktiveren Musters konserviert wurde.
Hinzu kommt, dass bei den Modalverben die sonst im Indikativ für das Präteritum typischen Suffixe PN2 im Präsens verwendet werden.
Formal handelt es sich also um eine erstarrte Präteritalbildung mit Präsensbedeutung, und aus diesem Grund nennt man diese Verben \textit{Präteritalpräsentien} (oder \textit{Praeterito-Präsentien}).%
\footnote{Das Verb \textit{wollen} hat sich historisch anders entwickelt, fügt sich aber im heutigen System dem hier beschriebenen Muster.}

Das heutige Präteritum und das Partizip dieser Verben wurde nach dem Muster der schwachen Verben nachgebildet.
Dazu wird der zusätzliche Umlaut, der bei \textit{dürfen}, \textit{können}, \textit{mögen} und \textit{müssen} auf der Pluralstufe liegt, rückgängig gemacht und für das Präteritum \textit{-te} sowie die Suffixe PN2 angehängt.
Im Partizip wird \mbox{\textit{ge-~-t}} zirkumfigiert (\textit{ge-durf-t} usw.).
Die weitergehende Stammänderung bei \textit{mögen} zu \textit{moch-te} ist eine zusätzliche historische Besonderheit, ähnlich, aber nicht genauso wie bei \textit{bringen} zu \textit{brach-te}.
Es ergibt sich Tabelle~\ref{tab:modpraet}.

\begin{table}[!htbp]
  \resizebox{\textwidth}{!}{
    \begin{tabular}{llllllll}
      \lsptoprule
      \multirow{2}{*}{\textbf{Sg}} & \textbf{1/3} & durf-te & konn-te & moch-te & muss-te & soll-te & woll-te \\
      & \textbf{2} & durf-te-st & konn-te-st & moch-te-st & muss-te-t & soll-te-st & woll-te-st \\
      \midrule
      \multirow{2}{*}{\textbf{Pl}} & \textbf{1/3} & durf-te-n & konn-te-n & moch-te-n & muss-te-n & soll-te-n & woll-te-n \\
      & \textbf{2} & durf-te-t & konn-te-t & moch-te-t & muss-te-t & soll-te-t & woll-te-t \\
      \lspbottomrule
    \end{tabular}
  }
  \caption{Präteritum der Modalverben}
  \label{tab:modpraet}
\end{table}

Der Konjunktiv der Modalverben wird ebenfalls im Grunde nach dem Muster der schwachen Verben gebildet.
Für den Konjunktiv Präsens wird der Pluralstamm des Präsens (\textit{dürf-}) mit dem \textit{-e} des Konjunktivs und den Suffixen PN2 kombiniert (\textit{ihr dürf-e-t}).
Der Konjunktiv Präteritum kombiniert denselben Stamm mit dem Präteritalsuffix \textit{-te}, dem Konjunktivsuffix \textit{-e} und den Endungen PN2 (\textit{ihr dürf-t-e-t}), vgl.\ Tabelle~\ref{tab:modvkonj} mit Beispielen.

\begin{table}[!htbp]
  \centering
  \begin{tabular}{llll}
    \lsptoprule
    \multicolumn{2}{c}{} & \textbf{Präsens} & \textbf{Präteritum} \\
    \midrule
    \multirow{2}{*}{\textbf{Sg}} & \textbf{1/3} & könn-e & könn-t-e \\
    & \textbf{2} & könn-e-st & könn-t-e-st \\
    \midrule
    \multirow{2}{*}{Pl} & \textbf{1/3} & könn-e-n & könn-t-e-n \\
    & \textbf{2} & könn-e-t & könn-t-e-t \\
    \lspbottomrule
  \end{tabular}
  \caption{Beispiele für den Konjunktiv der Modalverben (\textit{können})}
  \label{tab:modvkonj}
\end{table}

Lediglich der Konjunktiv Präteritum von \textit{mögen} ist leicht unregelmäßig: \textit{ihr möch-te-t}.
Aus diesem Konjunktiv Präteritum ist allerdings das defektive Verb zu \textit{ich möchte} hervorgegangen und die Formen sind daher wahrscheinlich als Konjunktiv Präteritum zu \textit{mögen} nur noch eingeschränkt verwendbar.
Insofern wäre \textit{mögen} selber defektiv, indem es keinen Konjunktiv Präteritum mehr hat.
Zusammenfassend kann man feststellen, dass die hier besprochenen Verben einem klaren Muster folgen, das auf eine sehr kleine Klasse von Wörtern beschränkt ist.
Außerdem kann man dieses Muster als Präteritalpräsens zusätzlich genauer bestimmen.

\index{Verb!Flexion!unregelmäßig}
\index{Verb!gemischt}
\index{Stärke!Verb}

Abschließend werden nun die besprochenen Verbklassen bezüglich des Grades ihrer Regelmäßigkeit eingeordnet und kurz die echten unregelmäßigen Verben besprochen.
Dies ist nötig, weil viel zu schnell von \textit{Unregelmäßigkeit} gesprochen wird, wo einfach nur eine speziellere Regularität zum Zuge kommt.
Vollständig regelmäßig sind zunächst einmal die schwachen Verben.
Ihre Flexion ist komplett vorhersagbar, sobald ihr (einziger) Stamm bekannt ist.
Dazu passt, dass sie die größte Klasse innerhalb der Verben bilden, und dass damit die Beschreibung der schwachen Flexion die weitest reichenden Regularitäten der Verbalflexion im Deutschen abdeckt.
Im Sinn von Abschnitt~\ref{sec:kernundperipherie} stellen wir fest, dass die schwachen Verben die höchste Typenhäufigkeit unter allen Verbklassen haben und damit den Kern des Systems bilden.

Kennt man hingegen einen Stamm eines starken Verbs, kann man es nur dann korrekt flektieren, wenn zusätzlich die Vokalreihe bekannt ist, der das Verb folgt.
Da auch den Vokalreihen ein System eigen ist, sind diese Verben aber eben nicht \textit{unregelmäßig}, sondern die Regularitäten, die sie betreffen, sind einfach nur von geringerer Reichweite.
Das System in den Vokalreihen zeigt sich vor allem daran, dass nicht beliebig viele Vokalreihen vorkommen (können), und dass bestimmte Vokalreihen bevorzugt sind.
So ist \zB die Reihe \textit{ei--i--i} wie in \textit{reiten} (kurzes /\textipa{I}/) oder \textit{bleiben} (langes /\textipa{i:}/) stark präferiert und bei ungefähr vierzig starken Verben zu finden.
Andererseits sind die Stufen klar mit bestimmten Funktionen (bzw.\ Positionen im Paradigma) verknüpft, so dass \zB niemals eine besondere Vokalstufe für den Konjunktiv existiert.
Man kann also von eingeschränkter Regelmäßigkeit sprechen oder -- wieder in Anlehnung an Abschnitt~\ref{sec:kernundperipherie} -- von einer eingeschränkten Typenhäufigkeit der starken Verben (und ihrer Untertypen je nach Vokalreihe).
Starke Verben sind damit weniger nah am Kern als schwache Verben.

Eine Sonderstellung innerhalb der starken Verben haben Verben wie \textit{bringen} (mit dem Präteritalstamm \textit{brach} wie in \textit{brach-te}) oder \textit{denken} (Präteritum \textit{dach-te}).
Sie haben zusätzlich zu den Vokalstufen weitere Stammveränderungen, nämlich den Verlust des Nasals im Präteritalstamm und Partizipstamm.
Außerdem haben sie zwar zwei Vokalstufen, bilden aber dennoch das Präteritum und das Partizip wie die schwachen Verben zusätzlich mit \textit{-te} bzw.\ \textit{ge-~-t}.
Die Gruppe dieser Verben wird daher gelegentlich als \textit{gemischte Verben} bezeichnet und damit durchaus sinnvollerweise zu einer Gruppe mit eingeschränkter Regelmäßigkeit erklärt.\index{Verb!gemischt}
Andere Beispiele für gemischte Verben (ohne Nasalverlust) sind \textit{brennen} (Präteritum \textit{brann-te}) oder \textit{senden} (Präteritum \textit{san-dte} mit lediglich orthographischem \textit{d}).
Oft existiert (wie bei \textit{senden}) eine vollständig schwache Variante (Präteritum \textit{sende-te}) parallel.

Die Modalverben bilden eine nochmals wesentlich kleinere Flexionsklasse als die starken Verben.
Sie folgen eigenen Regularitäten, unter anderem weil ihre Präsensformen formal wie die Präteritalformen starker Verben gebildet werden und sie sowohl Merkmale der starken Verben (Vokalreihen) als auch der schwachen Verben (Bildung von Präteritum und Konjunktiv) zeigen.
Trotzdem können wir diese wenigen Verben zu einer Gruppe zusammenfassen, die eigenen, eingeschränkten Regularitäten folgt und damit keineswegs unregelmäßig genannt werden sollte.
Mit einer gegenüber den starken Verben nochmals verringerten Typenhäufigkeit befinden sich die Modalverben recht weit vom Systemkern entfernt.

Darüberhinaus gibt es Verben wie \textit{sein}, das stark \textit{suppletiv} gebildet wird, also mehrere vollständig unterschiedliche Stämme hat und damit tatsächlich \textit{unregelmäßig} ist.\label{abs:suppletiv}
In seinen Paradigmen kommen mindestens vier völlig verschiedene Stämme zum Einsatz, und bei vielen Formen ist keine klare Grenze zwischen Stamm und Suffix auszumachen.
Tabelle~\ref{tab:sein} illustriert diese Verhältnisse.
Es gibt einen \textit{b}-haltigen Stamm (\textit{bin}), außerdem den \textit{sei}-Stamm, eventuell den davon zu unterscheidenden Stamm \textit{is} in \textit{is-t} und einen \textit{w}-haltigen Stamm in \textit{war} und \textit{ge-wes-en}.
Bis auf das Präsens ist die Verteilung der Suffixe durchgehend in Ordnung (nur PN2), und es gibt für jede Tempus\slash Modus-Kombination einen eigenen Stamm bzw. einen umgelauteten Stamm.
Sehr schön und fast wieder musterhaft im Stil der starken Verben sind der Indikativ Präteritum und der Konjunktiv Präteritum gebildet.

\begin{table}[!htbp]
  \centering
  \begin{tabular}{llllll}
    \lsptoprule
    \multicolumn{2}{c}{} & \multicolumn{2}{c}{\textbf{Indikativ}} & \multicolumn{2}{c}{\textbf{Konjunktiv}} \\
    \multicolumn{2}{c}{} & \textbf{Präsens} & \textbf{Präteritum} & \textbf{Präsens} & \textbf{Präteritum} \\
    \midrule
    \multirow{3}{*}{\textbf{Sg}} & \textbf{1} & bin & war & sei & wär(-e) \\
    & \textbf{2} & bi-st & war-st & sei-(e)st & wär(-e)-st \\
    & \textbf{3} & is-t & war & sei & wär(-e) \\
	\midrule
     \multirow{2}{*}{\textbf{Pl}} & \textbf{1/3} & sind & war-en & sei-e-n & wär-e-n \\
     & \textbf{2} & sei-d & war-t & sei(-e)-t & wär(-e)-t \\
    \lspbottomrule
  \end{tabular}
  \caption{Formen von \textit{sein}}
  \label{tab:sein}
\end{table}

Andere echte Unregelmäßigkeiten der Stammbildung finden sich bei \textit{haben} (vgl.\ Formen wie \textit{hab-e} und \textit{ha-st}) oder Verben wie \textit{bringen} (Präteritum \textit{brach-te}).
\label{abs:9z32fqgsve}Das Verb zu \textit{ich möchte} ist ein historisch aus dem Konjunktiv Präteritum von \textit{mögen} hervorgegangenes defektives Verb (s.\ oben).
Es hat nur finite Präsensformen, keinen Konjunktiv und auch keine infiniten Formen.
Von \textit{Unregelmäßigkeit} lohnt es sich also nur zu sprechen, wenn wie in diesen Fällen ein Verb (oder ganz allgemein ein Wort) zumindest partiell ein grammatisches Verhalten zeigt, das es mit keinem anderen teilt.
Die Typenhäufigkeit ist in diesen Fällen genau 1.


\Zusammenfassung{
  Nur Präsens und Präteritum sind \textit{morphologische} (\textit{synthetische}) Tempusformen, alle anderen Tempora sind \textit{analystische} Bildungen.
  Bei den beiden \textit{Konjunktiven} stimmen die morphologische Bildung (Präsens\slash Perfekt) und Semantik (quotativ\slash irreal) nicht überein.
  \textit{Genus Verbi} (Aktiv\slash Passiv) ist im Deutschen keine Flexionskategorie.
Es gibt zwei kombinierte Person\slash Numerus-Suffixreihen und ein Konjunktiv"=Suffix (\textit{-e}).
Tempus wird bei den schwachen Verben durch ein Suffix (\textit{-te}), bei den starken durch \textit{Vokalstufen} markiert.
\textit{Starke Verben} sind nicht unregelmäßig, sondern folgen spezielleren, aber nicht zufälligen Bildungsmustern.
\textit{Modalverben} sind \textit{Präteritalpräsentien}, weil sie ihre Präsensformen eher wie ein starkes Präteritum bilden.
}


\Uebungen

\begin{sloppypar}

\Uebung[\onestar] \label{u91} Erstellen Sie für die folgenden Sätze Tempusanalysen.
Stellen Sie dazu zuerst fest, (a) welche Tempora vorkommen.
Dann (b) überlegen Sie, welches Diagramm zu diesem Tempus gehört.
Schließlich (c) überlegen Sie, wie die Tempora (wenn es mehrere sind) interagieren bzw.\ einander die R-Punkte liefern.
Erst dann erstellen Sie das Diagramm.
\footnote{Siglen der Belege im DeReKo: NON09\slash JUN.14285, NON09\slash JAN.11778, A09\slash MAI.07721, NON09\slash FEB.03873, A00\slash FEB.10444}

\begin{enumerate}\Lf
  \item Niederösterreich lebt noch.
  \item Sie zogen jedoch wieder ohne Beute ab.
  \item Nachdem ein erster Angriff nicht erfolgreich gewesen war, setzte sich Näf beim zweiten Versuch zusammen mit Absalon von den Gegnern ab.
  \item Ab März beginnt dann die Pflanzzeit für Stauden.
  \item Dieselbe Vorgehensweise wird der Schulrat von Rossrüti wählen.
\end{enumerate}

\Uebung[\onestar] \label{u92} Finden Sie alle Tempusformen (im Sinn von Tabelle~\ref{tab:sechstempora}, S.~\pageref{tab:sechstempora}) in den folgenden Sätzen.
Sind die Tempora synthetisch oder analytisch gebildet?
Bestimmen Sie bei den analytischen Tempora, was die finiten Verbformen und was die infiniten sind, die zusammen das analytische Tempus ergeben.%
\footnote{Siglen der Belege im DeReKo: NON09\slash FEB.01018, A00\slash FEB.08209, A0\slash MAR.15912, HMP08\slash JUL.00395, RHZ09\slash MAR.15300, BVZ09\slash MAI.01348, RHZ09\slash JUN.22699, BRZ09\slash MAR.11529, NON09\slash FEB.03873, M09\slash MAR.23954, RHZ09\slash JUN.02920}

\begin{enumerate}\Lf
  \item Das heißt, zahlreiche Straßengegner kamen mit dem Auto.
  \item Die Kasse wird bei mir in ebenso guten Händen sein, wie sie es bis jetzt gewesen ist.
  \item Die Diskussion hat gezeigt, dass auch hier nicht mehr unbedingt eine heile Welt besteht.
  \item Einen solchen Verdacht hatte zuvor schon sein Sprecher geäußert - der hatte von DVDs statt Bier gesprochen.
  \item Sie ahnten wohl, was auf sie zukommt.
  \item Das Fahrzeug im Wert von 160.000 Euro war versperrt abgestellt gewesen.
  \item In Bad Ems wird dies sicher nicht der letzte Auftritt des \textit{Unterhaltungskanzlers} gewesen sein, dem es vortrefflich gelingt, sein Publikum bestens zu unterhalten.
  \item Es geht um das Duell zweier Schachspieler, die unterschiedlicher nicht sein können.
  \item Ab März beginnt dann die Pflanzzeit für Stauden.
  \item Sie hofft, dass es einmal auf den Bühnen der großen weiten Welt leuchten wird.
  \item Bei der Wahl wird der Wähler Personen aus zwei Listen wählen können.
\end{enumerate}

\Uebung[\tristar] \label{u93} Die Sätze in Übung \ref{u91} sind bewusst einfach gewählt.
Zum Transfer führen Sie die Aufgabenstellung von Übung \ref{u91} für die Sätze in Übung \ref{u92} durch.
Überlegen Sie sich, welche Ereignisse beschrieben werden.
Versuchen Sie dann, zu überlegen, wie sie in Relation stehen (es kommen $\ll$, $\sim$ und $=$ infrage).

\Uebung \label{u94} Kategorisieren Sie die Verben zu den in den folgenden Sätzen vorkommenden finiten und infiniten Verbformen.
Bestimmen Sie dazu (a) den Stamm und (b) ordnen Sie das Verb als schwaches, starkes, präteritalpräsentisches oder unregelmäßiges Verb ein.%
\footnote{Siglen der Belege im DeReKo: A01\slash AUG.22669, NON09\slash APR.11939, I96\slash MAR.09729, HMP10\slash JUN.01005, NON09\slash APR.03907, BRZ06\slash SEP.15899, RHZ02\slash APR.22373, A09\slash OKT.08867, M06\slash AUG.60965, HMP10\slash MAI.00115}

\begin{enumerate}\Lf
  \item Also pflege der Sohn einen anderen Führungsstil, sei wohl auch kompromissbereiter.
  \item Das Badener Spital sollte um 75 Millionen Euro völlig umgebaut werden.
  \item Könnet ihr denn nicht eine Stunde für mich wachen?
  \item Da wird Wäsche per Hand gewaschen, gesponnen und Seile gedreht.
  \item Er bäckt nun den Apfelkuchen nach meinem Rezept.
  \item Mehrere Ideen gibt es nun, wo die Geräte untergestellt werden könnten.
  \item Dabei drohte der "'Verkäufer`" dem alten Mann, ihn zu töten, wenn er die Decke nicht käuft.
  \item Die Wahlverlierer CDU und SPD rauften sich zusammen und schmiedeten einen Koalitionsvertrag.
  \item Sonst schwängen Machtbeziehungen mit, sonst wären unhinterfragt Klischees und Stereotypen wirksam.
  \item Du musst schlauer boxen.
\end{enumerate}

\Uebung \label{u95} Führen Sie Formenanalysen für die Verbformen aus Übung \ref{u94} durch:
Segmentieren Sie die Verbformen (nur Flexion).
Geben Sie bei Stämmen die Stufe an: schwach oder erste bis vierte Vokalstufe (s.\ Tabelle~\ref{tab:ablstuf}, S.~\pageref{tab:ablstuf}).
Geben Sie für die Suffixe an, um welche es sich handelt.
Es kommen infrage: Partizip (Zirkumfix), Präteritum der schwachen Verben \textit{-te}, Konjunktiv \textit{-e}, PN1 oder PN2.
Bei PN1 und PN2 geben Sie jeweils an, um welche Person/Numerus-Form es sich handelt.

\Uebung[\tristar] \label{u96} Verschaffen Sie sich einen Überblick über die Formen des Verbs \textit{wissen} und ordnen Sie es einem der bekannten Flexionstypen zu.
Überlegen Sie, was bezüglich unserer Darstellung bzw.\ Kategorisierung problematisch sein könnte.

\end{sloppypar}

\WeitereLiteratur

\begin{sloppypar}

\paragraph*{Einführungen und Gesamtdarstellungen}

Wie immer kann der \textit{Grundriss} \citep{Eisenberg1} zur Vertiefung verwendet werden, genauso \citet{Engel09}.
Zu allen Aspekten der deutschen Morphologie bietet \citet{HentschelVogel2009} gut lesbare Artikel.
Die hier vorgestellte Klassifikation der Wortarten ist eine Vereinfachung zu \citet{Engel09a} und \citet{Engel09}.
Etwas anders klassifiziert die Duden-Grammatik \citep{Duden8}.
Gut lesbare, allerdings nur auf Englisch verfügbare Einführungen in die Morphologie sind \citet{Katamba06} und \citet{Booj2007}.

\paragraph*{Wortbildung}

Zur Einführung in die Wortbildung kann \citet{Altmann2011} verwendet werden.
Eine Gesamtdarstellung der deutschen Wortbildung ist \citet{FB95}.
Weiterführende Lesevorschläge: 
\citet{BTh92} gegen die Annahme von nominalen Kopulativkomposita im Deutschen;
\citet{Gallmann1999} und \citet{NueblingSzczepaniak2009} zu den Fugenelementen;
\citet{EisenbergSayatz2002} zu Reihen von Wortbildungssuffixen.

\paragraph*{Flexion}

Einen Überblick über die Flexion des Deutschen bietet \citet{ThieroffVogel2009}.
Der Status von Komparation als Flexion bzw.\ Wortbildung wird \zB in der IDS-Grammatik \citep[47f.]{IDS} und Abschnitt~5.2 von \citet{Eisenberg1} sowie Abschnitt~12.3 aus \citet{Eisenberg2} besprochen.

Weiterführende Lesevorschläge zu Nomina:
\citet{Wiese2012} zur Substantivflexion;
\citet{KoepckeZubin1995} zum Genus;
\citet{Wegener2004} zu Pluralbildungen von Lehnwörtern;
\citet{Koepcke1995} und \citet{Thieroff2003} zu schwachen Maskulina;
\citet{Wiese2009} und \citet{Nuebling2011} zu Aspekten der Adjektivflexion;
\citet{Vogel1997} zu unflektierten Adjektiven;
\citet{Baerentzen2002} zu \textit{deren} und \textit{derer}.
Von besonderer Bedeutung ist schließlich die historische Betrachtung der Morphosyntax deutscher Nomina, da das System auch heute noch im Umbruch ist.
\citet{Demske00} bespricht hierzu eine Fülle von Daten.

Weiterführende Lesevorschläge zu Verben:
\citet{HS91} zur Subklassifikation der Verben nach ihren Valenzmustern;
\citet{Wiese08} zu Klassifikation des Ablauts.

\paragraph*{Tempus und Modus}

Ausführlichere Einführungen zum Tempus sind \citet{Rothstein2007} und \citet{Vater2007}.
Weiterführende Lesevorschläge:
\citet{Leibukt2011} zur sogenannten \textit{Höflichkeitsfunktion} des Konjunktivs;
\citet{Fabricius1997} zum Konjunktiv;
\citet{Fabricius2000} zur \textit{würde}-Paraphrase.

\end{sloppypar}
