\chapter{Phonologie}

\label{sec:phonologie}

Die im letzten Kapitel besprochene artikulatorische Phonetik beschreibt die physiologischen Grundlagen der Sprachproduktion.
Anhand des Vorrats an Zeichen im Alphabet der IPA haben wir außerdem definiert, welche Laute im in Deutschland gesprochenen Standarddeutschen vorkommen.
Die eigentliche Frage der systematischen Grammatik bezüglich der Lautgestalt von Wörtern und größeren Einheiten ist aber, nach welchen Regularitäten diese Laute verbunden werden, und welchen Stellenwert die einzelnen Segmente und Segmentverbindungen (wie \zB Silben) im gesamten Lautsystem haben.
In der Phonologie geht es daher um das \textit{Lautsystem} und seine Regularitäten.
In Abschnitt~\ref{sec:segmentalephol} wird der Status einzelner Laute und ihrer Vorkommen behandelt.
Es wird diskutiert, wie man Laute mit Merkmalen beschreiben kann und wie Laute im Lexikon gespeichert sind.
Schließlich werden einige konkrete phonologische Strukturbedingungen des Deutschen (wie die Auslautverhärtung) systematisch dargestellt.
Dann folgt eine recht ausführliche Analyse des Silbenbaus (Abschnitt~\ref{sec:phonotaktik}).
Abschließend gibt Abschnitt~\ref{sec:prosodie} einen Einblick in die Prosodie (die Betonungslehre) und die damit in phonologische Aspekte der Wortebene.

% ==================================




\section{Segmente}

\label{sec:segmentalephol}

\subsection{Segmente, Merkmale und Verteilungen}

\label{sec:segmenteverteilungen}
\label{sec:verteilungen}

Der zentrale Begriff in der Phonologie ist zunächst wie in der Phonetik der des \textit{Segments}, vgl.\ Definition~\ref{def:segment}.
Alternativ findet man auch den Begriff des \textit{Phonems}, auf den in Abschnitt~\ref{sec:phonphonem} kurz eingegangen wird.
Allerdings geht es in der Phonologie anders als in der Phonetik um den systematischen Stellenwert der Segmente, nicht um eine reine Beschreibung ihrer Lautgestalt.
Um sich den Übergang von der Phonetik zur Phonologie klar zu machen, ist der Begriff der \textit{Verteilung} hilfreich.
Schon in Abschnitt~\ref{sec:auslautverhaertungphonetik} wurde diskutiert, dass es bestimmte Positionen im Wort und in der Silbe gibt, an denen nur bestimmte Segmente vorkommen.
Im genannten Abschnitt ging es zunächst nur um die Beschreibung verschiedener Korrelationen von Schrift und Phonetik, in der Phonologie sind solche Phänomene hingegen von hohem theoretischen Stellenwert.
Das Beispiel war die Auslautverhärtung, die dazu führt, dass in der letzten Position der Silbe Plosive immer stimmlos sind (\textit{Bad} als \textipa{[ba:t]} und nicht *\textipa{[ba:d]}).
Man muss nun aber dennoch davon ausgehen, dass die betreffenden Wörter \textit{im Prinzip} (besser: \textit{im Lexikon}) einen stimmhaften Plosiv an der entsprechenden Stelle enthalten, denn wenn (\zB in Flexionsformen) ein weiterer Vokal folgt, ist der Plosiv stimmhaft, vgl.\ \textit{Bades} \textipa{[ba:d@s] nicht *\textipa{[ba:t@s]}}.
Ausgehend von dem Begriff der phonologischen Verteilung oder Distribution kann man in der Phonologie systematisch über solche Phänomene sprechen.

\Definition{Verteilung (Distribution)}{
Die Verteilung eines Segments ist die Menge der Umgebungen, in denen es vorkommt.
\index{Verteilung}
}

Die Beschreibung der Verteilung eines Segments nimmt typischerweise Bezug auf bestimmte Positionen in der Silbe oder im Wort, oder auf Positionen vor oder nach anderen Segmenten.
Eine für das phonologische System entscheidende Frage ist, ob zwei Segmente die gleiche Verteilung oder eine teilweise oder vollständig unterschiedliche Verteilung haben.
Die Beispiele in (\ref{ex:phol6438})--(\ref{ex:phol6440}) illustrieren drei Typen von Verteilungen anhand des Vergleiches von je zwei Segmenten.
(\ref{ex:phol6438}) zeigt, dass \textipa{[t]} und \textipa{[k]} eine \textit{vollständig übereinstimmende Verteilung} haben.
Sie kommen beide am Anfang und am Ende von Silben vor.
Hingegen haben \textipa{[h]} und \textipa{[N]} eine vollständig unterschiedliche  Verteilung, wie (\ref{ex:phol6439}) zeigt.
Am Anfang einer Silbe kommt nur \textipa{[h]} vor, am Ende einer Silbe kommt nur \textipa{[N]} vor.
Schließlich demonstriert (\ref{ex:phol6440}), dass \textipa{[s]} und \textipa{[z]} eine teilweise übereinstimmende Verteilung haben.
Am Anfang der ersten Silbe eines Wortes kommt nur \textipa{[z]} vor wie in (\ref{ex:phol6440a}), am Ende der letzten Silbe eines Wortes kommt nur \textipa{[s]} vor wie in (\ref{ex:phol6440b}), und am Anfang einer Silbe in der Wortmitte kommen beide vor, \textipa{[z]} aber nur nach langem Vokal oder Diphthong wie in (\ref{ex:phol6440c}).

\begin{exe}
  \ex\label{ex:phol6438}
    \begin{xlist}
      \ex{\label{ex:phol6438a} Tot \textipa{[to:t]}, Kot \textipa{[ko:t]}}
      \ex{\label{ex:phol6438b} Schott \textipa{[SOt]}, Schock \textipa{[SOk]}}
    \end{xlist}
  \ex{\label{ex:phol6439} Hang \textipa{[haN]}, *\textipa{[Nah]}}
  \ex\label{ex:phol6440}
    \begin{xlist}
      \ex{\label{ex:phol6440a} Sog \textipa{[zo:k]}, besingen \textipa{[b@zIN@n]}, *\textipa{[so:k]}}
      \ex{\label{ex:phol6440b} fließ \textipa{[fli:s]}, \textit{Boss} \textipa{[bOs]}, *\textipa{[fli:z]}}
      \ex{\label{ex:phol6440c} heißer \textipa{[h\t{aE}s5]}, heiser \textipa{[h\t{aE}z5]}, Base \textipa{[ba:z@]}, Basse \textipa{[bas@]}, *\textipa{[baz@]}}
    \end{xlist}
\end{exe}

Wie man an den Beispielen sieht, gibt es Segmente, anhand derer Wörter (wie \textit{heißer} und \textit{heiser}) unterschieden werden können, auch wenn die Wörter ansonsten völlig gleich lauten.
Dies geht genau deswegen, weil die zwei Segmente mindestens eine teilweise übereinstimmende Verteilung haben.
Zwei Wörter, die sich nur in einem Segment unterscheiden, nennt man \textit{Minimalpaar}, und ein Minimalpaar illustriert jeweils einen \textit{phonologischen Kontrast}.

\Definition{Phonologischer Kontrast}{
\label{def:phokonseg}
Zwei phonetisch unterschiedliche Segmente bzw.\ Merkmale stehen in einem phonologischen Kontrast, wenn sie eine teilweise oder vollständig übereinstimmende Verteilung haben und dadurch einen lexikalischen bzw.\ grammatischen Unterschied markieren können.
\index{Kontrast}
}

Ein phonologischer Kontrast besteht \zB zwischen \textipa{[t]} und \textipa{[k]}, weil wir Wörter anhand dieser Segmente unterscheiden können.
Das Gleiche gilt für \textipa{[s]} und \textipa{[z]} und viele andere Paare von Segmenten.
Es gilt aber nicht für \textipa{[h]} und \textipa{[N]}, weil diese beiden Segmente keine übereinstimmende Verteilung haben, wie in (\ref{ex:phol6439}) gezeigt wurde.
Man kann mit dem Unterschied zwischen \textipa{[h]} und \textipa{[N]} als nicht zwei verschiedene Wörter unterscheiden.
Diese Art der Verteilungen nennt man komplementär.

\Definition{Komplementäre Verteilung}{
Eine komplementäre Verteilung zweier Segmente liegt dann vor, wenn die beiden Segmente in keiner gemeinsamen Umgebung vorkommen.
Komplementär verteilte Segmente können prinzipiell keinen phonologischen Kontrast markieren.
\index{Verteilung!komplementär}
}

Über Verteilungen lässt sich schon anhand des bisher eingeführten Inventars von Beispielen noch mehr sagen.
Bei der bereits besprochenen Auslautverhärtung haben wir es mit Paaren von stimmlosen und stimmhaften Plosiven zu tun, die in bestimmten Umgebungen (im Silbenanlaut) einen Kontrast markieren, der aber in anderen Umgebungen (Silbenauslaut) verschwindet.
(\ref{ex:phol-5674-1})--(\ref{ex:phol-5674-3}) zeigen dies für \textipa{[g]} und \textipa{[k]}, \textipa{[d]} und \textipa{[t]} sowie \textipa{[b]} und \textipa{[p]}.

\begin{exe}
  \ex\label{ex:phol-5674-1}
  \begin{xlist}
    \ex{Weg \textipa{[ve:k]}, Weges \textipa{[ve:g@s]}}
    \ex{Bock \textipa{[bOk]}, Bockes \textipa{[bOk@s]}}
  \end{xlist}
  \ex\label{ex:phol-5674-2}
  \begin{xlist}
    \ex{Bad \textipa{[ba:t]}, Bades \textipa{[ba:d@s]}}
    \ex{Blatt \textipa{[blat]}, Blattes \textipa{[blat@s]}}
  \end{xlist}
  \ex\label{ex:phol-5674-3}
  \begin{xlist}
    \ex{Lab \textipa{[la:p]}, Labes \textipa{[la:b@s]}}
    \ex{Depp \textipa{[dEp]}, Deppen \textipa{[dEp@n]}}
  \end{xlist}
\end{exe}

Im Silbenauslaut des Deutschen gibt es prinzipiell keinen Unterschied zwischen stimmlosen und stimmhaften Plosiven.
Solche Effekte nennt man \textit{Neutralisierungen}.

\Definition{Neutralisierung}{
Eine Neutralisierung ist die Aufhebung eines phonologischen Kontrasts in einer bestimmten Position.
\index{Neutralisierung}
}

Im Silbenauslaut wird im Deutschen also der phonologische Kontrast zwischen \textipa{[g]} und \textipa{[k]}, zwischen \textipa{[d]} und \textipa{[t]} usw.\ neutralisiert.
Allgemein gesprochen wird der Kontrast zwischen stimmlosen und stimmhaften Plosiven in dieser Position neutralisiert.
Das erklärt die Formulierung 'Zwei phonetisch unterschiedliche Segmente \textit{bzw.\ Merkmale}\ldots' in Definition~\ref{def:phokonseg}.
Phonologische Kontraste bestehen im Prinzip zwischen Merkmalen und erst in zweiter Ordnung zwischen ganzen Segmenten.

Das Feststellen von Verteilungen ist allerdings kein Selbstzweck.
Durch die Untersuchung aller Verteilungen in einer Sprache konstruiert man das phonologische System (die phonologische Komponente der Grammatik).
Dabei geht es darum, die Formen zu ermitteln, die im Lexikon gespeichert werden müssen, und die Strukturbedingungen (wie die Auslautverhärtung) zu beschreiben, an die die Segmente in diesen Formen ggf.\ angepasst werden müssen.
Die \textit{lexikalisch gespeicherten Formen} und die \textit{phonologischen Strukturbedingungen} produzieren dann die konkreten phonetischen Verteilungen an der Oberfläche.

\subsection{Zugrundeliegende Formen und Strukturbedingungen}
\label{sec:pholfeat}
\label{sec:ur}

Wir bleiben jetzt beim Beispiel der Auslautverhärtung, um die Idee von lexikalisch zugrundeliegenden Formen und phonologischen Strukturbedingungen einzuführen.
Die Auslautverhärtung hat wie erwähnt zur Folge, dass für Obstruenten im Silbenauslaut der Stimmtonkontrast neutralisiert wird, denn alle Obstruenten im Silbenauslaut sind stimmlos.
Wenn man das gesamte Paradigma der Wörter in (\ref{ex:phol-5674-1}) bis (\ref{ex:phol-5674-3}) ansieht, fällt aber dennoch ein bedeutender Unterschied auf.
In manchen Wörtern steht im Silbenauslaut ein Konsonant, der in anderen Umgebungen stimmhaft ist, wie in \textipa{[ve:k]} und \textipa{[ve:g@s]}.
In anderen Wörtern steht ein stimmloser Konsonant, der auch in diesen anderen Umgebungen stimmlos bleibt, wie in \textipa{[bOk]} und \textipa{[bOk@s]}.
Es ist daher naheliegend, anzunehmen, dass Wörter wie \textit{Weg} (oder \textit{Bad}, \textit{Lab} usw.) eine \textit{zugrundeliegende Form} haben, in der der letzte Obstruent stimmhaft ist.
Diese zugrundeliegende Form ist eine der wesentlichen Informationen, die zum \textit{lexikalischen Wort} gehören.
Vgl.\ Abschnitt~\ref{sec:wortwortform}.

Die eigentliche Grammatik stellt allerdings allgemeine Anforderungen an die Wohlgeformtheit von Strukturen, hier die \textit{phonologischen Strukturbedingungen}.
Der \textit{Prozess} der Auslautverhärtung (als Veränderung der Merkmale eines Segments) ist in diesem Sinn das Ergebnis einer Anpassung von Silben an die Strukturbedingung, dass Silben nicht auf stimmhafte Obstruenten enden können.%
\footnote{Man kann die phonologische Grammatik in Form von \textit{Prozessen} bzw.\ \textit{Regeln} (im technischen Sinne) formulieren, die Formen als Eingabematerial nehmen und modifiziert als Ausgabematerial wieder ausgeben.\index{phonologischer Prozess}
Die Auslautverhärtung wäre dann einfach als eine Regel im technischen Sinn.
Alternativ kann man davon ausgehen, dass eine phonologische Grammatik aus Beschreibungen zulässiger Strukturen besteht, an die konkrete Formen angepasst werden.
Wie diese Anpassung vor sich geht, ist auch wieder eine sehr technische Frage.
Innerhalb einer phonembasierten Theorie (Abschnitt~\ref{sec:phonphonem}) bieten sich wieder andere Möglichkeiten, die Beziehung von Formen und Strukturbedingungen zu erfassen.
Die technischen Unterschiede sind für unsere Zwecke mehr als nachrangig.
Eine deskriptive Grammatik ist wahrscheinlich am besten bedient, wenn sie sich darauf beschränkt, zu beschreiben, wie Formen im Lexikon und an der Oberfläche aussehen, also systematische Beziehungen -- eben \textit{Regularitäten} (Abschnitt~\ref{sec:regulgen}) -- feststellt.}
Man könnte umgekehrt versuchen, eine Art \textit{Anlauterweichung} anzunehmen.
Die entsprechende Strukturbedingung wäre, dass Obstruenten stimmhaften sein müssen, wenn sie im Silbenanlaut stehen.
Dann gäbe es allerdings keine Formen wie \textit{Bockes} \textipa{[bOk@s]}, sondern es würde *\textipa{[bOg@s]} herauskommen.
Die zugrundeliegende Form muss also genau die phonologischen Informationen eines Wortes enthalten, die ausreichen, um zu erklären, wie die lautliche Gestalt des Wortes in allen möglichen Formen und Umgebungen aussieht.

\Definition{Zugrundeliegende Form und Strukturbedingung}{
\label{def:pholproz}
Die zugrundeliegende Form ist eine Folge von Segmenten, die im Lexikon gespeichert wird, und auf die alle zugehörigen phonetischen Formen zurückgeführt werden können.
Die Formen werden ggf. an die phonologischen Strukturbedingungen (die Regularitäten der phonologischen Grammatik) angepasst werden.
\index{zugrundeliegende Form}
\index{Strukturbedingung}
}

Die Phonologie stellt also eine Abstraktion gegenüber der Phonetik dar.
Die Phonetik eines Wortes beschreibt nur, wie es tatsächlich ausgesprochen wird, und jedes einzelne Wort einer Sprache kann ohne Betrachtung der anderen Wörter vollständig phonetisch beschrieben werden.
Die phonologische Repräsentation eines Wortes erfordert hingegen zusätzliches Wissen um Strukturbedingungen (\zB in Form der Auslautverhärtung), um aus ihr phonetische Formen abzuleiten.
Dieses Wissen erschließt sich durch die Betrachtung des gesamten Sprachsystems, also jedes Wortes in Bezug zu allen anderen Wörtern und in allen möglichen Umgebungen.
Anders gesagt müssen die Verteilungen der Segmente und der Wörter bekannt sein.

\begin{table}
  \resizebox{\textwidth}{!}{
    \begin{tabular}{ccc}
      \lsptoprule
      \multicolumn{2}{c}{\textbf{Grammatik}} & \textbf{Externe Systeme} \\ 
      \midrule
      \textbf{Lexikon} & \textbf{Phonologie} & \textbf{Phonetik} \\
      \midrule
      /~/& $\Rightarrow$ & \textipa{[~]}\\
      zugrundeliegende Form & Anpassung an Strukturbedingungen & phonetische Realisierung \\
      \lspbottomrule
    \end{tabular}
  }
  \caption{Lexikon, Phonologie und Phonetik}
  \label{tab:pholsystem}
\end{table}

Zugrundeliegende phonologische Formen schreibt man konventionellerweise nicht in \textipa{[~]} sondern in /~/, also \zB /\textipa{ve:g}/, /\textipa{ba:d}/ und /\textipa{la:b}/.
Schematisch kann man die Verhältnisse wie in Tabelle~\ref{tab:pholsystem} darstellen.
Mit \textit{externen Systemen} sind nicht zur Grammatik gehörige Systeme wie Gehör und Sprechapparat gemeint.
Durch die unterschiedlichen Klammern (/\textipa{ba:d}/, \textipa{[ba:t]}) kann man zugrundeliegende Formen und phonetische Realisierungen in Beziehung setzen.
Im nächsten Abschnitt werden beispielhaft einige Strukturbedingungen und Verteilungen besprochen, um zu illustrieren, wie ein phonologisches System rekonstruiert werden kann.

\subsection{Strukturbedingungen und Verteilungen}

\subsubsection{Auslautverhärtung}

\label{sec:prozauslautverh}

Die Auslautverhärtung lässt sich als Strukturbedingung unter Bezug auf phonetische bzw.\ phonologische Merkmale (Abschnitt~\ref{sec:photpholmerkmale}), bestimmte Positionen in Wort oder Silbe und die Oberklassen für Segmente (Abschnitt~\ref{sec:photoberklassen}) sehr einfach und kompakt beschreiben.

\Definition{Auslautverhärtung}{Segmente mit [\textsc{Obstruent}:~$+$] sind [\textsc{Stimme}:~$-$] am Silbenende.}

Wenn wir zugrundeliegende Formen entsprechend an diese Bedingung anpassen wollen, muss also die Silbenstruktur bekannt sein.
Um diese geht es in Abschnitt~\ref{sec:silben} noch im Detail, hier werden die Silbengrenzen einfach vorgegeben und durch Punkte markiert. 
Nur zur Veranschaulichung steht \phopro\ für \textit{wird realisiert als}.

\begin{exe}
  \ex\label{ex:phol6726}
  \begin{xlist}
    \ex{\label{ex:phol6726a} /\textipa{ba:d}/ \phopro \textipa{[ba:t]}}
    \ex{\label{ex:phol6726b} /\textipa{ba:d@s}/ \phopro \textipa{[b:ad@s]}}
    \ex{\label{ex:phol6726c} /\textipa{ba:t}/ \phopro \textipa{[ba:t]}}
  \end{xlist}
\end{exe}

Abhängig von der zugrundeliegenden Form und der Silbenstruktur muss eine Veränderung stattfinden -- oder eben nicht.
In (\ref{ex:phol6726a}) steht /\textipa{d}/ an das Silbenende.
Weil /\textipa{d}/ den Wert [\textsc{Obstruent}: $+$] hat, wird der Wert des Stimmton-Merkmals auf [\textsc{Stimme}: $-$] gesetzt.
In (\ref{ex:phol6726b}) ist die Silbenstruktur anders, die Bedingung für die Auslautverhärtung ist nicht erfüllt, und die Form bleibt unverändert.
In (\ref{ex:phol6726c}) steht zwar ein Obstruent /\textipa{t}/ am Silbenende, aber es muss keine Anpassung stattfinden, weil /\textipa{t}/ von vornherein [\textsc{Stimme}: $-$] ist.

\subsection{Gespanntheit, Betonung und Länge}

\label{sec:gespanntheit}

Die Formulierung von Strukturbedingung kann helfen, die Menge der Merkmale zu reduzieren, die man zugrundeliegend spezifizieren muss.
Solche Reduktionen sind typisch für die Phonologie im Gegensatz zur Phonetik, weil eine einfache Systembeschreibung aus allgemeinen Ökonomiegründen einer komplexeren prinzipiell vorzuziehen ist.
In Abschnitt~\ref{sec:photpholmerkmale} wurde die Vokallänge als gewöhnliches Merkmal (\textsc{Lang}) eingeführt.
Gleichzeitig wurde festgestellt, dass nur die Vokale \textipa{[i y u e \o\ E o a]} lange und kurze Varianten haben.
Bezüglich der Akzentuierung bzw.\ Betonung ist ebenfalls bereits klar, dass alle Vokale bis auf \textipa{[@ 5]} betonbar sind, und dass bei den Vokalen mit Längenunterschied die Länge an die Betonung gebunden ist.
Dieser Abschnitt verfolgt nun zwei Ziele.
Erstens wird das Merkmal \textsc{Gespannt} vorgeschlagen, um die Vokale zusammenzufassen, die lang und kurz vorkommen.
Zweitens wird dadurch das Merkmal \textsc{Lang} aus allen zugrundeliegenden Formen eliminiert und das Merkmal \textsc{Lage} wird auf drei Werte reduziert.
Für die genannten Vokale wird also zunächst das Merkmal \textsc{Gespannt} definiert.
Es ergibt sich das neue Vokalviereck in Tabelle~\ref{tab:vokalviereckmitgespannt}, das um den Preis erkauft wird, dass \textipa{[E]} und \textipa{[a]} jeweils bald als gespannte, bald als ungespannte Variante angesetzt werden.

\begin{exe}
  \ex \textsc{Gespannt}: $+$, $-$
\end{exe}

\begin{table}
  \centering
  \begin{tabular}{cp{2mm}p{2mm}cp{5mm}cp{5mm}cp{5mm}cp{5mm}cp{2mm}}
   \lsptoprule
   \multicolumn{2}{c}{} & \multicolumn{5}{c}{\textbf{vorne}} & \textbf{zentral} & \multicolumn{5}{c}{\textbf{hinten}} \\
   &&& && && && && & \\
   \multirow{3}{*}{\textbf{hoch}} &&& \Dim \rnode{i}{i} &&   &&   &&   &&   &\\
   &&& \Dim \rnode{y}{y} &&  \rnode{I}{\textipa{I}} & &   & &   && \Dim \rnode{u}{u} &\\
   &&& \Dim &&  \rnode{Y}{\textipa{Y}} &&   &&  \rnode{U}{\textipa{U}} &&  \Dim &\\
   &&& \Dim &&   &&   &&   &&  \Dim &\\
\cline{8-8}
   \multirow{3}{*}{\textbf{mittel}} &&& \Dim \rnode{e}{e} &&   && \multicolumn{1}{|c|}{\textipa{@}} &&   && \Dim \rnode{o}{o} &\\
   &&& \Dim \rnode{oe}{\textipa{\o}} &&  \rnode{OE}{\textipa{\oe}} && \multicolumn{1}{|c|}{\textipa{5}} &&   &&   &\\
\cline{8-8}
   &&& \Dim \rnode{E}{\textipa{E}} && \rnode{Eugs}{\textipa{E}} &&  &&   && \rnode{O}{\textipa{O}}  &\\
   \multirow{5}{*}{\textbf{tief}} &&&  &&   &&   &&   &&   &\\
   &&&   &&   &  &  \rnode{augs}{a} & &   &&   &\\
   &&&   &&   &&   &&   &&   &\\
   &&&   &&   &&\Dim \rnode{a}{a} &&   &&   &\\
   &&& && && && && & \\
  \lspbottomrule
  \end{tabular}
  \caption{Phonologisches Vokalviereck; Grau für \textsc{Gespannt}: $+$}
  \label{tab:vokalviereckmitgespannt}
  \ncline[nodesep=3pt]{-}{i}{I}
  \ncline[nodesep=3pt]{-}{y}{Y}
  \ncline[nodesep=3pt]{-}{e}{Eugs}
  \ncline[nodesep=3pt]{-}{E}{Eugs}
  \ncline[nodesep=3pt]{-}{oe}{OE}
  \ncline[nodesep=3pt]{-}{u}{U}
  \ncline[nodesep=3pt]{-}{o}{O}
  \ncline[nodesep=3pt]{-}{a}{augs}
\end{table}

Die Vokale in den ersten Silben von \textit{Liebe} \textipa{[li:b@]}, \textit{Tüte} \textipa{[ty:t@]}, \textit{Wut} \textipa{[vu:t]}, \textit{Weg} \textipa{[ve:k]}, \textit{schön} \textipa{[S\o:n]}, \textit{Käse} \textipa{[kE:z@]}, \textit{rot} \textipa{[ro:t]}, \textit{rate} \textipa{[Ka:t@]} gelten also gemäß dieser leicht veränderten Merkmalsmenge als \textit{gespannt}.
In diesen Beispielen sind sie betont und daher lang.
Ungespannte Vokale können zwar betont werden, aber sie werden dadurch nicht lang, \zB in \textit{Rinder} \textipa{[KInd5]}.
Formen wie *\textipa{[KI:nd5]} sind ausgeschlossen.
Man kann versuchen, die Kategorie der Gespanntheit mit einer erhöhten Muskelanspannung oder einer Veränderung der Position der Zungenwurzel in Verbindung zu bringen.
Aus Sicht der Phonologie ist der \textit{systematische} Aspekt aber wichtiger.
Für die gespannten Vokale gelten gemeinsame Strukturbedingungen, und daher sollte sie die Grammatik in jedem Fall als eine Gruppe auffassen, genauso wie die stimmhaften und stimmlosen Obstruenten usw.
Mit den Ortsmerkmalen der Vokale und der Lippenrundung kann man die gespannten (und damit längbaren) Vokale aber nicht von den anderen absondern. 
Die damit einhergehende partielle Ablösung von der reinen phonetischen Basis rechtfertigt auch die Annahme von je einem gespannten und einem ungespannten \textipa{[a]} und \textipa{[E]}.
Immerhin ist das gespannte \textipa{[a]} phonetisch nicht von dem ungespannten \textipa{[a]} unterscheidbar, und Gleiches gilt für gespanntes und ungespanntes \textipa{[E]}.

Weil die halbvorderen und halbhinteren Vokale jetzt durch die Gespanntheit von den vorderen und hinteren unterscheidbar werden, kann ein weiteres Merkmal in seinen möglichen Werten reduziert werden.

\begin{exe}
  \ex \textsc{Lage}: \textit{vorne}, \textit{zentral}, \textit{hinten}
\end{exe}

Die bemerkenswerten Zusammenhänge werden jetzt auf den Punkt gebracht und zusammengefasst.
Je nach Auffassung, was der Kernwortschatz ist, gilt im Kernwortschatz (auf jeden Fall aber im Erbwortschatz), dass gespannte Vokale immer betont und damit immer lang sind.
Innerhalb des Kernwortschatzes gibt es damit die in Tabelle~\ref{tab:vokalviereckmitgespannt} durch Striche markierten Paare aus langen gespannten betonten und kurzen ungespannten betonten oder unbetonten Vokalen.
Während die ungespannten betont oder unbetont auftreten können, sind die gespannten immer betont.

\begin{table}
	\centering
	\begin{tabular}{clcl}
		\lsptoprule
		\textbf{gespannt} & \textbf{Beispiel} & \textbf{ungespannt} & \textbf{Beispiel} \\
		\midrule
		\textipa{[i]}  & \textit{bieten} \textipa{[bi:t@n]} & \textipa{[I]} & \textit{bitten} \textipa{[bIt@n]} \\
		\textipa{[y]}  & \textit{fühlt} \textipa{[fy:lt]} & \textipa{[Y]} & \textit{füllt} \textipa{[fYlt]} \\
		\textipa{[u]}  & \textit{Mus} \textipa{[mu:s]} & \textipa{[U]} & \textit{muss} \textipa{[mUs]} \\
		\textipa{[e]}  & \textit{Kehle} \textipa{[ke:l@]} & \textipa{[E]} & \textit{Kelle} \textipa{[kEl@]} \\
		\textipa{[E]}  & \textit{stähle} \textipa{[StE:l@]} & \textipa{[E]} & \textit{Stelle} \textipa{[StEl@]} \\
		\textipa{[\o]} & \textit{Höhle} \textipa{[h\o:l@]} & \textipa{[\oe]} & \textit{Hölle} \textipa{[h\oe l@]} \\
		\textipa{[o]}  & \textit{Ofen} \textipa{[o:f@n]} & \textipa{[O]} & \textit{offen} \textipa{[Of@n]} \\
		\textipa{[a]}  & \textit{Wahn} \textipa{[va:n]} & \textipa{[a]} & \textit{wann} \textipa{[van]} \\
		\lspbottomrule
	\end{tabular}	
  \caption{Gespannte und ungespannte Vokale im Kernwortschatz}
  \label{tab:gespungesp}
\end{table}

\Satz{Gespanntheit im Kernwortschatz}{Im Kernwortschatz sind gespannte Vokale immer betont und lang.
Zu jedem gespannten Vokal gibt es einen entsprechenden ungespanten Vokal.
Der ungespannte ist betont oder unbetont, aber auf jeden Fall immer kurz.}

Im erweiterten Wortschatz, der mehr Vielsilbler enthält, gilt die eingangs erwähnte Strukturbedingung, dass bei den gespannten Vokalen die Betonung die Länge kontrolliert.
Beispiele für kurze unbetonte gespannte Vokale sind \textipa{[o]} und \textipa{[i]} in der jeweils ersten Silbe der Wörter \textit{Politik} \textipa{[politIk]} (bei manchen Sprechern \textipa{[politi:k]}), die \textipa{[o]}-Segmente in \textit{Phonologie} \textipa{[fonologi:]} und \textipa{[e]} in \textit{Methyl} \textipa{[mety:l]}.
Weil diese Wörter im alltäglichen Gebrauch durchaus häufig vorkommen, wird hier nicht von \textit{peripherem Wortschatz}, sondern vorsichtiger vom \textit{erweiterten Wortschatz} gesprochen.

\Satz{Gespanntheit im erweiterten Wortschatz}{Im erweiterten Wortschatz sind gespannte Vokale lang, wenn sie betont sind und kurz, wenn sie unbetont sind.
Es gibt keine ungespannten langen Vokale.}

Völlig außerhalb dieses Systems steht das Schwa und seine Variante \textipa{[5]}.

\Satz{Schwa}{Schwa ist immer kurz und nie betont.}

Damit müssen zugrundeliegenden Formen genau wie bei der Auslautverhärtung gemäß der neu eingeführten Strukturbedingungen angepasst werden.
Man erhält zum Beispiel (\ref{ex:phol013}).

\begin{exe}
  \ex\label{ex:phol013} \begin{xlist}
  	\ex /\textipa{veg}/ \phopro \textipa{[ve:g]}
  	\ex /\textipa{h\o l@}/ \phopro \textipa{[h\o:l@]} 
  	\ex /\textipa{of@n}/ \phopro \textipa{[o:f@n]}
  \end{xlist}
\end{exe}


\subsubsection{Verteilung von [ç] und [χ]}

\label{sec:prozichach}

Die sogenannten \textit{ich}- und \textit{ach}-Segmente sind komplementär verteilt.
Es gibt kein Wort, in dem sie einen lexikalischen Unterschied markieren.
Einige Beispielwörter, in denen \textipa{[\c{c}]} und \textipa{[X]} vorkommen, illustrieren dies in (\ref{ex:phol6110}).

\begin{exe}
  \ex\label{ex:phol6110}
  \begin{xlist}
    \ex{\label{ex:phol6110a} rieche, Bücher, schlich, Gerüche, Wehwehchen, röche, schlecht, Löcher}
    \ex{\label{ex:phol6110b} Tuch, Geruch, hoch, Loch, Schmach, Bach.}
  \end{xlist}
\end{exe}

Ausschlaggebend für das Vorkommen von \textipa{[\c{c}]} und \textipa{[X]} ist der unmittelbar vorangehende Kontext.
Nach /\textipa{i}/, /\textipa{y}/, /\textipa{I}/, /\textipa{Y}/, /\textipa{e}/, /\textipa{\o}/, /\textipa{E}/, /\textipa{E}/, /\textipa{\oe}/ kommt \textipa{[\c{c}]} vor, nach /\textipa{u}/, /\textipa{U}/, /\textipa{o}/, /\textipa{O}/ und /\textipa{a}/ hingegen \textipa{[X]}.
Nach Schwa kommt keins der beiden Segmente vor.
Ein Blick auf das phonologische Vokalviereck in Abbildung~\ref{tab:vokalviereckmitgespannt} zeigt sofort, was der relevante Merkmalsunterschied zwischen den beiden Gruppen von Vokalen ist.
Nach Vokalen, die [\textsc{Lage}: \textit{vorne}] sind, steht \textipa{[\c{c}]}.
Nach allen anderen Vokalen steht hingegen \textipa{[X]}.
Dabei handelt es sich um eine Angleichung des Artikulationsorts des Frikativs an den des Vokals, eine sogenannte \textit{Assimilation}.\index{Assimilation}

Es muss jetzt nur noch entschieden werden, wie die zugrundeliegende Form in diesem Fall aussieht.
Aufschlussreich ist hier die Betrachtung von Wörtern wie \textit{Milch} /\textipa{mIl\c{c}}/, \textit{Storch} /\textipa{StOK\c{c}}/ oder \textit{Röckchen} /\textipa{K\oe k\c{c}@n}/, in denen \textipa{[\c{c}]}, aber niemals \textipa{[X]} nach einem Konsonanten vorkommt.
Es ist also günstiger, anzunehmen, dass /\textipa{\c{c}}/ zugrundeliegt und \textipa{[X]} das phonetische Resultat einer Assimilation ist.
Das heißt, dass \textipa{[X]} kein zugrundeliegendes Segment ist und nicht in /~/ gehört.
Mit der entsprechenden Strukturbedingung ergeben sich die Beispiele wie in (\ref{ex:phol8011}).

\Definition{/ç/-Assimilation}{/ç/ kann nicht nach Vokalen stehen, die nicht [\textsc{Lage}: \textit{vorne}] sind.}

\begin{exe}
  \ex\label{ex:phol8011}
  \begin{xlist}
    \ex{/\textipa{I\c{c}}/ \phopro \textipa{[PI\c{c}]}}
    \ex{/\textipa{a\c{c}}/ \phopro \textipa{[PaX]}}
  \end{xlist}
\end{exe}

\subsubsection{/ʁ/-Vokalisierungen}

\label{sec:prozrvok}

In Abschnitt~\ref{sec:realisr} wurden verschiedene phonetische Korrelate von geschriebenem \textit{r} besprochen.
Die Schrift ist hier eigentlich besonders systematisch, denn orthographisches \textit{r} entspricht immer einem zugrundeliegenden /\textipa{K}/ (vgl.\ auch Abschnitt~\ref{sec:buchstabensegmente}).
In (\ref{ex:phol9906}) sind einige Beispiele zusammengestellt (inkl.\ Silbengrenzen), die dies illustrieren.

\begin{exe}
  \ex\label{ex:phol9906}
  \begin{xlist}
    \ex{geringer \textipa{[g@.KIN.5]}, geringere \textipa{[g@.KIN.@.K@]}}
    \ex{Bär \textipa{[b\t{E5}]}, Bären \textipa{[bE:.K@n]}}
    \ex{knarr \textipa{[kn\t{a@}]}, knarre \textipa{[kna.K@]}}
  \end{xlist}
\end{exe}

Wenn ein zugrundeliegendes /\textipa{K}/ am Silbenanfang steht, wird es als Konsonant \textipa{[K]} realisiert.
Demgegenüber findet am Silbenende immer eine Vokalisierung von /\textipa{K}/statt.
Nach gespannten Vokalen wird /\textipa{K}/ zu \textipa{[5]}, nach ungespannten zu \textipa{[@]}.
Nach (stets unbetontem) Schwa wird /\textipa{K}/ gar nicht realisiert, und Schwa wird zu \textipa{[5]}.
Diese Vorgänge formal genau aufzuschreiben, würde den hier gegebenen Rahmen sprengen.
Aus Sicht der Phonologie sind aber auf jeden Fall die Unterschiede zwischen \textipa{[@]} und \textipa{[5]} nicht sonderlich erheblich, stellen sie doch nur minimal unterschiedliche Färbungen des Schwa-Segments dar.
Die entsprechende Strukturbedingung und ihre Effekte werden daher nur grob in Definition~\ref{def:rvokalisierung} beschrieben.
Beispiele folgen in (\ref{ex:phol696969}).

\Definition{/ʁ/-Vokalisierung}{
\label{def:rvokalisierung}
Zugrundeliegendes /\textipa{K}/ kann nicht am Silbenende stehen.
Es wird als (ggf.\ gefärbtes) Schwa-Segment realisiert.}

\begin{exe}
  \ex \label{ex:phol696969}
  \begin{xlist}
  	\ex /\textipa{[g@KIN@K]}/ \phopro \textipa{[g@.KIN.5]}
  	\ex /\textipa{[bEK]}/ \phopro \textipa{[b\t{E5}]}
  	\ex /\textipa{[knaK]}/ \phopro \textipa{[kn\t{a@}]}
  \end{xlist}
\end{exe}



% ==================================




\section{Silben und Wörter}

\label{sec:phonotaktik}

\subsection{Phonotaktik}

Aufbauend auf die Beschreibung der einzelnen Segmente des Deutschen (Kapitel~\ref{sec:phonetik}) kann und sollte beschrieben werden, wie diese Segmente zu größeren Einheiten zusammengesetzt werden, wie also \textit{phonologische Struktur} (zum Strukturbegriff vgl.\ Abschnitt~\ref{sec:strukturen}, S.~\pageref{sec:strukturen}) aufgebaut wird.
Die Wörter in (\ref{ex:phol2852}) sind Phantasiewörter in Standardorthographie und hypothetischer phonetischer Umschrift.

\begin{exe}
  \ex\label{ex:phol2852}
  \begin{xlist}
    \ex{\label{ex:phol2852a} Nka \textipa{[Nka:]}, Totk \textipa{[tOtk]}, Pkafkme \textipa{[pkafkm@]}}
    \ex{\label{ex:phol2852b} Klie \textipa{[kli:]}, Filb \textipa{[fIlp]}, Renge \textipa{[KEN@]}}
  \end{xlist}
\end{exe}

Die hypothetischen Wörter in (\ref{ex:phol2852a}) unterscheiden sich deutlich von denen in (\ref{ex:phol2852b}).
Während die zweite Gruppe nämlich zumindest mögliche Wörter des Deutschen darstellt, enthält die erste Gruppe nur Wörter, die aus irgendeinem Grund niemals Wörter des Gegenwartsdeutschen sein könnten.
Der Grund dafür ist, dass die erste Gruppe \textit{phonotaktisch nicht wohlgeformte Wörter} enthält.
Es muss also Regularitäten geben, nach denen sich Segmente des Deutschen zu größeren Einheiten wie Silben und Wörtern zusammensetzen.

\Definition{Phonotaktik}{
Die Phonotaktik beschreibt die Regularitäten, nach denen Segmente zu größeren Strukturen zusammengesetzt werden.
Die Phonotaktik definiert Einheiten wie die \textit{Silbe} und das \textit{Wort}.
\index{Phonotaktik}
}

Die Silbe ist die Einheit, mittels derer die meisten Einschränkungen für mögliche Segmentfolgen formuliert werden können.
Dieser Abschnitt ist daher ausschließlich der Silbe gewidmet.

\subsection{Silben}

\label{sec:silben}

\index{Silbe}

Was Silben genau sind, ist nicht unbedingt leicht zu definieren.
Sie sind intuitiv Einheiten, die größer sein können (aber nicht müssen) als Segmente, aber kleiner sein können (nicht müssen) als Wörter.
Der Extremfall, bei dem Segment, Silbe und Wort zusammenfallen, ist im Deutschen nur möglich, wenn man den Glottalverschluss ignoriert (s.~\ref{sec:glottalverschluss}).
Selbst dann ist dieser Fall im Deutschen im normalen Wortschatz marginal und auf Interjektionen (Rufwörter) wie \textit{oh} \textipa{[Po:]} und \textit{ah} \textipa{[Pa:]} beschränkt.
Wenn man Diphthonge als ein Segment zählt, kommt das Substantiv \textit{Ei} \textipa{[P\t{aE}]} hinzu.
Auch in anderen Sprachen ist dieser Fall eher selten, vergleiche französische Substantive wie \textit{œufs} \textipa{[\o:]} `Eier' (nur im Plural) oder \textit{eau} \textipa{[o:]} `Wasser' sowie das schwedische Substantiv \textit{ö} \textipa{[\o:]} `Insel' (nur im Singular), die insgesamt auch innerhalb ihrer eigenen Sprachsysteme eher Exoten darstellen.%
\footnote{Auf jeden Fall entfällt in diesen Sprachen aber der Glottalverschluss.} 
Meistens bestehen Silben aus mehreren Segmenten, und Wörter bestehen häufig aus mehreren Silben.
Beispiele für einsilbige Wörter aus zwei Segmenten im Deutschen sind \textit{Schuh} \textipa{[Su:]} oder \textit{Tee} \textipa{[te:]}, Beispiele für zweisilbige Wörter aus zweisegmentigen Silben sind \textit{Tüte} \textipa{[ty:t@]} oder \textit{rege} \textipa{[Ke:g@]}.
Ein einsilbiges Wort mit deutlich mehr als zwei Segmenten ist \textit{Strauch} \textipa{[StK\t{aO}X]}. 
Die wesentliche Frage der Silbenphonologie wird sein, wie hoch die Komplexität solcher Strukturen maximal sein kann.

\index{Silbe!Klatschmethode}

In der Grundschuldidaktik wird oft über die \textit{Klatschmethode} versucht, Kindern ein Gefühl für Silben zu vermitteln.
Dabei wird gesagt, dass jedes Stück eines Wortes, zu dem man bei abgehacktem Sprechen einmal klatschen kann, eine Silbe sei.
Diese Methode ist problematisch, da sie sehr leicht absichtlich oder unabsichtlich sabotierbar ist:
Es ist für viele Sprecher vielleicht natürlicher, auf Wörter wie \textit{Mutter} \textipa{[mUt5]} nur einmal zu klatschen, da die Silbe mit dem \textipa{[5]} unbetont und phonologisch nicht sehr prominent ist.
Außerdem wird mit der Methode meist ein rein orthographisch-didaktisches Ziel ohne jede Sensibilität für Grammatik verfolgt, nämlich das Erlernen der Silbentrennung in der Schrift.
Die Regeln der orthographischen Silbentrennung im Deutschen erfordern aber subtilere Kenntnisse grammatischer Regularitäten, als sie die Klatschmethode vermitteln kann.
Ein Kind wird durch das Klatschen vielleicht intuitiv lernen, dass Wörter wie \textit{Kriecher}, \textit{Iglu} oder \textit{Mutter} aus genau zwei Silben bestehen.
Ob die Silbentrennung aber \textit{Krie-cher} oder \textit{Kriech-er}, \textit{I-glu} oder \textit{Ig-lu} und \textit{Mutt-er}, \textit{Mut-ter} oder \textit{Mu-tter} ist, ist prinzipiell durch Klatschen nicht erlernbar.
Daher müssen Lehrer bei solchen Übungen dann unnatürliche Aussprachen vormachen, \zB \textipa{[mUt]} -- \textipa{[ta]} oder gar \textipa{[mUt]} -- \textipa{[tEK]} statt korrekt \textipa{[mUt5]}.
Gerade dieses Abhacken macht \textit{Kriech-er} aber genauso plausibel wie \textit{Krie-cher}.
In Fall der zerhackten Aussprache in Fällen mit orthographischen Doppelkonsonanten wie \textipa{[mUt]} -- \textipa{[ta]} muss man zudem paradoxerweise bereits Kenntnisse der Orthographie und Silbentrennung besitzen, um das Wort überhaupt so aussprechen zu können.
Man dreht sich also im Kreis, und ein solider Lernerfolg durch das Klatschen ist daher nicht zu erwarten.%
\footnote{Aus meiner eigenen -- zugegebenermaßen länger zurückliegenden -- Grundschulerfahrung als Schüler mit zwei Lehrerinnen in zwei verschiedenen Bundesländern läuft die Unterrichtseinheit dann so ab, dass einige Kinder aus Haushalten mit hohem Bildungsniveau bereits seit längerem lesen können und die Silbentrennung durch Anschauung beim Lesen intuitiv gelernt haben.
Diese Kinder verstehen in den Augen des Lehrpersonals durch das Klatschen, wie Wörter zu trennen sind.
Alle andere Kinder gelten ohne ihr Verschulden als schwierig bzw.\ langsame Lerner.
Diese Beobachtung hat natürlich keinen Anspruch auf Allgemeingültigkeit.}

Trotz ihrer absoluten Unzulänglichkeit für den Grundschulunterricht veranschaulicht die Klatschmethode (recht umständlich) allerdings ein wichtiges Prinzip der Silbenbildung.
Silben bringen die Segmente in eine schematische Ordnung, die charakteristischen artikulatorischen Einheiten entspricht.
Diese artikulatorischen Einheiten sind Schübe, die im Prinzip einem Öffnen und Schließen des Vokaltraktes entsprechen.
An einsilbigen Wörtern wie \textit{Tag} \textipa{[ta:k]} oder \textit{gut} \textipa{[gU:t]} sieht man, dass sie mit einem Verschluss beginnen und mit einem Verschluss enden, während in der Mitte beim Vokal der Vokaltrakt geöffnet ist (genauer in Abschnitt~\ref{sec:sonoritaet}).
Im Kern der Silbe befindet sich passend dazu im Deutschen immer ein Vokal, also ein Segment bei dem sich die Artikulatoren nicht punktuell annähern (Abschnitt~\ref{sec:vokale}).
Die Klatschmethode kann man also auf die Anweisung reduzieren, bei jedem Vokal einmal zu klatschen.
Wie an den Zweifelsfällen weiter oben gezeigt wurde, löst das aber nicht das Problem, ob Konsonanten zwischen den Vokalen in mehrsilbigen Wörtern zur ersten oder zweiten Silbe gehören.

Komplizierter wird die Silbenphonologie dadurch, dass in den Formen eines Wortes die Silbengrenzen nicht konstant sind.
Anders gesagt ist die Silbenstruktur von Wörtern nicht im Lexikon festgelegt.
Die Beispiele (\ref{ex:phol1830}) zeigen dies.
In der Transkription werden die Silbengrenzen durch einen einfachen Punkt markiert.

\begin{exe}
  \ex\label{ex:phol1830}
  \begin{xlist}
    \ex{Ball \textipa{[bal]}, Bälle \textipa{[bE.l@]}}
    \ex{Knall \textipa{[knal]}, Knalls \textipa{[knals]}}
    \ex{Sturm \textipa{[St\t{U@}m]}, Stürme \textipa{[St\t{Y@}.m@]}}
    \ex{Mittelstürmer \textipa{[mI.t@l.St\t{Y@}.m5]}, Mittelstürmerin \textipa{[mI.t@l.St\t{Y@}.m@.KIn]}}
  \end{xlist}
\end{exe}

Ein Wort wie \textit{Ball} ist im Nominativ Singular einsilbig, und das \textipa{[l]} steht im Auslaut (am Ende) dieser einen Silbe.
Mit dem hinzutretenden \textipa{[@]} der Plural-Endung verändert sich auch die Silbenstruktur:
Das \textipa{[l]} steht im Anlaut (am Anfang) der zweiten Silbe.
Ähnliches passiert bei \textit{Sturm} und \textit{Stürme} mit dem \textipa{[m]}.
Bei \textit{Mittelstürmer} \textipa{[mI.t@l.St\t{Y@}.m5]} und \textit{Mittelstürmerin} \textipa{[mI.t@l.St\t{Y@}.m@.KIn]} wird es noch komplizierter, weil /\textipa{K}/ nur dann als Konsonant \textipa{[K]} realisiert wird, wenn noch ein Vokal folgt und das /\textipa{K}/ dadurch in den Silbenanlaut gerät (vgl.\ dazu genauer Abschnitt~\ref{sec:prozrvok}).
Wenn bei \textit{Ball} und \textit{Balls} aber ein \textipa{[s]} hinzutritt, bleibt das Wort einsilbig, und das \textipa{[s]} wird an die einzige Silbe hinten angehängt.
Die Silbenbildung, so wie sie hier betrachtet wird, ist also keine phonetische Fragestellung, sondern eine phonologische, weil ihre Beschreibung es erfordert, dass das Gesamtsystem (also \zB alle Formen eines Wortes) betrachtet werden.
Entsprechend wird Definition~\ref{def:silbe} gegeben.

\Definition{Silbe und Silbifizierung}{\label{def:silbe}
Silben sind die nächstgrößeren phonologischen Einheiten nach den Segmenten.
Die Segmente sind ihre kleinsten Konstituenten.
Die Silbenstruktur ist nicht im Lexikon abgelegt und wird durch einen Prozess zugewiesen (Silbifizierung).
\index{Silbe}
}

Mit Klatschen ist es also nicht getan.
Der analytische Einstieg in die Silbenstruktur des Deutschen gelingt am leichtesten über einsilbige Wörter.
Abschnitt~\ref{sec:einsilbler} leistet (nach der Einführung einiger technischer Begriffe in Abschnitt~\ref{sec:silbenstruktur}) daher zunächst eine einfache Beschreibung möglicher einsilbiger Wörter des Deutschen.
Die Verallgemeinerung zu mehrsilbigen Wörtern erfolgt nach einer theoretischen Ergänzung (Abschnitte~\ref{sec:sonoritaet}) in Abschnitt~\ref{sec:mehrsilbler}.






\subsection{Strukturformat für Silben}

\label{sec:silbenstruktur}

In diesem Abschnitt wird nur die Terminologie eingeführt, mit der man über Positionen in der Silbe redet.
Offensichtlich bilden Silben komplexere Strukturen aus, die sich um einen Vokal oder Diphthong im \textit{Kern} herum gruppieren.%
\footnote{Eine alternative Sichtweise würde bei Diphthongen das zweite Glied als Teil der Coda analysieren.
Für unsere Zwecke ist der sich ergebende theoretische Unterschied vernachlässigbar.}
Für die drei sich ergebenden Konstituenten der Silbe gibt es verschiedene Bezeichnungen, von denen hier \textit{Anfangsrand}, \textit{Kern} und \textit{Endrand} verwendet werden.
Aus Gründen, die erst in Abschnitt~\ref{sec:mehrsilbler} diskutiert werden, hat es sich als nützlich erwiesen, Kern und Endrand zu einer eigenen Konstituente, dem \textit{Reim} zusammenzufassen.
Neben Definition~\ref{def:kern} wird eine Baumdarstellung der allgemeinen Silbenstruktur in Abbildung~\ref{fig:silbenstruktur} und ein Beispiel (\textit{fremd}) in Abbildung~\ref{fig:phonstr} gegeben.

\Definition{Silbenstruktur}{
\label{def:kern}
\label{def:anfangsrand}
\label{def:endrand}
Der \textit{Silbenkern} (der \textit{Nukleus}) wird immer durch einen Vokal oder Diphthong gebildet.
Die Konsonanten vor dem Kern bilden den \textit{Anfangsrand} (den \textit{Onset}), die nach dem Kern den \textit{Endrand} (die \textit{Coda}).
Es gibt Silben mit leeren Anfangs- und\slash oder Endrändern, aber keine Silben ohne Kern.
Kern und Endrand bilden den \textit{Reim}.
\index{Silbe!Kern}
\index{Silbe!Anfangsrand}
\index{Silbe!Endrand}
\index{Silbe!Reim}
}

\begin{figure}
  \centering
  \Tree[2]{
    & \K{Silbe}\B{dl}\B{d} \\
    \K{Anfangsrand} & \K{Reim}\B{d}\B{dr} \\
    & \K{Kern} & \K{Endrand}\\
  }
  \caption{Silbenstruktur}
  \label{fig:silbenstruktur}
\end{figure}

\begin{figure}
  \centering
  \Tree[1.5]{
  & && \K{Silbe}\B{dll}\B{d} \\
  & \K{Anfangsrand}\B{ddl}\B{ddr} && \K{Reim}\B{d}\B{drr} \\
  &&& \K{Kern}\B{d} && \K{Endrand}\B{dl}\B{dr} \\
  \K{\textipa{[f]}} && \K{\textipa{[K]}} & \K{\textipa{[E]}} & \K{\textipa{[m]}} && \K{\textipa{[t]}} \\
  }
  \caption{Beispiel für Silbenstruktur}
  \label{fig:phonstr}
\end{figure}








\subsection{Einsilbler}

\label{sec:einsilbler}

In diesem Abschnitt werden einsilbige Wörter herangezogen, um die minimale und die maximale Komplexität deutscher Silben zu ermitteln.%
\footnote{Es ist für die meisten Menschen unmöglich, eine solche Beschreibung durch einmaliges oder zweimaliges Lesen zu memorieren.
Dies sollte beim Lesen zu beachtet werden, um Frustrationen zu vermeiden.
Es geht hier um eine möglichst genaue Darstellung der \textit{Methode}, mit der die möglichen Silbentypen einer Sprache ermittelt werden.
Eine bessere Übersicht bietet \textit{nach} der Lektüre dieses Abschnitts dann Abschnitt~\ref{sec:anfangsrandendrand}.}
Ein einsilbiges Wort wird üblicherweise \textit{Einsilbler} genannt.

\subsubsection{Anfangsrand}

In Abschnitt~\ref{sec:silben} wurde bereits festgestellt, dass Silben -- und damit Einsilbler -- mindestens aus einem Vokal im Silbenkern bestehen.
Gleichzeitig enthält eine Silbe immer genau einen (niemals zwei oder mehr) Vokale.
Diesem Vokal geht im Deutschen immer der Glottalverschluss voraus, wenn kein anderer Konsonant vorausgeht.
Maximal einfache Einsilbler sind also die (\ref{ex:phol777200}), wobei Diphthonge wie ein einfacher Vokal behandelt werden.

\begin{exe}
	\ex\label{ex:phol777200}
	\begin{xlist}
		\ex Ei \textipa{[P\t{aE}]}	
		\ex ah \textipa{[Pa:]}	
		\ex oh \textipa{[Po:]}	
	\end{xlist}
\end{exe}

Wir beginnen mit dem Anfangsrand und überlegen der Reihe nach, ob dort ein, zwei oder auch mehr Segmente stehen können, und falls es so ist, welche und in welcher Reihenfolge.
Der Anfangsrand kann durch ein einzelnes konsonantisches Segment einer beliebigen Artikulationsart besetzt werden.
In (\ref{ex:phol777201a}) sind es stimmlose und stimmhafte Plosive, in (\ref{ex:phol777201b}) stimmlose und stimmhafte Frikative (bis auf \textipa{[X]} bzw.\ \textipa{[\c{c}]}), in (\ref{ex:phol777201c}) Nasale (bis auf \textipa{[N]}) und in (\ref{ex:phol777201d}) der Approximant.
Der Nasal \textipa{[N]} sowie die Frikative \textipa{[\c{c}]} und \textipa{[X]} kommen prinzipiell im Anfangsrand nicht vor und werden aus allen weiteren Überlegungen über diese Position ausgeschlossen.%
\footnote{Beispielwörter, die in diesem Abschnitt unmögliche Kombinationen illustrieren sollen, werden pseudo-orthographisch und in IPA-Transkription wiedergegeben.
Der Asterisk * steht dabei immer nur vor der pseudo-orthographischen Version.
Es ist zu beachten, dass die entsprechenden Wörter nicht einfach nur durch Zufall nicht existieren.
Sie könnten vielmehr keine Wörter des Deutschen sein, weil das System die entsprechenden Silbenstrukturen nicht zulässt.}

\begin{exe}
	\ex\label{ex:phol777201}
	\begin{xlist}
		\ex{\label{ex:phol777201a} Kuh \textipa{[ku:]}, geh \textipa{[ge:]}}
		\ex{\label{ex:phol777201b} Schuh \textipa{[Su:]}, hau \textipa{[h\t{aO}]}, Reh \textipa{[Ke:]}, Vieh \textipa{[fi:]}, wo \textipa{[vo:]}}, *chie \textipa{[\c{c}i:]}\slash\textipa{[Xi:]}
		\ex{\label{ex:phol777201c} nie \textipa{[ni:]}, mäh \textipa{[mE:]}, *ngu \textipa{[Nu:]}}
		\ex{\label{ex:phol777201d} lau \textipa{[l\t{aO}]}}
	\end{xlist}
\end{exe}

Da Einsilbler also immer mindestens aus einem Vokal im Kern und einem Glottalverschluss bzw.\ anderen Konsonanten im Anfangsrand bestehen, beginnt die artikulatorische Bewegung mit einem Verschluss und führt auf jeden Fall zu einer maximalen Öffnung. 
Wenn im Anfangsrand \textit{zwei} Konsonanten stehen, sind die Kombinationsmöglichkeiten bereits erheblich eingeschränkt.
In unseren Überlegungen setzen wir jetzt jeweils (in dieser Reihenfolge) Plosive, Frikative, Nasale und Approximanten als zweites Segment ein und überlegen, welche Segmente dann davor stehen können.
Die Beispiele sind möglichst so gewählt, dass rechts vom Vokal nichts steht, aber wenn ein solches Beispiel zufällig nicht existiert, wird auf andere Einsilbler (im Notfall auf Mehrsilbler) ausgewichen.

Plosive an zweiter Position sind im zweisegmentalen Anfangsrand nahezu unmöglich (\ref{ex:phol777202a}) mit der Ausnahme von \textipa{[p]} und \textipa{[t]} nach \textipa{[S]} (\ref{ex:phol777202a}).
Es gibt jedoch sehr seltene Lehnwörter (meist keine Einsilbler), die abweichende Konsonantenverbindungen links vom Vokal enthalten.
Diese wenigen Ausnahmen wie in (\ref{ex:phol777202c}) sind wegen dieses ungewöhnlichen Silbenbaus nicht zum Kern des Systems zu rechnen (Abschnitt~\ref{sec:kern}).
Sie sind also nicht nur Lehnwörter, sondern auch Fremdwörter.
Wörter wie \textit{stygisch} sind im Übrigen nur dann betroffen, wenn \textipa{[st]} statt \textipa{[St]} gesprochen wird.

\begin{exe}
	\ex\label{ex:phol777202}
	\begin{xlist}
		\ex{\label{ex:phol777202a} *pteh \textipa{[pte:]}, *fpeh \textipa{[fpe:]}, *schguh \textipa{[Sgu:]}, *lta \textipa{[lta:]} usw.}
		\ex{\label{ex:phol777202b} spei \textipa{[Sp\t{aE}]}, steh \textipa{[Ste:]} }
		\ex{\label{ex:phol777202c} Pte(ranodon) \textipa{[pteKanodOn]}, chtho(nisch) \textipa{[Xto:nIS]}, sty(gisch) \textipa{[sty:gIS]}}
	\end{xlist}
\end{exe}

Frikative an zweiter Position kommen nur eingeschränkt vor.
Da wir \textipa{[\t{pf}]} wie in \textit{Pfau} und \textipa{[\t{ts}]} wie in \textit{zieh} sowie das seltene \textipa{[\t{tS}]} wie in \textit{Chips} als Affrikaten (also ein einziger Konsonant) auffassen (Abschnitte~\ref{sec:affrikatenhomorgan} und~\ref{sec:affrikaten}), fallen die Frikative \textipa{[f]}, \textipa{[s]}, \textipa{[S]}, \textipa{[h]}, \textipa{[z]} und \textipa{[J]} komplett als zweites Segment im Anfangsrand aus (\ref{ex:phol777203a}).%
\footnote{Außerdem kann die Kombination \textipa{[tJ]} bzw.\ \textipa{[t\c{c}a]} wie in \textit{tja} \textipa{[tJa]} (oder \textipa{[t\c{c}a]})  oder dem norddeutschen Namen \textit{Tjark} \textipa{[tJ\t{a@}k]} (oder \textipa{[t\s{c}\t{aE}k]}) zum System gerechnet werden oder nicht.
Wesentliches ändert sich dadurch nicht, und wir sehen hier davon ab.}
Es kommt \textipa{[K]} vor, aber nur nach den Plosiven \textipa{[f]}, \textipa{[S]} und \textipa{[v]} (\ref{ex:phol777203b}).
Außerdem findet man \textipa{[v]}, aber nur nach \textipa{[k]} und \textipa{[S]} (\ref{ex:phol777203c}).

\begin{exe}
	\ex\label{ex:phol777203}
	\begin{xlist}
		\ex{\label{ex:phol777203a} *ksie \textipa{[ksi:]}, *tfa \textipa{[tfa:]}, *gsau \textipa{[gz\t{aO}]} usw.}
		\ex{\label{ex:phol777203b} Pracht \textipa{[pKaXt]}, brüh \textipa{[bKy:]}, trau \textipa{[tK\t{aO}]}, dreh \textipa{[dKe:]}, kräh \textipa{[kKE:]},\\
		   grau \textipa{[gK\t{aO}]}, früh \textipa{[fKy:]}, Schrei \textipa{[SK\t{aE}]}, Wrack \textipa{[vKak]}}
		\ex{\label{ex:phol777203c} Qual \textipa{[kva:l]}, Schwur \textipa{[Sv\t{u5}]}}
	\end{xlist}
\end{exe}

Nasale an zweiter Position kommen ebenfalls kaum vor, sowohl nach Plosiven (\ref{ex:phol777204a}) als auch nach Frikativen (\ref{ex:phol777204b}).
Die einzigen systematischen Ausnahmen sind \textipa{[kn]} und selten \textipa{[gn]} (\ref{ex:phol777204c}) sowie \textipa{[Sn]} und \textipa{[Sm]} (\ref{ex:phol777204d}).%
\footnote{Wörter mit \textipa{[pn]} sind extrem seltene Lehnwörter wie \textit{Pneu}.
Das einzige häufiger vorkommende Erbwort mit \textipa{[gn]} in einem Anfangsrand ist \textit{Gnade}.
Alle anderen Wörter (\zB dialektal gefärbte wie \textit{Gnatz} und \textit{Gnitze} oder Lehnwörter wie \textit{Gnom} oder \textit{Gnosis}) sind selten.
Ob der Anlaut \textipa{[gn]} also zum Kern gehört, kann bezweifelt werden.}

\begin{exe}
	\ex\label{ex:phol777204}
	\begin{xlist}
		\ex{\label{ex:phol777204a} *pmeh \textipa{[pme:]}, *bnau \textipa{[bn\t{aO}]}, *tneh \textipa{[tne:]} usw.}
		\ex{\label{ex:phol777204b} *fnau \textipa{[fn\t{aO}]}, *smuh \textipa{[smu:]}, *rnie \textipa{[Kni:]} usw.}
		\ex{\label{ex:phol777204c} Knie \textipa{[kni:]} (Gnade \textipa{[gna:de]})}
		\ex{\label{ex:phol777204d} Schnee \textipa{[Sne:]}, schmäh \textipa{[SmE:]}}
	\end{xlist}
\end{exe}

Der einzige laterale Approximant des Deutschen (\textipa{[l]}) an zweiter Position ist im Wesentlichen auf zwei Fälle beschränkt.
Er steht nach allen Plosiven mit Ausnahme der alveolaren (\ref{ex:phol777205a}).
Außerdem findet man ihn nach den stimmlosen Frikativen \textipa{[f]} und \textipa{[S]} (\ref{ex:phol777205b}).

\begin{exe}
	\ex\label{ex:phol777205}
	\begin{xlist}
		\ex{\label{ex:phol777205a} Plan \textipa{[pla:n]}, blüh \textipa{[bly:]}, *tlüh \textipa{[tly:]}, *dlüh \textipa{[dly:]}, Klee \textipa{[kle:]}, glüh \textipa{[gly:]}}
		\ex{\label{ex:phol777205b} flieh \textipa{[fli:]}, Schlag \textipa{[Sla:k]}}
	\end{xlist}
\end{exe}

Die strukturellen Möglichkeiten für dreisegmentale Anfangsränder sind auf \textipa{[SpK]} und \textipa{[StK]} beschränkt (\ref{ex:phol777206a}).
Die wenigen (nicht einsilbigen) Wörter mit \textipa{[Spl]} (\ref{ex:phol777206b}) gehören wohl alle zur selben germanischen Grundform, sind dabei dialektal gefärbt bzw.\ aus dem Englischen entlehnt und können als peripher vernachlässigt werden.

\begin{exe}
	\ex\label{ex:phol777206}
	\begin{xlist}
		\ex{\label{ex:phol777206a} sprüh \textipa{[SpKy:]}, Stroh \textipa{[Stro:]}}
		\ex{\label{ex:phol777206b} Splitter \textipa{[SplIt5	]}, spleiß \textipa{[Spl\t{aE}s]}, Spliss \textipa{[SplIs]}}
	\end{xlist}
\end{exe}




\subsubsection{Endrand}

\label{sec:endranddeskriptiv}

Der Endrand wird jetzt etwas kompakter abgearbeitet als der Anfangsrand, zumal er insgesamt komplexer ist.
Auf die IPA-Transkription und die Auflistung strukturell unmöglicher Pseudo-Beispiele wird dabei aus Gründen der Übersichtlichkeit verzichtet.
Zunächst kann man feststellen, dass im Endrand wegen der Auslautverhärtung (Abschnitte~\ref{sec:auslautverhaertungphonetik} und~\ref{sec:prozauslautverh}) keine stimmhaften Obstruenten vorkommen können, und dass damit \textipa{[b d g v z J]} aus der Betrachtung ausgeschlossen werden können.
Das Gleiche gilt für \textipa{[h]}, das nur im Anfangsrand auftritt.
Ähnlich wie der einsegmentale Anfangsrand kann der einsegmentale Endrand von allen verbleibenden Konsonanten gefüllt werden.
Obwohl das \textipa{[K]} im Endrand phonetisch ein Vokal ist, behandeln wir es hier als Frikativ.
Mehr zu den Gründen findet sich dann in Abschnitt~\ref{sec:prozrvok}.

\begin{exe}
  \ex\label{ex:phol4711}
  \begin{xlist}
  	\ex ab, Hut, Rock
  	\ex auf, aus, Hasch, ich, Loch, Ohr
  	\ex Raum, Zaun, Fang
  	\ex voll
  \end{xlist}
\end{exe}

Bei den zweisegmentalen Endrändern verfahren wir genau wie bei den zweisegmentalen Anfangsrändern.
Wir gehen also von innen die Artikulationsarten (Plosive, Frikative, Nasale, Approximanten) durch und prüfen, inwiefern sie die Wahl des zweiten Segments einschränken.
Anders als im Anfangsrand sind zunächst Folgen aus zwei Plosiven möglich, allerdings von allen sechs möglichen nur \textipa{[pt]} und \textipa{[kt]}.

\begin{exe}
  \ex\label{ex:phol4712}
  \begin{xlist}
  	\ex Abt, schleppt, klappt
  	\ex Takt, Sekt, nackt, rückt
  \end{xlist}
\end{exe}

Die Auswahl des zweiten Segments ist bei Frikativen an erster Position stark eingeschränkt.
Es kann nur ein Plosiv folgen, im Fall von \textipa{[f s \c{c} X]} ist dies aber immer \textipa{[t]} wie in (\ref{ex:phol4713a}).
Völlig ausgeschlossen ist \textipa{[S]}.
\textipa{[K]} kann hingegen mit allen stimmlosen Plosiven kombiniert werden, s.\ (\ref{ex:phol4713b}).

\begin{exe}
  \ex\label{ex:phol4713}
  \begin{xlist}
  	\ex{\label{ex:phol4713a} Luft, Lust, Licht, Acht}
  	\ex{\label{ex:phol4713b} Korb, Ort, Mark}
  \end{xlist}
\end{exe}

Außerdem können alle Frikative bis auf \textipa{[s]} mit folgendem \textipa{[s]} kombiniert werden, vgl.\ (\ref{ex:phol47135}).

\begin{exe}
  \ex{\label{ex:phol47135} Laufs, Reichs, Bachs, Kurs}
\end{exe}

Nasale in erster Position kombinieren sich alle mit homorganen Plosiven, also solchen, die den gleichen Artikulationsort haben, vgl.\ (\ref{ex:phol4714}).
\textipa{[m]} und \textipa{[N]} können zusätzlich mit \textipa{[t]} verbunden werden.
Einzelne Verbindungen wie \textipa{[mp]} sind sehr selten.

\begin{exe}
  \ex\label{ex:phol4714}
  \begin{xlist}
    \ex{\label{ex:phol4714a} Lump, nimmt}
    \ex{\label{ex:phol4714a} Hund}
    \ex{\label{ex:phol4714a} krank, ringt}
  \end{xlist}
\end{exe}

In Kombination mit Frikativen sind \textipa{[nX]} und \textipa{[nK]} strikt ausgeschlossen.
Auch \textipa{[n\c{c}]} kommt nur in zwei nennenswert häufigen Wörtern vor, s. (\ref{ex:phol4715a}).
Etwas häufiger sind die Kombinationen \textipa{[nf]} uns \textipa{[ns]}, sehr selten hingegen wieder \textipa{[nS]}, was nur in zwei geläufigeren Wörtern vorkommt, s.\ (\ref{ex:phol4715c}).
\textipa{[ms]} wie in (\ref{ex:phol4715d}) und \textipa{[mS]} wie in (\ref{ex:phol4715e}) sind ähnlich rar, wobei \textipa{[mS]} durch Bildungen aus Eigennamen wie \textit{grimmsch} (in \textit{das grimmsche Wörterbuch}) gelegentlich vorkommen könnte.
\textipa{[Ns]} kommt durch Genitivbildungen von Substantiven häufiger vor, s.\ (\ref{ex:phol4715f}).

\begin{exe}
  \ex\label{ex:phol4715}
  \begin{xlist}
  	\ex{\label{ex:phol4715a} Mönch, manch}
  	\ex{\label{ex:phol4715b} Hanf, Senf, uns, eins, Gans}
  	\ex{\label{ex:phol4715c} Mensch, Punsch}
  	\ex{\label{ex:phol4715d} Ems, Wams, Gams}
  	\ex{\label{ex:phol4715e} Ramsch}
  	\ex{\label{ex:phol4715f} längs, rings, Hangs usw.}
  \end{xlist}
\end{exe}

\textipa{[mf]} und \textipa{[Nf]} sowie Kombinationen aus zwei Nasalen oder aus Nasal und Approximant sind gänzlich ausgeschlossen.
Das Problem bei der Sequenz aus Nasal und Frikativ im Endrand ist also vor allem die geringe Typenhäufigkeit einiger Kombinationen.
Ob man \zB für ein einzelnes Wort wie \textit{Ramsch} -- ggf.\ flankiert durch gespreizte Bildungen wie \textit{grimmsch} -- einen eigenen Silbentyp aufmachen möchte, ist wie bei ähnlichen Fällen im Anfangsrand kaum systematisch festzulegen.

Für den Approximant in erster Position ist die Angelegenheit etwas klarer.
Er kombiniert sich gut mit den drei Plosiven, vgl.\ (\ref{ex:phol4716a}).
Von den Frikativen kommen \textipa{[f s \c{c}]} in Frage wie in (\ref{ex:phol4716b}), \textipa{[S X K]} nicht.
Von den drei Nasalen können nur \textipa{[m n]} folgen, s.\ (\ref{ex:phol4716c}).
Dabei ist \textipa{[ln]} sehr charakteristisch für (meist mehrsilbige) Verbformen von Verben, deren Stamm (s.\ Abschnitt~\ref{sec:stamm}) auf \textipa{[l]} endet.

\begin{exe}
  \ex\label{ex:phol4716}
  \begin{xlist}
  	\ex{\label{ex:phol4716a} Alp, halb, Halt, bald, welk, Talg}
  	\ex{\label{ex:phol4716b} elf, Wolf, Hals, Fels, Milch, solch}
  	\ex{\label{ex:phol4716b} Qualm, Film, Köln, ähneln}
  \end{xlist}
\end{exe}

Wie am Anfang dieses Abschnitts erwähnt, sind Endränder im Deutschen potentiell komplexer als Anfangsränder.
Der Anfangsrand ist maximal zweisegmental mit ganz wenigen dreisegmentalen Ausnahmen.
Wörter wie wie \textit{qualmt} \textipa{[kvalmt]} oder \textit{qualmst} \textipa{[kvalmst]} zeigen, dass es drei- und viersegmentale Endränder gibt.
Bis auf wenige Ausnahmen lassen sich diese aber als Erweiterungen des zweisegmentalen verstehen.
Wenn zum Beispiel \textipa{[ns]} zulässig ist (\textit{grins}) und \textipa{[st]} ebenso (\textit{ist}), ist automatisch auch \textipa{[nst]} zulässig.
Man kann also die zulässigen Kombinationen verketten, wenn das zweite Segment der ersten und das erste der zweiten identisch sind.
Für \textit{strolchst} ergibt das die Verkettung in (\ref{ex:phol4717}).

\begin{exe}
  \ex\label{ex:phol4717}
  \begin{xlist}
  	\ex{\label{ex:phol4717a} \textipa{[lm]} ist zulässig (\textit{Qualm})}
  	\ex{\label{ex:phol4717b} \textipa{[ms]} ist zulässig (\textit{Gams})}
  	\ex{\label{ex:phol4717b} \textipa{[st]} ist zulässig (\textit{ist})}
  	\ex{\label{ex:phol4717b} $\Rightarrow$ \textipa{[lmst]} ist zulässig (\textit{qualmst})} 	
  \end{xlist}
\end{exe}

Diese Regularität ist nicht ohne Ausnahmen, vor allem wenn das erste Segment der Approximant ist.
Zum Beispiel sind \textipa{[lmp]} und \textipa{[lnS]} nicht möglich.
Hier wird offensichtlich eine Komplexität überschritten, wahrscheinlich gekoppelt an die ohnehin geringe Häufigkeit der Kombinationen \textipa{[lm]} und \textipa{[lm]}.
Das Gesamtsystem -- inkl.\ des hier unterschlagenen \textipa{[s]} nach Plosiven wie in \textit{Schnaps}, \textit{zwecks} und auch \textit{Huts} -- wird in Abschnitt~\ref{sec:anfangsrandendrand} nochmals aufgerollt.
Falls der in diesem Abschnitt abgelieferte deskriptive Befund unübersichtlich erscheint, sei ebenfalls auf die weitere Systematisierung in besagtem Abschnitt verwiesen, die eine deutliche Reduktion der Komplexität aus der Darstellung herausnimmt.
Hier sollte vor allem klar aufgezeigt werden, dass die Besetzung der Ränder nicht beliebig ist und offensichtlich vertrackteren Bedingungen unterliegt.
In Abschnitt~\ref{sec:sonoritaet} wird für die weitere Systematisierung mit der Einführung der Sonoritätshierarchie ein wichtiger Grundstein gelegt.



\subsection{Sonorität}

\label{sec:sonoritaet}

Wie in Abschnitt~\ref{sec:einsilbler} gezeigt wurde, sind an den Rändern der Silbe nicht beliebige Kombinationen von Konsonanten möglich.
Dabei fällt ein Muster auf.
Während am Silbenanfang \zB \textipa{[kn]} (\textit{Knie}) aber nicht \textipa{[nk]} möglich ist, ist es am Silbenende genau umgekehrt (\textit{Zank}).
Gleiches gilt für \textipa{[pl]} (\textit{Plan}) und \textipa{[lp]} (\textit{Alp}) usw.
Es ergibt sich eine Art spiegelbildlicher Ordnung vom Vokal zu den Außenrändern.
Diese Ordnung zeigt sich nach aktuellem Kenntnisstand in allen Sprachen der Welt, und man erklärt sie mit Hilfe des Konstrukts der \textit{Sonorität} (ungefähr \textit{Klangfülle}).
Für unsere Zwecke reicht es, festzustellen, dass (in dieser Reihenfolge) Plosive (P), Frikative (F), Nasale (N), Approximanten (A) und Vokale (V) eine Skala mit ansteigender Sonorität bilden (Abbildung~\ref{fig:sonoritaetshierarchie}).

\index{Sonorität!Hierarchie}

\begin{figure}
  \centering
  \begin{tabular}{l|ccccc|r}
    \cline{2-6}
    (minimal sonor) & \rnode{HP}{P} & \rnode{HF}{F} & \rnode{HN}{N} & \rnode{HA}{A} & \rnode{HV}{V} & (maximal sonor) \\
    \cline{2-6}
  \end{tabular}
  \ncline[nodesep=3pt]{->}{HP}{HF}
  \ncline[nodesep=3pt]{->}{HF}{HN}
  \ncline[nodesep=3pt]{->}{HN}{HA}
  \ncline[nodesep=3pt]{->}{HA}{HV}
  \caption{Sonoritätshierarchie}
  \label{fig:sonoritaetshierarchie}
\end{figure}

Innerhalb der Silbe gibt es das universelle Bildungsprinzip der \textit{Sonoritätskontur}, welches regelt, dass die Sonorität zum Vokal hin ansteigt und dann wieder abfällt (Abbildung~\ref{fig:sonhier}).
Dies gilt natürlich nur, wenn die Silbe mindestens ein weiteres Segment außer dem Vokal enthält.
Eine Silbe, die nur aus einem Plosiv und einem Vokal besteht, zeigt einen Sonoritätsanstieg, aber keinen Sonoritätsabfall.
Bei einer Silbe, die nur aus einem Vokal und einem Plosiv besteht, ist es umgekehrt.
Es gibt also Silben, die nur einen Ausschnitt aus der Sonoritätskontur realisieren (nur Anstieg oder nur Abfall), aber einen Sonoritätsabfall gefolgt von einem Anstieg gibt es innerhalb einzelner Silben nicht.
In Tabelle~\ref{tab:silbenbau} werden zur Illustration einige deutsche Wörter in das Schema eingeordnet.%
\footnote{Die in Abschnitt~\ref{sec:endranddeskriptiv} beschriebenen möglichen Zweierkombinationen im Endrand passen genau zu dieser Generalisierung, denn es gibt die Sequenz Frikativ-Plosiv, aber nicht Plosiv-Frikativ usw.}
Das ideale Bild der Sonoritätskontur wird dabei weitgehend bestätigt.
Die einzige Ausnahme ist das Auftreten von \textit{s}-Lauten am äußersten Silbenrand jenseits von Plosiven (\textit{sprüh}, \textit{Raps}).
Da Frikative eine höhere Sonorität haben als Frikative, steigt in diesen Fällen die Sonorität zum Rand hin wieder an.
Diese Ausnahme für \textit{s}-Segmente findet man in vielen Sprachen der Welt, und in Abschnitt~\ref{sec:anfangsrandendrand} wird dafür plädiert, diese Segmente als \textit{extrasilbisch} (außerhalb der normalen Silbenstruktur stehend) zu analysieren.
Außerdem kann die Sonorität auch gleich bleiben, so dass sich \textit{Plateaus} aus zwei Plosiven (\textit{Abt}), zwei Frikativen (\textit{Reichs}) usw.\ bilden.
Abbildung~\ref{fig:sonhiers} zeigt ein modifiziertes Sonoritätsschema, das extrasilbische \textit{s}-Segmente und Plateaus berücksichtigt.

\begin{figure}
  \centering
  \begin{tabular}{ccccccccccc}
  &&&& \rnode{V}{V} &&&& \\
  &&& \rnode{L1}{A} && \rnode{L2}{A} &&& \\
  && \rnode{N1}{N} &&&& \rnode{N2}{N} && \\
  & \rnode{F1}{F} &&&&&& \rnode{F2}{F} & \\
  \rnode{P1}{P} &&&&&&&& \rnode{P2}{P} \\
  \end{tabular}
  \ncline[nodesep=3pt]{->}{P1}{F1}
  \ncline[nodesep=3pt]{->}{F1}{N1}
  \ncline[nodesep=3pt]{->}{N1}{L1}
  \ncline[nodesep=3pt]{->}{L1}{V}
  \ncline[nodesep=3pt]{->}{V}{L2}
  \ncline[nodesep=3pt]{->}{L2}{N2}
  \ncline[nodesep=3pt]{->}{N2}{F2}
  \ncline[nodesep=3pt]{->}{F2}{P2}
  \caption{Sonorität für die Segmentklassen in der schematischen Silbe}
  \label{fig:sonhier}
\end{figure}

\begin{table}
  \centering
    \begin{tabular}{cccccccccccp{0.5mm}l}
      \lsptoprule
      \textbf{(S)} & \textbf{P} & \textbf{F} & \textbf{N} & \textbf{A} & \textbf{V} & \textbf{A} & \textbf{N} & \textbf{F} & \textbf{P} & \textbf{(S)} && \\
      \midrule
	& k &&&& \textipa{\o:} &&&&&&& \textit{Kö} \\
	&&& n && \textipa{i:} &&&&&&& \textit{nie} \\
	& k && n && \textipa{i:} &&&&&&& \textit{Knie} \\
	& d & \textipa{K} &&& \textipa{o:} &&&&&&& \textit{droh} \\
	\textipa{S} & t &&&& \textipa{e:} &&&&&&& \textit{steh} \\
	\textipa{S} &&& n && \textipa{e:} &&&&&&& \textit{Schnee} \\
	\textipa{S} & p & \textipa{K} &&& \textipa{y:} &&&&&&& \textit{sprüh} \\
	&&&&&&&&&& \\
	& (\textipa{P}) &&&& a &&&& p &&& \textit{ab} \\
	& (\textipa{P}) &&&& a && n &&&&& \textit{an} \\
	& (\textipa{P}) &&&& a &&& \textipa{X} & t &&& \textit{acht} \\
	& (\textipa{P}) &&&& a & l & m &&&&& \textit{Alm} \\
	&&&&&&&&&& \\
	&& \textipa{K} &&& a &&&& p & s && \textit{Raps}\\
	&& \textipa{K} &&& a && m & s & t &&& \textit{rammst} \\
	\textipa{S} & t & \textipa{K} &&& \textipa{O} & l && \textipa{\c{c}s} & t &&& \textit{strolchst} \\
      \lspbottomrule
    \end{tabular}
  \caption{Einordnung einiger Konsonatengruppen in das Silbenschema}
  \label{tab:silbenbau}
\end{table}

\begin{figure}
  \centering
  \begin{tabular}{ccccccccccccccc}
    V &&&&&& \rnode{xV}{V} &&&& \\
    A &&&&& \rnode{xL1}{A} && \rnode{xL2}{A} &&&& \\
    N &&&& \rnode{xN1}{N} &&&& \rnode{xN2}{N} &&& \\
    F &\rnode{xS1}{(S)} && \rnode{xF1}{F} &&&&&& \rnode{xF2}{F} & \rnode{xF3}{F} && \rnode{xS2}{(S)} \\
    P &&\rnode{xP1}{P} &&&&&&&&& \rnode{xP2}{P} & \\
  \end{tabular}
  \ncline[nodesep=3pt]{->}{xS1}{xP1}
  \ncline[nodesep=3pt]{->}{xP1}{xF1}
  \ncline[nodesep=3pt]{->}{xF1}{xN1}
  \ncline[nodesep=3pt]{->}{xN1}{xL1}
  \ncline[nodesep=3pt]{->}{xL1}{xV}
  \ncline[nodesep=3pt]{->}{xV}{xL2}
  \ncline[nodesep=3pt]{->}{xL2}{xN2}
  \ncline[nodesep=3pt]{->}{xN2}{xF2}
  \ncline[nodesep=3pt]{->}{xF2}{xF3}
  \ncline[nodesep=3pt]{->}{xF3}{xP2}
  \ncline[nodesep=3pt]{->}{xP2}{xS2}
  \caption{Sonoritätsverlauf mit Rand-Frikativen und Plateau}
  \label{fig:sonhiers}
\end{figure}

Was die Sonorität aus phonetisch-artikulatorischer (oder perzeptorischer) Sicht genau ist, ist eine schwierige Frage.
Stimmhaftigkeit ist ein wichtiger Faktor für eine hohe Sonorität.
Darüber hinaus kann als Faustregel gelten, dass, je enger die durch die Artikulatoren hergestellte Annäherung ist, die Sonorität umso geringer ist.
Dies entspricht dem artikulatorischen Schema des Öffnens und Schließens des Vokaltrakts (Abschnitt~\ref{sec:silben}).
Die Überlegungen zur Sonorität schließen mit Definition~\ref{def:sonoritaet} und der Sonoritätsanalyse eines der komplexesten Einsilblers des Deutschen (\textit{strolchst}) in Abbildung~\ref{fig:sonhiersstrolchst}

\Definition{Sonoritätskontur}{\label{def:sonoritaet}
Segmente können auf einer Sonoritätsskala eingeordnet werden.
Alle zulässigen Silbenstrukturen repräsentieren maximal einen Anstieg der Sonorität zur Mitte der Silbe und einen Abfall der Sonorität zum Ende der Silbe.
Es gibt innerhalb einer Silbe keinen Sonoritätsanstieg nach einem Sonoritätsabfall.
Die einzige Ausnahme bilden \textit{s}-Frikative am äußersten Silbenrand. 
\index{Sonorität}
}

\begin{figure}
  \centering
  \begin{tabular}{ccccccccc}
    V &&&& \rnode{V1}{\textipa{O}} &&&& \\
    A &&&&& \rnode{L21}{\textipa{l}} &&& \\
    N &&&&&&&& \\
    F & \rnode{S11}{\textipa{S}} && \rnode{F11}{\textipa{K}} &&& \rnode{F21}{\textipa{\c{c}}} & \rnode{F31}{\textipa{s}} & \\
    P && \rnode{P11}{\textipa{t}} &&&&&& \rnode{P21}{\textipa{t}} \\
  \end{tabular}
  \ncline[nodesep=3pt]{->}{S11}{P11}
  \ncline[nodesep=3pt]{->}{P11}{F11}
  \ncline[nodesep=3pt]{->}{F11}{V1}
  \ncline[nodesep=3pt]{->}{V1}{L21}
  \ncline[nodesep=3pt]{->}{L21}{F21}
  \ncline[nodesep=3pt]{->}{F21}{F31}
  \ncline[nodesep=3pt]{->}{F31}{P21}
  \caption{Sonorität am Beispiel von \textit{strolchst}}
  \label{fig:sonhiersstrolchst}
\end{figure}



\subsection{Die Systematik von Anfangsrand und Endrand}

\label{sec:anfangsrandendrand}

In diesem Abschnitt werden der Anfangsrand und der Endrand im Einsilbler für den Kernwortschatz mit dem Wissen um die Sonoritätshierarchie abschließend beschrieben.
Die hauptsächliche Vereinfachung und Systematisierung des Anfangsrandes wird dadurch erreicht, dass \textipa{[S]} in Anfangsrändern mit scheinbar  zwei oder drei Segmenten eliminiert wird.
In Abschnitt~\ref{sec:sonoritaet} wurde festgestellt, dass \textipa{[S]} vor Plosiven (\textit{Sprung}, \textit{Stuhl}) die Sonoritätshierarchie verletzt.
Vor Frikativen (\textit{Schwung}, \textit{Schrank}) entsteht zumindest bei einer grob gestaffelten Sonoritätsskala ein Sonoritätsplateau.
Lediglich in mehregmentigen Anfangsrändern mit Nasal oder Approximant an zweiter Stelle (\textit{Schmal}, \textit{Schlund}) verhält sich \textipa{[S]} konform zur Sonoritätshierarchie.
Gleichzeitig sind die einzigen Anfangsränder mit drei Segmenten solche, bei denen das erste Segment \textipa{[S]} ist.
Das Segment \textipa{[S]} verhält sich im Silbenbau offensichtlich besonders, und es erschwert die Systematisierung.
Daher behandeln wir \textipa{[S]} in komplexen Anfangsrändern jetzt als \textit{extrasilbisch}.
Damit soll nicht gemeint sein, dass es nicht zur Silbe gehört, sondern vielmehr, dass es sich in einer besonderen Position vor dem Anfangsrand befindet, die von anderen Segmenten nicht besetzt werden kann, und die nicht der Sonoritätskontur unterliegt.

\newcommand{\Rxx}[3]{\POS[]+(#1,-1.2)\ar@{-}[#3]-(#2,1.2)}
\newcommand{\Rxxx}[4]{\POS[]+(#1,-1.2)\ar@{-}[#4]-(#2,#3)}

\begin{figure}
  \centering
  \Tree[3]{
     \K{\small\textbf{extrasilbisch}} & \K{\small\textbf{Anfangsrand}} && \K{\small\textbf{Kern}} \\
     \K{\textipa{S}}\R[--]{r} & \K{p t}\R{ddddrr}^{\text{Plosiv}}              &&           \\
     & \K{b d k g}\Rxx{6}{9.75}{r}                &&           \\
     \K{\textipa{S}}\R[--]{r} & \K{\textipa{K} v}\R{ddrr}_{\text{Frikativ}}             &&           \\
     & \K{\textipa{f S h z J}}\Rxxx{6}{9}{7.6}{ur}           &&           \\
     &          && \K{~~~Vokal} \\
     \K{\textipa{S}}\R[--]{r} & \K{m n}\R{urr}_{\text{Nasal}}  &&           \\
     &                                         &&           \\
     \K{\textipa{S}}\R[--]{r} & \K{l}\R{uuurr}_{\text{Approximant}}         &&           \\
  }
  \caption{Struktur des simplexen Anfangsrands (nur Systemkern)}
  \label{fig:anfangsrandsimplex}
\end{figure}

Die maximale Komplexität des Anfangsrands besteht also in zwei Segmenten (duplex), und scheinbare Fälle von drei Segmenten (\textipa{[SpK]} und \textipa{[StK]}) bestehen aus zwei Segmenten mit extrasilbischem \textipa{[S]}.
Außerdem wurde in Abschnitt~\ref{sec:einsilbler} bereits stark bezweifelt, dass Anlaute wie \textipa{[gn]} in \textit{Gnade} und \textipa{[Spl]} wie in \textit{spleißen} zum Kern des Systems gerechnet werden müssen.
Wenn man nun alles weglässt, was weglassbar ist, und \textipa{[S]} den genannten Sonderstatus zuweist, dampft die Beschreibung des simplexen Anfangsrands auf Abbildung~\ref{fig:anfangsrandsimplex} und die des duplexen Anfangsrands auf Abbildung~\ref{fig:anfangsrandduplex} ein.
Die Abbildungen sind von rechts nach links zu lesen.
In Abbildung~\ref{fig:anfangsrandduplex} beginnt man mit dem Vokal im Kern, und die nach links weisenden Kanten zeigen Besetzungsmöglichkeiten für das erste Segment links.
Es kommen Frikative, Nasale und der Approximant in Frage.
Mögliche Frikative sind \textipa{[K]} und \textipa{[v]}.
Wenn ein \textipa{[K]} steht, kann ein Plosiv oder ein Frikativ davor stehen, usw.

\begin{figure}
  \centering
  \Tree[3]{
     \K{\small\textbf{extrasilbisch}} & \K{\small\textbf{Anfangsrand}}\Below{\small\textbf{erste Position}} & \K{\small\textbf{Anfangsrand}}\Below{\small\textbf{zweite Position}} && \K{\small\textbf{Kern}} \\
     \K{\textipa{S}}\R[--]{r} & \K{p t}\Rxx{3}{3}{ddr}^{\text{Plosiv}} &&\\
     & \K{\textipa{b d k g}}\Rxx{6}{11.6}{r} & \\
  	 && \K{\textipa{K}}\Rxx{3.7}{3.7}{ddrr}^{\text{Frikativ}} \\
     & \K{f v}\Rxx{3}{3}{ur}_{\text{Frikativ}} & &\\
     & \K{k}\Rxx{3}{5}{r}^{\text{Plosiv}} & \K{v}\Rxxx{3.7}{0.8}{0.825}{ur}         && \K{~~~Vokal} \\
     & \K{k}\Rxx{3.7}{3.7}{r}^{\text{Plosiv}} & \K{n}\R{urr}^{\text{Nasal}}  &&           \\
     & \K{p b k g}\Rxx{7}{3.7}{dr}^{\text{Plosiv}} &                                         &&           \\
     && \K{l}\R{uuurr}_{\text{Approximant}}         &&           \\
     & \K{f}\Rxx{5}{3.7}{ur}_{\text{Frikativ}} &                                         &&           \\
  }
  \caption{Struktur des duplexen Anfangsrands (nur Systemkern)}
  \label{fig:anfangsrandduplex}
\end{figure}

Abbildung~\ref{fig:anfangsrandsimplex} zeigt übersichtlich, dass im simplexen Anfangsrand alle Konsonanten stehen können, bis auf die im Anfangsrand generell nicht lizenzierten \textipa{[s \c{c} X N]}.
Extrasilbisches \textipa{[S]} kann vor Konsonanten aller Artikulationsarten stehen, im Fall von Plosiven aber nicht vor stimmhaften und nicht vor \textipa{[k]}.
Ebenso sind die Frikative \textipa{[f S J h]} nicht mit extrasilbischem \textipa{[S]} kombinierbar.
Abbildung~\ref{fig:anfangsrandduplex} zeigt, dass die Kombinationsmöglichkeiten im duplexen Anfangsrand stark eingeschränkt sind.
Plosive an zweiter Position sind ausgeschlossen.
Häufig (im Sinn einer Typenhäufigkeit, s.\ Abschnitt~\ref{sec:kern}) sind nur Kombinationen aus Plosiv und \textipa{[K]} sowie (bereits weniger häufig) Plosiv und \textipa{[l]}.
Die Präferenz für diese Kombination hat genau wie die Sonoritätskontur Einzelsprachen übergreifende Züge.
Man fasst daher \textit{r}- und \textit{l}-Segmente zu den sogenannten \textit{Liquiden} (oder \textit{Fließlaute}) zusammen, um ihrem ähnlichen Verhalten beim Silbenbau Rechnung zu tragen.
In der Beschreibung des Endrands wird sich diese Klassenbildung sofort weiter auszahlen.
\index{Liquid}

\Definition{Liquid}{Liquide sind \textit{l}- und \textit{r}-Segmente.
Die Gruppierung erfolgt für das Deutsche auf Basis phonologischer, nicht aber phonetischer Kriterien.}

\Satz{Präferenz in duplexen Anfangsrändern}{Die präferierte Struktur für duplexe Anfangsränder ist die aus einem Plosiv und Liquid.}

Der Endrand lässt sich zwar auch in so einem Diagramm darstellen, das aber ungleich komplizierter würde als Abbildung~\ref{fig:anfangsrandsimplex} und Abbildung~\ref{fig:anfangsrandduplex}.
Die Beobachtungen aus Abschnitt~\ref{sec:endranddeskriptiv} legen eine tabellarische Darstellung wie in Tabelle~\ref{tab:endrand} nahe.
Die Tabelle definiert in jeder Zeile ein im Endrand zulässiges Segmentpaar.
Die Paare können wie in Abschnitt~\ref{sec:endranddeskriptiv} beschrieben bis auf wenige Ausnahmen verkettet werden.
Wenn also \textipa{[fs]} und \textipa{[st]} möglich sind, ist \idR auch \textipa{[fst]} möglich (\textit{läufst}).

Gegenüber Abschnitt~\ref{sec:endranddeskriptiv} wurden einige Systematisierungen vorgenommen.
Zunächst verteilen sich die Liquiden \textipa{[l]} und \textipa{[K]} (phonetisch Teil eines sekundären Diphthongs) völlig gleich und wurden daher zusammengefasst.
Sie lassen sich mit allen anderen Segmenten außer \textipa{[X]} kombinieren (dazu s.\ ).


\begin{table}
  \begin{tabular}{ccccccccp{1cm}l}
%   \lsptoprule
%   \textbf{A} & \textbf{N} & \textbf{\textipa{K}} & \textbf{F} & \textbf{s} & \textbf{P} & \textbf{t} & \textbf{(s)} && \textbf{Beispiel} \\
%   \midrule
     \textipa{K} l & m           &             &                               &   &   &   &   && \textit{Qualm}, \textit{warm}   \\
     \textipa{K} l & n           &             &                               &   &   &   &   && \textit{Köln}, \textit{Garn}    \\
     \textipa{K} l &             &             & f                             &   &   &   &   && \textit{elf}, \textit{darf}     \\
     \textipa{K} l &             &             & \textipa{S}                   &   &   &   &   && \textit{falsch}, \textit{Lurch}     \\
     \textipa{K} l &             &             & \textipa{\c{c}}               &   &   &   &   && \textit{Milch}, \textit{Lurch}   \\
     \textipa{K} l &             &             &                               & s &   &   &   && \textit{Fels}, \textit{Kurs}    \\
     \textipa{K} l &             &             &                               &   & p &   &   && \textit{halb}, \textit{Korb}    \\
     \textipa{K} l &             &             &                               &   & k &   &   && \textit{Kalk}, \textit{Mark}    \\
     \textipa{K} l &             &             &                               &   &   & t &   && \textit{alt}, \textit{Ort}     \\
%   \midrule
                   & \textipa{N} &             &                               & s &   &   &   && \textit{längs}   \\
                   & \textipa{N} &             &                               &   & k &   &   && \textit{krank}   \\
                   & \textipa{N} &             &                               &   &   & t &   && \textit{hängt}   \\
                   & n           &             & f                             &   &   &   &   && \textit{Hanf}    \\
                   & n           &             & \textipa{S}                   &   &   &   &   && \textit{Mensch}  \\
                   & n           &             &                               & s &   &   &   && \textit{eins}    \\
                   & n           &             &                               &   &   & t &   && \textit{Hund}    \\
                   & m           &             & \textipa{S}                   &   &   &   &   && \textit{Ramsch}  \\
                   & m           &             &                               & s &   &   &   && \textit{Gams}    \\
                   & m           &             &                               &   & p &   &   && \textit{Lump}    \\
                   & m           &             &                               &   &   & t &   && \textit{klemmt}  \\
%   \midrule
                   &             &             & f \textipa{S \c{c}}           & s &   &   &   && \textit{Laufs}, \textit{Reichs}, \textit{Bachs}  \\
                   &             &             & f \textipa{S \c{c}}           &   &   & t &   && \textit{Luft}, \textit{reicht}, \textit{acht}   \\
                   &             &             &                               & s &   & t &   && \textit{Ast}    \\
%   \midrule
                   &             &             &                               &   & p k & t &  && \textit{Abt}, \textit{nackt}    \\
%   \midrule
                   &             &             &                               &   & p k &   & s && \textit{Schnaps}, \textit{zwecks} \\
                   &             &             &                               &   &     & t & s && \textit{Huts}   \\
%    \lspbottomrule
  \end{tabular}
  \caption{Struktur des komplexen Endrands: mögliche Segmentpaare}
  \label{tab:endrand}
\end{table}


\subsection{Einsilbler und Mehrsilbler}

\label{sec:mehrsilbler}
\index{Silbe!Silbifizierung}

Nach den Silben ist die nächsthöhere Ebene der phonologischen Strukturbildung das phonologische Wort.
Der Grund, warum man eine nächsthöhere Einheit nach der Silbe innerhalb der Phonologie annehmen möchte, ist, dass es phonologische Regularitäten gibt, die sich nicht nur mit Bezug auf Segmente und einzelne Silben beschreiben lassen.

\Definition{Phonologisches Wort}{
\label{def:phonwort}
Ein phonologisches Wort ist die kleinste phonologische Struktur, die Silben als Konstituenten hat, und bezüglich derer eigene Regularitäten feststellbar sind.
\index{Wort!phonologisch}
}

Diese Definition kommt sehr formal daher.
Denken wir aber an den Grammatikbegriff aus Definition~\ref{def:grammatik} (S.~\pageref{def:grammatik}), dann ist die Einschränkung \textit{bezüglich derer eigene Regularitäten feststellbar sind} ausgesprochen instruktiv.
Wenn es nämlich phonologische Regularitäten gibt, die sich nicht effektiv und angemessen mittels Segmenten oder Silben beschreiben lassen, müssen wir eine andere (größere) Einheit annehmen, bezüglich derer wir sie beschreiben können.
Eine solche Regularität wird in (\ref{ex:phol0815}) illustriert, wobei in der orthographischen Variante und der Transkription die Silbengrenzen markiert sind.

\begin{exe}
  \ex\label{ex:phol0815}
  \begin{xlist}
  	\ex{\label{ex:phol0815a} Mie.te \textipa{[mi:.t@]}, Mi.tte \textipa{[mi.t@]}}
  	\ex{\label{ex:phol0815b} Knie \textipa{[kni:]}, *Kni \textipa{[knI]} }
  	\ex{\label{ex:phol0815c} schief \textipa{[Si:f]}, Schiff \textipa{[SIf]}}
  \end{xlist}
\end{exe}

\textbf{?? HIER WEITER}


% ==========================================================




\section{Wortakzent}

\label{sec:prosodie}

\index{Prosodie}

\subsection{Prosodie}

Außer den Regularitäten der Silbenstruktur in Mehrsilblern gibt es andere phonologische Phänomene, die auf der Wortebene beschrieben werden müssen.
Das wichtigste Beispiel ist die \textit{Akzentzuweisung}, also umgangssprachlich die \textit{Betonung} einer Silbe innerhalb eines Wortes.
In (\ref{ex:phol8735}) ist der Akzent in einigen Wörtern markiert.
Das Zeichen \Akz\ steht jeweils vor der akzentuierten (betonten) Silbe.

\begin{exe}
  \ex\label{ex:phol8735}
  \begin{xlist}
    \ex{\label{ex:phol8735a} \Akz Spiel, \Akz Spiele, \Akz Spielerin, be\Akz spielen}
    \ex{\label{ex:phol8735b} \Akz Fußball, \Akz Fußballerin, \Akz Fitness, \Akz Fitnesstrainerin}
    \ex{\label{ex:phol8735c} \Akz rot, \Akz rötlich, \Akz roter}
    \ex{\label{ex:phol8735d} \Akz fahren, um\Akz fahren, \Akz umfahren}
    \ex{\label{ex:phol8735e} wahr\Akz scheinlich, \Akz damals, \Akz übrigens, vie\Akz lleicht}
    \ex{\label{ex:phol8735f} \Akz wo, wa\Akz rum, wes\Akz halb}
    \ex{\label{ex:phol8735g} \Akz August, Au\Akz gust}
    \ex{\label{ex:phol8735h} \Akz fahren, Fahre\Akz rei, \Akz drängeln, Dränge\Akz lei}
  \end{xlist}
\end{exe}

Die \textit{Akzentlehre} nennt man \textit{Prosodie}, und wir besprechen hier aus Platzgründen nur den Bereich der Wortbetonung und \zB nicht die Satzbetonung.
Bis zu Abschnitt~\ref{sec:prosphonwort}) nehmen wir außerdem an, dass die Definition des phonologischen Worts (Definition~\ref{def:phonwort}) für die Betrachtung des Wortakzents ausreicht.
Jedes phonologische Wort hat also eine Silbe, die durch eine besondere Hervorhebung gekennzeichnet ist.
Phonetisch besteht diese Hervorhebung aus einem Bündel von Eigenschaften wie Lautstärke, Länge, Tonhöhe und Beeinflussung der Qualität der Vokale sowie der umliegenden Segmente.
Es gilt, dass jedes nicht zusammengesetzte Wort des deutschen Kernwortschatzes genau eine Akzentsilbe hat (\textit{\Akz Ball}, \textit{\Akz Tante}, \textit{\Akz schneite}, \textit{\Akz rot}, \textit{\Akz unter} usw.).
Zusammengesetzte Wörter oder längere Wörter haben genau einen \textit{Hauptakzent} (\textit{\Akz untergehen}, \textit{\Akz Wirtschaftswunder}, \textit{Tautolo\Akz gie} usw.).
Zusätzlich findet man in diesen Wörtern aber \textit{Nebenakzente} (im Vergleich zu Akzentsilben weniger stark akzentuierte Silben) in den zuletzt erwähnten Wörtern.

\Definition{Akzent}{
\label{def:akzent}
Akzent ist die Prominenzmarkierung, die einer Silbe im phonologischen Wort zugewiesen wird.
Akzent wird durch verschiedene phonetische Mittel (wie Lautstärke, Tonhöhe usw.) phonetisch realisiert.
\index{Akzent}
}

Die Frage ist, nach welchen Regularitäten der Akzent auf die Wörter verteilt wird.
Manche Sprachen sind sehr systematisch bzw.\ starr bezüglich der Akzentposition.
Im Polnischen liegt der Akzent immer auf der zweitletzten Wortsilbe, s.\ (\ref{ex:phol8254}).
Im Tschechischen hingegen wird immer die erste Silbe akzentuiert, vgl.\ (\ref{ex:phol8255}).%
\footnote{Für die slawischen Beispiele danke ich Götz Keydana.}

\begin{exe}
  \ex{\label{ex:phol8254} \Akz okno (Fenster), nagroma\Akz dzenie (Ansammlung)}
  \ex{\label{ex:phol8255} \Akz okno (Fenster), \Akz nahromad\v{e}n\'i (Ansammlung)}
\end{exe}

Solche Sprachen haben einen sogenannten \textit{metrischen Akzent}.
Einen streng \textit{lexikalischen Akzent} hat dagegen das Russische.
Hier ist der Akzent für jedes Wort im Lexikon festgelegt, und man kann allein durch die Position des Akzents zwei Wörter mit völlig verschiedener Bedeutung unterscheiden, s.\ (\ref{ex:phol8256}).

\begin{exe}
  \ex{\label{ex:phol8256} \Akz muka (Qual), mu\Akz ka (Mehl)}
\end{exe}

Bevor die Frage geklärt wird, wie sich der Akzent im Deutschen verhält, wird ein einfacher Test auf den Akzentsitz vorgestellt.
Dabei bedient man sich der Tatsache, dass Sprecher zur besonderen Hervorhebung einzelner Wörter in einem Satz eine besonders starke Betonung einsetzen können.
In den Beispielen in (\ref{ex:fokus}) ist jeweils das betonte Wort in Großbuchstaben gesetzt.
Zusätzlich markiert in den Beispielen das Akzentzeichen, auf welcher Silbe der Höhepunkt der Betonung genau liegt.

\begin{exe}
  \ex\label{ex:fokus}
  \begin{xlist}
    \ex{Sie hat das \Akz AUTO gewaschen.}
    \ex{Sie hat das Auto GE\Akz WASCHEN.}
  \end{xlist}
\end{exe}

Von der Bedeutung her ergibt sich typischerweise durch die Betonung eines Wortes ein ähnlicher Effekt, als würde man jeweils die Formel \textit{und nichts anderes} hinzufügen, als würde man also die sogenannten \textit{Alternativen} zum betonten Wort ausdrücklich ausschließen.

\begin{exe}
  \ex\label{ex:fokus-deutlich}
  \begin{xlist}
    \ex{Sie hat das \Akz AUTO (und nichts anderes) gewaschen.}
    \ex{Sie hat das Auto GE\Akz WASCHEN (und nichts anderes damit gemacht).}
  \end{xlist}
\end{exe}

Bei dieser Betonung eines Wortes tritt die Akzentsilbe besonders deutlich hörbar hervor.
Es wird sozusagen stellvertretend für das ganze Wort die Akzentsilbe betont.
In \textit{Auto} ist es die Silbe \textipa{[\t{aO}]}, in \textit{gewaschen} die Silbe \textipa{[vaS]} usw.
Damit hat man einen einfachen Test an der Hand, mit dem man in Zweifelsfällen den Wortakzent lokalisieren kann.

\subsection{Wortakzent im Deutschen}

\label{sec:deutscherwortakzent}

Es ist nun die Frage zu beantworten, welchem Akzenttyp (metrisch oder lexikalisch) das Deutsche folgt.
Die Frage wird unterschiedlich beantwortet, aber es lassen sich für die Wörter des Kernwortschatzes relativ klare Regularitäten erkennen, die auf einen tendenziell stark metrischen Akzent für das Deutsche hinweisen.
Leider benötigen wir zur Beschreibung der wichtigsten Regularität einen Begriff, den wir noch nicht eingeführt haben, nämlich den des \label{abs:3453457}Wortstamms (vgl.\ Abschnitt~\ref{sec:stamm}).
In den Beispielen in (\ref{ex:phol8735a}) bleibt der Akzent in allen Wörtern immer auf der Silbe \textit{spiel}.
Ob nun der Plural \textit{Spiele} gebildet wird, die Form \textit{Spielerin} oder ob ein morphologisches Element vorangestellt wird wie in \textit{bespielen}, der Akzent bleibt auf dem sogenannten Stamm dieser Wörter, nämlich \textit{spiel}.
Ganz ähnlich verhält es sich mit \textit{rot} in (\ref{ex:phol8735c}).
Im Deutschen gibt es die starke Tendenz, den Wortstamm zu betonen zu betonen.
Ist der Stamm zweisilbig wie in \textit{Tüte}, \textit{wichtig}, \textit{jemand} oder \textit{unter}, wird typischerweise die erste Silbe betont.%
\footnote{Bei mehr als zweisilbigen Stämmen wird die Angelegenheit allerdings komplizierter.}

\Satz{Stammbetonung}{
Im Kernwortschatz (s.\ Abschnitt~\ref{sec:kern}) wird der Stamm akzentuiert.
Zweisilbige Stämme werden auf der ersten Silbe akzentuiert.
\index{Akzent!Stamm--}
}

Wörter wie \textit{Fußball} und \textit{Fitnesstrainerin} aus (\ref{ex:phol8735b}) sind aus zwei Stämmen zusammengesetzt und werden \textit{Komposita} genannt (vgl.\ Abschnitt~\ref{sec:komp}).
In ihnen wird immer der erste Stamm betont.

\Satz{Betonung in Komposita}{
In Komposita wird der erste Bestandteil akzentuiert.
\index{Akzent!in Komposita}
}

Mit dem Betonungstest aus Abschnitt~\ref{sec:akzentsitztest} kann für beliebig lange Komposita festgestellt werden, dass der Akzent immer auf ihrem ersten Bestandteil liegt, vgl.\ (\ref{ex:fokuskomp}).

\begin{exe}
  \ex\label{ex:fokuskomp}
  \begin{xlist}
    \ex{Sie hat das \Akz AUTODACH (und nichts anderes) gewaschen.}
    \ex{Sie hat am \Akz LANGSTRECKENLAUF (und nichts anderem) teilgenommen.}
    \ex{Sie hat sich an dem \Akz BUSHALTESTELLENUNTERSTAND (und nichts anderem) verletzt.}
  \end{xlist}
\end{exe}

Im Falle von \textit{\Akz umfahren} und \textit{um\Akz fahren} aus (\ref{ex:phol8735d}) liegt wieder eine andere Situation vor.
Das Element \textit{um-} ist einmal betont, einmal nicht.
Diese Wörter haben allerdings auch nicht dieselbe Bedeutung.
\textit{\Akz umfahren} bedeutet soviel wie \textit{niederfahren}, \textit{um\Akz fahren} bedeutet soviel wie \textit{herumfahren}.
Es gibt weitere morphologische und syntaktische Unterschiede zwischen den beiden verschiedenen \textit{um}-Elementen, die in \ref{sec:derivohnewaw} genauer beschrieben werden.
In \textit{\Akz umfahren} handelt es sich bei \textit{um} um eine sogenannte \textit{Verbpartikel}, in \textit{um\Akz fahren} um ein \textit{Verbpräfix}.
Die anderen, meist nachgestellten Ableitungselemente wie \textit{-heit}, \textit{-keit}, \textit{-in} usw.\ verändern die Stammbetonung Betonung nicht, verhalten sich diesbezüglich also eher wie Verbpräfixe als wie Verbpartikeln.
Lediglich \textit{-ei} und \textit{-erei} ziehen den Akzent auf die letzte Silbe, vgl.\ (\ref{ex:phol8735h}).

\Satz{Präfix- und Partikelbetonung}{
\label{satz:pholvprtprf}
Verbpartikeln ziehen den Akzent auf sich, Verbpräfixe nicht.
\index{Akzent!Präfixe und Partikeln}
}

Neben diesen regelhaften Fällen (metrischer Akzent) gibt es eine gewisse Menge von Wörtern, die nicht regelhaft akzentuiert werden (lexikalischer Akzent).
Neben Lehnwörtern, die offensichtlich einen lexikalischen Akzent haben (wie \textit{\Akz August} und \textit{Au\Akz gust}) gibt es eine Reihe von Wörtern wie \textit{vie\Akz lleicht}, die sich unregelmäßig zu verhalten scheinen und nicht auf der ersten Stammsilbe betont werden.
Dazu gehören auch die Fragewörter \textit{wa\Akz rum}, \textit{wes\Akz halb} usw.
Es spricht allerdings auch überhaupt nichts dagegen, ein überwiegend metrisches Akzentsystem anzunehmen, innerhalb dessen es gewisse lexikalische Ausnahmen gibt.
Außerdem gibt es manche Wörter, die gar keinen Akzent zu tragen scheinen.
Bei einsilbigen Wörtern stellt sich die Frage nach dem Akzentsitz normalerweise nicht, weil die einzige Silbe des Worts den Akzent trägt.
Bestimmte Pronomen, wie das \textit{es} in (\ref{ex:phol9101}) sind aber prinzipiell nicht betonbar.
Wenn man dieses \textit{es} zu betonen versucht, wird der Satz ungrammatisch.
Zu solchen \textit{Explitivpronomina} vgl.\ auch Abschnitt~\ref{sec:expletiva}.

\begin{exe}
  \ex\label{ex:phol9101}
  \begin{xlist}
    \ex[]{Es schneit.}
    \ex[*]{\Akz ES schneit.}
  \end{xlist}
\end{exe}

Eine sich aus der Abfolge von betonten und unbetonten Silben ergebende Einheit wird hier aus Platzgründen nur sehr kurz behandelt, obwohl sie auch in der Morphologie (zumindest des Kernwortschatzes) weitreichendes Erklärungspotential hat, nämlich der Fuß.%
\footnote{In Teil~\ref{part:schrift} kommen wir nochmal auf Füße zurück.}
Wenn man längere phonologische Wörter daraufhin untersucht, wie akzentuierte (inkl.\ Nebenakzente) und nicht-akzentuierte Silben einander folgen, stellt man fest, dass im Deutschen das mit Abstand häufigste Muster eine Folge von betonter und unbetonter Silbe ist (\textit{\Akz um.ge.\Akz fah.ren}, \textit{\Akz Kin.der}, \textit{\Akz Kin.der.\Akz gar.ten} und viele der oben genannten Beispiele).
Manchmal liegt der umgekehrte Fall vor, also eine Abfolge unbetont vor betont (\textit{vie.\Akz lleicht} usw.).
Noch seltener kommt es zu Abfolgen von zwei unbetonten vor einer betonten Silbe (\textit{Po.li.\Akz tik}).
Der umgekehrte Fall von einer betonten vor zwei unbetonten Silben ergibt sich sogar regelhaft in bestimmten Formen von Verben und Adjektiven (\textit{\Akz reg.ne.te}, \textit{\Akz röt.li.che}).
Diese rhythmischen Verhältnisse sind als \textit{Füße} -- Abfolgen von betonten und unbetonten Silben -- beschreibbar.
Gemäß Tabelle~\ref{tab:dtfuesse}, die einige wichtige Fußtypen zusammenfasst, wäre dann das prototypische Wort des Kernwortschatzes trochäisch.

\begin{table}
\centering
\begin{tabular}{lll}
  \lsptoprule
  \textbf{Fuß} & \textbf{Muster} & \textbf{Beispiel} \\
  \midrule
  Trochäus & \Akz -- & \Akz Mu.tter \\
  Daktylus & \Akz -- -- & \Akz reg.ne.te \\
  Jambus & -- \Akz & vie.\Akz lleicht \\
  Anapäst & -- -- \Akz & Po.li.\Akz tik \\
  \lspbottomrule
\end{tabular}
\caption{Namen verschiedener Fußtypen mit Beispielen}
\label{tab:dtfuesse}
\end{table}



\subsection{Einfügung des Glottalverschlusses}

\label{sec:glottalverschluss}

\textbf{?? INTEGRATION PRÜFEN}

Jetzt kann, nachdem auch der Akzent besprochen wurde, noch die Regularität der \textipa{[P]}-Einfügung, die in Abschnitt~\ref{sec:photlaryngale} sehr kurz angesprochen wurde, genau angegeben werden.
Es handelt sich um eine Interaktion von segmentaler Phonologie, Silbifizierung und Prosodie.
Statt mühsam einen phonologischen Prozess zu formulieren, erfassen wir die Regularität in einem Satz.

\Satz{[ʔ]-Einfügung}{
\label{satz:glottalstoprule}
Der laryngale Plosiv \textipa{[P]} ist nicht zugrundeliegend und wird im Zuge der Akzentzuweisung und der Silbifizierung in den leeren Onset von Silben eingefügt, die entweder (1) am Wortanfang stehen oder (2) im Wortinneren stehen und betont sind.
}

Silben, die eigentlich einen leeren Onset haben (also mit Vokal anlauten) werden um dieses Segment unter genau benennbaren phontaktischen und prosodischen Bedingungen ergänzt.
Die Beispiele in (\ref{ex:phol1249}) in phonetischer Umschrift mit Silbengrenzen und \textipa{[\textprimstress]} für den Akzent zeigen die Wirkung dieser Regularität.

\begin{exe}
  \ex\label{ex:phol1249}
  \begin{xlist}
    \ex{Aue \textipa{[\textprimstress P\t{aO}.@]}}
    \ex{Chaos \textipa{[\textprimstress ka:.Os]}}
    \ex{Chaot \textipa{[ka.\textprimstress Po:t]}}
    \ex{beäugen \textipa{[be.\textprimstress P\t{O\oe}.g@n]}}
    \ex{vereisen \textipa{[f5.\textprimstress P\t{aE}z@n]}}
    \ex{unterweisen \textipa{[PUnt5.\textprimstress v\t{aE}z@n]}}
  \end{xlist}
\end{exe}



\subsection{Prosodisches und phonologisches Wort}

\label{sec:prosphonwort}

Abschließend soll noch anhand eines Phänomens darauf hingewiesen werden, warum es naheliegend ist, zwischen \textit{phonologischem Wort} und \textit{prosodischem Wort} unterschieden wird.
Zur Illustration dienen die Beispiele in (\ref{ex:phol8945}), in denen der Hauptakzent und die Silbengrenzen notiert wurden.

\begin{exe}
  \ex\label{ex:phol8945}
  \begin{xlist}
    \ex{Leser \textipa{[\textprimstress le:.z5]}}
    \ex{Leserin \textipa{[\textprimstress le:.z@.KIn]}}
    \ex{Leseranfrage \textipa{[\textprimstress le:.z5.Pan.fKa:.g@]}}
    \ex{(wenn) Leser anfragen \textipa{[\textprimstress le:.z5 \textprimstress Pan.fKa:.g@n]}}
  \end{xlist}
\end{exe}

\textbf{?? ab hier nochmal prüfen}

Im Fall von \textit{Le.ser} und \textit{Le.se.rin} wird offensichtlich gemäß den Regularitäten, die in Abschnitt~\ref{sec:silbifizierung} beschrieben wurden, silbifiziert.
Wegen der Bedingung Onset-Maximierung gerät dabei das /\textipa{K}/ von \textit{Leserin} in den Onset der letzten Silbe und wird folgerichtig nicht vokalisiert, so wie es bei \textit{Leser} passiert.
Bei \textit{Leseranfrage} ist es anders, denn obwohl dem /\textipa{K}/ ein Vokal folgt, wird /\textipa{K}/ nicht in den Anlaut eingeordnet, sondern bleibt in der Silbe \textipa{[z5]} und wird vokalisiert.
Es heißt also nicht *\textipa{[le:.z@.Kan.fKa:.g@]}.

Einerseits gilt also innerhalb eines Wortes wie \textit{Leserin} die Onset-Maximierung, andererseits aber scheint sie in einem Wort wie \textit{Leseranfrage} nicht vollständig zu gelten.
Es muss sich also bei Komposita wir \textit{Leseranfrage} um zwei phonologische Wörter handeln, denn die Silbifizierung verläuft genauso wie in (\textit{wenn}) \textit{Leser anfragen}, wobei es sich eindeutig um zwei verschiedene Wörter handelt.
Trotzdem verhalten sich \textit{Leseranfragen} und (\textit{wenn}) \textit{Leser anfragen} phonologisch nicht genau gleich.
Im Kompositum \textit{Leseranfragen} gibt es nur einen Hauptakzent (auf der ersten Silbe), während in \textit{Leser anfragen} jedes Wort einen Hauptakzent erhält.
Prosodisch verhält sich ein Kompositum also wie ein Wort und hat einen Hauptakzent, phonotaktisch-segmental verhält es sich allerdings wie zwei Wörter, denn an der Grenze zwischen den Gliedern des Kompositums findet keine normale wortinterne Silbifizierung statt.
Daher benötigt man eigentlich zwei Wort-Ebenen in der Phonologie, das phonologische Wort und das prosodische Wort.

\Definition{Phonologisches und prosodisches Wort}{
\label{def:phonoprosowort}
Das phonologische Wort ist die aus Füßen (in vereinfachter Darstellung aus Silben) bestehende Einheit, innerhalb derer die Regularitäten der segmentalen Phonologie und der Phonotaktik wirken.
Das prosodische Wort ist die aus phonologischen Wörtern bestehende Einheit, innerhalb derer prosodische Regularitäten (Akzentzuweisung) wirken.
\index{Wort!phonologisch}
\index{Wort!prosodisch}
}

Es gibt natürlich viele Fälle, in denen das phonologische Wort gleich dem prosodischen Wort ist, aber gerade bei Komposita (und \zB Fügungen aus Verbpartikel und Verb) muss man davon ausgehen, dass das phonologische Wort kleiner ist als das prosodische.




% ==========================================================



\section[Phone und Phoneme]{\Opsional Phone und Phoneme}

\label{sec:phonphonem}

In diesem Abschnitt soll kurz auf einige oft erwähnte phonologische Begriffe -- vor allem auf den des Phonems -- eingegangen werden.
Dabei soll gezeigt werden, warum eine einfache Phonemtheorie bestimmte Probleme mit sich bringt, zumal wenn sie ohne phonologische Merkmale formuliert wird.

Zugrundeliegende Formen und phonologische Prozesse gibt es in der Phonemtheorie zunächst nicht.
Segmente werden lediglich danach klassifiziert, ob sie distinktiv sind oder nicht.
Als Basisbegriff wird das Phon als phonetisch realisiertes Segment definiert, also als das, was wir in [~] schreiben.
In \textipa{[ta:k]} sind drei Phone zu beobachten, nämlich \textipa{[t]}, \textipa{[a:]} und \textipa{[k]}.

\Definition{Phon}{
\label{def:phon}
Das Phon ist eine segmentale phonetische Realisierung.
\index{Phon}
}

Der Begriff des Phonems baut dann auf dem des Phons auf, denn die Phoneme sind Abstraktionen von Phonen.
Wenn nämlich mehrere Phone distinktiv sind, gehören sie zu verschiedenen Phonemen, sonst sind sie lediglich Realisierungen eines einzigen abstrakten Phonems.
Als Beispiel kann man wieder \textipa{[\c{c}]} und \textipa{[X]} heranziehen (vgl.\ Abschnitt~\ref{sec:prozichach}).
Diese beiden Phone können keine Bedeutungen unterscheiden (es gibt keine Minimalpaare, vgl.\ Abschnitt~\ref{sec:verteilungen}) und können daher als Realisierungen eines abstrakten Phonems /\textipa{x}/ angesehen werden.
Man würde sagen, \textipa{[\c{c}]} und \textipa{[X]} sind Allophone eines Phonems /x/.
Wie man das Phonem nennt, ist dabei egal.
Man könnte es auch /P\Tidx{42}/ oder /\#/ nennen, solange nicht schon ein anderes Phonem so benannt wurde.

\Definition{Phonem}{
\label{def:phonem}
Ein Phonem ist eine Abstraktion von (potentiell) mehreren Phonen, die nicht distinktiv sind.
Die verschiedenen möglichen Phone zu einem Phonem werden Allophone genannt.
\index{Phonem}
}

Als Beispiel wird (\ref{ex:phol2209}) gegeben.

\begin{exe}
  \ex\label{ex:phol2209}
  \begin{xlist}
    \ex{\label{ex:phol2209a} \textit{ich}: Phone: \textipa{[I\c{c}]}, Phoneme: /\textipa{Ix}/}
    \ex{\label{ex:phol2209b} \textit{ach}: Phone: \textipa{[aX]}, Phoneme: /\textipa{ax}/}
  \end{xlist}
\end{exe}

An dieser Theorie ist im Prinzip nichts Falsches, sie ist lediglich explanatorisch schwächer als die bisher vorgestellte Theorie.
Die Phoneme sind zunächst nur abstrakte Größen, die nicht als Mengen von Merkmalen, sondern über die Distinktivität definiert werden.
Selbst wenn man Merkmalsanalysen hinzufügt, fehlt das Konzept des phonologischen Prozesses.
Phonologische Alternationen können also nicht effektiv als Prozess (Änderung von Werten phonologischer Merkmale) beschrieben werden.

Man kann dies an der Auslautverhärtung gut demonstrieren.
In der hier benutzten Darstellung lässt sich die Auslautverhärtung kompakt als Prozess der Änderung eines Merkmals unter einer bestimmten Bedingung formulieren (vgl.\ Abschnitt~\ref{sec:prozauslautverh}).
In einer reinen Phonemtheorie müsste man sagen, dass das Phonem /\textipa{b}/ je nach Umgebung zwei Allophone hat, nämlich Allophon \textipa{[p]} im Silbenauslaut und Allophon \textipa{[b]} in allen anderen Positionen.
Dasselbe müsste man für /\textipa{d}/ und /\textipa{g}/ (und ihre Allophone) wiederholen, wobei die eigentliche Regularität, die wir in einem einfachen Prozess dargestellt haben, nicht erfasst wird.

Als abschließendes Beispiel soll gezeigt werden, dass sich die fehlende Merk\-mals\-ana\-lyse noch auf ganz andere Weise bemerkbar macht.
Die Phone \textipa{[h]} und \textipa{[N]} sind im Deutschen zueinander nicht distinktiv (vgl.\ Abschnitt~\ref{sec:verteilungen}, vor allem (\ref{ex:phol6439}) auf S.~\pageref{ex:phol6439}).
Man könnte sie daher ohne weiteres als Allophone eines abstrakten Phonems /\textipa{h}/ auffassen.
Dieses Phonem hätte zwei Allophone, nämlich \textipa{[h]} im Onset und \textipa{[N]} in Coda.
Wegen der geringen phonetischen Ähnlichkeit dieser potentiellen Allophone (vgl.\ die Merkmale der Segmente in Tabelle~\ref{tab:pholkonsmerk}) erscheint dies zunächst absurd.
Darüber hinaus stehen diese Segmente aber strukturell auch in keinerlei Beziehung, es ist sozusagen offensichtlicher Zufall, dass sie komplementär verteilt sind.
Bei \textipa{[\c{c}]} und \textipa{[X]} ist die komplementäre Verteilung hingegen eindeutig nicht zufällig, wie in Abschnitt~\ref{sec:prozichach} demonstriert wurde.
Daher fügt man für die Phonembildung als Lösungsversuch gerne die Bedingung hinzu, dass Allophone eines Phonems phonetisch ähnlich sein sollen.
Wenn es aber keine Merkmalsanalysen gibt, weiß man nicht so recht, was phonetische Ähnlichkeit eigentlich sein soll.

Außerdem kann man zeigen, dass phonetische Ähnlichkeit generell kein gutes Kriterium ist, wenn die strukturelle Analyse eine Allophon-Beziehung zwischen zwei Phonen nahelegt.
Nach Vokalen müsste man \zB annehmen, dass \textipa{[@]} und \textipa{[5]} als Allophone eines Phonems /\textipa{r}/ vorkommen.
Ebenso wäre im Onset \textipa{[K]} ein Allophon von /\textipa{r}/ (vgl.\ Abschnitt~\ref{sec:prozrvok}).%
\footnote{Hier wird absichtlich /\textipa{r}/ als Symbol für das Phonem verwendet, um deutlich zu machen, dass es sich eben nicht um eine zugrundeliegende Form handelt und man daher irgendein Symbol nehmen kann.
Hier ist es eben dasjenige, das der Schreibung entspricht.}
Phonetisch ähnlich sind sich \textipa{[@]} und \textipa{[K]} aber in keiner Weise.
Es zeigt sich also, dass die noch gebräuchliche Rede von Phonemen und Allophonen zwar nicht falsch ist, aber in vielen Punkten gegenüber der hier verwendeten Darstellung Nachteile mit sich bringt.

\Zusammenfassung

\begin{enumerate}
  \item Die Phonologie beschäftigt sich mit den phonetischen Unterschieden, die eine systematische grammatische Funktion haben.
  \item Nicht jedes Segment (=~jeder Laut) kommt in den gleichen Umgebungen vor, und man kann Segmente danach einteilen, ob sie in vollständig identischen, teilweise identischen oder gänzlich verschiedenen Umgebungen vorkommen.
  \item Solche Verteilungen kann man auch für Merkmale (statt ganzer Segmente) ermitteln, \zB kommen stimmhafte Obstruenten im Deutschen nicht im Silbenauslaut vor.
  \item Phonologische Prozesse (wie die Auslautverhärtung oder die Frikativierung von /\textipa{Ig}/ zu \textipa{[i\c{c}]}) verändern die im Lexikon abgelegten Segmentfolgen je nachdem, in welcher Umgebung sie realisiert werden.
  \item Silbenstrukturen sind nicht im Lexikon festgelegt, sondern werden den Wörtern durch einen Prozess zugewiesen.
  \item Alle Silben folgen der Sonoritätshierarchie sowie weiteren sprachspezifischen Bedingungen (\zB Beschränkung der Plateaubildungen).
  \item \textbf{?? TODO}
  \item \textbf{?? TODO}
  \item Der Wortakzent ist die Hervorhebung einer Silbe im Wort durch Lautstärke, Länge usw.
  \item Das Deutsche ist dominant trochäisch mit der Betonung auf der ersten Silbe des Wortstamms.
\end{enumerate}

\Uebungen

\Uebung \label{u41} Finden Sie deutsche Minimalpaare für die folgenden Kontraste in der Art des ersten Beispiels.

\begin{enumerate}\Lf
  \item{/\textipa{t}/, /\textipa{d}/ : \textit{Tank}, \textit{Dank}}
  \item{/\textipa{n}/, /\textipa{s}/}
  \item{/\textipa{v}/, /\textipa{m}/}
  \item{/\textipa{X}/, /\textipa{N}/}
  \item{/\textipa{K}/, /\textipa{h}/}
  \item{/\textipa{s}/, /\textipa{k}/}
  \item{/\textipa{\t{pf}}/, /\textipa{s}/}
  \item{/\textipa{\t{aE}}/, /\textipa{\t{aO}}/}
  \item{/\textipa{i:}/, /\textipa{I}/}
\end{enumerate}

\Uebung \label{u42} Zeichnen Sie die Paare von nicht umgelauteten Vokalen und umgelauteten Vokalen in ein Vokalviereck und beschreiben Sie das Phänomen Umlaut dann mittels phonologischer Merkmale.
Die Vokalpaare mit und ohne Umlaut finden Sie in \textit{Fuß} -- \textit{Füße}, \textit{Genuss} -- \textit{Genüsse}, \textit{rot} -- \textit{röter}, \textit{Koffer} -- \textit{Köfferchen}, \textit{Schlag} -- \textit{Schläge}, \textit{Bach} -- \textit{Bäche}.
Zusatzaufgabe: Versuchen Sie, den Umlaut /\textipa{\t{aO}}/ -- /\textipa{\t{O\oe}}/ in die Beschreibung zu integrieren.

\Uebung[\tristar] \label{u43} Diese Übung bezieht sich auf Abschnitt~\ref{sec:prozichach}.

\begin{enumerate}\Lf
  \item Überlegen Sie, wie sich im Fall von Lehnwörtern wie \textit{Chemie} oder \textit{Chuzpe} die teilweise üblichen Realisierungen wie \textipa{[\c{c}emi:]} und \textipa{[XU\t{ts}p@]} in das phonologische System des Deutschen integrieren.
  \item Wie beurteilen Sie unter dem Gesichtspunkt des phonologischen Systems des Deutschen die Strategien, statt \textipa{[\c{c}emi:]} entweder \textipa{[Semi:]} oder \textipa{[kemi:]} zu realisieren?
  \item Bedenken Sie die Tatsache, dass für \textit{Chuzpe} niemals \textipa{[SU\t{ts}p@]} oder \textipa{[kU\t{ts}p@]} realisiert werden.
    Was sagt Ihnen das über die Integration des Wortes \textit{Chuzpe} in den deutschen Wortschatz (im Vergleich zu \textit{Chemie})?
\end{enumerate}

\Uebung \label{u44} Zerteilen Sie die folgenden Wörter in ihre Silben (Silbifizierung) und zeichnen Sie eine Sonoritätskurve wie in Abbildung~\ref{fig:sonhiers-strolchst}.
Geben Sie an, welche Bedingungen des Silbifizierungsprozesses (Abschnitt~\ref{sec:silbifizierung}) erfüllt werden und welche nicht.

\begin{enumerate}\Lf
  \item Strumpf
  \item wringen
  \item winkte
  \item Quarkspeise
  \item Leser
  \item Leserin
  \item zusätzlich
  \item zusätzliche
  \item Hammer
  \item Fenster
  \item Iglu
  \item komplett
\end{enumerate}

\Uebung \label{u45} Entscheiden Sie, wo die folgenden Wörter ihren Akzent haben (ggf.\ unter Zuhilfenahme des Betonungstests).
Überlegen Sie, ob sie damit den Regeln aus Abschnitt~\ref{sec:prosodie} folgen.

\begin{enumerate}\Lf
  \item freches
  \item Klingel
  \item Opa
  \item nachdem
  \item Auto
  \item Autoreifen
  \item Beendigung
  \item Melone
  \item rötlich
  \item Rötlichkeit
  \item Pöbelei
  \item respektabel
  \item Schulentwicklungsplan
\end{enumerate}

\Uebung[\tristar] \label{u46} Beschreiben Sie die Silbenstruktur in Wörtern wie \textit{Herbst}, \textit{lebst}, \textit{kriegst} usw.
Was fällt auf?

\Uebung[\tristar] \label{u47} In (\ref{ex:phol8882}) auf Seite \pageref{ex:phol8882} wird behauptet, dass \textipa{[s5]} im Deutschen kein Einsilbler sein kann.
Nennen Sie zwei Gründe, warum das so ist.

\WeitereLiteratur

\paragraph*{Phonetik}

Eine sehr ausführliche Einführung in die artikulatorische Phonetik ist \citet{Laver94}.
Einführende Darstellungen der deutschen Phonetik finden sich \zB in \citet{RRKWS09} und \citet{Wiese10}.
Eine ausführliche Beschreibung der deutschen Standardvarietäten (Deutschland, Österreich, Schweiz), der wir hier überwiegend gefolgt sind, gibt \citet{Krech-ea2009}.
Ein weiteres Nachschlagewerk mit kleinen Unterschieden in der Darstellung zu \citealp{Krech-ea2009} ist \citet{Mangold06}.

\paragraph*{Phonologie}

\label{abs:pholliteratur}

Der hier zur Phonologie besprochene Stoff findet sich mit teilweise erheblichen Abweichungen in der Darstellung \zB in \citet{Hall00} und \citet{Wiese10}.
In eine grammatische Gesamtbeschreibung eingebunden sind Kapitel~3 und~4 im \textit{Grundriss} \citep{Eisenberg1}.
Eine Einführung, die eher strukturalistisch argumentiert, ist \citet{Ternes2012}.
Als anspruchsvolle Gesamtdarstellung der deutschen Phonologie kann \citet{Wiese00} verwendet werden.
