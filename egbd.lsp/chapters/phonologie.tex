\chapter{Phonologie}

\label{sec:phonologie}

Die im letzten Kapitel besprochene artikulatorische Phonetik beschreibt die physiologischen Grundlagen der Sprachproduktion.
Anhand des Vorrats an Zeichen im Alphabet der IPA haben wir außerdem definiert, welche Laute im in Deutschland gesprochenen Standarddeutschen vorkommen.
Die eigentliche Frage der systematischen Grammatik bezüglich der Lautgestalt von Wörtern und größeren Einheiten ist aber, nach welchen Regularitäten diese Laute verbunden werden, und welchen Stellenwert die einzelnen Segmente und Segmentverbindungen (wie \zB Silben) im gesamten Lautsystem haben.
In der Phonologie geht es daher um das \textit{Lautsystem} und seine Regularitäten.
Anders als in vielen anderen Einführungen (und in der ersten Auflage dieses Buches) wird hier mit der Analyse des Silbenbaus begonnen (Abschnitt~\ref{sec:phonotaktik}). 
Abschließend gibt Abschnitt~\ref{sec:prosodie} einen Einblick in die Prosodie (die Betonungslehre) und die wichtigen phonologischen Aspekte auf der Wortebene.
In Abschnitt~\ref{sec:segmentalephol} wird der Status einzelner Laute und ihrer Vorkommen behandelt.
Es wird diskutiert, wie man Laute mit Merkmalen beschreiben kann und wie Laute im Lexikon gespeichert sind.
Schließlich werden einige konkrete phonologische Prozesse des Deutschen wie die Auslautverhärtung diskutiert.





\section{Silben und Wörter}

\label{sec:phonotaktik}

\subsection{Phonotaktik}

Aufbauend auf die Beschreibung der einzelnen Segmente des Deutschen (Kapitel~\ref{sec:phonetik}) kann und sollte beschrieben werden, wie diese Segmente zu größeren Einheiten zusammengesetzt werden, wie also \textit{phonologische Struktur} (zum Strukturbegriff vgl.\ Abschnitt~\ref{sec:strukturen}, S.~\pageref{sec:strukturen}) aufgebaut wird.
Die Wörter in (\ref{ex:phol2852}) sind Phantasiewörter in Standardorthographie und hypothetischer phonetischer Umschrift.

\begin{exe}
  \ex\label{ex:phol2852}
  \begin{xlist}
    \ex{\label{ex:phol2852a} Nka \textipa{[Nka:]}, Totk \textipa{[tOtk]}, Pkafkme \textipa{[pkafkm@]}}
    \ex{\label{ex:phol2852b} Klie \textipa{[kli:]}, Filb \textipa{[fIlp]}, Renge \textipa{[KEN@]}}
  \end{xlist}
\end{exe}

Die hypothetischen Wörter in (\ref{ex:phol2852a}) unterscheiden sich deutlich von denen in (\ref{ex:phol2852b}).
Während die zweite Gruppe nämlich zumindest mögliche Wörter des Deutschen darstellt, enthält die erste Gruppe nur Wörter, die aus irgendeinem Grund niemals Wörter des Gegenwartsdeutschen sein könnten.
Der Grund dafür ist, dass die erste Gruppe \textit{phonotaktisch nicht wohlgeformte Wörter} enthält.
Es muss also Regularitäten geben, nach denen sich Segmente des Deutschen zu größeren Einheiten wie Silben und Wörtern zusammensetzen.

\Definition{Phonotaktik}{
Die Phonotaktik beschreibt die Regularitäten, nach denen Segmente zu größeren Strukturen zusammengesetzt werden.
Die Phonotaktik definiert Einheiten wie die \textit{Silbe} und das \textit{Wort}.
\index{Phonotaktik}
}

Die Silbe ist die Einheit, mittels derer die meisten Einschränkungen für mögliche Segmentfolgen formuliert werden können.
Dieser Abschnitt ist daher ausschließlich der Silbe gewidmet.

\subsection{Silben}

\label{sec:silben}

\index{Silbe}

Was Silben genau sind, ist nicht unbedingt leicht zu definieren.
Sie sind intuitiv Einheiten, die größer sein können (aber nicht müssen) als Segmente, aber kleiner sein können (nicht müssen) als Wörter.
Der Extremfall, bei dem Segment, Silbe und Wort zusammenfallen, ist im Deutschen nur möglich, wenn man den Glottalverschluss ignoriert (s.~\ref{sec:glottalverschluss}).
Selbst dann ist dieser Fall im Deutschen im normalen Wortschatz marginal und auf Interjektionen (Rufwörter) wie \textit{oh} \textipa{[Po:]} und \textit{ah} \textipa{[Pa:]} beschränkt.
Wenn man Diphthonge als ein Segment zählt, kommt das Substantiv \textit{Ei} \textipa{[P\t{aE}]} hinzu.
Auch in anderen Sprachen ist dieser Fall eher selten, vergleiche französische Substantive wie \textit{œufs} \textipa{[\o:]} `Eier' (nur im Plural) oder \textit{eau} \textipa{[o:]} `Wasser' sowie das schwedische Substantiv \textit{ö} \textipa{[\o:]} `Insel' (nur im Singular), die insgesamt auch innerhalb ihrer eigenen Sprachsysteme eher Exoten darstellen.%
\footnote{Auf jeden Fall entfällt in diesen Sprachen aber der Glottalverschluss.} 
Meistens bestehen Silben aus mehreren Segmenten, und Wörter bestehen häufig aus mehreren Silben.
Beispiele für einsilbige Wörter aus zwei Segmenten im Deutschen sind \textit{Schuh} \textipa{[Su:]} oder \textit{Tee} \textipa{[te:]}, Beispiele für zweisilbige Wörter aus zweisegmentigen Silben sind \textit{Tüte} \textipa{[ty:t@]} oder \textit{rege} \textipa{[Ke:g@]}.
Ein einsilbiges Wort mit deutlich mehr als zwei Segmenten ist \textit{Strauch} \textipa{[StK\t{aO}X]}. 
Die wesentliche Frage der Silbenphonologie wird sein, wie hoch die Komplexität solcher Strukturen maximal sein kann.

\index{Silbe!Klatschmethode}

In der Grundschuldidaktik wird oft über die \textit{Klatschmethode} versucht, Kindern ein Gefühl für Silben zu vermitteln.
Dabei wird gesagt, dass jedes Stück eines Wortes, zu dem man bei abgehacktem Sprechen einmal klatschen kann, eine Silbe sei.
Diese Methode ist problematisch, da sie sehr leicht absichtlich oder unabsichtlich sabotierbar ist:
Es ist für viele Sprecher vielleicht natürlicher, auf Wörter wie \textit{Mutter} \textipa{[mUt5]} nur einmal zu klatschen, da die Silbe mit dem \textipa{[5]} unbetont und phonologisch nicht sehr prominent ist.
Außerdem wird mit der Methode meist ein rein orthographisch-didaktisches Ziel ohne jede Sensibilität für Grammatik verfolgt, nämlich das Erlernen der Silbentrennung in der Schrift.
Die Regeln der orthographischen Silbentrennung im Deutschen erfordern aber subtilere Kenntnisse grammatischer Regularitäten, als sie die Klatschmethode vermitteln kann.
Ein Kind wird durch das Klatschen vielleicht intuitiv lernen, dass Wörter wie \textit{Kriecher}, \textit{Iglu} oder \textit{Mutter} aus genau zwei Silben bestehen.
Ob die Silbentrennung aber \textit{Krie-cher} oder \textit{Kriech-er}, \textit{I-glu} oder \textit{Ig-lu} und \textit{Mutt-er}, \textit{Mut-ter} oder \textit{Mu-tter} ist, ist prinzipiell durch Klatschen nicht erlernbar.
Daher müssen Lehrer bei solchen Übungen dann unnatürliche Aussprachen vormachen, \zB \textipa{[mUt]} -- \textipa{[ta]} oder gar \textipa{[mUt]} -- \textipa{[tEK]} statt korrekt \textipa{[mUt5]}.
Gerade dieses Abhacken macht \textit{Kriech-er} aber genauso plausibel wie \textit{Krie-cher}.
In Fall der zerhackten Aussprache in Fällen mit orthographischen Doppelkonsonanten wie \textipa{[mUt]} -- \textipa{[ta]} muss man zudem paradoxerweise bereits Kenntnisse der Orthographie und Silbentrennung besitzen, um das Wort überhaupt so aussprechen zu können.
Man dreht sich also im Kreis, und ein solider Lernerfolg durch das Klatschen ist daher nicht zu erwarten.%
\footnote{Aus meiner eigenen -- zugegebenermaßen länger zurückliegenden -- Grundschulerfahrung als Schüler mit zwei Lehrerinnen in zwei verschiedenen Bundesländern läuft die Unterrichtseinheit dann so ab, dass einige Kinder aus Haushalten mit hohem Bildungsniveau bereits seit längerem lesen können und die Silbentrennung durch Anschauung beim Lesen intuitiv gelernt haben.
Diese Kinder verstehen in den Augen des Lehrpersonals durch das Klatschen, wie Wörter zu trennen sind.
Alle andere Kinder gelten ohne ihr Verschulden als schwierig bzw.\ langsame Lerner.
Diese Beobachtung hat natürlich keinen Anspruch auf Allgemeingültigkeit.}

Trotz ihrer absoluten Unzulänglichkeit für den Grundschulunterricht veranschaulicht die Klatschmethode (recht umständlich) allerdings ein wichtiges Prinzip der Silbenbildung.
Silben bringen die Segmente in eine schematische Ordnung, die charakteristischen artikulatorischen Einheiten entspricht.
Diese artikulatorischen Einheiten sind Schübe, die im Prinzip einem Öffnen und Schließen des Vokaltraktes entsprechen.
An einsilbigen Wörtern wie \textit{Tag} \textipa{[ta:k]} oder \textit{gut} \textipa{[gU:t]} sieht man, dass sie mit einem Verschluss beginnen und mit einem Verschluss enden, während in der Mitte beim Vokal der Vokaltrakt geöffnet ist (genauer in Abschnitt~\ref{sec:sonoritaet}).
Im Kern der Silbe befindet sich passend dazu im Deutschen immer ein Vokal, also ein Segment bei dem sich die Artikulatoren nicht punktuell annähern (Abschnitt~\ref{sec:vokale}).
Die Klatschmethode kann man also auf die Anweisung reduzieren, bei jedem Vokal einmal zu klatschen.
Wie an den Zweifelsfällen weiter oben gezeigt wurde, löst das aber nicht das Problem, ob Konsonanten zwischen den Vokalen in mehrsilbigen Wörtern zur ersten oder zweiten Silbe gehören.

Komplizierter wird die Silbenphonologie dadurch, dass in den Formen eines Wortes die Silbengrenzen nicht konstant sind.
Anders gesagt ist die Silbenstruktur von Wörtern nicht im Lexikon festgelegt.
Die Beispiele (\ref{ex:phol1830}) zeigen dies.
In der Transkription werden die Silbengrenzen durch einen einfachen Punkt markiert.

\begin{exe}
  \ex\label{ex:phol1830}
  \begin{xlist}
    \ex{Ball \textipa{[bal]}, Bälle \textipa{[bE.l@]}}
    \ex{Knall \textipa{[knal]}, Knalls \textipa{[knals]}}
    \ex{Sturm \textipa{[St\t{U@}m]}, Stürme \textipa{[St\t{Y@}.m@]}}
    \ex{Mittelstürmer \textipa{[mI.t@l.St\t{Y@}.m5]}, Mittelstürmerin \textipa{[mI.t@l.St\t{Y@}.m@.KIn]}}
  \end{xlist}
\end{exe}

Ein Wort wie \textit{Ball} ist im Nominativ Singular einsilbig, und das \textipa{[l]} steht im Auslaut (am Ende) dieser einen Silbe.
Mit dem hinzutretenden \textipa{[@]} der Plural-Endung verändert sich auch die Silbenstruktur:
Das \textipa{[l]} steht im Anlaut (am Anfang) der zweiten Silbe.
Ähnliches passiert bei \textit{Sturm} und \textit{Stürme} mit dem \textipa{[m]}.
Bei \textit{Mittelstürmer} \textipa{[mI.t@l.St\t{Y@}.m5]} und \textit{Mittelstürmerin} \textipa{[mI.t@l.St\t{Y@}.m@.KIn]} wird es noch komplizierter, weil /\textipa{K}/ nur dann als Konsonant \textipa{[K]} realisiert wird, wenn noch ein Vokal folgt und das /\textipa{K}/ dadurch in den Silbenanlaut gerät (vgl.\ dazu genauer Abschnitt~\ref{sec:prozrvok}).
Wenn bei \textit{Ball} und \textit{Balls} aber ein \textipa{[s]} hinzutritt, bleibt das Wort einsilbig, und das \textipa{[s]} wird an die einzige Silbe hinten angehängt.
Die Silbenbildung, so wie sie hier betrachtet wird, ist also keine phonetische Fragestellung, sondern eine phonologische, weil ihre Beschreibung es erfordert, dass das Gesamtsystem (also \zB alle Formen eines Wortes) betrachtet werden.
Entsprechend wird Definition~\ref{def:silbe} gegeben.

\Definition{Silbe und Silbifizierung}{\label{def:silbe}
Silben sind die nächstgrößeren phonologischen Einheiten nach den Segmenten.
Die Segmente sind ihre kleinsten Konstituenten.
Die Silbenstruktur ist nicht im Lexikon abgelegt und wird durch einen Prozess zugewiesen (Silbifizierung).
\index{Silbe}
}

Mit Klatschen ist es also nicht getan.
Der analytische Einstieg in die Silbenstruktur des Deutschen gelingt am leichtesten über einsilbige Wörter.
Abschnitt~\ref{sec:einsilbler} leistet (nach der Einführung einiger technischer Begriffe in Abschnitt~\ref{sec:silbenstruktur}) daher zunächst eine einfache Beschreibung möglicher einsilbiger Wörter des Deutschen.
Die Verallgemeinerung zu mehrsilbigen Wörtern erfolgt nach einer theoretischen Ergänzung (Abschnitte~\ref{sec:sonoritaet}) in Abschnitt~\ref{sec:mehrsilbler}.






\subsection{Strukturformat für Silben}

\label{sec:silbenstruktur}

In diesem Abschnitt wird nur die Terminologie eingeführt, mit der man über Positionen in der Silbe redet.
Offensichtlich bilden Silben komplexere Strukturen aus, die sich um einen Vokal oder Diphthong im \textit{Kern} herum gruppieren.%
\footnote{Eine alternative Sichtweise würde bei Diphthongen das zweite Glied als Teil der Coda analysieren.
Für unsere Zwecke ist der sich ergebende theoretische Unterschied vernachlässigbar.}
Für die drei sich ergebenden Konstituenten der Silbe gibt es verschiedene Bezeichnungen, von denen hier \textit{Anfangsrand}, \textit{Kern} und \textit{Endrand} verwendet werden.
Aus Gründen, die erst in Abschnitt~\ref{sec:mehrsilbler} diskutiert werden, hat es sich als nützlich erwiesen, Kern und Endrand zu einer eigenen Konstituente, dem \textit{Reim} zusammenzufassen.
Neben Definition~\ref{def:kern} wird eine Baumdarstellung der allgemeinen Silbenstruktur in Abbildung~\ref{fig:silbenstruktur} und ein Beispiel (\textit{fremd}) in Abbildung~\ref{fig:phonstr} gegeben.

\Definition{Silbenstruktur}{
\label{def:kern}
\label{def:anfangsrand}
\label{def:endrand}
Der \textit{Silbenkern} (der \textit{Nukleus}) wird immer durch einen Vokal oder Diphthong gebildet.
Die Konsonanten vor dem Kern bilden den \textit{Anfangsrand} (den \textit{Onset}), die nach dem Kern den \textit{Endrand} (die \textit{Coda}).
Es gibt Silben mit leeren Anfangs- und\slash oder Endrändern, aber keine Silben ohne Kern.
Kern und Endrand bilden den \textit{Reim}.
\index{Silbe!Kern}
\index{Silbe!Anfangsrand}
\index{Silbe!Endrand}
\index{Silbe!Reim}
}

\begin{figure}
  \centering
  \Tree[2]{
    & \K{Silbe}\B{dl}\B{d} \\
    \K{Anfangsrand} & \K{Reim}\B{d}\B{dr} \\
    & \K{Kern} & \K{Endrand}\\
  }
  \caption{Silbenstruktur}
  \label{fig:silbenstruktur}
\end{figure}

\begin{figure}
  \centering
  \Tree[1.5]{
  & && \K{Silbe}\B{dll}\B{d} \\
  & \K{Anfangsrand}\B{ddl}\B{ddr} && \K{Reim}\B{d}\B{drr} \\
  &&& \K{Kern}\B{d} && \K{Endrand}\B{dl}\B{dr} \\
  \K{\textipa{[f]}} && \K{\textipa{[K]}} & \K{\textipa{[E]}} & \K{\textipa{[m]}} && \K{\textipa{[t]}} \\
  }
  \caption{Beispiel für Silbenstruktur}
  \label{fig:phonstr}
\end{figure}








\subsection{Einsilbler}

\label{sec:einsilbler}

In diesem Abschnitt werden einsilbige Wörter herangezogen, um die minimale und die maximale Komplexität deutscher Silben zu ermitteln.%
\footnote{Es ist für die meisten Menschen unmöglich, eine solche Beschreibung durch einmaliges oder zweimaliges Lesen zu memorieren.
Dies sollte beim Lesen zu beachtet werden, um Frustrationen zu vermeiden.
Es geht hier um eine möglichst genaue Darstellung der \textit{Methode}, mit der die möglichen Silbentypen einer Sprache ermittelt werden.
Eine bessere Übersicht bietet \textit{nach} der Lektüre dieses Abschnitts dann Abschnitt~\ref{sec:anfangsrandendrand}.}
Ein einsilbiges Wort wird üblicherweise \textit{Einsilbler} genannt.

\subsubsection{Anfangsrand}

In Abschnitt~\ref{sec:silben} wurde bereits festgestellt, dass Silben -- und damit Einsilbler -- mindestens aus einem Vokal im Silbenkern bestehen.
Gleichzeitig enthält eine Silbe immer genau einen (niemals zwei oder mehr) Vokale.
Diesem Vokal geht im Deutschen immer der Glottalverschluss voraus, wenn kein anderer Konsonant vorausgeht.
Maximal einfache Einsilbler sind also die (\ref{ex:phol777200}), wobei Diphthonge wie ein einfacher Vokal behandelt werden.

\begin{exe}
	\ex\label{ex:phol777200}
	\begin{xlist}
		\ex Ei \textipa{[P\t{aE}]}	
		\ex ah \textipa{[Pa:]}	
		\ex oh \textipa{[Po:]}	
	\end{xlist}
\end{exe}

Wir beginnen mit dem Anfangsrand und überlegen der Reihe nach, ob dort ein, zwei oder auch mehr Segmente stehen können, und falls es so ist, welche und in welcher Reihenfolge.
Der Anfangsrand kann durch ein einzelnes konsonantisches Segment einer beliebigen Artikulationsart besetzt werden.
In (\ref{ex:phol777201a}) sind es stimmlose und stimmhafte Plosive, in (\ref{ex:phol777201b}) stimmlose und stimmhafte Frikative (bis auf \textipa{[X]} bzw.\ \textipa{[\c{c}]}), in (\ref{ex:phol777201c}) Nasale (bis auf \textipa{[N]}) und in (\ref{ex:phol777201d}) der Approximant.
Der Nasal \textipa{[N]} sowie die Frikative \textipa{[\c{c}]} und \textipa{[X]} kommen prinzipiell im Anfangsrand nicht vor und werden aus allen weiteren Überlegungen über diese Position ausgeschlossen.%
\footnote{Beispielwörter, die in diesem Abschnitt unmögliche Kombinationen illustrieren sollen, werden pseudo-orthographisch und in IPA-Transkription wiedergegeben.
Der Asterisk * steht dabei immer nur vor der pseudo-orthographischen Version.
Es ist zu beachten, dass die entsprechenden Wörter nicht einfach nur durch Zufall nicht existieren.
Sie könnten vielmehr keine Wörter des Deutschen sein, weil das System die entsprechenden Silbenstrukturen nicht zulässt.}

\begin{exe}
	\ex\label{ex:phol777201}
	\begin{xlist}
		\ex{\label{ex:phol777201a} Kuh \textipa{[ku:]}, geh \textipa{[ge:]}}
		\ex{\label{ex:phol777201b} Schuh \textipa{[Su:]}, hau \textipa{[h\t{aO}]}, Reh \textipa{[Ke:]}, Vieh \textipa{[fi:]}, wo \textipa{[vo:]}}, *chie \textipa{[\c{c}i:]}\slash\textipa{[Xi:]}
		\ex{\label{ex:phol777201c} nie \textipa{[ni:]}, mäh \textipa{[mE:]}, *ngu \textipa{[Nu:]}}
		\ex{\label{ex:phol777201d} lau \textipa{[l\t{aO}]}}
	\end{xlist}
\end{exe}

Da Einsilbler also immer mindestens aus einem Vokal im Kern und einem Glottalverschluss bzw.\ anderen Konsonanten im Anfangsrand bestehen, beginnt die artikulatorische Bewegung mit einem Verschluss und führt auf jeden Fall zu einer maximalen Öffnung. 
Wenn im Anfangsrand \textit{zwei} Konsonanten stehen, sind die Kombinationsmöglichkeiten bereits erheblich eingeschränkt.
In unseren Überlegungen setzen wir jetzt jeweils (in dieser Reihenfolge) Plosive, Frikative, Nasale und Approximanten als zweites Segment ein und überlegen, welche Segmente dann davor stehen können.
Die Beispiele sind möglichst so gewählt, dass rechts vom Vokal nichts steht, aber wenn ein solches Beispiel zufällig nicht existiert, wird auf andere Einsilbler (im Notfall auf Mehrsilbler) ausgewichen.

\textbf{Plosive an zweiter Position} sind im zweisegmentalen Anfangsrand nahezu unmöglich (\ref{ex:phol777202a}) mit der Ausnahme von \textipa{[p]} und \textipa{[t]} nach \textipa{[S]} (\ref{ex:phol777202a}).
Es gibt jedoch sehr seltene Lehnwörter (meist keine Einsilbler), die abweichende Konsonantenverbindungen links vom Vokal enthalten.
Diese wenigen Ausnahmen wie in (\ref{ex:phol777202c}) sind wegen dieses ungewöhnlichen Silbenbaus nicht zum Kern des Systems zu rechnen (Abschnitt~\ref{sec:kern}).
Sie sind also nicht nur Lehnwörter, sondern auch Fremdwörter.
Wörter wie \textit{stygisch} sind im Übrigen nur dann betroffen, wenn \textipa{[st]} statt \textipa{[St]} gesprochen wird.

\begin{exe}
	\ex\label{ex:phol777202}
	\begin{xlist}
		\ex{\label{ex:phol777202a} *pteh \textipa{[pte:]}, *fpeh \textipa{[fpe:]}, *schguh \textipa{[Sgu:]}, *lta \textipa{[lta:]} usw.}
		\ex{\label{ex:phol777202b} spei \textipa{[Sp\t{aE}]}, steh \textipa{[Ste:]} }
		\ex{\label{ex:phol777202c} Pte(ranodon) \textipa{[pteKanodOn]}, chtho(nisch) \textipa{[Xto:nIS]}, sty(gisch) \textipa{[sty:gIS]}}
	\end{xlist}
\end{exe}

\textbf{Frikative an zweiter Position} kommen nur eingeschränkt vor.
Da wir \textipa{[\t{pf}]} wie in \textit{Pfau} und \textipa{[\t{ts}]} wie in \textit{zieh} sowie das seltene \textipa{[\t{tS}]} wie in \textit{Chips} als Affrikaten (also ein einziger Konsonant) auffassen (Abschnitte~\ref{sec:affrikatenhomorgan} und~\ref{sec:affrikaten}), fallen die Frikative \textipa{[f]}, \textipa{[s]}, \textipa{[S]}, \textipa{[h]}, \textipa{[z]} und \textipa{[J]} komplett als zweites Segment im Anfangsrand aus (\ref{ex:phol777203a}).%
\footnote{Außerdem kann die Kombination \textipa{[tJ]} bzw.\ \textipa{[t\c{c}a]} wie in \textit{tja} \textipa{[tJa]} (oder \textipa{[t\c{c}a]})  oder dem norddeutschen Namen \textit{Tjark} \textipa{[tJ\t{aE}k]} (oder \textipa{[t\s{c}\t{aE}k]}) zum System gerechnet werden oder nicht.
Wesentliches ändert sich dadurch nicht, und wir sehen hier davon ab.}
Es kommt \textipa{[K]} vor, aber nur nach den Plosiven \textipa{[f]}, \textipa{[S]} und \textipa{[v]} (\ref{ex:phol777203b}).
Außerdem findet man \textipa{[v]}, aber nur nach \textipa{[k]} und \textipa{[S]} (\ref{ex:phol777203c}).

\begin{exe}
	\ex\label{ex:phol777203}
	\begin{xlist}
		\ex{\label{ex:phol777203a} *ksie \textipa{[ksi:]}, *tfa \textipa{[tfa:]}, *gsau \textipa{[gz\t{aO}]} usw.}
		\ex{\label{ex:phol777203b} Pracht \textipa{[pKaXt]}, brüh \textipa{[bKy:]}, trau \textipa{[tK\t{aO}]}, dreh \textipa{[dKe:]}, kräh \textipa{[kKE:]}, grau \textipa{[gK\t{aO}]}, früh \textipa{[fKy:]}, Schrei \textipa{[SK\t{aE}]}, Wrack \textipa{[vKak]}}
		\ex{\label{ex:phol777203c} Qual \textipa{[kva:l]}, Schwur \textipa{[Sv\t{u5}]}}
	\end{xlist}
\end{exe}

\textbf{Nasale an zweiter Position} kommen ebenfalls kaum vor, sowohl nach Plosiven (\ref{ex:phol777204a}) als auch nach Frikativen (\ref{ex:phol777204b}).
Die einzigen systematischen Ausnahmen sind \textipa{[kn]} und selten \textipa{[gn]} (\ref{ex:phol777204c}) sowie \textipa{[Sn]} und \textipa{[Sm]} (\ref{ex:phol777204d}).%
\footnote{Wörter mit \textipa{[pn]} sind extrem seltene Lehnwörter wie \textit{Pneu}.
Das einzige häufiger vorkommende Erbwort mit \textipa{[gn]} in einem Anfangsrand ist \textit{Gnade}.
Alle anderen Wörter (\zB dialektal gefärbte wie \textit{Gnatz} und \textit{Gnitze} oder Lehnwörter wie \textit{Gnom} oder \textit{Gnosis}) sind selten.
Ob der Anlaut \textipa{[gn]} also zum Kern gehört, kann bezweifelt werden.}

\begin{exe}
	\ex\label{ex:phol777204}
	\begin{xlist}
		\ex{\label{ex:phol777204a} *pmeh \textipa{[pme:]}, *bnau \textipa{[bn\t{aO}]}, *tneh \textipa{[tne:]} usw.}
		\ex{\label{ex:phol777204b} *fnau \textipa{[fn\t{aO}]}, *smuh \textipa{[smu:]}, *rnie \textipa{[Kni:]} usw.}
		\ex{\label{ex:phol777204c} Knie \textipa{[kni:]} (Gnade \textipa{[gna:de]})}
		\ex{\label{ex:phol777204d} Schnee \textipa{[Sne:]}, schmäh \textipa{[SmE:]}}
	\end{xlist}
\end{exe}

Der einzige \textbf{laterale Approximant des Deutschen (\textipa{[l]}) an zweiter Position} ist im Wesentlichen auf zwei Fälle beschränkt.
Er steht nach allen Plosiven mit Ausnahme der alveolaren (\ref{ex:phol777205a}).
Außerdem findet man ihn nach den stimmlosen Frikativen \textipa{[f]} und \textipa{[S]} (\ref{ex:phol777205b}).

\begin{exe}
	\ex\label{ex:phol777205}
	\begin{xlist}
		\ex{\label{ex:phol777205a} Plan \textipa{[pla:n]}, blüh \textipa{[bly:]}, *tlüh \textipa{[tly:]}, *dlüh \textipa{[dly:]}, Klee \textipa{[kle:]}, glüh \textipa{[gly:]}}
		\ex{\label{ex:phol777205b} flieh \textipa{[fli:]}, Schlag \textipa{[Sla:k]}}
	\end{xlist}
\end{exe}

Die strukturellen Möglichkeiten für dreisegmentale Anfangsränder sind auf \textipa{[SpK]} und \textipa{[StK]} beschränkt (\ref{ex:phol777206a}).
Die wenigen (nicht einsilbigen) Wörter mit \textipa{[Spl]} (\ref{ex:phol777206b}) gehören wohl alle zur selben germanischen Grundform, sind dabei dialektal gefärbt bzw.\ aus dem Englischen entlehnt und können als peripher vernachlässigt werden.

\begin{exe}
	\ex\label{ex:phol777206}
	\begin{xlist}
		\ex{\label{ex:phol777206a} sprüh \textipa{[SpKy:]}, Stroh \textipa{[Stro:]}}
		\ex{\label{ex:phol777206b} Splitter \textipa{[SplIt5	]}, spleiß \textipa{[Spl\t{aE}s@n]}, Spliss \textipa{[SplIs]}}
	\end{xlist}
\end{exe}

\subsubsection{Endrand}

Der Endrand wird jetzt etwas kompakter abgearbeitet, und auf die IPA-Transkription und die Auflistung strukturell unmöglicher Pseudo-Beispiele wird dabei aus Gründen der Übersichtlichkeit verzichtet.
Zunächst kann man feststellen, dass im Endrand wegen der Auslautverhärtung (Abschnitte~\ref{sec:auslautverhaertungphonetik} und~\ref{sec:prozauslautverh}) keine stimmhaften Obstruenten vorkommen können, und dass damit \textipa{[b d g v z J]} aus der Betrachtung ausgeschlossen werden können.
Das Gleiche gilt für \textipa{[h]}, das nur im Anfangsrand auftritt.
Ähnlich wie der einsegmentale Anfangsrand kann der einsegmentale Endrand von allen verbleibenden Konsonanten gefüllt werden.
Obwohl das \textipa{[K]} im Endrand phonetisch ein Vokal ist, behandeln wir es hier als Frikativ.
Mehr zu den Gründen findet sich dann in Abschnitt~\ref{sec:prozrvok}.

\begin{exe}
  \ex\label{ex:phol4711}
  \begin{xlist}
  	\ex ab, Hut, Rock
  	\ex auf, aus, Hasch, ich, Loch, Ohr
  	\ex Raum, Zaun, Fang
  	\ex voll
  \end{xlist}
\end{exe}

Bei den zweisegmentalen Endrändern verfahren wir genau wie bei den zweisegmentalen Anfangsrändern.
Wir gehen also von innen die Artikulationsarten (Plosive, Frikative, Nasale, Approximanten) durch und prüfen, inwiefern sie die Wahl des zweiten Segments einschränken.
Anders als im Anfangsrand sind zunächst \textbf{Folgen aus zwei Plosiven} möglich, allerdings von allen sechs möglichen nur \textipa{[pt]} und \textipa{[kt]}.

\begin{exe}
  \ex\label{ex:phol4712}
  \begin{xlist}
  	\ex Abt, schleppt, klappt
  	\ex Takt, Sekt, nackt, rückt
  \end{xlist}
\end{exe}

Die Auswahl des zweiten Segments ist bei \textbf{Frikativen an erster Position} stark eingeschränkt.
Es kann nur ein Plosiv folgen, im Fall von \textipa{[f s \c{c} X]} ist dies aber immer \textipa{[t]} wie in (\ref{ex:phol4713a}).
Völlig ausgeschlossen ist \textipa{[S]}.
\textipa{[K]} kann hingegen mit allen stimmlosen Plosiven kombiniert werden, s.\ (\ref{ex:phol4713b}).

\begin{exe}
  \ex\label{ex:phol4713}
  \begin{xlist}
  	\ex{\label{ex:phol4713a} Luft, Lust, Licht, Acht}
  	\ex{\label{ex:phol4713b} Korb, Ort, Mark}
  \end{xlist}
\end{exe}

\textbf{Nasale in erster Position} kombinieren nur mit homorganen Plosiven, also solchen, die den gleichen Artikulationsort haben, vgl.\ (\ref{ex:phol4714}).
Vor allem \textipa{[mp]} ist allerdings sehr selten.

\begin{exe}
  \ex{\label{ex:phol4714} Lump, Hund, krank}
\end{exe}

In Kombination mit Frikativen sind \textipa{[nX]} und \textipa{[nK]} strikt ausgeschlossen.
Auch \textipa{[n\c{c}]} kommt nur in zwei nennenswert häufigen Wörtern vor, s. (\ref{ex:phol4715a}).
Etwas häufiger sind die Kombinationen \textipa{[nf]} uns \textipa{[ns]}, sehr selten hingegen wieder \textipa{[nS]}, was nur in zwei geläufigeren Wörtern vorkommt, s.\ (\ref{ex:phol4715c}).
\textipa{[ms]} wie in (\ref{ex:phol4715d}) und \textipa{[mS]} wie in (\ref{ex:phol4715e}) sind ähnlich rar, wobei \textipa{[mS]} durch Bildungen aus Eigennamen wie \textit{grimmsch} (in \textit{das grimmsche Wörterbuch}) gelegentlich vorkommen könnte.
\textipa{[Ns]} kommt durch Genitivbildungen von Substantiven häufiger vor, s.\ (\ref{ex:phol4715f}).

\begin{exe}
  \ex\label{ex:phol4715}
  \begin{xlist}
  	\ex{\label{ex:phol4715a} Mönch, manch}
  	\ex{\label{ex:phol4715b} Hanf, Senf, uns, eins, Gans}
  	\ex{\label{ex:phol4715c} Mensch, Punsch}
  	\ex{\label{ex:phol4715d} Ems, Wams, Gams}
  	\ex{\label{ex:phol4715e} Ramsch}
  	\ex{\label{ex:phol4715f} längs, rings, Hangs usw.}
  \end{xlist}
\end{exe}

\textipa{[mf]} und \textipa{[Nf]} sowie Kombinationen aus zwei Nasalen oder aus Nasal und Approximant sind gänzlich ausgeschlossen.
Das Problem bei der Sequenz aus Nasal und Frikativ im Endrand ist also vor allem die geringe Typenhäufigkeit einiger Kombinationen.
Ob man \zB für ein einzelnes Wort wie \textit{Ramsch} -- ggf.\ flankiert durch gespreizte Bildungen wie \textit{grimmsch} -- einen eigenen Silbentyp aufmachen möchte, ist wie bei ähnlichen Fällen im Anfangsrand kaum systematisch festzulegen.

Für den \textbf{Approximant in erster Position} ist die Angelegenheit etwas klarer.
Er kombiniert sich gut mit den drei Plosiven, vgl.\ (\ref{ex:phol4716a}).
Von den Frikativen kommen \textipa{[f s \c{c}]} in Frage wie in (\ref{ex:phol4716b}), \textipa{[S X K]} nicht.
Von den drei Nasalen können nur \textipa{[m n]} folgen, s.\ (\ref{ex:phol4716c}).
Dabei ist \textipa{[ln]} sehr charakteristisch für (meist mehrsilbige) Verbformen von Verben, deren Stamm (s.\ Abschnitt~\ref{sec:stamm}) auf \textipa{[l]} endet.

\begin{exe}
  \ex\label{ex:phol4716}
  \begin{xlist}
  	\ex{\label{ex:phol4716a} Alp, halb, Halt, bald, welk, Talg}
  	\ex{\label{ex:phol4716b} elf, Wolf, Hals, Fels, Milch, solch}
  	\ex{\label{ex:phol4716b} Qualm, Film, Köln, ähneln}
  \end{xlist}
\end{exe}


\textbf{?? HIER WEITER}

Falls der in diesem Abschnitt abgelieferte deskriptive Befund unübersichtlich erscheint, sei auf die weitere Systematisierung in Abschnitt~\ref{sec:anfangsrandendrand} verwiesen, die eine deutliche Reduktion der Komplexität aus der Darstellung herausnimmt.
In Abschnitt~\ref{sec:sonoritaet} wird dafür mit der Einführung der Sonoritätshierarchie ein wichtiger Grundstein gelegt.







\subsection{Sonorität}

\label{sec:sonoritaet}

Wie in Abschnitt~\ref{sec:einsilbler} gezeigt wurde, sind an den Rändern der Silbe nicht beliebige Kombinationen von Konsonanten möglich.
Dabei fällt ein Muster auf.
Während am Silbenanfang \zB \textipa{[kn]} (\textit{Knie}) aber nicht \textipa{[nk]} möglich ist, ist es am Silbenende genau umgekehrt (\textit{Zank}).
Gleiches gilt für \textipa{[pl]} (\textit{Plan}) und \textipa{[lp]} (\textit{Alp}) usw.
Es ergibt sich eine Art spiegelbildlicher Ordnung vom Vokal zu den Außenrändern.
Diese Ordnung zeigt sich nach aktuellem Kenntnisstand in allen Sprachen der Welt, und man erklärt sie mit Hilfe des Konstrukts der \textit{Sonorität} (ungefähr \textit{Klangfülle}).
Für unsere Zwecke reicht es, festzustellen, dass (in dieser Reihenfolge) Plosive (\textit{P}), Frikative (\textit{F}), Nasale (\textit{N}), Approximanten (\textit{A}) und Vokale (\textit{V}) eine Skala mit ansteigender Sonorität bilden (Abbildung~\ref{fig:sonoritaetshierarchie}).

\index{Sonorität!Hierarchie}

\begin{figure}
  \centering
  \begin{tabular}{l|ccccc|r}
    \cline{2-6}
    (minimal sonor) & \rnode{HP}{P} & \rnode{HF}{F} & \rnode{HN}{N} & \rnode{HA}{A} & \rnode{HV}{V} & (maximal sonor) \\
    \cline{2-6}
  \end{tabular}
  \ncline[nodesep=3pt]{->}{HP}{HF}
  \ncline[nodesep=3pt]{->}{HF}{HN}
  \ncline[nodesep=3pt]{->}{HN}{HA}
  \ncline[nodesep=3pt]{->}{HA}{HV}
  \caption{Sonoritätshierarchie}
  \label{fig:sonoritaetshierarchie}
\end{figure}

Innerhalb der Silbe gibt es das universelle Bildungsprinzip der \textit{Sonoritätskontur}, welches regelt, dass die Sonorität zum Vokal hin ansteigt und dann wieder abfällt (Abbildung~\ref{fig:sonhier}).
Dies gilt natürlich nur, wenn die Silbe mindestens zwei Segmente enthält.
Eine Silbe aus Plosiv und Vokal zeigt einen Sonoritätsanstieg aber keinen Sonoritätsabfall.
Bei einer Silbe aus Vokal und Plosiv ist es umgekehrt.
Es gibt also Silben, die nur einen Ausschnitt aus der Sonoritätskontur realisieren (nur Anstieg oder nur Abfall), aber einen Sonoritätsabfall gefolgt von einem Anstieg gibt es innerhalb einzelner Silben nicht.
In Tabelle~\ref{tab:silbenbau} werden zur Illustration einige deutsche Wörter in das Schema eingeordnet. 
Das ideale Bild der Sonoritätskontur wird dabei weitgehend bestätigt.
Die einzige Ausnahme ist das Auftreten von \textit{s}-Lauten am äußersten Silbenrand jenseits von Plosiven (\textit{sprüh}, \textit{Raps}).
Da Frikative eine höhere Sonorität haben als Frikative, steigt in diesen Fällen die Sonorität zum Rand hin wieder an.
Diese Ausnahme für \textit{s}-Segmente findet man in vielen Sprachen der Welt, und in Abschnitt~\ref{sec:anfangsrandendrand} wird dafür plädiert, diese Segmente als \textit{extrasilbisch} (außerhalb der normalen Silbenstruktur stehend) zu analysieren.
Abbildung~\ref{fig:sonhiers} zeigt ein modifiziertes Sonoritätsschema, das extrasilbische \textit{s}-Segmente berücksichtigt.

\begin{figure}
  \centering
  \begin{tabular}{ccccccccccc}
  &&&& \rnode{V}{V} &&&& \\
  &&& \rnode{L1}{A} && \rnode{L2}{A} &&& \\
  && \rnode{N1}{N} &&&& \rnode{N2}{N} && \\
  & \rnode{F1}{F} &&&&&& \rnode{F2}{F} & \\
  \rnode{P1}{P} &&&&&&&& \rnode{P2}{P} \\
  \end{tabular}
  \ncline[nodesep=3pt]{->}{P1}{F1}
  \ncline[nodesep=3pt]{->}{F1}{N1}
  \ncline[nodesep=3pt]{->}{N1}{L1}
  \ncline[nodesep=3pt]{->}{L1}{V}
  \ncline[nodesep=3pt]{->}{V}{L2}
  \ncline[nodesep=3pt]{->}{L2}{N2}
  \ncline[nodesep=3pt]{->}{N2}{F2}
  \ncline[nodesep=3pt]{->}{F2}{P2}
  \caption{Sonorität für die Segmentklassen in der schematischen Silbe}
  \label{fig:sonhier}
\end{figure}

\begin{table}
  \centering
    \begin{tabular}{cccccccccccp{0.5mm}l}
      \lsptoprule
      \textbf{(S)} & \textbf{P} & \textbf{F} & \textbf{N} & \textbf{A} & \textbf{V} & \textbf{A} & \textbf{N} & \textbf{F} & \textbf{P} & \textbf{(S)} && \\
      \midrule
	& k &&&& \textipa{\o:} &&&&&&& \textit{Kö} \\
	&&& n && \textipa{i:} &&&&&&& \textit{nie} \\
	& k && n && \textipa{i:} &&&&&&& \textit{Knie} \\
	& d & \textipa{K} &&& \textipa{o:} &&&&&&& \textit{droh} \\
	\textipa{S} & t &&&& \textipa{e:} &&&&&&& \textit{steh} \\
	\textipa{S} &&& n && \textipa{e:} &&&&&&& \textit{Schnee} \\
	\textipa{S} & p & \textipa{K} &&& \textipa{y:} &&&&&&& \textit{sprüh} \\
	&&&&&&&&&& \\
	& (\textipa{P}) &&&& a &&&& p &&& \textit{ab} \\
	& (\textipa{P}) &&&& a && n &&&&& \textit{an} \\
	& (\textipa{P}) &&&& a &&& \textipa{X} & t &&& \textit{acht} \\
	& (\textipa{P}) &&&& a & l & m &&&&& \textit{Alm} \\
	&&&&&&&&&& \\
	&& \textipa{K} &&& a &&&& p & s && \textit{Raps}\\
	&& \textipa{K} &&& a && m & s & t &&& \textit{rammst} \\
	\textipa{S} & t & \textipa{K} &&& \textipa{O} & l && \textipa{\c{c}s} & t &&& \textit{strolchst} \\
      \lspbottomrule
    \end{tabular}
  \caption{Einordnung einiger Konsonatengruppen in das Silbenschema}
  \label{tab:silbenbau}
\end{table}

\begin{figure}
  \centering
  \begin{tabular}{ccccccccccccccc}
    V &&&&&& \rnode{xV}{V} &&&& \\
    A &&&&& \rnode{xL1}{A} && \rnode{xL2}{A} &&&& \\
    N &&&& \rnode{xN1}{N} &&&& \rnode{xN2}{N} &&& \\
    F &\rnode{xS1}{(S)} && \rnode{xF1}{F} &&&&&& \rnode{xF2}{F} & \rnode{xF3}{F} && \rnode{xS2}{(S)} \\
    P &&\rnode{xP1}{P} &&&&&&&&& \rnode{xP2}{P} & \\
  \end{tabular}
  \ncline[nodesep=3pt]{->}{xS1}{xP1}
  \ncline[nodesep=3pt]{->}{xP1}{xF1}
  \ncline[nodesep=3pt]{->}{xF1}{xN1}
  \ncline[nodesep=3pt]{->}{xN1}{xL1}
  \ncline[nodesep=3pt]{->}{xL1}{xV}
  \ncline[nodesep=3pt]{->}{xV}{xL2}
  \ncline[nodesep=3pt]{->}{xL2}{xN2}
  \ncline[nodesep=3pt]{->}{xN2}{xF2}
  \ncline[nodesep=3pt]{->}{xF2}{xF3}
  \ncline[nodesep=3pt]{->}{xF3}{xP2}
  \ncline[nodesep=3pt]{->}{xP2}{xS2}
  \caption{Sonoritätsverlauf mit Rand-Frikativen und Plateau}
  \label{fig:sonhiers}
\end{figure}

Was die Sonorität aus phonetisch-artikulatorischer (oder perzeptorischer) Sicht genau ist, ist eine schwierige Frage.
Stimmhaftigkeit ist ein wichtiger Faktor für eine hohe Sonorität.
Darüber hinaus kann als Faustregel gelten, dass, je enger die durch die Artikulatoren hergestellte Annäherung ist, die Sonorität umso geringer ist.
Dies entspricht dem artikulatorischen Schema des Öffnens und Schließens des Vokaltrakts (Abschnitt~\ref{sec:silben}).
Die Überlegungen zur Sonorität schließen mit Definition~\ref{def:sonoritaet} und der Sonoritätsanalyse eines der komplexesten Einsilblers des Deutschen (\textit{strolchst}) in Abbildung~\ref{fig:sonhiersstrolchst}

\Definition{Sonoritätskontur}{\label{def:sonoritaet}
Segmente können auf einer Sonoritätsskala eingeordnet werden.
Alle zulässigen Silbenstrukturen repräsentieren maximal einen Anstieg der Sonorität zur Mitte der Silbe und einen Abfall der Sonorität zum Ende der Silbe.
Es gibt innerhalb einer Silbe keinen Sonoritätsanstieg nach einem Sonoritätsabfall.
Die einzige Ausnahme bilden \textit{s}-Frikative am äußersten Silbenrand. 
\index{Sonorität}
}

\begin{figure}
  \centering
  \begin{tabular}{ccccccccc}
    V &&&& \rnode{V1}{\textipa{O}} &&&& \\
    A &&&&& \rnode{L21}{\textipa{l}} &&& \\
    N &&&&&&&& \\
    F & \rnode{S11}{\textipa{S}} && \rnode{F11}{\textipa{K}} &&& \rnode{F21}{\textipa{\c{c}}} & \rnode{F31}{\textipa{s}} & \\
    P && \rnode{P11}{\textipa{t}} &&&&&& \rnode{P21}{\textipa{t}} \\
  \end{tabular}
  \ncline[nodesep=3pt]{->}{S11}{P11}
  \ncline[nodesep=3pt]{->}{P11}{F11}
  \ncline[nodesep=3pt]{->}{F11}{V1}
  \ncline[nodesep=3pt]{->}{V1}{L21}
  \ncline[nodesep=3pt]{->}{L21}{F21}
  \ncline[nodesep=3pt]{->}{F21}{F31}
  \ncline[nodesep=3pt]{->}{F31}{P21}
  \caption{Sonorität am Beispiel von \textit{strolchst}}
  \label{fig:sonhiersstrolchst}
\end{figure}








\subsection{Die Systematik von Anfangsrand und Endrand}

\label{sec:anfangsrandendrand}

In diesem Abschnitt werden der Anfangsrand und der Endrand im Einsilbler für den Kernwortschatz mit dem Wissen um die Sonoritätshierarchie abschließend beschrieben.
Die hauptsächliche Vereinfachung und Systematisierung des Anfangsrandes wird dadurch erreicht, dass \textipa{[S]} in Anfangsrändern mit scheinbar  zwei oder drei Segmenten eliminiert wird.
In Abschnitt~\ref{sec:sonoritaet} wurde festgestellt, dass \textipa{[S]} vor Plosiven (\textit{Sprung}, \textit{Stuhl}) die Sonoritätshierarchie verletzt.
Vor Frikativen (\textit{Schwung}, \textit{Schrank}) entsteht zumindest bei einer grob gestaffelten Sonoritätsskala ein Sonoritätsplateau.
Lediglich in mehregmentigen Anfangsrändern mit Nasal oder Approximant an zweiter Stelle (\textit{Schmal}, \textit{Schlund}) verhält sich \textipa{[S]} konform zur Sonoritätshierarchie.
Gleichzeitig sind die einzigen Anfangsränder mit drei Segmenten solche, bei denen das erste Segment \textipa{[S]} ist.
Das Segment \textipa{[S]} verhält sich im Silbenbau offensichtlich besonders, und es erschwert die Systematisierung.
Daher behandeln wir \textipa{[S]} in komplexen Anfangsrändern jetzt als \textit{extrasilbisch}.
Damit soll nicht gemeint sein, dass es nicht zur Silbe gehört, sondern vielmehr, dass es sich in einer besonderen Position vor dem Anfangsrand befindet, die von anderen Segmenten nicht besetzt werden kann, und die nicht der Sonoritätskontur unterliegt.
Die maximale Komplexität des Anfangsrands besteht also in zwei Segmenten (duplex), und scheinbare Fälle von drei Segmenten (\textipa{[SpK]} und \textipa{[StK]}) bestehen aus zwei Segmenten mit extrasilbischem \textipa{[S]}.
Außerdem wurde in Abschnitt~\ref{sec:einsilbler} bereits stark bezweifelt, dass Anlaute wie \textipa{[gn]} in \textit{Gnade} und \textipa{[Spl]} wie in \textit{spleißen} zum Kern des Systems gerechnet werden müssen.
Wenn man nun alles weglässt, was weglassbar ist, und \textipa{[S]} den genannten Sonderstatus zuweist, dampft die Beschreibung des simplexen Anfangsrands auf Abbildung~\ref{fig:anfangsrandsimplex} und die des duplexen Anfangsrands auf Abbildung~\ref{fig:anfangsrandduplex} ein.

\newcommand{\Rxx}[3]{\POS[]+(#1,-1.2)\ar@{-}[#3]-(#2,1.2)}
\newcommand{\Rxxx}[4]{\POS[]+(#1,-1.2)\ar@{-}[#4]-(#2,#3)}

\begin{figure}
  \centering
  \Tree[3]{
     \K{\small\textbf{extrasilbisch}} & \K{\small\textbf{Anfangsrand}} && \K{\small\textbf{Kern}} \\
     \K{\textipa{S}}\R[--]{r} & \K{p t}\R{ddddrr}^{\text{Plosiv}}              &&           \\
     & \K{Rest}\Rxx{5}{9.75}{r}                &&           \\
     \K{\textipa{S}}\R[--]{r} & \K{\textipa{K} v}\R{ddrr}_{\text{Frikativ}}             &&           \\
     & \K{Rest}\Below{(außer \textipa{\c{c} X s})}\Rxxx{5}{9}{7.6}{ur}           &&           \\
     &          && \K{~~~Vokal} \\
     \K{\textipa{S}}\R[--]{r} & \K{m n}\Below{(nicht \textipa{N})}\R{urr}_{\text{Nasal}}  &&           \\
     &                                         &&           \\
     \K{\textipa{S}}\R[--]{r} & \K{l}\R{uuurr}_{\text{Approximant}}         &&           \\
  }
  \caption{Struktur des simplexen Anfangsrands (nur Systemkern)}
  \label{fig:anfangsrandsimplex}
\end{figure}

\begin{figure}
  \centering
  \Tree[3]{
     \K{\small\textbf{extrasilbisch}} & \K{\small\textbf{Anfangsrand}}\Below{\small\textbf{erste Position}} & \K{\small\textbf{Anfangsrand}}\Below{\small\textbf{zweite Position}} && \K{\small\textbf{Kern}} \\
     \K{\textipa{S}}\R[--]{r} & \K{p t}\Rxx{3}{3}{ddr}^{\text{Plosiv}} &&\\
     & \K{Rest}\Rxx{6}{11.6}{r} & \\
  	 && \K{\textipa{K}}\Rxx{3.7}{3.7}{ddrr}^{\text{Frikativ}} \\
     & \K{f v}\Rxx{3}{3}{ur}_{\text{Frikativ}} & &\\
     & \K{k}\Rxx{3}{5}{r}^{\text{Plosiv}} & \K{v}\Rxxx{3.7}{0.8}{0.825}{ur}         && \K{~~~Vokal} \\
     & \K{k}\Rxx{3.7}{3.7}{r}^{\text{Plosiv}} & \K{n}\R{urr}^{\text{Nasal}}  &&           \\
     & \K{p b k g}\Rxx{7}{3.7}{dr}^{\text{Plosiv}} &                                         &&           \\
     && \K{l}\R{uuurr}_{\text{Approximant}}         &&           \\
     & \K{f}\Rxx{5}{3.7}{ur}_{\text{Frikativ}} &                                         &&           \\
  }
  \caption{Struktur des duplexen Anfangsrands (nur Systemkern)}
  \label{fig:anfangsrandduplex}
\end{figure}

Abbildung~\ref{fig:anfangsrandsimplex} zeigt übersichtlich, dass im simplexen Anfangsrand alle Konsonanten stehen können bis auf die im Anfangsrand generell nicht lizenzierten \textipa{[\c{c} X s N]}.
Extrasilbisches \textipa{[S]} kann vor Konsonanten aller Artikulationsarten stehen, innerhalb der Plosive aber nicht von stimmhaften und nicht vor \textipa{[k]}.
Ebenso sind die Frikative \textipa{[f S h J]} nicht mit extrasilbischem \textipa{[S]} kombinierbar.
Abbildung~\ref{fig:anfangsrandduplex} zeigt, dass die Kombinationsmöglichkeiten im duplexen Anfangsrand stark eingeschränkt sind.
Plosive an zweiter Position sind ausgeschlossen.
Häufig (im Sinn einer Typenhäufigkeit, s.\ Abschnitt~\ref{sec:kern}) sind desweiteren nur Kombinationen aus Plosiv und \textipa{[K]} sowie (bereits weniger häufig) Plosiv und \textipa{[l]}.
Die Präferenz für diese Kombination hat genau wie die Sonoritätskontur universelle Züge.
Man fasst gelegentlich \textit{r}- und \textit{l}-Segmente zu den sogenannten \textit{Liquiden} (oder \textit{Fließlaute}) zusammen, um ihrem ähnlichen Verhalten beim Silbenbau Rechnung zu tragen.

\subsection{Mehrsilbler}

\label{sec:mehrsilbler}
\index{Silbe!Silbifizierung}




% ==========================================================




\section{Wortakzent}

\label{sec:prosodie}

\index{Prosodie}

\subsection{Prosodie}

Nach den Silben ist die nächsthöhere Ebene der phonologischen Strukturbildung das phonologische Wort.%
\footnote{Unter anderem wird die Satzprosodie, also die besonderen Betonungs- und vor allem Tonhöhenverläufe in bestimmten Satzarten, aus Platzgründen nicht besprochen.}
Der Grund, warum man eine nächsthöhere Einheit nach der Silbe innerhalb der Phonologie annehmen möchte, ist, dass es ganz bestimmte phonologische Prozesse gibt, die sich nicht im Rahmen der Silbe behandeln lassen.
Das wichtigste Beispiel ist die Akzentzuweisung, also umgangssprachlich die Betonung einer Silbe innerhalb eines Wortes.
Das phonologische Wort ist die relevante Einheit der Prosodie.

Bisher haben wir noch gar keine Definition des Wortes (\zB eine morphologische Definition) gegeben.
Aus Sicht der Phonologie gibt es eine einfache Möglichkeit, eine solche Definition aufzustellen. 

\Definition{Phonologisches Wort}{
\label{def:phonwort}
Ein phonologisches Wort ist die kleinste phonologische Struktur, die Silben als Konstituenten hat, und bezüglich derer eigene Regularitäten feststellbar sind.
\index{Wort!phonologisch}
}

Diese Definition kommt sehr formal daher.
Denken wir aber an die Definition von Grammatik (Definition~\ref{def:grammatik}, S.~\pageref{def:grammatik}) zurück, so ist die Einschränkung \textit{bezüglich derer eigene Regularitäten feststellbar sind} ausgesprochen instruktiv.
Wenn es nämlich phonologische Regularitäten gibt, die sich nicht effektiv und angemessen mittels Segmenten oder Silben beschreiben lassen, müssen wir eine andere (größere) Einheit annehmen, bezüglich derer wir diese Regularitäten beschreiben können.
Solche Reguläritäten betreffen wie gesagt den Wortakzent.
In (\ref{ex:phol8735}) sind einige Wörter bezüglich ihres Akzents markiert, das Zeichen \Akz\ steht vor der akzentuierten (betonten) Silbe.

\begin{exe}
  \ex\label{ex:phol8735}
  \begin{xlist}
    \ex{\label{ex:phol8735a} \Akz Spiel, \Akz Spiele, \Akz Spielerin, be\Akz spielen}
    \ex{\label{ex:phol8735b} \Akz Fußball, \Akz Fußballerin, \Akz Fitness, \Akz Fitnesstrainerin}
    \ex{\label{ex:phol8735c} \Akz rot, \Akz rötlich, \Akz roter}
    \ex{\label{ex:phol8735d} \Akz fahren, um\Akz fahren, \Akz umfahren}
    \ex{\label{ex:phol8735e} wahr\Akz scheinlich, \Akz damals, \Akz übrigens, vie\Akz lleicht}
    \ex{\label{ex:phol8735f} \Akz wo, wa\Akz rum, wes\Akz halb}
    \ex{\label{ex:phol8735g} \Akz August, Au\Akz gust}
    \ex{\label{ex:phol8735h} \Akz fahren, Fahre\Akz rei, \Akz drängeln, Dränge\Akz lei}
  \end{xlist}
\end{exe}

Jedes Wort hat eine Silbe, die durch eine besondere Hervorhebung markiert werden kann.
Phonetisch besteht diese Hervorhebung nicht unbedingt in einer lauteren Aussprache, sondern aus einem Bündel von Eigenschaften, das Lautstärke, Länge, Tonhöhe und Beeinflussung der Qualität der Vokale und der umliegenden Segmente beinhaltet.
Es gilt, dass jedes simplexe Wort des deutschen Kernwortschatzes genau eine Akzentsilbe hat (\textit{\Akz Ball}, \textit{\Akz Tante}, \textit{\Akz schneite}, \textit{\Akz rot}, \textit{\Akz unter} usw.).
Komplexe Wörter oder längere Wörter des Nicht-Kernwortschatzes haben genau eine Haupt-Akzentsilbe (\textit{\Akz untergehen}, \textit{\Akz Wirtschaftswunder}, \textit{Tautolo\Akz gie} usw.).
Zusätzlich findet man Nebenakzente (im Vergleich zu Akzentsilben weniger stark akzentuierte Silben) in den zuletzt erwähnten Wörtern.
Die Frage ist nun, nach welchen Regularitäten dieser Akzent auf die Wörter verteilt wird (vgl.\ Abschnitt~\ref{sec:deutscherwortakzent}).
Auf jeden Fall ist der Akzent eine weitere Motivation der Definition des phonologischen Wortes (Definition~\ref{def:phonwort}).
Die Akzentzuweisung ist eine der Regularitäten, für die man die Einheit des phonologischen Wortes benötigt.

\Definition{Akzent}{
\label{def:akzent}
Akzent ist die Prominenzmarkierung, die einer Silbe im phonologischen Wort zugewiesen wird.
Akzent wird durch verschiedene phonetische Mittel (wie Lautstärke, Tonhöhe usw.) phonetisch realisiert.
\index{Akzent}
}

\enlargethispage{1\baselineskip}
Manche Sprachen sind sehr systematisch bzw.\ starr bezüglich der Akzentposition.
Im Polnischen liegt der Akzent immer auf der zweitletzten Wortsilbe, s.\ (\ref{ex:phol8254}).
Im Tschechischen hingegen wird immer die erste Silbe akzentuiert, vgl.\ (\ref{ex:phol8255}).%
\footnote{Für die slawischen Beispiele danke ich Götz Keydana.}

\begin{exe}
  \ex{\label{ex:phol8254} \Akz okno (Fenster), nagroma\Akz dzenie (Ansammlung)}
  \ex{\label{ex:phol8255} \Akz okno (Fenster), \Akz nahromad\v{en\'i} (Ansammlung)}
\end{exe}

Solche Sprachen haben einen sogenannten metrischen Akzent.
Einen streng lexikalischen Akzent hat dagegen das Russische.
Hier ist der Akzent für jedes Wort im Lexikon festgelegt, und man kann allein durch die Position des Akzents ein Minimalpaar erzeugen, wie in (\ref{ex:phol8256}).

\begin{exe}
  \ex{\label{ex:phol8256} \Akz muka (Qual), mu\Akz ka (Mehl)}
\end{exe}

Bevor die Frage geklärt wird, wie sich der Akzent im Deutschen verhält, wird in Abschnitt~\ref{sec:akzentsitztest} ein einfacher Test auf den Akzentsitz vorgestellt.

\subsection{Test zur Ermittlung des Wortakzents}

\label{sec:akzentsitztest}


\index{Akzent!Wort--}

Es gibt eine einfache Methode, den Akzentsitz in Wörtern zu ermitteln.
Will ein Sprachbenutzer einzelne Wörter in einem Satz besonders hervorheben (fokussieren), besteht im Deutschen die Möglichkeit, dies mittels einer sehr starken Betonung zu erreichen.

\begin{exe}
  \ex\label{ex:fokus}
  \begin{xlist}
    \ex{Sie hat das \Akz AUTO gewaschen.}
    \ex{Sie hat das Auto GE\Akz WASCHEN.}
  \end{xlist}
\end{exe}

In den Beispielen in (\ref{ex:fokus}) ist jeweils das fokussierte Wort in Großbuchstaben gesetzt.
Zusätzlich markiert in den Beispielen das Akzentzeichen, auf welcher Silbe der Höhepunkt der Betonung genau liegt.
Von der Bedeutung her ergibt sich typischerweise durch die Fokussierung eines Wortes ein ähnlicher Effekt, als würde man jeweils die Formel \textit{und nichts anderes} hinzufügen, als würde man also die sogenannten Alternativen zum fokussierten Wort ausdrücklich ausschließen.

\begin{exe}
  \ex\label{ex:fokus-deutlich}
  \begin{xlist}
    \ex{Sie hat das \Akz AUTO (und nichts anderes) gewaschen.}
    \ex{Sie hat das Auto GE\Akz WASCHEN (und nichts anderes damit gemacht).}
  \end{xlist}
\end{exe}

Bei der Fokusbetonung tritt die Akzentsilbe durch eine Anhebung der Tonhöhe besonders deutlich hörbar hervor.
Damit liegt also ein einfacher Test vor, mit dem man in Zweifelsfällen den Wortakzent lokalisieren kann.

\subsection{Wortakzent im Deutschen}

\label{sec:deutscherwortakzent}

Es ist nun die Frage zu beantworten, welchem Akzenttypus (metrisch oder lexikalisch) das Deutsche folgt.
Die Frage wird unterschiedlich beantwortet, aber es lassen sich für die Wörter des Kernwortschatzes relativ klare Regularitäten erkennen, die auf einen tendenziell stark metrischen Akzent für das Deutsche hinweisen.
Leider benötigen wir zur Beschreibung der wichtigsten Regularität einen Begriff, den wir noch nicht eingeführt haben, nämlich den des \label{abs:3453457}Wortstamms (vgl.\ Abschnitt~\ref{sec:stamm}).
In den Beispielen in (\ref{ex:phol8735a}) bleibt der Akzent in allen Wörtern immer auf der Silbe \textit{spiel}.
Ob nun der Plural \textit{Spiele} gebildet wird, die Form \textit{Spielerin} oder ob ein morphologisches Element vorangestellt wird wie in \textit{bespielen}, der Akzent bleibt auf dem Kern dieser Wörter, nämlich \textit{spiel}.
Ganz ähnlich verhält es sich mit \textit{rot} in (\ref{ex:phol8735c}).
Der hier informell Kern genannte Teil dieser Wörter ist der Wortstamm, und im Deutschen gibt es die starke Tendenz, diesen zu betonen.

\Satz{Stammbetonung}{
Im Kernwortschatz wird die erste Silbe des Stamms akzentuiert.
\index{Akzent!Stamm--}
}

Mit Kernwortschatz sind die Wörter im Lexikon gemeint, die sich nach den allgemeinen Regeln des Sprachsystems verhalten.
Es gibt auch Wörter (sehr häufig, aber nicht immer Lehnwörter), die spezielleren, in ihrer Gültigkeit stark eingeschränkten Regularitäten folgen (s.\ \textit{August} usw.\ weiter unten).

Wörter wie \textit{Fußball} und \textit{Fitnesstrainerin} aus (\ref{ex:phol8735b}) sind aus zwei Stämmen zusammengesetzt und werden Komposita genannt (vgl.\ Abschnitt~\ref{sec:komp}).
In ihnen wird immer der erste Bestandteil betont.

\Satz{Betonung in Komposita}{
In Komposita wird der erste Bestandteil akzentuiert.
\index{Akzent!in Komposita}
}

Mit dem Fokussierungstest aus Abschnitt~\ref{sec:akzentsitztest} kann für beliebig lange Komposita festgestellt werden, dass der Akzent immer auf ihrem ersten Bestandteil liegt, vgl.\ (\ref{ex:fokuskomp}).

\begin{exe}
  \ex\label{ex:fokuskomp}
  \begin{xlist}
    \ex{Sie hat das \Akz AUTODACH (und nichts anderes) gewaschen.}
    \ex{Sie hat am \Akz LANGSTRECKENLAUF (und nichts anderem) teilgenommen.}
    \ex{Sie hat sich an dem \Akz BUSHALTESTELLENUNTERSTAND (und nichts anderem) verletzt.}
  \end{xlist}
\end{exe}

Im Falle von \textit{\Akz umfahren} und \textit{um\Akz fahren} aus (\ref{ex:phol8735d}) liegt wieder eine andere Situation vor.
Das Element \textit{um-} ist einmal betont, einmal nicht.
Diese Wörter weisen allerdings auch einen Bedeutungsunterschied auf:
\textit{\Akz umfahren} bedeutet soviel wie \textit{niederfahren}, \textit{um\Akz fahren} bedeutet soviel wie \textit{um etwas herumfahren}.
Es gibt weitere morphologische und syntaktische Unterschiede zwischen den beiden verschiedenen \textit{um}-Elementen, die in \ref{sec:derivohnewaw} genauer beschrieben werden.
In \textit{\Akz umfahren} handelt es sich bei \textit{um} um eine sogenannte Verbpartikel, in \textit{um\Akz fahren} um ein Verbpräfix.

\Satz{Präfix- und Partikelbetonung}{
\label{satz:pholvprtprf}
Verbpartikeln ziehen den Akzent auf sich, Verbpräfixe nicht.
\index{Akzent!Präfixe und Partikeln}
}

Die anderen, meist nachgestellten Ableitungselemente wie \textit{-heit}, \textit{-keit}, \textit{-in} usw.\ belassen den Akzent fast alle auf dem Stamm, verhalten sich diesbezüglich also eher wie Verbpräfixe als wie Verbpartikeln.
Lediglich \textit{-ei} und \textit{-erei} ziehen den Akzent auf die letzte Silbe, vgl.\ (\ref{ex:phol8735h}).

Neben diesen regelhaften Fällen (metrischer Akzent) gibt es eine gewisse Menge von Wörtern, die nicht regelhaft akzentuiert werden (lexikalischer Akzent).
Neben Lehnwörtern, die offensichtlich einen lexikalischen Akzent haben (wie \textit{\Akz August} und \textit{Au\Akz gust}) gibt es eine Reihe von Wörtern wie \textit{vie\Akz lleicht}, die sich unregelmäßig zu verhalten scheinen und nicht stamminitial betont werden.
Dazu gehören auch die Fragewörter \textit{wa\Akz rum}, \textit{wes\Akz halb} usw.
Es spricht allerdings auch überhaupt nichts dagegen, ein überwiegend metrisches Akzentsystem anzunehmen, innerhalb dessen es gewisse lexikalische Ausnahmen gibt.

Außerdem gibt es manche Wörter, die gar keinen Akzent zu tragen scheinen.
Bei einsilbigen Wörtern stellt sich die Frage nach dem Akzentsitz normalerweise nicht, weil die einzige Silbe des Worts den Akzent trägt.
Bestimmte Pronomen, wie das \textit{es} in (\ref{ex:phol9101}) sind aber prinzipiell unbetonbar.
Wenn man dieses \textit{es} zu fokussieren versucht, wird der Satz ungrammatisch.

\begin{exe}
  \ex\label{ex:phol9101}
  \begin{xlist}
    \ex[]{Es schneit.}
    \ex[*]{\Akz ES schneit.}
  \end{xlist}
\end{exe}

Eine weitere wichtige Einheit wird hier aus Platzgründen nur sehr kurz behandelt, obwohl sie auch in der Morphologie (zumindest des Kernwortschatzes) weitreichendes Erklärungspotential hat, nämlich der Fuß.%
\footnote{In Teil~\ref{part:schrift} kommen wir nochmal auf Füße zurück.}
Wenn man phonologische Wörter daraufhin untersucht, wie akzentuierte (inkl.\ Nebenakzente) und nicht-akzentuierte Silben einander folgen, stellt man fest, dass im Deutschen das mit Abstand häufigste Muster eine Folge von betonter und unbetonter Silbe ist (\textit{\Akz um.ge.\Akz fah.ren}, \textit{\Akz Kin.der}, \textit{\Akz Kin.der.\Akz gar.ten} und viele der oben genannten Beispiele).
Manchmal liegt der umgekehrte Fall vor, also eine Abfolge unbetont vor betont (\textit{vie.\Akz lleicht} usw.).
Noch seltener kommt es zu Abfolgen von zwei unbetonten vor einer betonten Silbe (\textit{Po.li.\Akz tik}).
Der umgekehrte Fall von einer betonten vor zwei unbetonten Silben ergibt sich sogar regelhaft in bestimmten Beugungsformen und durch Wortableitungen (\textit{\Akz reg.ne.te}, \textit{\Akz röt.li.che}).

Diese rhythmischen Verhältnisse sind mit Bezug auf Füße -- Abfolgen von betonten und unbetonten Silben -- analysierbar%
\footnote{Eigentlich bestehen prosodische Wörter dann aus Füßen, nicht aus Silben.}
Gemäß Tabelle~\ref{tab:dtfuesse}, die einige wichtige Fußtypen zusammenfasst, wäre dann das prototypische Wort des Kernwortschatzes trochäisch.
Ob die anderen Fußtypen wirklich als phonologische Größen für das Deutsche angenommen werden müssen, ist eine Frage von einigem theoretischen Gehalt, die hier nicht geklärt werden kann.

\begin{table}
\centering
\begin{tabular}{lll}
  \lsptoprule
  \textbf{Fuß} & \textbf{Muster} & \textbf{Beispiel} \\
  \midrule
  Trochäus & \Akz -- & \Akz Mu.tter \\
  Daktylus & \Akz -- -- & \Akz reg.ne.te \\
  Jambus & -- \Akz & vie.\Akz lleicht \\
  Anapäst & -- -- \Akz & Po.li.\Akz tik \\
  \lspbottomrule
\end{tabular}
\caption{Namen verschiedener Fußtypen mit Beispielen}
\label{tab:dtfuesse}
\end{table}

\subsection{Prosodisches und phonologisches Wort}

\label{sec:prosphonwort}

Abschließend soll noch anhand eines Phänomens darauf hingewiesen werden, warum oft zwischen phonologischem Wort und prosodischem Wort unterschieden wird.
Zur Illustration dienen die Beispiele in (\ref{ex:phol8945}) inkl.\ IPA, wobei Betonung (Hauptakzent) und Silbengrenzen markiert wurden.

\begin{exe}
  \ex\label{ex:phol8945}
  \begin{xlist}
    \ex{Leser \textipa{[\textprimstress le:.z5]}}
    \ex{Leserin \textipa{[\textprimstress le:.z@.KIn]}}
    \ex{Leseranfrage \textipa{[\textprimstress le:.z5.Pan.fKa:.g@]}}
    \ex{(wenn) Leser anfragen \textipa{[\textprimstress le:.z5 \textprimstress Pan.fKa:.g@n]}}
  \end{xlist}
\end{exe}

Im Fall von \textit{Le.ser} und \textit{Le.se.rin} wird offensichtlich gemäß den Regularitäten, die in Abschnitt~\ref{sec:silbifizierung} beschrieben wurden, silbifiziert.
Wegen der Bedingung Onset-Maximierung gerät dabei das /\textipa{K}/ von \textit{Leserin} in den Onset der letzten Silbe und wird folgerichtig nicht vokalisiert, so wie es bei \textit{Leser} passiert.
Bei \textit{Leseranfrage} ist es anders, denn obwohl dem /\textipa{K}/ ein Vokal folgt, wird /\textipa{K}/ nicht in den Anlaut eingeordnet, sondern bleibt in der Silbe \textipa{[z5]} und wird vokalisiert.
Es heißt also nicht *\textipa{[le:.z@.Kan.fKa:.g@]}.

Einerseits gilt also innerhalb eines Wortes wie \textit{Leserin} die Onset-Maximierung, andererseits aber scheint sie in einem Wort wie \textit{Leseranfrage} nicht vollständig zu gelten.
Es muss sich also bei Komposita wir \textit{Leseranfrage} um zwei phonologische Wörter handeln, denn die Silbifizierung verläuft genauso wie in (\textit{wenn}) \textit{Leser anfragen}, wobei es sich eindeutig um zwei verschiedene Wörter handelt.
Trotzdem verhalten sich \textit{Leseranfragen} und (\textit{wenn}) \textit{Leser anfragen} phonologisch nicht genau gleich.
Im Kompositum \textit{Leseranfragen} gibt es nur einen Hauptakzent (auf der ersten Silbe), während in \textit{Leser anfragen} jedes Wort einen Hauptakzent erhält.
Prosodisch verhält sich ein Kompositum also wie ein Wort und hat einen Hauptakzent, phonotaktisch-segmental verhält es sich allerdings wie zwei Wörter, denn an der Grenze zwischen den Gliedern des Kompositums findet keine normale wortinterne Silbifizierung statt.
Daher benötigt man eigentlich zwei Wort-Ebenen in der Phonologie, das phonologische Wort und das prosodische Wort.

\Definition{Phonologisches und prosodisches Wort}{
\label{def:phonoprosowort}
Das phonologische Wort ist die aus Füßen (in vereinfachter Darstellung aus Silben) bestehende Einheit, innerhalb derer die Regularitäten der segmentalen Phonologie und der Phonotaktik wirken.
Das prosodische Wort ist die aus phonologischen Wörtern bestehende Einheit, innerhalb derer prosodische Regularitäten (Akzentzuweisung) wirken.
\index{Wort!phonologisch}
\index{Wort!prosodisch}
}

Es gibt natürlich viele Fälle, in denen das phonologische Wort gleich dem prosodischen Wort ist, aber gerade bei Komposita (und \zB Fügungen aus Verbpartikel und Verb) muss man davon ausgehen, dass das phonologische Wort kleiner ist als das prosodische.




% ==========================================================




\section{Segmente}

\label{sec:segmentalephol}

\subsection{Segmente, Merkmale und Verteilungen}

\label{sec:segmenteverteilungen}
\label{sec:verteilungen}

Der zentrale Begriff in der Phonologie ist zunächst wie in der Phonetik der des Segments, vgl.\ Definition~\ref{def:segment}.
Alternativ findet man auch den Begriff des \textit{Phonems}, auf den in Abschnitt~\ref{sec:phonphonem} kurz eingegangen wird.
Allerdings geht es in der Phonologie anders als in der Phonetik um den systematischen Stellenwert der Segmente, nicht um eine oberflächliche Beschreibung ihrer Lautgestalt.

Für den Übergang von der Phonetik zur Phonologie ist der Begriff der \textit{Verteilung} wichtig.
Schon in Abschnitt~\ref{sec:auslautverhaertungphonetik} wurde diskutiert, dass es bestimmte Positionen im Wort oder der Silbe gibt, in denen nur bestimmte Segmente vorkommen.
Dort ging es nur um die Beschreibung verschiedener Korrelationen von Schrift und Phonetik, in der Phonologie haben einige dieser Phänomene aber einen hohen theoretischen Stellenwert.
Das Beispiel war die sog.\ Auslautverhärtung, die dazu führt, dass in der letzten Position der Silbe Plosive immer stimmlos sind (\textit{Bad} als \textipa{[ba:t]}).
Man muss nun aber dennoch davon ausgehen, dass die betreffenden Wörter im Prinzip einen stimmhaften Plosiv an der entsprechenden Stelle enthalten, denn wenn (\zB in Flexionsformen) ein weiterer Vokal folgt, wird der Plosiv wieder stimmhaft, vgl.\ \textit{Bades} \textipa{[ba:d@s]}.
Ausgehend von dem Begriff der phonologischen Verteilung oder Distribution kann man in der Phonologie systematisch über solche Phänomene sprechen.

\Definition{Verteilung (Distribution)}{
Die Verteilung eines Segments ist die Menge der Umgebungen, in denen es vorkommt.
\index{Verteilung}
}

Die Beschreibung der Verteilung eines Segments nimmt typischerweise Bezug auf bestimmte Positionen in der Silbe oder im Wort, oder auf Positionen vor oder nach anderen Segmenten.
Wir können uns nun fragen, wie Segmente zueinander in Beziehung stehen, je nachdem welche Verteilung sie haben.
Konkret ist die entscheidende Frage, ob zwei Segmente dieselbe Verteilung oder eine teilweise oder vollständig unterschiedliche Verteilung haben.
Die Beispiele in (\ref{ex:phol6438})--(\ref{ex:phol6440}) illustrieren drei Typen von Verteilungen anhand des Vergleiches von je zwei Segmenten.

\begin{exe}

  \ex{\label{ex:phol6438} \textipa{[t]} und \textipa{[k]} haben eine vollständig übereinstimmende Verteilung.}
    \begin{xlist}
      \ex{Am Anfang einer Silbe kommen beide vor:\\
      \textit{Tante} \textipa{[tant@]} und \textit{Kante} \textipa{[kant@]}}
      \ex{Am Ende einer Silbe kommen ebenfalls beide vor:\\
      \textit{Schott} \textipa{[SOt]} und \textit{Schock} \textipa{[SOk]}}
    \end{xlist}

  \ex{\label{ex:phol6439} \textipa{[h]} und \textipa{[N]} haben eine vollständig unterschiedliche Verteilung.}
    \begin{xlist}
      \ex{Am Anfang einer Silbe kommt nur \textipa{[h]} vor:\\
      \textit{Hang} \textipa{[haN]} und \textit{behend} \textipa{[b@hEnd]} (niemals *\textipa{[NaN]})}
      \ex{Am Ende einer Silbe kommt nur \textipa{[N]} vor:\\
      \textit{Hang} \textipa{[haN]} und \textit{denken} \textipa{[dENk@n]} (niemals *\textipa{[hah]})}
    \end{xlist}

  \ex{\label{ex:phol6440} \textipa{[s]} und \textipa{[z]} haben eine teilweise übereinstimmende Verteilung.}
    \begin{xlist}
      \ex{Am Anfang der ersten Silbe eines Wortes kommt nur \textipa{[z]} vor:\\
      \textit{Sog} \textipa{[zo:k]} und \textit{besingen} \textipa{[b@zIN@n]} (niemals *\textipa{[so:k]})}
      \ex{Am Ende der letzten Silbe eines Wortes kommt nur \textipa{[s]} vor:\\
      \textit{Vließ} \textipa{[fli:s]} und \textit{Boss} \textipa{[bOs]} (niemals *\textipa{[fli:z]})}
      \ex{Am Anfang einer Silbe in der Wortmitte kommen beide vor, \textipa{[z]} aber nur nach langem Vokal oder Diphthong:\\
      \textit{heißer} \textipa{[h\t{aE}s5]} und \textit{heiser} \textipa{[h\t{aE}z5]}\\
      \textit{Base} \textipa{[ba:z@]} und \textit{Basse} \textipa{[bas@]} (niemals *\textipa{[baz@]})}
    \end{xlist}

\end{exe}

Wie man an den entsprechenden Beispielen sieht, gibt es Segmente, anhand derer Wörter (wie \textit{heißer} und \textit{heiser}) unterschieden werden können, auch wenn die Wörter ansonsten völlig gleich lauten.
Dies geht natürlich nur, wenn die zwei Segmente mindestens eine teilweise übereinstimmende Verteilung haben.
Zwei Wörter, die sich nur in einem Segment unterscheiden, nennt man Minimalpaar, und Minimalpaare illustrieren einen phonologischen Kontrast.

Ähnlich kann man auch für einzelne Merkmale argumentieren.
\textbf{?? TODO}

\Definition{Phonologischer Kontrast}{
\label{def:phokonseg}
Zwei phonetisch unterschiedliche Segmente oder Merkmale stehen in einem phonologischen Kontrast, wenn sie eine teilweise oder vollständig übereinstimmende Verteilung haben und dadurch einen lexikalischen bzw.\ grammatischen Unterschied markieren können.
\index{Kontrast}
}

Ein phonologischer Kontrast besteht also \zB zwischen \textipa{[t]} und \textipa{[k]}, weil wir Wörter anhand dieser Segmente unterscheiden können.
Das Gleiche gilt für \textipa{[s]} und \textipa{[z]} und viele andere Paare von Segmenten.
Es gilt aber nicht für \textipa{[h]} und \textipa{[N]}, weil diese beiden Segmente keine übereinstimmende Verteilung haben, wie in (\ref{ex:phol6439}) gezeigt wurde.
Wie wollte man mit \textipa{[h]} und \textipa{[N]} zwei verschiedene Wörter unterscheiden?
Sobald ein \textipa{[h]} nicht im Silbenanlaut steht, kommen keine akzeptablen Wörter des Deutschen heraus, so wie \textipa{[SVUh]}.
Steht allerdings \textipa{[N]} nicht im Silbenauslaut, kommen ebenfalls keine akzeptablen Wörter dabei heraus, so wie \textipa{[Nand]}.
Sind zwei Segmente in einer Sprache so verteilt wie \textipa{[h]} und \textipa{[N]}, dann können sie niemals einen phonologischen Kontrast markieren.
Diese Art der Verteilungen nennt man komplementär.

\Definition{Komplementäre Verteilung}{
Eine komplementäre Verteilung ist eine Verteilung zweier Segmente, die keinerlei Überschneidung hat.
Komplementär verteilte Segmente können prinzipiell keinen phonologischen Kontrast markieren.
\index{Verteilung!komplementär}
}

Über Verteilungen lässt sich schon anhand des bisher eingeführten Inventars von Beispielen noch mehr sagen.
Bei der bereits besprochenen Auslautverhärtung haben wir es mit Paaren von stimmlosen und stimmhaften Plosiven zu tun, die in bestimmten Umgebungen (im Silbenanlaut) einen Kontrast markieren, der aber in anderen Umgebungen (Silbenauslaut) verschwindet.
(\ref{ex:phol-5674-1})--(\ref{ex:phol-5674-3}) zeigen dies für \textipa{[g]} und \textipa{[k]}, \textipa{[d]} und \textipa{[t]} sowie \textipa{[b]} und \textipa{[p]}.

\begin{exe}
  \ex\label{ex:phol-5674-1}
  \begin{xlist}
    \ex{(der) Zwerg \textipa{[\t{ts}v\t{E@}k]}, (des) Zwerges \textipa{[\t{ts}v\t{E@}g@s]}}
    \ex{(der) Fink \textipa{[fINk]}, (des) Finken \textipa{[fINk@n]}}
  \end{xlist}
  \ex\label{ex:phol-5674-2}
  \begin{xlist}
    \ex{(das) Bad \textipa{[ba:t]}, (des) Bades \textipa{[ba:d@s]}}
    \ex{(das) Blatt \textipa{[blat]}, (des) Blattes \textipa{[blat@s]}}
  \end{xlist}
  \ex\label{ex:phol-5674-3}
  \begin{xlist}
    \ex{(das) Lab \textipa{[la:p]}, (des) Labes \textipa{[la:b@s]}}
    \ex{(der) Depp \textipa{[dEp]}, (des) Deppen \textipa{[dEp@n]}}
  \end{xlist}
\end{exe}

Im Silbenauslaut des Deutschen gibt es prinzipiell keinen Unterschied zwischen stimmlosen und stimmhaften Plosiven.
Solche Effekte nennt man Neutralisierungen.

\Definition{Neutralisierung}{
Eine Neutralisierung ist die positionsspezifische Aufhebung eines phonologischen Kontrasts.
\index{Neutralisierung}
}

Im Silbenauslaut wird im Deutschen also der phonologische Kontrast zwischen \textipa{[g]} und \textipa{[k]}, zwischen \textipa{[d]} und \textipa{[t]} usw.\ neutralisiert.
Allgemein gesprochen wird der Kontrast zwischen stimmlosen und stimmhaften Plosiven in dieser Position neutralisiert.

Das Feststellen von Verteilungen ist allerdings kein Selbstzweck.
Durch die Untersuchung aller Verteilungen in einer Sprache konstruiert man das phonologische System (die phonologische Komponente der Grammatik).
Dabei geht es darum, die Formen zu ermitteln, die im Lexikon gespeichert werden müssen, und die Prozesse (wie die Auslautverhärtung) zu beschreiben, denen die Segmente in diesen Formen unterzogen werden.
Die gespeicherten Formen und die phonologischen Prozesse führen dann zu den phonetisch beobachtbaren Verteilungen an der Oberfläche.

\subsection{Gespanntheit}

\label{sec:gespanntheit}

Außerdem ist noch die \textit{Gespanntheit} zu diskutieren.
Phonetisch ist diese schwer festzumachen, und es ist ggf.\ der Vorwurf gerechtfertigt, dass wir hier sehr viel Phonologie in die Phonetik mit hineinnehmen.
Die Vokale \textipa{[i]}, \textipa{[e]}, \textipa{[u]}, \textipa{[o]} und manchmal auch \textipa{[a]} (\textit{Liebe}, \textit{Weg}, \textit{Wut}, \textit{rot}, \textit{rate}) und ihre gerundeten Entsprechungen gelten als \textit{gespannt}.
Man kann, die Kategorie der Gespanntheit mit einem höheren Luftdruck, erhöhter Muskelanspannung oder einer Veränderung der Position der Zungenwurzel in Verbindung zu bringen.
Eine einfache und im Selbstversuch zu erkundende Zuordnung gibt es aber nicht, und wir diskutieren stattdessen die konkreten Auswirkungen der Gespanntheit weiter unten in Zusammenhang mit der \textit{Vokallänge}.
Diese wird orthographisch uneinheitlich markiert (s.\ Abschnitt~\ref{sec:laengeschreib}), wie sich in (\ref{ex:phot6669}) zeigt.

\begin{exe}
  \ex\label{ex:phot6669}
  \begin{xlist}
    \ex{Mus \textipa{[mu:s]}}
    \ex{muss \textipa{[mUs]}}
    \ex{Ofen \textipa{[Po:f@n]}}
    \ex{offen \textipa{[POf@n]}}
    \ex{Wahn \textipa{[va:n]}}
    \ex{wann \textipa{[van]}}
    \ex{bieten \textipa{[bi:t@n]}}
    \ex{bitten \textipa{[bIt@n]}}
    \ex{fühlt \textipa{[fy:lt]}}
    \ex{füllt \textipa{[fYlt]}}
    \ex{wenig \textipa{[ve:nI\c{c}]}}
    \ex{besonders \textipa{[b@zOnd5s]}}
    \ex{Höhle \textipa{[h\o:l@]}}
    \ex{Hölle \textipa{[h\oe l@]}}
    \ex{Täler \textipa{[tE:l5]}}
    \ex{Teller \textipa{[tEl5]}}
  \end{xlist}
\end{exe}

\textit{Länge} bedeutet hier wirklich erst einmal nur, dass der Vokal mit einer längeren zeitlichen Dauer ausgesprochen wird.
Man markiert Länge in der Transkription mit einem \textipa{[:]} nach dem Vokal.
Die Verteilung von langen und kurzen Vokalen wird mit der Wortliste in (\ref{ex:phot6669}) umfassend, aber nicht vollständig illustriert.
Länge und Gespanntheit hängen nämlich wie schon angedeutet zusammen.

Zunächst einmal gibt es zu fast allen gespannten Vokalen eine ungespannte Variante, s.\ Tabelle~\ref{tab:gespungesp}.
Das Hauptproblem ist zunächst einmal, dass gespanntes und ungespanntes \textipa{[a]} sich artikulatorisch und akustisch nicht unterscheiden.
Außerdem 

Außerdem sind gespannte Vokale immer lang, wenn sie betont werden, und es gibt keine anderen langen Vokale im Deutschen.
Ungespannte Vokale können natürlich auch betont werden, aber sie werden eben nicht lang, \zB betontes \textipa{[I]} in \textit{Rinder} \textipa{[KInd5]} (nicht *\textipa{[kI:nd5]}, *\textipa{[kind5]} oder *\textipa{[ki:nd5]}).
Da im Kernwortschatz (Abschnitt~\ref{sec:kern}) gespannte Vokale immer betont sind, muss der Nicht-Kernwortschatz hinzugezogen werden, um gespannte unbetonte -- und damit kurze -- Vokale zu illustrieren.
Beispiele sind \textipa{[o]} und \textipa{[i]} in der jeweils ersten Silbe der Wörter \textit{Politik} \textipa{[politIk]} (bei manchen Sprechern \textipa{[politi:k]}) oder \textipa{[e]} in \textit{Methyl} \textipa{[mety:l]}.

\begin{table}
	\centering
	\begin{tabular}{clcl}
		\lsptoprule
		\textbf{gespannt} & \textbf{Beispiel} & \textbf{ungespannt} & \textbf{Beispiel} \\
		\midrule
		\textipa{[i]}  & \textit{bieten} \textipa{[bi:t@n]} & \textipa{[I]} & \textit{bitten} \textipa{[bIt@n]} \\
		\textipa{[y]}  & \textit{fühlt} \textipa{[fy:lt]} & \textipa{[Y]} & \textit{füllt} \textipa{[fYlt]} \\
		\textipa{[u]}  & \textit{Mus} \textipa{[mu:s]} & \textipa{[U]} & \textit{muss} \textipa{[mUs]} \\
		\textipa{[e]}  & \textit{Kehle} \textipa{[ke:l@]} & \textipa{[E]} & \textit{Kelle} \textipa{[kEl@]} \\
		\textipa{[E]}  & \textit{stähle} \textipa{[StE:l@]} & \textipa{[E]} & \textit{Stelle} \textipa{[StEl@]} \\
		\textipa{[\o]} & \textit{Höhle} \textipa{[h\o l@]} & \textipa{[\oe]} & \textit{Hölle} \textipa{[h\oe l@]} \\
		\textipa{[o]}  & \textit{Ofen} \textipa{[o:f@n]} & \textipa{[O]} & \textit{offen} \textipa{[Of@n]} \\
		\textipa{[a]}  & \textit{Wahn} \textipa{[va:n]} & \textipa{[a]} & \textit{wann} \textipa{[van]} \\
		\lspbottomrule
	\end{tabular}	
  \caption{Gespannte Vokale mit ihren ungespannten Gegenstücken}
  \label{tab:gespungesp}
\end{table}


Es gilt also Satz~\ref{satz:gespanntlang}.

\Satz{Länge und Gespanntheit}{\label{satz:gespanntlang} Nur betonte gespannte Vokale sind lang.}


\subsection{Affrikaten}

\label{sec:affrikaten}

\textbf{?? ERWEITERN, INTEGRATION PRÜFEN}

Die Affrikaten sind hier aus Platzgründen weitgehend aus der Diskussion ausgespart worden.
Eine wichtige Frage ist allerdings, ob in der Phonologie Affrikaten wie \textipa{[\t{ts}]} als ein Segment behandelt werden sollen, oder als eine Folge aus zwei Segmenten (hier \textipa{[t]} und \textipa{[s]}).
Der Weg zur Lösung dieser Frage führt über die Verteilung der Affrikaten.
Wenn Fremdwörter (bzw.\ Wörter jenseits des Kernwortschatzes, vgl.\ Abschnitt~\ref{sec:nichtkernschreib}) einmal ausgeklammert werden (\zB \textit{Chips} oder \textit{tschechisch}), ergibt sich ein interessantes Bild für die drei primären Kandidaten für Affrikaten.
Vgl.\ dazu die Beispiele in (\ref{ex:phol81209}).

\begin{exe}
  \ex\label{ex:phol81209} 
  \begin{xlist}
    \ex{\label{ex:phol81209a} Zange, Platz}
    \ex{\label{ex:phol81209b} Pfund, Napf}
    \ex{\label{ex:phol81209c} --, Matsch}
  \end{xlist}
\end{exe}

Während /\textipa{\t{ts}}/ und /\textipa{\t{pf}}/ im Onset und in der Coda von Silben vorkommen können, kann /\textipa{\t{tS}}/ nur im Auslaut vorkommen.
Weil sich /\textipa{\t{ts}}/ und /\textipa{\t{pf}}/ also verteilen wie andere stimmlose Obstruenten, kann man sie parallel zu diesen als ein Segment behandeln, aber /\textipa{\t{tS}}/ eher nicht.

Bei /\textipa{\t{pf}}/ kommt hinzu, dass das /\textipa{f}/ als einzelnes Segment in dieser Position eine weitere Verletzung des Sonoritätskontur mit sich brächte.
Durch die Auffassung, dass /\textipa{\t{pf}}/ zusammen ein Segment darstellt, verhindert man dies.

\subsection{Phonologische Prozesse}
\label{sec:pholfeat}
\label{sec:ur}

\subsubsection{Zugrundeliegende Formen und Prozesse}

Wir kommen jetzt noch einmal zum Beispiel der Auslautverhärtung zurück.
Diese hat wie erwähnt zur Folge, dass es bei deutschen Obstruenten im Silbenauslaut keinen Kontrast bezüglich der Stimmhaftigkeit gibt, denn alle Obstruenten im Silbenauslaut sind stimmlos.

Wenn man das gesamte Paradigma der Wörter in (\ref{ex:phol-5674-1}) bis (\ref{ex:phol-5674-3}) ansieht, fällt aber dennoch ein bedeutender Unterschied auf.
In manchen Wörtern steht im Silbenauslaut ein Konsonant, der in anderen Umgebungen stimmhaft ist, wie in \textipa{[\t{ts}v\t{E@}k]} und \textipa{[\t{ts}v\t{E@}g@s]}.
In anderen Wörtern steht ein stimmloser Konsonant, der auch in diesen anderen Umgebungen stimmlos bleibt, wie in \textipa{[fINk]} und \textipa{[fINk@n]}.
Es ist daher sinnvoll, anzunehmen, dass Wörter wie \textit{Zwerg} (oder \textit{Bad}, \textit{Lab} usw.) eine \textit{zugrundeliegende Form} haben, in der der letzte Obstruent stimmhaft ist.
Dazu gibt es einen \textit{phonologischen Prozess}, der diese stimmhaften Konsonanten zu stimmlosen macht, wenn sie in den Silbenauslaut geraten.%
Der Prozess ist in diesem Beispiel eben die Auslautverhärtung.
Man könnte umgekehrt versuchen, eine Art \textit{Inlauterweichung} anzunehmen, die zugrundeliegend stimmlose Obstruenten zu stimmhaften macht, wenn diese nicht im Silbenauslaut stehen.
Dieser Prozess würde dann aber auch in Formen wie \textit{Finken} stattfinden, und es würde\Ast\textipa{[fIN@n]} dabei herauskommen.
Die zugrundeliegende Form muss also genau die phonologischen Informationen eines Wortes enthalten, die ausreichen, um zu erklären, wie die lautliche Gestalt des Wortes in allen möglichen Formen und Umgebungen aussieht.

\Definition{Zugrundeliegende Form und phonologischer Prozess}{
\label{def:pholproz}
Die zugrundeliegende Form ist eine Folge von Segmenten, die im Lexikon gespeichert wird, und aus der alle zugehörigen phonetischen Formen gemäß dem System der phonologischen Prozesse (den Regularitäten der Phonologie) erzeugt werden können.
\index{zugrundeliegende Form}
\index{Prozess!phonologisch}
}

Es ist hoffentlich deutlich geworden, warum die Phonologie eine Abstraktion gegenüber der Phonetik darstellt.
Die Phonetik eines Wortes beschreibt nur, wie es tatsächlich ausgesprochen wird.
Die phonologische Repräsentation eines Wortes erfordert aber zusätzliches Wissen um Prozesse wie die Auslautverhärtung, um aus ihr (ggf.\ abstraktere) phonetische Formen abzuleiten.
Dieses zusätzliche Wissen zur Ermittlung der phonologischen Formen können wir nur gewinnen, wenn wir das gesamte Sprachsystem betrachten, also jedes Wort in Bezug zu allen anderen Wörtern und in allen möglichen Umgebungen.
Anders gesagt müssen die Verteilungen der Segmente und der Wörter bekannt sein.

Zugrundeliegende phonologische Formen schreibt man konventionellerweise nicht in \textipa{[~]} sondern in /~/, also \zB /\textipa{\t{ts}vEKg}/, /\textipa{ba:d}/ und /\textipa{la:b}/.%
\footnote{Die Form /\textipa{\t{ts}vEKg}/ steht hier absichtlich, es handelt sich bei dem /\textipa{K}/ nicht um einen Fehler, wie in Abschnitt~\ref{sec:phonologischeprozesse} erklärt wird.}
Schematisch kann man die Verhältnisse wie in Tabelle~\ref{tab:pholsystem} darstellen, wobei die Prozesse durch den Doppelpfeil $\Rightarrow$ angedeutet werden.
Mit externen Systemen sind nicht zur Grammatik gehörige Systeme wie Gehör und Sprechapparat gemeint.
Wir schreiben später /\textipa{ba:d}/$\Rightarrow$\textipa{[ba:t]}, um zugrundeliegende Form und phonetische Realisierungen in Beziehung zu setzen.

\begin{table}
  \resizebox{\textwidth}{!}{
    \begin{tabular}{ccc}
      \lsptoprule
      \multicolumn{2}{c}{\textbf{Grammatik}} & \textbf{Externe Systeme} \\ 
      \midrule
      \textbf{Lexikon} & \textbf{Phonologie} & \textbf{Phonetik} \\
      \midrule
      /~/& $\Rightarrow$ & \textipa{[~]}\\
      zugrundeliegende Form & phonologische Prozesse & phonetische Realisierung \\
      \lspbottomrule
    \end{tabular}
  }
  \caption{Lexikon, Phonologie und Phonetik}
  \label{tab:pholsystem}
\end{table}

In den Unterabschnitten~\ref{sec:prozauslautverh} bis \ref{sec:prozrvok} werden einige segmentale phonologische Prozesse des Deutschen besprochen.
In Abschnitt~\ref{sec:silbifizierung} wird auch die Silbenbildung als Prozess beschrieben.

\subsubsection{Auslautverhärtung}

\label{sec:prozauslautverh}

Die Auslautverhärtung lässt sich mit den jetzt entwickelten Beschreibungswerkzeugen sehr einfach und kompakt formulieren.
Neben einer quasi-formalen Notation wird eine Übersetzung in natürliche Sprache angegeben.
Vor $\Rightarrow$ steht jeweils das Material, auf das der Prozess angewendet wird, rechts das Material, das der Prozess ausgibt.
Man spricht auch vom Input (linke Seite) und Output (rechte Seite) des Prozesses.

\PholProz{Auslautverhärtung (AV)}{\label{pp:auslautverhaertung}\index{Auslautverhärtung}
 [\textsc{Son}: $-$] \PhPr{AV} [\textsc{Stimme}: $-$] in Coda}

Es wird also gesagt, dass zugrundeliegende Segmente, die [\textsc{Son}: $-$] sind, als [\textsc{Stimme}: $-$] realisiert werden, wenn sie am Silbenende stehen.
Es ist dabei völlig gleichgültig, ob das Segment vorher stimmhaft war oder nicht, und deswegen muss links von $\Rightarrow$ auch nichts über das Merkmal \textsc{Stimme} ausgesagt werden.

Wenn wir diesen Prozess auf zugrundeliegende Formen anwenden, muss also zunächst der Silbifizierungsprozess (hier abgekürzt mit SI) durchgeführt werden, dann kann der Prozess der Auslautverhärtung entsprechende stimmhafte Nicht-Sonoranten stimmlos machen.%
\footnote{Die Silbengrenzen werden in diesem Abschnitt zur besonderen Verdeutlichung in den Phonetik-Klammern auch vor und nach dem Wort durch einen Punkt markiert.}

\begin{exe}
  \ex\label{ex:phol6726}
  \begin{xlist}
    \ex{\label{ex:phol6726a} /\textipa{ba:d}/ \PhPr{SI} \textipa{[.b:ad.]} \PhPr{AV} \textipa{[.ba:t.]}}
    \ex{\label{ex:phol6726b} /\textipa{ba:d@s}/ \PhPr{SI} \textipa{[.b:a.d@s.]}}
    \ex{\label{ex:phol6726c} /\textipa{ba:t}/ \PhPr{SI} \textipa{[.b:at.]} \PhPr{AV} \textipa{[.ba:t.]}}
  \end{xlist}
\end{exe}

Abhängig von der zugrundeliegenden Form und der Silbifizierung hat die Auslautverhärtung eine Wirkung oder nicht.
In (\ref{ex:phol6726a}) gerät /\textipa{d}/ durch die Silbifizierung in den Silbenauslaut (Coda), und weil /\textipa{d}/ den Wert [\textsc{Son}: $-$] hat, greift die Auslautverhärtung und ändert das Merkmal [\textsc{Stimme}: $+$] zu [\textsc{Stimme}: $-$] (hier hilft ggf.\ ein Blick zurück in Abschnitt~\ref{sec:pholfeat}, vor allem Abbildung~\ref{fig:artart} und Tabelle~\ref{tab:pholkonsmerk}).
In (\ref{ex:phol6726b}) wird anders silbifiziert (Onset-Maximierung, vgl.\ Abschnitt~\ref{sec:silbifizierung}), und daher ist die Bedingung für die Auslautverhärtung (der Nicht-Sonorant soll am Silbenende stehen) nicht erfüllt, und sie findet nicht statt.
In (\ref{ex:phol6726c}) steht zwar ein Nicht-Sonorant /\textipa{t}/ am Silbenende, aber die Auslautverhärtung hat keine Wirkung, weil /\textipa{t}/ von vornherein [\textsc{Stimme}: $-$] ist.

\subsubsection{Verteilung von [ç] und [χ]}

\label{sec:prozichach}

Die sogenannten \textit{ich}- und \textit{ach}-Segmente sind komplementär verteilt.
Es gibt kein Wort, in dem sie einen lexikalischen Unterschied markieren können.
Schauen wir uns zunächst einige Beispiele für Wörter an, in denen \textipa{[\c{c}]} (\ref{ex:phol6110a}) und \textipa{[X]} (\ref{ex:phol6110b}) vorkommen.

\begin{exe}
  \ex\label{ex:phol6110}
  \begin{xlist}
    \ex{\label{ex:phol6110a} rieche, Bücher, schlich, Gerüche, Wehwehchen, röche, schlecht, Löcher}
    \ex{\label{ex:phol6110b} Tuch, Geruch, hoch, Loch, Schmach, Bach.}
  \end{xlist}
\end{exe}

Ausschlaggebend für das Vorkommen von \textipa{[\c{c}]} und \textipa{[X]} ist der unmittelbar vorangehende Kontext.
Nach /\textipa{i:}/, /\textipa{y:}/, /\textipa{I}/, /\textipa{Y}/, /\textipa{e:}/, /\textipa{\o}/, /\textipa{E:}/, /\textipa{E}/, /\textipa{\oe}/ kommt \textipa{[\c{c}]} vor, nach /\textipa{u:}/, /\textipa{U}/, /\textipa{o:}/, /\textipa{O}/, /\textipa{a:}/ und /\textipa{a}/ hingegen \textipa{[X]} (nach Schwa kommt keins der beiden Segmente vor).
Ein Blick auf das Vokalviereck (Abbildung~\ref{fig:vokaltrapatr}, S.~\pageref{fig:vokaltrapatr}) zeigt sofort, was der relevante Merkmalsunterschied ist.
Nach Vokalen, die [\textsc{Hinten}: $-$] sind, steht \textipa{[\c{c}]}, nach Vokalen, die [\textsc{Hinten}: $+$] sind, steht hingegen \textipa{[X]}.
Die relevanten Merkmale der beiden Frikative sind die in (\ref{ex:phol1190}).

\begin{exe}
  \ex\label{ex:phol1190}
  \begin{xlist}
    \ex{\textipa{[\c{c}]} $=$ [\textsc{Kons}: $+$, \textsc{Appr}: $-$, \textsc{Son}: $-$, \textsc{Kont}: $+$, \textsc{Ort}: \textit{dor}, \textsc{Hinten}: $-$]}
    \ex{\textipa{[X]} $=$ [\textsc{Kons}: $+$, \textsc{Appr}: $-$, \textsc{Son}: $-$, \textsc{Kont}: $+$, \textsc{Ort}: \textit{dor}, \textsc{Hinten}: $+$]}
  \end{xlist}
\end{exe}

Hier wird ein Vorteil der zunächst vielleicht etwas umständlich wirkenden phonologischen Merkmale deutlich.
Dank des sowohl vokalischen als auch konsonantischen Merkmals \textsc{Hinten} kann die Frage der Realisierung von \textipa{[\c{c}]} und \textipa{[X]} als Prozess beschrieben werden, der den Wert des Merkmals \textsc{Hinten} beim Frikativ an den entsprechenden Wert des vorangehenden Vokals angleicht bzw.\ assimiliert.
Assimilation heißt hier nichts anderes, als dass der Wert eines Merkmals mit dem eines anderen gleichgesetzt wird, was durch eine Variable (hier \textit{x}) angezeigt werden kann.
Alle Merkmale, über die auf der rechten Seite keine Angaben gemacht werden, bleiben wie sie sind.

\PholProz{\textsc{Hinten}-Assimilation (HA)}{\index{hinten!Assimilation}
[\textsc{Son}: $-$, \textsc{Kont}: $+$, \textsc{Ort}: \textit{dor}] \PhPr{HA} [\textsc{Hinten}: \textit{x}]\\
nach [\textsc{Kons}:$-$, \textsc{Hinten}: \textit{x}]
}

Es muss jetzt nur noch entschieden werden, ob in der zugrundeliegenden Form für \textipa{[\c{c}]} und \textipa{[X]} gar kein Wert für \textsc{Hinten} gespeichert ist, oder ob vielleicht einer der beiden möglichen Werte ($+$ oder $-$) zugrundeliegt und in einem der beiden Fälle geändert wird.
Aufschlussreich ist hier die Betrachtung von Wörtern wie \textit{Milch} /\textipa{mIl\c{c}}/, \textit{Storch} /\textipa{StOK\c{c}}/ oder \textit{Röckchen} /\textipa{K\oe k\c{c}@n}/, in denen \textipa{[\c{c}]} (aber niemals \textipa{[X]}) nach einem Konsonanten vorkommt.
Es ist also besser, anzunehmen, dass /\textipa{\c{c}}/ zugrundeliegt und \textipa{[X]} das phonetische Resultat einer Assimilation ist.
Aus diesem Grund wurde in Abschnitt~\ref{sec:konsonantenmerkmale} das Segment \textipa{[X]} auch nicht in /~/ gesetzt.
Es ist kein zugrundeliegendes Segment.
Damit ergeben sich die Anwendungen des Prozesses wie in (\ref{ex:phol8011}).

\begin{exe}
  \ex\label{ex:phol8011}
  \begin{xlist}
    \ex{/\textipa{I\c{c}}/ \PhPr{HA} \textipa{[PI\c{c}]}}
    \ex{/\textipa{a\c{c}}/ \PhPr{HA} \textipa{[PaX]}}
  \end{xlist}
\end{exe}

\subsubsection{Frikativierung von /g/}

\label{sec:prozgfrik}

Im Standard wird /\textipa{Ig}/ als \textipa{[I\c{c}.]} realisiert.
Das /\textipa{g}/ wird also zum Frikativ, und kein anderer Vokal außer /\textipa{I}/ hat diese Wirkung auf das /\textipa{g}/.
Der Prozess wird als /\textipa{g}/-Frikativierung oder /\textipa{g}/-Spirantisierung bezeichnet.
In (\ref{ex:phol8482}) sind die einzigen Merkmale von /\textipa{g}/ und /\textipa{\c{c}}/ gegenübergestellt, die sich in ihren Werten unterscheiden.

\begin{exe}
  \ex\label{ex:phol8482}
  \begin{xlist}
    \ex{/\textipa{g}/ $=$ [\textsc{Kont}: $-$, \textsc{Stimme}: $+$]}
    \ex{/\textipa{\c{c}}/ $=$ [\textsc{Kont}: $+$, \textsc{Stimme}: $-$]}
  \end{xlist}
\end{exe}

Die Änderung dieser Werte ist offensichtlich nicht gut als Assimilation an die Merkmale von /\textipa{I}/ zu beschreiben.
Der Prozess hat vielmehr etwas Willkürliches an sich.
Daher können wir ihn auch unter Bezugnahme auf ganze Segmente formulieren und müssen diese nicht unbedingt in Merkmale aufschlüsseln.%
\footnote{Man kann den Verlust der Stimmhaftigkeit auch der Auslautverhärtung überlassen.
Dies hat aber weitere Implikationen bezüglich der Reihenfolge, in der die Prozesse stattfinden müssen, weswegen dies hier nicht besprochen wird.}

\PholProz{/g/-Frikativierung (GF)}{\textipa{Ig} \PhPr{GF} \textipa{I\c{c}} in Coda}

Die Formulierung des Prozesses enthält eine wichtige Einschränkung, nämlich dass der Prozess nur am Silbenende stattfindet.
In (\ref{ex:phol9990}) sind einige Beispiele angegeben, in denen diese Einschränkung zusammen mit dem Silbifizierungsprozess interessante Resultate erzeugt.

\begin{exe}
  \ex\label{ex:phol9990}
  \begin{xlist}
    \ex{/\textipa{ve:nIg}/ \PhPr{SI} \textipa{[.ve:.nIg.]} \PhPr{GF} \textipa{[.ve:.nI\c{c}.]}}
    \ex{/\textipa{ve:nIg@}/ \PhPr{SI} \textipa{[.ve:.nI.g@.]} \PhPr{GF} \textipa{[.ve:.nI.g@.]}}
  \end{xlist}
\end{exe}

Wie schon bei der Auslautverhärtung (Abschnitt~\ref{sec:prozauslautverh}) kann die Silbifizierung die Anwendbarkeit anderer Prozesse beeinflussen.
Weil im Wort \textit{wenige} das /\textipa{g}/ in den Onset der letzten Silbe gerät (und nicht in die Coda wie bei \textit{wenig}), kann die \textit{g}-Frikativierung nicht eintreten, denn sie ist beschränkt auf die Codaposition.

\subsubsection{/ʁ/-Vokalisierungen}

\label{sec:prozrvok}

Mit der Diskussion der /\textipa{K}/-Vo\-ka\-li\-sie\-rung (RV) schließt jetzt der Abschnitt über die phonologischen Prozesse.
In Abschnitt~\ref{sec:realisr} wurden verschiedene phonetische Korrelate von geschriebenem \textit{r} besprochen.
Die Schrift ist hier eigentlich besonders systematisch, denn orthographisches \textit{r} entspricht immer einem zugrundeliegenden /\textipa{K}/ (vgl.\ auch Abschnitt~\ref{sec:buchstabensegmente}).
In (\ref{ex:phol9906}) sind einige Beispiele zusammengestellt, die dies illustrieren.

\begin{exe}
  \ex\label{ex:phol9906}
  \begin{xlist}
    \ex{geringer \textipa{[.g@.KIN.5.]}, geringere \textipa{[.g@.KIN.@.K@.]}}
    \ex{Bär \textipa{[.b\t{E5}.]}, Bären \textipa{[.bE:.K@n.]}}
    \ex{knarr \textipa{[.kn\t{a@}.]}, knarre \textipa{[.kna.K@.]}}
  \end{xlist}
\end{exe}

Wenn ein zugrundeliegendes /\textipa{K}/ im Onset steht, wird es als konsonantisches \textipa{[K]} realisiert.
Demgegenüber müssen für /\textipa{K}/ in Codas drei Fälle unterschieden werden.
Erstens gibt es eine dem Schwa ähnliche Realisierung von /\textipa{@K}/, nämlich \textipa{[5]}.
Dieses steht niemals in einer akzentuierten Silbe, da Schwa niemals in solchen Silben vorkommt.
Bei allen anderen Vokalen muss zwischen langen und kurzen Vokalen unterschieden werden.
Ein langer Vokal vor /\textipa{K}/ verliert an Länge, und das /\textipa{K}/ wird als \textipa{[5]} realisiert.
Nach kurzem Vokal wird /\textipa{K}/ schließlich als \textipa{[@]} realisiert.
Wegen der komplizierten Verhältnisse versuchen wir im Fall der /\textipa{K}/-Vokalisierung nicht, den Prozess vollständig mit Merkmalen zu beschreiben und geben einfach die drei möglichen Varianten an.

\PholProz{/ʁ/-Vokalisierung}{\index{r-Vokalisierung}
\begin{tabular}{ccc}
  \textipa{@K} & \PhPr{RV} & \textipa{5} am Silbenende\\
  {}\textipa{K} & \PhPr{RV} & \textipa{5} nach [\textsc{Lang}: $+$] am Silbenende\\
  {}\textipa{K} & \PhPr{RV} & \textipa{@} nach [\textsc{Lang}: $-$] am Silbenende
\end{tabular}
}

Interessant ist, dass in allen diesen Fällen die Coda der Silbe letztendlich nicht besetzt wird, sondern im Nukleus ein sekundärer Diphthong entsteht.
Der Begriff des sekundären Diphthongs wurde in Abschnitt~\ref{sec:realisr} bereits benutzt, jetzt können wir genauer angeben, was darunter zu verstehen ist.
Es handelt sich um Diphthonge, die auf die Vokalisierung eines zugrundeliegenden Konsonanten zurückgehen.

\subsubsection{Einfügung des Glottalverschlusses}

\label{sec:glottalverschluss}

Jetzt kann, nachdem auch der Akzent besprochen wurde, noch die Regularität der \textipa{[P]}-Einfügung, die in Abschnitt~\ref{sec:photlaryngale} sehr kurz angesprochen wurde, genau angegeben werden.
Es handelt sich um eine Interaktion von segmentaler Phonologie, Silbifizierung und Prosodie.
Statt mühsam einen phonologischen Prozess zu formulieren, erfassen wir die Regularität in einem Satz.

\Satz{[ʔ]-Einfügung}{
\label{satz:glottalstoprule}
Der laryngale Plosiv \textipa{[P]} ist nicht zugrundeliegend und wird im Zuge der Akzentzuweisung und der Silbifizierung in den leeren Onset von Silben eingefügt, die entweder (1) am Wortanfang stehen oder (2) im Wortinneren stehen und betont sind.
}

Silben, die eigentlich einen leeren Onset haben (also mit Vokal anlauten) werden um dieses Segment unter genau benennbaren phontaktischen und prosodischen Bedingungen ergänzt.
Die Beispiele in (\ref{ex:phol1249}) in phonetischer Umschrift mit Silbengrenzen und \textipa{[\textprimstress]} für den Akzent zeigen die Wirkung dieser Regularität.

\begin{exe}
  \ex\label{ex:phol1249}
  \begin{xlist}
    \ex{Aue \textipa{[\textprimstress P\t{aO}.@]}}
    \ex{Chaos \textipa{[\textprimstress ka:.Os]}}
    \ex{Chaot \textipa{[ka.\textprimstress Po:t]}}
    \ex{beäugen \textipa{[be.\textprimstress P\t{O\oe}.g@n]}}
    \ex{vereisen \textipa{[f5.\textprimstress P\t{aE}z@n]}}
    \ex{unterweisen \textipa{[PUnt5.\textprimstress v\t{aE}z@n]}}
  \end{xlist}
\end{exe}

\section[Phone und Phoneme]{\Opsional Phone und Phoneme}

\label{sec:phonphonem}

In diesem Abschnitt soll kurz auf einige oft erwähnte phonologische Begriffe -- vor allem auf den des Phonems -- eingegangen werden.
Dabei soll gezeigt werden, warum eine einfache Phonemtheorie bestimmte Probleme mit sich bringt, zumal wenn sie ohne phonologische Merkmale formuliert wird.

Zugrundeliegende Formen und phonologische Prozesse gibt es in der Phonemtheorie zunächst nicht.
Segmente werden lediglich danach klassifiziert, ob sie distinktiv sind oder nicht.
Als Basisbegriff wird das Phon als phonetisch realisiertes Segment definiert, also als das, was wir in [~] schreiben.
In \textipa{[ta:k]} sind drei Phone zu beobachten, nämlich \textipa{[t]}, \textipa{[a:]} und \textipa{[k]}.

\Definition{Phon}{
\label{def:phon}
Das Phon ist eine segmentale phonetische Realisierung.
\index{Phon}
}

Der Begriff des Phonems baut dann auf dem des Phons auf, denn die Phoneme sind Abstraktionen von Phonen.
Wenn nämlich mehrere Phone distinktiv sind, gehören sie zu verschiedenen Phonemen, sonst sind sie lediglich Realisierungen eines einzigen abstrakten Phonems.
Als Beispiel kann man wieder \textipa{[\c{c}]} und \textipa{[X]} heranziehen (vgl.\ Abschnitt~\ref{sec:prozichach}).
Diese beiden Phone können keine Bedeutungen unterscheiden (es gibt keine Minimalpaare, vgl.\ Abschnitt~\ref{sec:verteilungen}) und können daher als Realisierungen eines abstrakten Phonems /\textipa{x}/ angesehen werden.
Man würde sagen, \textipa{[\c{c}]} und \textipa{[X]} sind Allophone eines Phonems /x/.
Wie man das Phonem nennt, ist dabei egal.
Man könnte es auch /P\Tidx{42}/ oder /\#/ nennen, solange nicht schon ein anderes Phonem so benannt wurde.

\Definition{Phonem}{
\label{def:phonem}
Ein Phonem ist eine Abstraktion von (potentiell) mehreren Phonen, die nicht distinktiv sind.
Die verschiedenen möglichen Phone zu einem Phonem werden Allophone genannt.
\index{Phonem}
}

Als Beispiel wird (\ref{ex:phol2209}) gegeben.

\begin{exe}
  \ex\label{ex:phol2209}
  \begin{xlist}
    \ex{\label{ex:phol2209a} \textit{ich}: Phone: \textipa{[I\c{c}]}, Phoneme: /\textipa{Ix}/}
    \ex{\label{ex:phol2209b} \textit{ach}: Phone: \textipa{[aX]}, Phoneme: /\textipa{ax}/}
  \end{xlist}
\end{exe}

An dieser Theorie ist im Prinzip nichts Falsches, sie ist lediglich explanatorisch schwächer als die bisher vorgestellte Theorie.
Die Phoneme sind zunächst nur abstrakte Größen, die nicht als Mengen von Merkmalen, sondern über die Distinktivität definiert werden.
Selbst wenn man Merkmalsanalysen hinzufügt, fehlt das Konzept des phonologischen Prozesses.
Phonologische Alternationen können also nicht effektiv als Prozess (Änderung von Werten phonologischer Merkmale) beschrieben werden.

Man kann dies an der Auslautverhärtung gut demonstrieren.
In der hier benutzten Darstellung lässt sich die Auslautverhärtung kompakt als Prozess der Änderung eines Merkmals unter einer bestimmten Bedingung formulieren (vgl.\ Abschnitt~\ref{sec:prozauslautverh}).
In einer reinen Phonemtheorie müsste man sagen, dass das Phonem /\textipa{b}/ je nach Umgebung zwei Allophone hat, nämlich Allophon \textipa{[p]} im Silbenauslaut und Allophon \textipa{[b]} in allen anderen Positionen.
Dasselbe müsste man für /\textipa{d}/ und /\textipa{g}/ (und ihre Allophone) wiederholen, wobei die eigentliche Regularität, die wir in einem einfachen Prozess dargestellt haben, nicht erfasst wird.

Als abschließendes Beispiel soll gezeigt werden, dass sich die fehlende Merk\-mals\-ana\-lyse noch auf ganz andere Weise bemerkbar macht.
Die Phone \textipa{[h]} und \textipa{[N]} sind im Deutschen zueinander nicht distinktiv (vgl.\ Abschnitt~\ref{sec:verteilungen}, vor allem (\ref{ex:phol6439}) auf S.~\pageref{ex:phol6439}).
Man könnte sie daher ohne weiteres als Allophone eines abstrakten Phonems /\textipa{h}/ auffassen.
Dieses Phonem hätte zwei Allophone, nämlich \textipa{[h]} im Onset und \textipa{[N]} in Coda.
Wegen der geringen phonetischen Ähnlichkeit dieser potentiellen Allophone (vgl.\ die Merkmale der Segmente in Tabelle~\ref{tab:pholkonsmerk}) erscheint dies zunächst absurd.
Darüber hinaus stehen diese Segmente aber strukturell auch in keinerlei Beziehung, es ist sozusagen offensichtlicher Zufall, dass sie komplementär verteilt sind.
Bei \textipa{[\c{c}]} und \textipa{[X]} ist die komplementäre Verteilung hingegen eindeutig nicht zufällig, wie in Abschnitt~\ref{sec:prozichach} demonstriert wurde.
Daher fügt man für die Phonembildung als Lösungsversuch gerne die Bedingung hinzu, dass Allophone eines Phonems phonetisch ähnlich sein sollen.
Wenn es aber keine Merkmalsanalysen gibt, weiß man nicht so recht, was phonetische Ähnlichkeit eigentlich sein soll.

Außerdem kann man zeigen, dass phonetische Ähnlichkeit generell kein gutes Kriterium ist, wenn die strukturelle Analyse eine Allophon-Beziehung zwischen zwei Phonen nahelegt.
Nach Vokalen müsste man \zB annehmen, dass \textipa{[@]} und \textipa{[5]} als Allophone eines Phonems /\textipa{r}/ vorkommen.
Ebenso wäre im Onset \textipa{[K]} ein Allophon von /\textipa{r}/ (vgl.\ Abschnitt~\ref{sec:prozrvok}).%
\footnote{Hier wird absichtlich /\textipa{r}/ als Symbol für das Phonem verwendet, um deutlich zu machen, dass es sich eben nicht um eine zugrundeliegende Form handelt und man daher irgendein Symbol nehmen kann.
Hier ist es eben dasjenige, das der Schreibung entspricht.}
Phonetisch ähnlich sind sich \textipa{[@]} und \textipa{[K]} aber in keiner Weise.
Es zeigt sich also, dass die noch gebräuchliche Rede von Phonemen und Allophonen zwar nicht falsch ist, aber in vielen Punkten gegenüber der hier verwendeten Darstellung Nachteile mit sich bringt.

\Zusammenfassung

\begin{enumerate}
  \item Die Phonologie beschäftigt sich mit den phonetischen Unterschieden, die eine systematische grammatische Funktion haben.
  \item Nicht jedes Segment (=~jeder Laut) kommt in den gleichen Umgebungen vor, und man kann Segmente danach einteilen, ob sie in vollständig identischen, teilweise identischen oder gänzlich verschiedenen Umgebungen vorkommen.
  \item Solche Verteilungen kann man auch für Merkmale (statt ganzer Segmente) ermitteln, \zB kommen stimmhafte Obstruenten im Deutschen nicht im Silbenauslaut vor.
  \item Phonologische Prozesse (wie die Auslautverhärtung oder die Frikativierung von /\textipa{Ig}/ zu \textipa{[i\c{c}]}) verändern die im Lexikon abgelegten Segmentfolgen je nachdem, in welcher Umgebung sie realisiert werden.
  \item Silbenstrukturen sind nicht im Lexikon festgelegt, sondern werden den Wörtern durch einen Prozess zugewiesen.
  \item Alle Silben folgen der Sonoritätshierarchie sowie weiteren sprachspezifischen Bedingungen (\zB Beschränkung der Plateaubildungen).
  \item \textbf{?? TODO}
  \item \textbf{?? TODO}
  \item Der Wortakzent ist die Hervorhebung einer Silbe im Wort durch Lautstärke, Länge usw.
  \item Das Deutsche ist dominant trochäisch mit der Betonung auf der ersten Silbe des Wortstamms.
\end{enumerate}

\Uebungen

\Uebung \label{u41} Finden Sie deutsche Minimalpaare für die folgenden Kontraste in der Art des ersten Beispiels.

\begin{enumerate}\Lf
  \item{/\textipa{t}/, /\textipa{d}/ : \textit{Tank}, \textit{Dank}}
  \item{/\textipa{n}/, /\textipa{s}/}
  \item{/\textipa{v}/, /\textipa{m}/}
  \item{/\textipa{X}/, /\textipa{N}/}
  \item{/\textipa{K}/, /\textipa{h}/}
  \item{/\textipa{s}/, /\textipa{k}/}
  \item{/\textipa{\t{pf}}/, /\textipa{s}/}
  \item{/\textipa{\t{aE}}/, /\textipa{\t{aO}}/}
  \item{/\textipa{i:}/, /\textipa{I}/}
\end{enumerate}

\Uebung \label{u42} Zeichnen Sie die Paare von nicht umgelauteten Vokalen und umgelauteten Vokalen in ein Vokalviereck und beschreiben Sie das Phänomen Umlaut dann mittels phonologischer Merkmale.
Die Vokalpaare mit und ohne Umlaut finden Sie in \textit{Fuß} -- \textit{Füße}, \textit{Genuss} -- \textit{Genüsse}, \textit{rot} -- \textit{röter}, \textit{Koffer} -- \textit{Köfferchen}, \textit{Schlag} -- \textit{Schläge}, \textit{Bach} -- \textit{Bäche}.
Zusatzaufgabe: Versuchen Sie, den Umlaut /\textipa{\t{aO}}/ -- /\textipa{\t{O\oe}}/ in die Beschreibung zu integrieren.

\Uebung[\tristar] \label{u43} Diese Übung bezieht sich auf Abschnitt~\ref{sec:prozichach}.

\begin{enumerate}\Lf
  \item Überlegen Sie, wie sich im Fall von Lehnwörtern wie \textit{Chemie} oder \textit{Chuzpe} die teilweise üblichen Realisierungen wie \textipa{[\c{c}emi:]} und \textipa{[XU\t{ts}p@]} in das phonologische System des Deutschen integrieren.
  \item Wie beurteilen Sie unter dem Gesichtspunkt des phonologischen Systems des Deutschen die Strategien, statt \textipa{[\c{c}emi:]} entweder \textipa{[Semi:]} oder \textipa{[kemi:]} zu realisieren?
  \item Bedenken Sie die Tatsache, dass für \textit{Chuzpe} niemals \textipa{[SU\t{ts}p@]} oder \textipa{[kU\t{ts}p@]} realisiert werden.
    Was sagt Ihnen das über die Integration des Wortes \textit{Chuzpe} in den deutschen Wortschatz (im Vergleich zu \textit{Chemie})?
\end{enumerate}

\Uebung \label{u44} Zerteilen Sie die folgenden Wörter in ihre Silben (Silbifizierung) und zeichnen Sie eine Sonoritätskurve wie in Abbildung~\ref{fig:sonhiers-strolchst}.
Geben Sie an, welche Bedingungen des Silbifizierungsprozesses (Abschnitt~\ref{sec:silbifizierung}) erfüllt werden und welche nicht.

\begin{enumerate}\Lf
  \item Strumpf
  \item wringen
  \item winkte
  \item Quarkspeise
  \item Leser
  \item Leserin
  \item zusätzlich
  \item zusätzliche
  \item Hammer
  \item Fenster
  \item Iglu
  \item komplett
\end{enumerate}

\Uebung \label{u45} Entscheiden Sie, wo die folgenden Wörter ihren Akzent haben (ggf.\ unter Zuhilfenahme des Fokussierungstests).
Überlegen Sie, ob sie damit den Regeln aus Abschnitt~\ref{sec:prosodie} folgen.

\begin{enumerate}\Lf
  \item freches
  \item Klingel
  \item Opa
  \item nachdem
  \item Auto
  \item Autoreifen
  \item Beendigung
  \item Melone
  \item rötlich
  \item Rötlichkeit
  \item Pöbelei
  \item respektabel
  \item Schulentwicklungsplan
\end{enumerate}

\Uebung[\tristar] \label{u46} Beschreiben Sie die Silbenstruktur in Wörtern wie \textit{Herbst}, \textit{lebst}, \textit{kriegst} usw.
Was fällt auf?

\Uebung[\tristar] \label{u47} In (\ref{ex:phol8882}) auf Seite \pageref{ex:phol8882} wird behauptet, dass \textipa{[s5]} im Deutschen kein Einsilbler sein kann.
Nennen Sie zwei Gründe, warum das so ist.

\WeitereLiteratur

\paragraph*{Phonetik}

Eine sehr ausführliche Einführung in die artikulatorische Phonetik ist \citet{Laver94}.
Einführende Darstellungen der deutschen Phonetik finden sich \zB in \citet{RRKWS09} und \citet{Wiese10}.
Eine ausführliche Beschreibung der deutschen Standardvarietäten (Deutschland, Österreich, Schweiz), der wir hier überwiegend gefolgt sind, gibt \citet{Krech-ea2009}.
Ein weiteres Nachschlagewerk mit kleinen Unterschieden in der Darstellung zu \citealp{Krech-ea2009} ist \citet{Mangold06}.

\paragraph*{Phonologie}

\label{abs:pholliteratur}

Der hier zur Phonologie besprochene Stoff findet sich mit kleinen Abweichungen \zB in \citet{Hall00} und \citet{Wiese10}.
In eine grammatische Gesamtbeschreibung eingebunden sind Kapitel~3 und~4 im \textit{Grundriss} \citep{Eisenberg1}.
Eine Einführung, die eher strukturalistisch argumentiert, ist \citet{Ternes2012}.
Als anspruchsvolle Gesamtdarstellung der deutschen Phonologie kann \citet{Wiese00} verwendet werden.
