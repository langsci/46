\chapter{Wortklassen}

\label{sec:wortklassen}

\section{Wörter}

\label{sec:woerter}

Mit diesem Kapitel beginnt die Betrachtung der Wörter im Rahmen der Grammatik.
Daher soll zuerst überlegt werden, was Wörter sind.
In \ref{sec:definitionsproblemewort} wird kurz die Problematik der Definition des Wortes diskutiert.
In \ref{sec:klassifikationsmethoden} werden grundsätzliche Prinzipien der Wortklassifizierung diskutiert, und in \ref{sec:wortklassendesdeutschen} wird schließlich eine Klassifikation der Wörter des Deutschen vorgeschlagen.
In den nächsten Kapiteln wird dann ausführlich die Beziehung von Form und Funktion bei einzelnen Wortklassen diskutiert.
Wie schon in Abschnitt~\ref{sec:kategorien} beschrieben handelt es sich bei der Definition von Wortklassen um eine Kategorienbildung innerhalb des Lexikons.
Nach bestimmten Kriterien (idealerweise nach Merkmalen und ihren Werten) werden Wörter in eine überschaubare Menge von Klassen (und ggf.\ Unterklassen) eingeteilt.
Dies hat den Zweck, dass möglichst viele Regularitäten des Sprachsystems über größere Klassen von Wörtern formuliert werden können, statt dass man für jedes Wort einzeln festlegen müsste, wie es sich verhält.
Was ist aber überhaupt ein Wort?
Darum geht es jetzt zunächst.

\subsection{Definitionsprobleme}

\label{sec:definitionsproblemewort}

\index{Wort}

Mit Definition~\ref{def:phonwort} in Kapitel~\ref{sec:phonologie} auf S.~\pageref{def:phonwort} haben wir schon eine (rein phonologische) Definition des Wortes gegeben.
Das \textit{phonologische Wort} ist gemäß dieser Definition die kleinste Struktur, die aus Silben besteht und bezüglich derer eigene phonologische Regularitäten erkennbar sind, wie \zB die Akzentzuweisung.
Dieser Stil, Definitionen zu formulieren, ist äußerst elegant, weil dabei ausschließlich formale Kriterien verwendet werden.
Viel problematischer wäre es zum Beispiel, Wörter als \textit{Bedeutungsträger} zu definieren.
Es wäre dann zu fragen, ob Wörter wie \textit{und} oder \textit{doch}, oder \textit{es} in Satz (\ref{ex:wk9191}) wirklich eine Bedeutung haben.

\begin{exe}
  \ex{\label{ex:wk9191} Es kommt eine Sendung auf Kurzwelle.}
\end{exe}

Vielleicht kann man auch diesen eine Bedeutung zusprechen, aber der Bedeutungsbegriff, den man dann anwenden müsste, wäre ungleich komplexer als jeder intuitive Bedeutungsbegriff.
Anders gesagt ist das Problem der Definition von Wörtern als Bedeutungsträger, dass sie die Definition des Bedeutungsbegriffs voraussetzt, die aber sicherlich noch problematischer ist als die Definition des Wortes.

Für die Beschreibung des Aufbaus der Wörter sowie ihres Verhaltens in der Syntax wäre es hilfreich, eine Definition des Wortes zu finden, die nicht nur auf rein phonologische Größen Bezug nimmt.
Anders gesagt:
Man möchte nicht die wichtigste grundlegende Einheit der Morphologie und der Syntax mittels einer phonologischen Definition einführen.
Leider ist die Definition des Wortes notorisch schwierig, und jede Definition muss in der einen oder anderen Hinsicht unzulänglich werden.
Es sei hier daher darauf hingewiesen, dass auch die folgende Kette von tentativen Definitionen keine echte Definition ergibt und als eine von Zirkularität nicht ganz freie Heuristik angesehen werden muss.
Eine formale Möglichkeit, das Wort ohne direkten Bezug zur Phonologie zu definieren, wäre der explizite Bezug auf Kombinationsregeln der Wort-Einheit, die nichts mit Phonologie zu tun haben.

\Definition{Wort (falsch)}{
\label{def:woerter1}
Wörter sind die kleinsten Einheiten, die nach nicht"=phonologischen Regularitäten zu Strukturen zusammengefügt werden.
}

Die Intention hinter dieser Definition ist leicht ersichtlich.
Dass zum Beispiel in (\ref{ex:wa7197}) die Segmentfolge \textit{der} (nicht \textit{die}) mit \textit{Satz} kombiniert werden muss, hat auf keinen Fall phonologische Gründe.
Die Struktur, die hier aufgebaut wird, folgt anderen Regularitäten (und zwar morphologischen und syntaktischen).

\begin{exe}
  \ex\label{ex:wa7197}
  \begin{xlist}
    \ex[]{Der Satz ist eine grammatische Einheit.}
    \ex[*]{Die Satz ist eine grammatische Einheit.}
  \end{xlist}
\end{exe}

Der Nachteil an dieser Definition ist aber, dass sie eher auf Einheiten zutrifft, die kleiner als das sind, was gemeinhin als Wort bezeichnet wird.
Es folgt ein Beispiel zur Illustration.

\Enl

\begin{exe}
  \ex
  \begin{xlist}
    \ex[]{Staat-es}
    \ex[*]{Tür-es}
  \end{xlist}
\end{exe}

Man sieht sofort, dass auch die Bestandteile des Wortes nach Regularitäten zusammengesetzt werden, die nichts mit Phonologie zu tun haben.
Der Bestandteil \textit{-es} ist mit \textit{Tür} nicht kombinierbar, mit \textit{Staat} aber schon, obwohl aus phonologischer Sicht gegen die Segmentkombination /\textipa{ty:K@s}/ im Deutschen nichts einzuwenden wäre.
Es gibt also in der sogenannten \textit{Flexion} auch eigene Regularitäten.%
\footnote{Eigene Regularitäten gibt es auch im Bereich der Wortbildung (vgl.\ Kapitel~\ref{sec:wortbildung}).}
\label{arbref:9234645}Da man \textit{-es} nicht gerne als Wort, sondern eher als Wortbestandteil bezeichnen möchte, kann die Ebene der kleinsten nicht"=phonologischen Einheiten also nicht die der Wörter sein.
Es wäre nun denkbar, zunächst die Ebene der Wortbestandteile (als Morphologie) zu definieren, um dann darauf aufzubauen.

\Definition{Morphologie (Versuch)}{Die Morphologie ist die grammatische Ebene der kleinsten Einheiten, die nach eigenen, nicht-phonologischen Regularitäten kombiniert werden.
}

Damit hätten wir also die Ebene, die die Kombinierbarkeit von \textit{-es} mit \textit{Staat} und \textit{Tür} regelt.
Darauf könnte die nächste Definition aufgesetzt werden.

\Definition{Syntax (Versuch)}{Die Syntax ist die grammatische Strukturebene der kleinsten Einheiten, die nach eigenen, nicht-morphologischen Regularitäten kombiniert werden.
}

Das Wort könnte man nun als Einheit auf dieser Ebene verorten.

\Definition{Syntaktisches Wort (Versuch)}{
Ein syntaktisches Wort ist die kleinste grammatische Einheit, bezüglich derer auf der Ebene der Syntax kombinatorische Regularitäten beobachtet werden können.
}

Diese Definitionen sind mit zahlreichen Problemen behaftet, auf die nicht im Einzelnen eingegangen werden muss.
Vor allem aber verwäscht ihre Aussagekraft, je höher wir die Ebenen aufeinanderstapeln.
Trotzdem ist der formale Stil dieser tentativen Definitionen nicht von der Hand zu weisen.
Wörter sind (so wie Segmente, Silben, Wortbestandteile oder Sätze) in einer bestimmten formalen Schicht des Sprachsystems offensichtlich existent.
Es gibt zwar in gewissem Maß Interaktionen zwischen den Ebenen, aber man hat es trotzdem mit verschiedenen Gesetzmäßigkeiten zu tun.
Im nächsten Abschnitt wird deshalb argumentiert, dass eine pragmatische Festlegung dessen, was wir als Wort betrachten wollen, nicht notwendigerweise problematisch ist.

Wenn wir die weiter oben geleisteten Bemühungen um eine Definition des Wortes ansehen, werden wir feststellen, dass dort von Anfang an so argumentiert und definiert wurde, dass dem Autor offensichtlich genau klar war, was ein Wort ist oder sein soll.
Es sollte sozusagen eine exakte Definition für den Begriff des Wortes gefunden werden, wobei alle Beteiligten bereits wussten, was man unter einem Wort verstehen möchte.
Dies ist gut an den Formulierungen wie der folgenden zu erkennen:
`Da man \textit{-es} nicht gerne als Wort, sondern eher als Wortbestandteil bezeichnen möchte, kann die Ebene der kleinsten nicht"=phonologischen Einheiten also nicht die der Wörter sein.' (S.~\pageref{arbref:9234645}).
Ohne formal penibel Ebenen über Ebenen zu definieren, ist uns bei aufmerksamer Betrachtung relativ schnell klar, welche Einheiten nach ihren eigenen Regularitäten kombiniert werden.
Wir können also einfach diese Einheiten auflisten und ihr Verhalten beschreiben.

Auch wenn wir eine sehr formale Grammatik konstruieren oder auf Computern implementieren würden, müssten wir uns alle grundlegende Fragen (über \textit{das wahre Wesen der Wörter} usw.) nicht unbedingt stellen.
Man definiert dabei üblicherweise Listen von den Wörtern, also ein Lexikon.
Man weist diesen Wörtern Merkmale und Werte zu, und formuliert die Kombinationsregeln (die Syntax).
Solange das, was dabei herauskommt, die zu beschreibende Sprache erfolgreich nachbildet, gibt es keinen prinzipiellen Einwand gegen ein solch pragmatisches Vorgehen.
Nicht anders geht übrigens auch die angewandte Grammatik vor:
Anhand einer Liste von Wörtern (dem Wörterbuch) und einer Grammatik, die sich auf diese Liste bezieht, ist es im Prinzip möglich, eine Sprache zu lernen.%
\footnote{Selbstverständlich ist für ein flüssiges und idiomatisch gutes Sprechen sowie das Beherrschen von Gebrauchsbedingungen in einer Fremdsprache weit mehr erforderlich als eine Grammatik und ein Wörterbuch.
Große Teile der rein formalen Seite der Sprache sind aber mit den genannten Hilfsmitteln erlernbar.}
Kaum jemand, der ein Wörterbuch benutzt, wird dabei zuerst in der Einleitung nachlesen wollen, welche formale Definition des Wortes in diesem Wörterbuch zur Anwendung kommt.
Auf Basis dieser Nicht-Definition des Wortes können wir also trotzdem gut weiterarbeiten.
Im folgenden Abschnitt wird eine weitere Differenzierung im Bereich der Wörter eingeführt.

\subsection{Wörter und Wortformen}

\label{sec:woerterwortformen}

Das, was wir oben als \textit{syntaktisches Wort} bezeichnet haben, ist im Prinzip nicht das Wort, wie es im Lexikon abgelegt werden muss.
Nehmen wir wieder einige Wörter aus dem \textsc{Kasus}-\textsc{Numerus}-Paradigma.%
\footnote{Hier wird zur Verdeutlichung der altertümliche Dativ auf \textit{-e} angegeben.}

\begin{exe}
  \ex\label{ex:wa8817}
  \begin{xlist}
    \ex{(das) Kind = [\textsc{Genus}: \textit{neut}, \textsc{Kasus}: \textit{nom}, \textsc{Numerus}: \textit{sg}, \ldots]}
    \ex{(das) Kind = [\textsc{Genus}: \textit{neut}, \textsc{Kasus}: \textit{akk}, \textsc{Numerus}: \textit{sg}, \ldots]}
    \ex{(dem) Kinde = [\textsc{Genus}: \textit{neut}, \textsc{Kasus}: \textit{dat}, \textsc{Numerus}: \textit{sg}, \ldots]}
    \ex{(des) Kindes = [\textsc{Genus}: \textit{neut}, \textsc{Kasus}: \textit{gen}, \textsc{Numerus}: \textit{sg}, \ldots]}
    \ex{(die) Kinder = [\textsc{Genus}: \textit{neut}, \textsc{Kasus}: \textit{nom}, \textsc{Numerus}: \textit{pl}, \ldots]}
  \end{xlist}
\end{exe}

Die zu einem Paradigma gehörenden Formen haben sowohl eine Reihe von in ihrem Wert gleichbleibenden Merkmalen (hier \textsc{Genus}), aber auch eine Reihe von Merkmalen mit unterschiedlichen Werten (hier \textsc{Kasus} und \textsc{Numerus}).\index{Kasus}\index{Numerus}\index{Paradigma}
Durch beide Arten von Werten wird das syntaktische Verhalten der Wörter gesteuert.
Es gibt Kontexte (\textit{Syntagmen} im Sinne von Abschnitt~\ref{sec:paradigmasyntagma}), in denen jeweils nur eine Form des Paradigmas verwendet werden kann.\index{Syntagma}

\begin{exe}
  \ex\label{ex:wa8818}
  \begin{xlist}
    \ex{Das \_\_\_\ ist frei in seinen Entscheidungen.}
    \ex{Wir sehen das \_\_\_.}
    \ex{Glaube dem \_\_\_\ nicht.}
    \ex{Die Würde des \_\_\_\ ist unantastbar.}
    \ex{Das Beste im Leben sind leider immer noch die \_\_\_.}
  \end{xlist}
\end{exe}

Wenn diese Kontexte in (\ref{ex:wa8818}) mit einer Form aus (\ref{ex:wa8817}) ergänzt werden sollen, kommt jeweils nur eine infrage.
Bezüglich ihrer syntaktischen Kombinierbarkeit sind die Formen also durchaus verschieden, sie müssen demnach unterschiedliche syntaktische Wörter sein. 
Trotzdem wollen wir die Formen in (\ref{ex:wa8817}) als lexikalisch zusammengehörig beschreiben, also im Lexikon nur eine Repräsentation für alle diese Formen ablegen.
Dazu trennen wir den konkreten syntaktischen Wortbegriff vom abstrakteren lexikalischen Wortbegriff.

\Definition{Wortform (syntaktisches Wort)}{
\label{def:wortform}
Eine Wortform ist eine nicht weiter teilbare Einheit, wie sie in syntaktischen Strukturen vorkommt.
Die Werte der Merkmale von Wortformen sind gemäß ihrem Paradigma vollständig spezifiziert.
\index{Wort!syntaktisch}
}

Wortformen sind also all die (minimalen) Einheiten, die in syntaktischen Kontexten vorkommen.
Sie haben die nötigen Werte für ihre Merkmale und die dazu passende Form.
Das (lexikalische) Wort ist die Abstraktion davon.
Das ist vergleichbar mit der zugrundeliegenden Form der Phonologie (\ref{sec:segmentemerkmaleverteilungen}), die ebenfalls genau die Information enthält, die benötigt wird, um die phonetischen Realisierungen eines Segments in allen möglichen Kontexten abzuleiten.

\Enl[1]

\newcommand{\DefWort}{Das Wort ist die Repräsentation von paradigmatisch verbundenen Wortformen.
Beim Wort sind Werte nur für Merkmale spezifiziert, die in allen Wortformen des Paradigmas dieselben Werte haben.
Die restlichen Werte werden gemäß der Position im Paradigma bei den Wortformen gesetzt.}

\Definition{Wort (lexikalisches Wort)}{
\label{def:wort}
\DefWort
\index{Wort!lexikalisch}
}

Das lexikalische Wort -- oder einfach \textit{Wort} -- zu den Wortformen in (\ref{ex:wa8817}) wäre demnach die abstrakte Repräsentation, für die \zB der nicht veränderliche Teil der Formen (falls vorhanden) sowie die Bedeutung spezifiziert werden muss.
Zudem wären alle Merkmale (mit oder ohne Wert) angegeben, die zu Wörtern des Paradigmas gehören.
Werte für Merkmale dürfen beim lexikalischen Wort allerdings nur dann abgelegt werden, wenn sie in allen zugehörigen Wortformen gleich sind.

Die Repräsentation eines lexikalischen Wortes könnte also wie in (\ref{ex:wa4282}) aussehen.

\begin{exe}
  \ex{\label{ex:wa4282} Kind (lexikalisches Wort) =\\
  {}[\textsc{Segmente}: \textipa{/kInd/}, \textsc{Genus}: \textit{neut}, \textsc{Kasus}, \textsc{Numerus}, \ldots]}
\end{exe}

Die Merkmale \textsc{Kasus} und \textsc{Numerus} haben keine Werte, weil diese gemäß der Position im Paradigma angepasst werden. 
Es wird jetzt das Merkmal \textsc{Segmente} verwendet, um die zugrundeliegende phonologische Form des lexikalischen Wortes anzugeben.
Damit ist geklärt, was mit einer lexikalischen Wortklassifikation überhaupt klassifiziert werden soll.
Es sind nämlich Wörter, nicht etwa Wortformen.



\Zusammenfassung{
Den Wortbegriff aus ersten Anschauungen heraus zu definieren, ist vermutlich unmöglich und für die Grammatik nicht unbedingt nötig.
Phonologisch gesehen bestehen Wörter aus Segmenten bzw.\ Silben.
Es gibt aber strukturelle Prozesse in Wörtern (\zB die Bildung von \textit{Staat-es} zu \textit{Staat}), die nicht durch phonologische Regularitäten erklärbar sind.
Das \textit{lexikalische Wort} oder nur \textit{Wort} ist die lexikalische Abstraktion ggf.\ vieler möglicher \textit{Wortformen} oder \textit{syntaktischer Wörter}.
}


\section{Klassifikationsmethoden}

\label{sec:klassifikationsmethoden}

\subsection{Semantische Klassifikation}

\label{sec:semantischeklassifikation}

\index{Wortklasse!semantisch}

In der Grundschuldidaktik wird der Wortschatz gerne in Klassen wie \textit{Dingwort} bzw.\ \textit{Namenwort}, \textit{Tätigkeitswort} (oder gar \textit{Tuwort}), \textit{Eigenschaftswort} (oder \textit{Wiewort}) usw.\ eingeteilt.
Es ist relativ eindeutig, dass dabei \textit{Bedeutungsklassen} gebildet werden.
Es werden semantische Charakteristika der Wörter zu ihrer Definition herangezogen.
\textit{Dingwörter} bezeichnen Dinge, \textit{Tätigkeitswörter} bezeichnen Tätigkeiten, \textit{Eigenschaftswörter} bezeichnen Eigenschaften usw.
Wir müssen uns an dieser Stelle fragen, ob diese Art der Klassifikation zielführend ist, ob wir sie also übernehmen möchten.
Schon beim Dingwort könnten findige Schüler einwenden, dass Abstrakta wie \textit{Idee}, \textit{Angst}, \textit{Schuld} keine Dinge bezeichnen, aber in die Klasse der Dingwörter eingeordnet werden.

Beim \textit{Tätigkeitswort} ist es ebenso einfach, auf die Mängel der Definition hinzuweisen, wie an den Beispielen in (\ref{ex:wa8354}) gezeigt werden soll.

\begin{exe}
  \ex\label{ex:wa8354}
  \begin{xlist}
    \ex{\label{ex:wa-8354a} Simone schießt auf das Tor.}
    \ex{\label{ex:wa-8354b} Barbara schläft.}
    \ex{\label{ex:wa-8354c} Das Foulspiel durch Inka wurde nicht geahndet.}
  \end{xlist}
\end{exe}

Schon in (\ref{ex:wa-8354a}), einem wahrscheinlich gemeinhin für eindeutig gehaltenen Fall eines Tätigkeitswortes, könnten wir uns überlegen, ob wirklich das Verb (\textit{schießt}) die Tätigkeit bezeichnet, oder ob nicht vielmehr \textit{schießt auf das Tor} die Bezeichnung der Tätigkeit ist.
In Beispiel (\ref{ex:wa-8354b}) ist es angesichts des Verbs \textit{schläft} schwierig, von einer Tätigkeit zu sprechen, weil dem Schlaf eine für Tätigkeiten typische Komponente der Aktivität fehlt.
Völlig zusammenbrechen muss die semantische Definition der Tätigkeitswörter allerdings angesichts von (\ref{ex:wa-8354c}), weil hier das Substantiv (also das vermeintliche Dingwort) \textit{Foulspiel} offensichtlich eine Tätigkeit bzw.\ Handlung beschreibt, aber kein Verb ist.

Einige weitere der zahlreichen Probleme kann man an den sogenannten \textit{Eigenschaftswörtern} (also Adjektiven wie \textit{rot} oder \textit{schnell}) illustrieren.
Vielleicht kann man sagen, \textit{rot} (oder besser \textit{Rotsein}) bezeichne eine Eigenschaft.
Ist es aber nicht genauso eine Eigenschaft von Dingen, ein Fußball oder eine Eckfahne zu sein?
Noch weiter gedacht, sind es nicht ebenso Eigenschaften von Dingen, dass sie laufen, stehen, fliegen, spielen usw.?
Obwohl also die Definition des Eigenschaftswortes zunächst intuitiv plausibel erscheint, hängt sie doch davon ab, dass wir aus einem diffusen Grund in den zuletzt genannten Fällen (also bei Substantiven und Verben) nicht von Eigenschaften sprechen.
Als weiteres Problem sollen die Sätze in (\ref{ex:wa9292}) diskutiert werden.

\begin{exe}
  \ex \label{ex:wa9292}
  \begin{xlist}
    \ex{\label{ex:wa-9292a} Der schnelle Ball ging ins Netz.}
    \ex{\label{ex:wa-9292b} Der Ball ging schnell ins Netz.}
  \end{xlist}
\end{exe}

\index{Adjektiv}

Hier kommt zweimal das Adjektiv \textit{schnell} vor, einmal bezieht es sich aber auf das Substantiv \textit{Ball} (klassische adjektivische Verwendung), gibt also (wenn man so will) eine Eigenschaft an.
In (\ref{ex:wa-9292b}) allerdings bezieht es sich auf das Verb \textit{ging} (\textit{ins Netz}).
Von wem oder was beschreibt das Adjektiv hier aber eine Eigenschaft?
Oder ist es in diesem Fall doch kein Adjektiv?
Konsistente Antworten auf diese Fragen sind im Rahmen der semantischen Klassifikation mit Sicherheit nicht zu finden.

Abschließend sei noch auf Beispiel (\ref{ex:wa9191}) verwiesen.

\begin{exe}
  \ex{\label{ex:wa9191} Der ehemalige Trainer des FFC freut sich immer noch über jeden Sieg.}
\end{exe}

In diesem Satz ist \textit{ehemalige} zweifelsfrei ein Adjektiv, aber es bezeichnet kaum eine Eigenschaft.
Was genau mit \textit{ehemalige} hier gemeint ist, kann man erst in Zusammenhang mit dem Substantiv \textit{Trainer des FFC} überhaupt erschließen.
Selbst dann kann man aber nicht gut sagen, der Trainer des FFC habe die Eigenschaft der Ehemaligkeit.

Es sollte klar geworden sein, dass eine semantische Klassifizierung zu massiven Problemen führt, wenn die Kriterien für die Klassenzuordnung der Wörter präzise angegeben werden sollen.
Im nächsten Abschnitt wird deswegen eine andere Art der Klassifikation beschrieben.
Diese wird auch unserem Plan gerecht, dass Grammatik hier möglichst von ihrer formalen Seite und weitgehend ohne Berücksichtigung der Bedeutung betrachtet werden soll (vgl.\ Abschnitt~\ref{sec:sprachealssymbolsystem}).

\subsection{Paradigmatische Klassifikation}

\label{sec:paradigmatischeklassifikation}

Eine sehr exakte Unterscheidung von Wortklassen ist über die Zugehörigkeit zu morphologischen Paradigmen der Wörter möglich (vgl.\ Abschnitt~\ref{sec:paradigmasyntagma}).
Wörter, die in den gleichen Paradigmen stehen, gehören dabei zu einer Klasse.
Um dies wieder am Beispiel zu illustrieren, folgen (\ref{ex:wa9393}) bis (\ref{ex:wa9395}).

\begin{exe}
  \ex \label{ex:wa9393}
  \begin{xlist}
    \ex{Ich pfeife.\\
      Du pfeifst.\\
      Die Schiedsrichterin pfeift.}
    \ex{Ich schlafe.\\
      Du schläfst.\\
      Die Schiedsrichterin schläft.}
  \end{xlist}
  \ex \label{ex:wa9394}
  \begin{xlist}
    \ex{ein schneller Ball\\
      der schnelle Ball\\
      schneller Ball}
    \ex{ein leckerer Kuchen\\
      der leckere Kuchen\\
      leckerer Kuchen}
  \end{xlist}
  \ex \label{ex:wa9395}
  \begin{xlist}
    \ex{der Berg\\
      des Berges\\
      die Berge}
    \ex{der Mensch\\
      des Menschen\\
      die Menschen}
    \ex{der Staat\\
      des Staates\\
      die Staaten}
  \end{xlist}
\end{exe}

Die Beispiele illustrieren bestimmte Paradigmen.\index{Paradigma}
In (\ref{ex:wa9393}) ist es das Paradigma der (singularischen) Personalformen (\textit{ich}, \textit{du}, \textit{die Schiedsrichterin}/\textit{sie}) der Verben.\index{Verb}
In (\ref{ex:wa9394}) ist es ein spezielles Paradigma der Adjektive, bei dem sich die Formen abhängig von der Wahl des Artikels (\textit{ein}, \textit{der} bzw.\ kein Artikel) unterscheiden.\index{Adjektiv}
Schließlich wird in (\ref{ex:wa9395}) das Kasus-Numerus-Paradigma der Substantive (bzw.\ ein Ausschnitt daraus) illustriert.\index{Substantiv}
Mittels der in den Beispielen gezeigten unterschiedlichen Paradigmen könnten wir also bereits Verben, Adjektive und Substantive definitorisch voneinander abgrenzen.

Ein Sachverhalt bezüglich der Formen in Paradigmen sollte noch beachtet werden.
In zwei von drei Fällen gibt es bei den Wörtern in (\ref{ex:wa9393}) bis (\ref{ex:wa9395}) uneinheitliche Formenunterschiede.
Bei beiden Verben in (\ref{ex:wa9393}) sind zwar die Endungen dieselben (\textit{-e}, \textit{-st}, \textit{-t}).
Während sich aber der Bestandteil \textit{pfeif-} nicht ändert, ändert sich sehr wohl die Form von \textit{schlaf-} (erste Person) zu \textit{schläf-} (zweite und dritte Person).
Bei den Substantiven in (\ref{ex:wa9395}) ändern sich zwar die Bestandteile \textit{Berg-}, \textit{Mensch-} und \textit{Staat-} nicht, dafür sind aber die Endungen nicht einheitlich:
Beim Genitiv Singular (\textit{des Berg-es} usw.) kommen \textit{-es} und \textit{-en} vor, im Nominativ Plural (\textit{die Berg-e} usw.) finden wir \textit{-e} und \textit{-en}.
Die Paradigmen sind also nicht etwa bestimmte Formenreihen in dem Sinn, dass die Bildung der Formen innerhalb des Paradigmas immer mit denselben Mitteln geschieht.
Vielmehr sind sie Formenreihen in dem Sinn, dass die verschiedenen Formen des Paradigmas bestimmte Merkmalswerte aufweisen, wobei sich manchmal auch die Form ändert.
Mehr zu der Beziehung von formalen Mitteln und Merkmalen findet sich in Kapitel~\ref{sec:morphologie}.

\Np

\Satz{Formen im morphologischen Paradigma}{
Die Formänderungen in einem Paradigma müssen nicht bei allen Wörtern im Paradigma dem gleichen Muster folgen.
Die Merkmalszuweisungen sind aber einheitlich.
\index{Paradigma}
}

Man kann nun die paradigmatische Wortklassifkation in einem Satz zusammenfassen.

\Satz{Wortklassifikation nach morphologischen Paradigmen}{
Eine Wortklassifkation nach morphologischen Paradigmen weist Wörter Wortklassen zu, je nachdem in welchen morphologischen Paradigmen die Wörter vorkommen.
\index{Wortklasse!morphologisch}
}

\index{Paradigma}

Eine Einschränkung muss an dieser Stelle gemacht werden, auch um Beispiel (\ref{ex:wa-8354c}) von S.~\pageref{ex:wa-8354c} aus Abschnitt~\ref{sec:semantischeklassifikation} zu erläutern.
Sehen wir uns die Beispiele in (\ref{ex:wa001}) an.

\begin{exe}
  \ex\label{ex:wa001}\begin{xlist}
    \ex{Wir sind des Wanderns müde.}
    \ex{Sie wandern.}
  \end{xlist}
\end{exe}

Die beiden Wortformen \textit{Wanderns} und \textit{wandern} gehören offensichtlich in irgendeiner Art und Weise zusammen, was an der Bedeutung und der Form leicht abzulesen ist.
Außerdem können offensichtlich sehr viele Verben in einer Weise wie \textit{Wanderns} verwendet werden.
Man kann einfach \textit{Laufens}, \textit{Lachens}, \textit{Nachdenkens} usw.\ an Stelle von \textit{Wanderns} einsetzen, um dies nachzuvollziehen.
Trotzdem wäre es nicht angemessen, die Formen \textit{wandern} (eine Verbform) und \textit{Wanderns} (eine Substantivform) als Formen eines Paradigmas aufzufassen.
Wenn wir dies täten, könnten wir zwischen Verben und Substantiven nicht mehr eindeutig trennen, obwohl diese Trennung für unsere Grammatik essentiell ist.

Auch dieses Problem führt uns zurück zu Abschnitt~\ref{sec:paradigmasyntagma}.
Die Definition des Paradigmas und der (lexikalischen) Kategorie war an das Vorhandensein bestimmter Merkmale geknüpft.
Wortformen eines Paradigmas müssen in jedem Fall bestimmte Merkmale haben (bei den Substantiven \zB \textsc{Genus}).
Im Paradigma ändern sich dann für bestimmte Merkmale die Werte in systematischer Weise (\zB \textsc{Kasus} im Kasus-Paradigma der Substantive).
Die Formen \textit{wandern} und \textit{Wanderns} unterscheiden sich aber signifikant in ihrer grundlegenden Merkmalsausstattung.
\textit{Wanderns} hat typisch nominale Merkmale wie \textsc{Genus} und \textsc{Kasus}, die \textit{wandern} fehlen -- und umgekehrt.

\begin{exe}
  \ex{(wir) wandern = [\textsc{Temp}: \textit{präs}, \textsc{Mod}: \textit{ind}, \textsc{Per}: \textit{1}, \textsc{Num}: \textit{pl}, \ldots]}
  \ex{(des) Wanderns = [\textsc{Gen}: \textit{neut}, \textsc{Kas}: \textit{gen}, \textsc{Num}: \textit{sg}, \ldots]}
\end{exe}

\index{Wortbildung}\index{Flexion}

Die Beziehung zwischen den beiden Wörtern kann also eigentlich keine paradigmatische im engeren Sinne sein.
Trotzdem ist \textit{Wanderns} offensichtlich in irgendeiner Form von \textit{wandern} abgeleitet.
Ableitungen wie diese werden in Kapitel~\ref{sec:wortbildung} ausführlich besprochen.

Es ist also in vielen Fällen möglich, über einen genau eingegrenzten morphologischen Paradigmenbegriff Wörter in Klassen einzuteilen.
Allerdings sollen meist auch Wortklassen unterschieden werden, deren zugehörige Wörter in keinem morphologischen Paradigma stehen.
Weil sie sich im Satzkontext ganz anders verhalten, unterscheidet man zum Beispiel gerne Adverben wie \textit{möglicherweise} von Präpositionen wie \textit{durch} und Komplementierern wie \textit{dass}.
Sie alle stehen aber nicht in irgendeinem morphologischen Paradigma.
Für die Unterscheidung dieser Klassen müssen andere Kriterien gefunden werden.

\subsection{Syntagmatische Klassifikation}

\label{sec:syntagmatischeklassifikation}

Neben der paradigmatischen Klassifizierung kann die syntagmatische herangezogen werden, um Wörter zu klassifizieren.
Die Beispiele in (\ref{ex:wa6622}) und (\ref{ex:wa6623}) illustrieren das Prinzip.

\begin{exe}
  \ex\label{ex:wa6622}
  \begin{xlist}
    \ex[]{\label{ex:wa-6622a} Alexandra spielt schnell und präzise.}
    \ex[*]{\label{ex:wa-6622b} Alexandra spielt schnell obwohl präzise.}
    \ex[]{\label{ex:wa-6622c} Alexandra und Dzenifer spielen eine gute Saison.}
    \ex[*]{\label{ex:wa-6622d} Alexandra obwohl Dzenifer spielen eine gute Saison.}
  \end{xlist}
  \ex\label{ex:wa6623}
  \begin{xlist}
    \ex[]{\label{ex:wa-6623a} Alexandra spielt herausragend, obwohl der Leistungsdruck hoch ist.}
    \ex[*]{\label{ex:wa-6623b} Alexandra spielt herausragend, und der Leistungsdruck hoch ist.}
  \end{xlist}
\end{exe}

In diesen Beispielen geht es um die Wörter \textit{und} und \textit{obwohl}.
Beide sind in ihrer Form nicht veränderlich, und sie stehen in keinem morphologischen Paradigma.
Dies bedeutet, dass bei ihnen zwischen Wort und Wortform nur ein theoretischer, aber kein sichtbarer Unterschied besteht.
Trotzdem unterscheiden sie sich in der Art, wie sie in syntaktischen Strukturen verwendet werden.
In (\ref{ex:wa6622}) erkennt man, dass \textit{und} Wörter wie \textit{schnell} und \textit{präzise} oder \textit{Alexandra} und \textit{Dzenifer} verbinden kann, was mit dem Wort \textit{obwohl} nicht möglich ist.
In (\ref{ex:wa6623}) ist der umgekehrte Fall illustriert, nämlich dass \textit{obwohl} einen Nebensatz wie \textit{obwohl der Leistungsdruck hoch ist} einleiten kann, \textit{und} dies aber nicht kann.

Wichtig ist hier wiederum, nicht anzunehmen, es handle sich um einen reinen Effekt der Bedeutung.
Natürlich haben die Sätze in (\ref{ex:wa6622}) und (\ref{ex:wa6623}), die mit \Ast\ gekennzeichnet sind, keine rekonstruierbare Bedeutung.
Das ist allerdings bei (\ref{ex:wa6624}) auch der Fall.

\begin{exe}
  \ex\label{ex:wa6624}{Der Marmorkuchen spielt schnell und präzise.}
\end{exe}

Der Unterschied zwischen den nicht akzeptablen Sätzen in (\ref{ex:wa6622}) und (\ref{ex:wa6623}) auf der einen Seite und Satz (\ref{ex:wa6624}) auf der anderen Seite ist, dass (\ref{ex:wa-6622b}), (\ref{ex:wa-6622d}) und (\ref{ex:wa-6623b}) bereits auf der grammatischen Ebene scheitern, während (\ref{ex:wa6624}) grammatisch in Ordnung, aber auf der Bedeutungsebene schlecht ist.
Die Wörter sind in (\ref{ex:wa-6622b}), (\ref{ex:wa-6622d}) und (\ref{ex:wa-6623b}) zu einer Struktur zusammengefügt, die so niemals vorkommen würde.
Dass sie keine Bedeutung haben, ist eher eine Folge davon, dass sie grammatisch nicht in Ordnung sind.
Man kann also über die syntaktische Verteilung (\textit{Distribution}) diejenigen Wörter klassifizieren, die nicht in einem morphologischen Paradigma stehen.\index{Distribution}

\Satz{Wortklassifikation nach syntaktischer Verteilung}{Eine Wortklassifkation nach syntaktischer Verteilung weist Wörter Wortklassen zu, je nachdem, in welchen Positionen in syntaktischen Strukturen sie vorkommen können.
}

Im Prinzip sollen sich natürlich alle diese syntaktischen Eigenschaften der Wörter auch aus ihren Merkmalen und Werten ergeben, wobei hier nur nicht genug Raum bleibt, die entsprechenden Analysen konsequent durchzuführen.
Insofern ist jede Klassifikation von Wörtern letztlich eine Klassifikation nach Merkmalen und Werten.
Das morphologische und das syntaktische Kriterium benutzen wir im nächsten Abschnitt, um eine grobe Einteilung innerhalb des Lexikons vorzunehmen.


\Zusammenfassung{
Für den praktischen Gebrauch ist es schwierig, Wortklassen als \textit{Bedeutungsklassen} (semantisch) zu definieren.
Man kann Wörter erfolgreich danach in Klassen zusammenfassen, welchen Veränderungen von Merkmalswerten (und welchen Veränderungen von Formen) sie unterzogen werden (\textit{paradigmatische Klassifikation}).
Außerdem bietet sich an, Wörter danach zu klassifizieren, wie sie sich mit anderen Wörtern kombinieren lassen (\textit{syntagmatische Klassifikation}).
}


\section{Wortklassen des Deutschen}

\label{sec:wortklassendesdeutschen}

\subsection{Filtermethode}

\label{sec:filtermethode}

Es sollte bis hierher klar geworden sein, dass Wörter eine reiche Ausstattung mit Merkmalen haben, und dass abhängig von diesen Merkmalen auch ein vielfältiges paradigmatisches und syntaktisches Verhalten einhergeht.
Dies hat zur Folge, dass eine Klassifikation von Wörtern in große Klassen immer ein sehr grobkörniges Bild ergibt.
Wenn wir es auf die Spitze trieben, könnten wir sicherlich einige hundert Wortklassen definieren, da sich Wörter im Detail sehr individuell verhalten.
Allerdings ist der Nutzen von Wortklassen einerseits der, dass wichtige Generalisierungen für möglichst große Klassen von Wörtern formuliert werden können.
Es ist unstrittigerweise in vielen Kontexten zielführend, von \textit{den Verben} zu sprechen, und eventuelle Unterklassen außer Acht zu lassen.
Andererseits haben Wortklassen für den Menschen, der eine Grammatik oder eine grammatische Theorie anwendet, eine wichtige konzeptuelle Bedeutung.
Am deutlichsten wird diese beim Lernen einer Fremdsprache.
Wie sollte man eine Sprache lernen, wenn man von Anfang an jede kleinste grammatische Unterscheidung berücksichtigen würde?
Viel einfacher ist es, sich zunächst grobe Verallgemeinerungen einzuprägen und Details nach und nach zu lernen.
Durch die bewusst grobe Klassenbildung können wir später Generalisierungen effizient und elegant bezüglich ganzer Wortklassen beschreiben und alle Abweichungen oder Verfeinerungen, die sich für einzelne Wörter ergeben, als Ausnahme behandeln.
Hier wird eine von vielen möglichen Klassifikationen konstruiert.
Die Methode folgt dabei einem \textit{Filtermodell}, bei dem die Menge aller Wörter jeweils auf Basis eines einzigen definitorischen Kriteriums (das auf ein Wort entweder zutrifft oder nicht) in zwei Teile teilt.

Dieses Vorgehen ist an die Klassifikation von \citet{Engel09} angelehnt.
Im Unterschied zu \citet{Engel09} erlauben wir die mehrfache Klassifikation im Sinne einer Unterklassifikation bereits klassifizierter Wörter.
Dies hat zur Folge, dass für jeden Filter angegeben werden muss, auf welche Restmenge er anzuwenden ist.
Die Restmenge wird der \textit{Anwendungsbereich} genannt und steht in den Filter-Diagrammen ganz oben.
In den folgenden Abschnitten werden die Filter -- und damit die Wortklassen -- einzeln eingeführt und erläutert.
Definition~\ref{def:wfilter} fasst die Filtermethode zusammen.

\Definition{Wortklassenfilter}{
\label{def:wfilter}
Ein Wortklassenfilter ist eine Bedingung bezüglich des morphologischen (paradigmatischen) oder des syntaktischen Verhaltens von Wörtern, die auf jedes Wort entweder zutrifft oder nicht.
Anhand mehrerer Filter werden Wörter der Reihe nach in je zwei Klassen eingeteilt (\textit{Filter trifft zu} oder \textit{Filter trifft nicht zu}), die durch folgende Filter weiter klassifiziert werden können.
Damit ergibt sich eine hierarchische Gliederung des Lexikons.
}

\subsection{Flektierbare Wörter}

\label{sec:flektierbarewoerterwortklassen}

\index{Wort!flektierbar}

Um flektierbare und nicht flektierbare Wörter ging es bereits in Abschnitt~\ref{sec:kategorien}.
Dort wurde vorgeschlagen, das Lexikon grob danach zu teilen, ob die Wörter ein \textsc{Numerus}-Merkmal haben oder nicht, und so die flektierbaren von den nicht flektierbaren Wörtern zu trennen.
Das eigentliche Konzept eines flektierbaren Wortes ist normalerweise nicht, dass es ein \textsc{Numerus}-Merkmal hat, sondern dass es paradigmatische Änderungen seiner Werte erfährt, und zwar bis auf wenige Ausnahmen in Verbindung mit Änderungen seiner Form.\index{Numerus}
Allerdings ist \textit{Flektierbarkeit} an sich nicht als Teil der formalen Merkmale eines Wortes definierbar.
Im Deutschen haben aber alle flektierbaren Wörter ein \textsc{Numerus}-Merkmal.%
\footnote{Die Verben haben in ihren infiniten Formen kein \textsc{Numerus}-Merkmal, aber alle Verben (im Sinn \textit{lexikalischer Wörter}) können auch finit flektieren (s.\ Abschnitt~\ref{sec:finit}).}
Dass dies so ist, ist kein Zufall, sondern hat seine Wurzeln in den Kongruenzverhältnissen des Deutschen.
Kongruenz ist laut Abschnitt~\ref{sec:rektionkongruenz} eine Übereinstimmung der Werte von Merkmalen bestimmter Einheiten in einer Struktur.
In Strukturen mit einem finiten Verb und einem von diesem regierten Nominativ herrscht Person- und Numerus-Kongruenz, und innerhalb einer zusammengehörenden Gruppe aus Nomina wie \textit{dieser leckere Keks} oder \textit{diese leckeren Kekse} herrscht Numeruskongruenz.%
\footnote{Da Nomina in der ersten und zweiten Person immer Pronomina sind, die nur alleine auftreten (\textit{ich}, \textit{du} usw.), ist Person-Kongruenz innerhalb von nominalen Gruppen kein sichtbares Phänomen.
Vgl.\ auch Abschnitt~\ref{sec:person}.}
Es folgt, dass sowohl Verben als auch Nomina eine Singular- und eine Pluralform haben müssen, um überhaupt kongruieren zu können.
In Abschnitt~\ref{sec:numerus} wird argumentiert, dass die Unterscheidung von Singular und Plural semantisch bei den Nomina motiviert ist.
Die Kongruenz innerhalb der nominalen Gruppen und ihre Kongruenz mit dem Verb sind Mittel, um die Satzstruktur besser zu markieren.
Filter~\ref{wfilt:flektierbare} wird entsprechend formuliert.

\WFiltTree{Flektierbares und unflektierbares Wort}{wfilt:flektierbare}{Wort}{Hat das Wort potentiell ein \textsc{Numerus}-Merkmal?}{flektierbar}{unflektierbar}

\subsection{Verben und Nomina}

\label{sec:verbennominawortklassen}

\index{Verb}
\index{Nomen}

Verben und Nomina haben zwar beide die Merkmale \textsc{Numerus} und \textsc{Person}, aber ansonsten durchaus unterschiedliche Merkmalsausstattungen.
Verben haben keinen Kasus und kein Genus, Nomina kein Tempus und keinen Modus (Indikativ oder Konjunktiv).
Wir führen den Begriff der \textit{Finitheit} ein, den wir später in Kapitel~\ref{sec:verben} noch benötigen, und knüpfen ihn an das \textsc{Tempus}-Merkmal.
Zwar könnte man sich genausogut auf \textsc{Modus} beziehen, weil beide immer zusammen auftreten, aber eine hinreichende Definition lässt sich auch mit nur einem der beiden Merkmale geben.
In Kapitel~\ref{sec:verben} wird ausführlich die Funktion von Tempus (und Modus) eingeführt.
Außerdem werden Gründe dafür genannt, dass im Deutschen nur \textit{Präsens} (eigentlich ohne Gegenwartsbezug, aber trotzdem oft \textit{Gegenwartsform} genannt) und \textit{Präteritum} (mit Vergangenheitsbezug) Tempusformen im eigentlichen Sinn darstellen.
In (\ref{ex:wk9123}) je ein Beispiel für Präsens und Präteritum gegeben.

\begin{exe}
  \ex\label{ex:wk9123}
  \begin{xlist}
    \ex{Barbara läuft. (Präsens)}
    \ex{Barbara lief. (Präteritum)}
  \end{xlist}
\end{exe}

\Definition{Finitheit}{
\label{def:finitheit}
Ein Verb ist finit flektiert, wenn es ein Merkmal \textsc{Tempus} hat, und infinit flektiert, wenn es keins hat.\index{Tempus}
\index{Finitheit}
}

\WFiltTree{Verb und Nomen}{wfilt:verbennomina}{flektierbares Wort}{Kann das Wort im finiten Paradigma stehen?}{Verb}{Nomen}

\subsection{Substantive}

\label{sec:substantivewortklassen}

Der Begriff \textit{Nomen} wird hier als Oberbegriff verwendet, der \textit{Substantive}, \textit{Adjektive}, \textit{Artikel} und \textit{Pronomina} umfasst.
In anderen Traditionen steht \textit{Nomen} für \textit{Substantiv}, also nur für die oft sogenannten \textit{Hauptwörter}.
Filter~\ref{wfilt:subst} (genau wie Filter~\ref{wfilt:adjartpron} in Abschnitt~\ref{sec:adjektivewortklassen}) hat die Funktion, innerhalb der Oberklasse der Nomina weiter zu gliedern.
Das Substantiv ist leicht als der lexikalische Träger des \textsc{Genus}-Merkmals zu identifizieren, das bei ihm nicht paradigmatisch variiert.

\index{Substantiv}
\WFiltTree{Substantiv}{wfilt:subst}{Nomen}{Hat das Nomen einen festen Wert für sein \textsc{Genus}-Merkmal?}{Substantiv}{anderes Nomen}

Der unveränderliche Wert für \textsc{Genus} bei Substantiven ist einfach zu illustrieren.\index{Genus}
In (\ref{ex:wk8103456}) ändern die Adjektive (\textit{stark}) und die Artikel (\textit{die}, \textit{der}, \textit{das}) jeweils ihren \textsc{Genus}-Wert (und dabei auch ihre Form) abhängig vom Substantiv (\textit{Gewichtheberin}, \textit{Versuch}, \textit{Gewicht}).
Artikelwörter und Adjektive kongruieren also nur mit dem Substantiv in ihrem Genus.
Der Wert des Merkmals \textsc{Genus} ist damit beim Substantiv fest, bei den anderen Nomina aber nicht.
Pronomina haben normalerweise verschiedene Genus-Formen wie \textit{dieser}, \textit{dieses} und \textit{diese}.

\begin{exe}
  \ex\label{ex:wk8103456}
  \begin{xlist}
    \ex{Die stärkste Gewichtheberin wurde Weltmeisterin.}
    \ex{Der stärkste Versuch war der zweite.}
    \ex{Das schwerste Gewicht wurde von Tatjana gerissen.}
  \end{xlist}
\end{exe}

\subsection{Adjektive}

\label{sec:adjektivewortklassen}

In den folgenden Sätzen finden wir jeweils das gleiche Substantiv und das gleiche davorstehende Adjektiv (\textit{groß}).
Dennoch ändert sich die Form der Adjektive, je nachdem, ob ein Artikel davor steht, bzw.\ welcher Artikel es ist.

\begin{exe}
  \ex
  \begin{xlist}
    \ex{Kein großer Ball wurde gespielt.}
    \ex{Der große Ball wurde gespielt.}
  \end{xlist}
  \ex
  \begin{xlist}
    \ex{Keine großen Bälle wurden gespielt.}
    \ex{Die großen Bälle wurden gespielt.}
    \ex{Große Bälle wurden gespielt.}
  \end{xlist}
\end{exe}

Man spricht hier vom \textit{Stärkeparadigma} der Adjektive.
Man kann diese Formen sehr umständlich als Raster mit insgesamt 48 Formen beschreiben, aber eigentlich sind die verschiedenen Stärkeformen recht einfach verteilt (Abschnitt~\ref{sec:adjektivflexion}).

Der Filter trennt diejenigen nicht-substantivischen Nomina, die diesem speziellen Stärkeparadigma folgen (also Adjektive) von den verbleibenden Nomina.
Die verbleibenden Nomina sind genau diejenigen, die syntaktisch noch vor der Gruppe aus Adjektiv und Substantiv stehen können, nämlich Artikel und Pronomina.
Dadurch erklärt sich die etwas umständliche Klausel in der Formulierung des Filters \textit{in Abhängigkeit von davor stehenden anderen nicht-substantivischen Nomina}.
Die Artikel und Pronomina werden erst in Abschnitt~\ref{sec:artikelpronomen} voneinander getrennt.

\Np

\index{Adjektiv}
\index{Pronomen}
\index{Stärke!Adjektiv}
\WFiltTree{Adjektiv}{wfilt:adjartpron}{anderes Nomen\\ nach Filter \ref{wfilt:subst}}{Flektiert das Nomen nach dem Stärkeparadigma?}{Adjektiv}{Artikel\slash Pronomen}

\subsection{Präpositionen}

\label{sec:praepositionenwortklassen}

\index{Präposition}

Mit der Abgrenzung der Präpositionen beginnt die Unterklassifizierung der nichtflektierbaren Wörter. 
Bereits in Kapitel~\ref{sec:grundbegriffe} haben wir Valenz und Rektion definiert und dabei an Verben illustriert.\index{Valenz}
Dass auch Präpositionen Valenz und Rektion haben, kann man an den folgenden Sätzen leicht sehen.

\begin{exe}
  \ex
  \begin{xlist}
    \ex{Mit dem kaputten Rasen ist nichts mehr anzufangen.}
    \ex{Angesichts des kaputten Rasens wurde das Spiel abgesagt.}
  \end{xlist}
\end{exe}

Welchen Wert für \textsc{Kasus} das Substantiv \textit{Rasen} (und die mit ihm zusammenhängenden Nomina wie Adjektive und Artikel) haben, hängt hier von der Präposition ab, die davorsteht.
Die Präposition \textit{mit} regiert den Dativ, \textit{angesichts} regiert den Genitiv.
Keine andere Art von unflektierbaren Wörtern verhält sich so, und Filter~\ref{wfilt:prep} bezieht sich daher auf dieses Verhalten.

\Np

\WFiltTree{Präposition}{wfilt:prep}{nicht flektier-\\bares Wort}{Hat das Wort eine einstellige Valenz und Kasusrektion?}{Präposition}{anderes}

\subsection{Komplementierer}

\label{sec:komplementiererwortklassen}

\index{Komplementierer}

Der nächste Filter verlangt nach einer Definition des Nebensatz-Begriffs, auch wenn ausführlich über Nebensätze erst in Kapitel~\ref{sec:saetze} gesprochen wird.
Definition~\ref{def:nebensatz} ist also als vorläufig zu betrachten.

\Definition{Nebensatz}{
\label{def:nebensatz}
Ein Nebensatz ist eine syntaktische Struktur, die ein finites Verb enthält, das an letzter Stelle steht, innerhalb derer typischerweise alle Ergänzungen und Angaben dieses Verbes enthalten sind und die syntaktisch abhängig ist (nicht alleine stehen kann).
\index{Nebensatz}
}

Hier einige Beispiele, um die Definition zu illustrieren.
Die Nebensätze sind dabei in [~] gesetzt.

\Enl

\begin{exe}
  \ex[]{\label{ex:wa9911} Ich glaube, [dass dieser Nebensatz ein Verb enthält].}
  \ex[]{\label{ex:wa9912}{} [Während die Spielzeit läuft], zählt jedes Tor.}
  \ex[]{\label{ex:wa9913} Es fällt ihnen schwer [zu laufen].}
  \ex[*]{\label{ex:wa9914} [Obwohl kein Tor fiel].}
\end{exe}

In (\ref{ex:wa9911}) ist die Definition des Nebensatzes erfüllt, weil \textit{enthält} ein \textsc{Tempus}-Merkmal hat und damit finit ist.
Außerdem sind alle Ergänzungen des Verbs (der Nominativ \textit{dieser Nebensatz} und der Akkusativ \textit{ein Verb}) enthalten.
In (\ref{ex:wa9912}) ist es ähnlich.
In (\ref{ex:wa9913}) hingegen ist \textit{zu laufen} ein Infinitiv (ohne Tempusflexion) und ist daher nicht finit.
Die Struktur \textit{zu laufen} kann daher nach der hier vertretenen Definition kein Nebensatz sein.%
\footnote{Es gibt auch andere Ansätze, in denen \textit{zu}-Infinitive wie Nebensätze behandelt werden.
Ein guter Grund dafür ist, dass sich diese Infinitive im übergeordneten Satz relativ frei verhalten und dabei ähnliche grammatische Funktionen wie Nebensätze haben können.
Das wird in Abschnitt~\ref{sec:kohaerenz} besprochen.}
(\ref{ex:wa9914}) demonstriert schließlich, dass ein Nebensatz wie \textit{obwohl kein Tor fiel} normalerweise nicht alleine stehen kann.
Nebensatzeinleiter sind laut Filter~\ref{wfilt:subjunktion} \textit{Komplementierer} und werden auch \textit{subordinierende Konjunktionen} oder \textit{Subjunktoren} genannt.

\WFiltTree{Komplementierer}{wfilt:subjunktion}{anderes aus Filter \ref{wfilt:prep}}{Kann das Wort einen Nebensatz einleiten?}{Komplementierer}{Partikel\slash Adverb}

Weil \textit{dass} in (\ref{ex:wa9911}) und (\ref{ex:wa9912}) Nebensatzstrukturen einleitet, ist es gemäß Filter \ref{wfilt:subjunktion} ein Komplementierer.
In (\ref{ex:wa9914}) kommt \textit{obwohl} als Komplementierer vor, auch wenn insgesamt die Struktur nicht grammatisch ist.
In (\ref{ex:wa9913}) ist \textit{zu} kein Komplementierer, weil die eingeleitete Struktur nicht Definition~\ref{def:nebensatz} erfüllt.

\subsection{Adverben und Partikeln}

\label{sec:adverbenpartikelnwortklassen}

\index{Partikel}

Die Unterscheidung von Adverben und Partikeln ist eine delikate Angelegenheit.
Syntaktisch gesehen sind Adverben flexibler im Satz positionierbar als Partikeln. 
Dazu definieren wir zuerst den Begriff \textit{Vorfeldbesetzer} bzw.\ \textit{Vorfeldfähigkeit}.\index{Vorfeld}
Warum man hier den Terminus \textit{Vorfeld} benutzt, wird in Kapitel~\ref{sec:saetze} genauer erklärt.
Auch ohne die zugehörigen Konzepte genau zu kennen, kann das Kriterium aber trotzdem definiert werden.

\Definition{Vorfeldbesetzer und Vorfeldfähigkeit}{
\label{def:vorfeldfaehig}
Vorfeldbesetzer sind Wörter, die einen unabhängigen Aussagesatz einleiten und dabei alleine vor dem finiten Verb stehen können.
\index{Vorfeld!Fähigkeit}
}

Auf die in den folgenden Sätzen am Satzanfang stehenden Wörter trifft diese Definition offensichtlich jeweils zu oder nicht zu.
Das \textit{doch} in Satz (\ref{ex:wa2829d}) soll dabei verstanden werden wie das nicht betonbare \textit{doch} in (\ref{ex:wa282999}).

\begin{exe}
  \ex\label{ex:wa2829}
  \begin{xlist}
    \ex[]{Gestern hat der FCR Duisburg gewonnen.}
    \ex[]{Erfreulicherweise hat der FCR Duisburg gestern gewonnen.}
    \ex[]{Oben finden wir andere Beispiele.}
    \ex[*]{Doch ist das aber nicht das Ende der Saison.\label{ex:wa2829d}}
    \ex[*]{Und ist die Saison zuende.}
  \end{xlist}
  \ex\label{ex:wa282999} Das ist aber doch nicht das Ende der Saison.
\end{exe}

Die Beispiele in (\ref{ex:wa2829}) sind hilfreich für die Klassifizierung der in ihnen vorkommenden satzeinleitenden Wörter.
\textit{Gestern}, \textit{erfreulicherweise} und \textit{oben} sind gemäß Filter \ref{wfilt:advpart} Adverben, \textit{doch} und \textit{und} jedoch Partikeln.

\Np

\WFiltTree{Adverb}{wfilt:advpart}{Partikel\slash Adverb}{Ist das Wort ein Vorfeldbesetzer?}{Adverb}{Partikel}

\subsection{Kopulapartikeln}

\label{sec:kopulapartikelwortklassen}

\index{Adverb}
\index{Kopulapartikel}

Die Beispielsätze in (\ref{ex:wa2220}) zeigen Kopulapartikeln, die jeweils mit einem sogenannten \textit{Kopulaverb} (KoV) wie \textit{sein}, \textit{bleiben} oder \textit{werden} auftreten.\index{Kopula}

\begin{exe}
  \ex\label{ex:wa2220}
  \begin{xlist}
    \ex Hamlet ist meschugge.
    \ex Quitt bin ich mit dir noch lange nicht.
  \end{xlist}
\end{exe}

Man kann diese Wörter auch als \textit{nur prädikativ verwendbare Adjektive} bezeichnen.
Adjektive können dieselbe, aber eben auch andere Positionen im Satz einnehmen wie Kopulapartikeln, vgl.\ (\ref{ex:wa0815}) und (\ref{ex:wa0816}).
Den Kopulapartikeln fehlt damit jegliche Flektierbarkeit, was der Grund für die hier vertretene Einordnung als eigene Klasse und eben nicht als Adjektiv ist.

\begin{exe}
  \ex\label{ex:wa0815}
  \begin{xlist}
  	\ex[ ]{\label{ex:wa0815a} Tatjana ist stark.}
  	\ex[ ]{\label{ex:wa0815b} Die starke Tatjana ist Weltmeisterin.}
  \end{xlist}
  \ex\label{ex:wa0816}
  \begin{xlist}
  	\ex[ ]{\label{ex:wa0816a} Der Staat ist pleite.}
  	\ex[*]{\label{ex:wa0816b} Der pleite Staat.}
  \end{xlist}
\end{exe}

\WFiltTree{Kopulapartikel}{wfilt:advkopulaele}{Partikel}{Tritt das Wort prototypisch mit einem Kopulaverb auf?}{Kopulapartikel}{anderes}

\subsection{Satzäquivalente}

\label{sec:satzaequivalentewortklassen}

\index{Satzäquivalent}

Mit den Satzäquivalenten sind wir bei einer relativ kleinen Klasse angekommen, die wahrscheinlich eher für gesprochene Sprache -- zumindest aber für dialogische Sprache -- typisch ist.
Wörter, die traditionell auch als \textit{Interjektionen} bezeichnet werden, gehören in die Gruppe der Satzäquivalente, also \textit{Ja!} oder \textit{Ohje!}

\WFiltTree{Satzäquivalent}{wfilt:satzaeq}{anderes aus Filter \ref{wfilt:advkopulaele}}{Ist die Partikel wie ein Satz unabhängig verwendbar?}{Satzäquivalent}{anderes}

\subsection{Konjunktionen}

\label{sec:konjunktionenwortklassen}

\index{Konjunktion}

Wie in Kapitel~\ref{sec:konstituentenstruktur} und Kapitel~\ref{sec:phrasen} ausführlich gezeigt wird, können Wörter wie \textit{und} oder \textit{oder} jede Art von syntaktischer Konstituente verbinden (bis auf einige Partikeln).
Das Ergebnis der Verbindung verhält sich syntaktisch genauso, wie sich auch die verbundenen Konstituenten verhalten.
Einige Beispiele sind in (\ref{ex:wa7392}) angegeben, die verbundenen Konstituenten stehen jeweils in [~].
Bei den verbindenden Wörtern spricht man von \textit{Konjunktionen} (Filter~\ref{wfilt:konjpart}), traditionell auch von \textit{koordinierenden Konjunktionen}.

\begin{exe}
  \ex\label{ex:wa7392} 
  \begin{xlist}
    \ex{[Dzsenifer] und [eine andere Spielerin] haben Tore geschossen.}
    \ex{Sätze können wir [aufschreiben] oder [laut aussprechen].}
    \ex{Spielt bitte [konzentriert] und [offensiv].}
  \end{xlist}
\end{exe}

In der übrig bleibenden Kategorie der restlichen Partikeln finden sich jetzt Wörter wie \textit{wie}, \textit{als}, \textit{eben} oder \textit{doch}.
Auch diese verhalten sich unterschiedlich, aber eine Restmenge bleibt realistisch gesehen immer, und für eine grobe Einteilung können wir an dieser Stelle die Klassifikation abschließen.

\WFiltTree{Konjunktion}{wfilt:konjpart}{anderes aus Filter \ref{wfilt:satzaeq}}{Kann die Partikel zwei gleichartige Konstituenten verbinden?}{Konjunktion}{andere Partikel}

\subsection{Gesamtübersicht}

\label{sec:gesamtuebersichtwortklassen}

In Abbildung~\ref{fig:wa} wird die Klassifikation anhand der Filter zusammengefasst.
Zu beachten ist, dass diese Klassifikation weder die einzige noch die in einem absoluten Sinn richtige ist.
Jede Klassifikation von Wörtern ist, wie eingangs schon erwähnt, ein Kompromiss zwischen Genauigkeit und Brauchbarkeit.
Im Wesentlichen leistet unsere Klassifikation aber eine Rekonstruktion der traditionellen Wortarten auf Basis einer einigermaßen genauen definitorischen Basis.

\newcommand{\Bjn}{\B{dl}_{\textnormal{Ja}}\B{dr}^{\textnormal{Nein}}}
\newcommand{\Bjnd}{\B{dll}_{\textnormal{Ja}}\B{drr}^{\textnormal{Nein}}}

\begin{figure}[!htbp]
  \centering
  \resizebox{\textwidth}{!}{
    \Treek[0.5]{2.5}{
    &&&& \K{\textbf{Wort}}\B{d} \\
    &&&& \K{(F1) hat \textsc{Numerus}?}\B{dlll}_{\textnormal{Ja}}\B{drrr}^{\textnormal{Nein}} \\
    & \K{flektierbar}\B{d} &&&&&& \K{nicht flektierbar}\B{d} \\
    & \K{(F2) finit flektierbar?}\Bjn &&&&&& \K{(F5) Valenz+Kasusrektion?}\Bjn \\
    \K{\textbf{Verb}} && \K{Nomen}\B{d} &&&& \K{\textbf{Präposition}} && \K{andere}\B{d} \\
    && \K{(F3) \textsc{Genus} fest?}\Bjn &&&&&& \K{(F6) nebensatzeinleitend?}\Bjn \\
    & \K{\textbf{Substantiv}} && \K{andere}\B{d} &&&& \K{\textbf{Komplemen-}}\Below{\textbf{tierer}} && \K{Partikel/Adverb}\B{d} \\
    &&& \K{(F4) Stärkeflexion?}\Bjn &&&&&& \K{(F7) Vorfeldbesetzer?}\Bjnd \\
    && \K{\textbf{Adjektiv}} && \K{\textbf{Artikel\slash}}\Below{\textbf{Pronomen}} &&& \K{Adverb\slash Kopulapartikel}\B{d} &&&& \K{Partikel}\B{d} \\
    &&&&&&& \K{(F8) immer mit KoV?}\Bjn &&&& \K{(F9) satzersetzend?}\Bjn \\
    &&&&&& \K{\textbf{Kopulapartikel}} && \K{\textbf{Adverb}} && \K{\textbf{Satzäquivalent}} && \K{andere}\B{d} \\
    &&&&&&&&&&&& \K{(F10) Konstituentenverbinder?}\Bjn \\
    &&&&&&&&&&& \K{\textbf{Konjunktion}} && \K{\textbf{Rest}}
    }
  }
  \caption{Entscheidungsbaum für die Wortklassen}
  \label{fig:wa}
\end{figure}


\Zusammenfassung{
  \textit{Flektierbare Wörter} im Deutschen haben immer ein \textsc{Numerus}-Merkmal.
  \textit{Nomen} ist ein Oberbegriff für \textit{Substantive}, \textit{Adjektive}, \textit{Artikel} und \textit{Pronomina}.
  Nur Verben haben ein \textsc{Tempus}-Merkmal.
  \textit{Adverben} können im \textit{Vorfeld} stehen, \textit{Partikeln} nicht.
  \textit{Konjunktionen} und \textit{Komplementierer} bilden zwei völlig verschiedene Klassen, anders als die traditionelle Rede von den \textit{beiordnenden} und \textit{nebenordnenden Konjunktionen} suggeriert.
}


\Uebungen

\Uebung \label{u51} Überlegen Sie, wie gut die folgenden Wortklassen semantisch definierbar wären:

\begin{enumerate}\Lf
  \item Präpositionen: \textit{mit}, \textit{an}, \textit{neben} usw.
  \item Komplementierer: \textit{während}, \textit{obwohl}, \textit{dass}, \textit{ob} usw.
  \item Adverben: \textit{schnell}, \textit{gestern}, \textit{bedauerlicherweise}, \textit{oben} usw.
\end{enumerate}

\Uebung[\tristar] \label{u52} Im Folgenden finden Sie Wörter, die sich in ihrem morphologischen oder syntaktischen Verhalten wesentlich unterscheiden, obwohl wir sie in eine Klasse einsortiert haben.
Suchen Sie nach syntaktischen Kriterien, diese Wörter zu unterscheiden (also die Klassifikation zu verfeinern).

\begin{enumerate}\Lf
  \item Adverben: \textit{quitt/meschugge} ggü. \textit{gerne}
  \item Adverben: \textit{erfreulicherweise} ggü. \textit{gerne}
  \item Artikel/Pronomen: \textit{ich/du/\ldots} ggü. \textit{der/das/die}
  \item Artikel/Pronomen: \textit{kein/keine} ggü. \textit{dieser/dieses/diese}
\end{enumerate}

Tipp zu \textit{erfreulicherweise} und \textit{gerne}:
Prüfen Sie, wie gut die Wörter als Antworten auf Fragen fungieren können.

\Uebung[\tristar] \label{u53} Überlegen Sie, was die syntaktischen Verwendungsbesonderheiten der folgenden Wörter ist.

\begin{enumerate}\Lf
  \item statt
  \item außer, bis auf
  \item wie, als
\end{enumerate}

\Uebung \label{u54} Wörter können verschiedene Bedeutungen haben, obwohl sie die gleiche Form haben (\zB \textit{Bank}).
Natürlich kommt es auch vor, dass gleichlautende Wörter, die semantisch oder funktional verschieden sind, auch in verschiedene Wortklassen einzuordnen sind.
Finden Sie Verwendungen/Beispiele von \textit{eben} als (1) Adjektiv, (2) Adverb, (3) Partikel.
Finden Sie jeweils ein anderes Wort, das \textit{eben} (nur) in dieser Klassenzugehörigkeit ersetzen kann.

Setzen Sie außerdem die Partikel \textit{eben} in die folgenden Muster ein und finden Sie zwei andere Partikeln, die \textit{eben} jeweils nur in genau einem dieser Kontexte ersetzen können.

\begin{exe}
  \ex{Und \_ dieser Test hat die Studierenden so verwirrt.}
  \ex{Diese Tests sind \_ schwierig.}
\end{exe}

\Uebung \label{u55} Wenden Sie die Filter auf die Wörter der folgenden Wortformen an und klassifizieren Sie sie.
Interpretieren Sie die hier gegebene Form jeweils als die Nennform bzw.\ die Form, wie sie im Wörterbuch stehen würde.
Rechnen Sie damit, dass einige Wörter mehrfach klassifiziert werden müssen.

\begin{enumerate}\Lf
  \item reihenweise
  \item Trikot
  \item während
  \item etwas
  \item aber
  \item rennen
  \item hallo
  \item mit
  \item erstaunt
  \item Abseits
  \item ob
  \item abseits
  \item jedoch
  \item rötlich
  \item es
  \item lediglich
  \item durch
  \item einzelnen
  \item gelungen
  \item damit
  \item etwa
  \item unsererseits
  \item gewann
  \item Gewand
  \item nicht
  \item mitnichten
\end{enumerate}

\Np

\Uebung \label{u56} Bestimmen Sie die Wortklassen der Wörter in folgendem Text.%
\footnote{\url{https://de.wikipedia.org/wiki/Kritischer_Rationalismus}}

\begin{quote}\small
  Der Kritische Rationalismus setzt sich mit der Frage auseinander, wie wissenschaftliche oder gesellschaftliche (aber prinzipiell auch alltägliche) Probleme undogmatisch, planmäßig (\textit{methodisch}) und vernünftig (\textit{rational}) untersucht und geklärt werden können. Dabei sucht er nach einem Ausweg aus der Wahl zwischen Wissenschaftsgläubigkeit (Szientismus) und der Auffassung, dass wissenschaftliches Wissen auf positiven Befunden aufbauen muss (Positivismus) auf der einen Seite, sowie andererseits dem Standpunkt, dass Wahrheit vom Blickwinkel abhängig ist (Relativismus) und dass Wissen der Willkür preisgegeben ist, wenn Beweise unmöglich sind (Wahrheitsskeptizismus).

Der Kritische Rationalismus übernimmt die im Alltagsverstand selbstverständliche Überzeugung, dass es die Welt wirklich gibt, und dass sie vom menschlichen Erkenntnis­vermögen unabhängig ist. Das bedeutet beispielsweise, dass sie nicht zu existieren aufhört, wenn man die Augen schließt. Der Mensch aber ist in seiner Erkenntnisfähigkeit dieser Welt durch seine Wahrnehmung begrenzt, so dass er sich keine endgültige Gewissheit darüber verschaffen kann, dass seine Erfahrungen und Meinungen mit der tatsächlichen Wirklichkeit übereinstimmen (Kritischer Realismus). Er muss daher davon ausgehen, dass jeder seiner Problemlösungsversuche falsch sein kann (Fallibilismus). Das Bewusstsein der Fehlbarkeit führt einerseits zu der Forderung nach der ständigen kritischen Prüfung von Überzeugungen und Annahmen, andererseits zum methodischen und rationalen Vorgehen bei der Lösung von Problemen (Methodischer Rationalismus).
\end{quote}
