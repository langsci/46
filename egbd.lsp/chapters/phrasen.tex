\chapter{Phrasen}

\label{sec:phrasen}

\index{Phrasenschema}
\index{Baumdiagramm}

Dieses Kapitel ist sehr einfach aufgebaut.
Es wird für die wichtigen Phrasentypen des Deutschen der jeweilige Aufbau besprochen und das allgemeine Phrasenschema explizit gegeben.
Die Bäume werden dabei prinzipiell so angefertigt, dass die Wortklassen der Wörter als erste Analyseebene eingefügt werden.
Dann wird nach den vorhandenen Phrasenschemata eine Phrasenanalyse von unten nach oben aufgebaut.
Die Wortklassen und alle Merkmale eines Wortes bestimmen sein syntaktisches Verhalten, und daher ist das Herausfinden der Wortklasse aller Wörter eines Satzes Grundvoraussetzung für seine Analyse.
Der Baum baut sich über dem Kopf gerade nach oben auf, zu ihm dependente Konstituenten werden seitlich hinzugefügt.
Insgesamt kann so in der praktischen Analyse immer mit einem linear aufgeschriebenen Satz begonnen werden, über dem dann der Baum konstruiert wird.
In Abschnitt~\ref{sec:syntaxana} wird zum Schluss dieses Kapitels noch eine kleine Anleitung gegeben, wie Analysen von Konstituentenstrukturen in diesem Sinn systematisch konstruiert werden können.

\begin{figure}
  \centering
  \Tree[1.5]{
    \K{PP}\B{dd}\B{drr} &&&& \\
    && \K{NP}\B{dl}\B{d} && \K{PP}\B{d}\B{dr} \\
    \K{\textbf{P}}\B{d} & \K{Art}\B{d} & \K{\textbf{N}}\B{d} && \K{\textbf{P}}\B{d} & \K{NP}\TRi[-9]\\
    \K{\textit{für}} & \K{\textit{einen}} & \K{\textit{Keks}} && \K{\textit{für}} & \K{\textit{einen Keks}} &\K{}\\
  }
  \caption[Vollständiger und abgekürzter Baum]{Beispiel für einen abgekürzten Baum (rechts) und das vollständige Äquivalent (links)}
  \label{fig:treeabbr}
\end{figure}

Wir benutzen in Baumdiagrammen außerdem eine oft gesehene Notation mit Dreiecken.
Mit diesen Dreiecken kann man die Strukturen abkürzen, die zum gegenwärtigen Zeitpunkt noch nicht analysierbar sind, oder die gerade nicht im Detail von Interesse sind.
Abbildung~\ref{fig:treeabbr} zeigt ein Beispiel, bei dem eine PP einmal vollständig analysiert wurde.
Außerdem wird dieselbe Struktur in einer abgekürzten Form dargestellt, bei der die interne Struktur der NP nicht weiter analysiert wird.
Das Dreieck zeigt an, dass die NP zwar eine Struktur hat, diese aber hier nicht dargestellt wird.
In der Klammerschreibweise entsprechen die Dreiecke einfach weggelassenen Klammern, und gerade hier ist das strategische Weglassen von Klammern oft essentiell, um die Übersichtlichkeit zu gewährleisten.
Zum Beispiel kann man mit (\ref{ex:phr1623783}) gut illustrieren, dass \textit{wahrscheinlich dem Arzt heimlich das Bild schnell verkauft} im Ganzen eine Konstituente ist, und dass innerhalb dieser Konstituente \textit{dem Arzt} und \textit{das Bild} Konstituenten aus je zwei Wörtern sind.
Da \textit{wahrscheinlich}, \textit{heimlich}, \textit{schnell} und \textit{verkauft} Konstituenten aus je einem Wort sind, kann die Klammerung entfallen.
Mit einer Näherungsweise vollständigen Klammerung ergäbe sich die recht unübersichtliche Darstellung in (\ref{ex:phr1623784a}), mit beschrifteten Klammern die nahezu unbrauchbare in (\ref{ex:phr1623784b}).

\begin{exe}
  \ex{\label{ex:phr1623783} Ischariot hat [wahrscheinlich [dem Arzt] heimlich [das Bild] schnell verkauft].}
  \ex\label{ex:phr1623784}
  \begin{xlist}
    \ex{\label{ex:phr1623784a} [[Ischariot] [hat] [[wahrscheinlich] [[dem] [Arzt]] [heimlich] [[das] [Bild]] [schnell] [verkauft]]].}
    \ex{\label{ex:phr1623784b} [\Sub{S} [\Sub{NP} Ischariot] [\Sub{V} hat] [\Sub{VP} [\Sub{Adv} wahrscheinlich] [\Sub{NP} [\Sub{Art} dem] [\Sub{Subst} Arzt]] [\Sub{A} heimlich] [\Sub{NP} [\Sub{Art} das] [\Sub{Subst} Bild]] [\Sub{A} schnell] [\Sub{V} verkauft]]].}
  \end{xlist}
\end{exe}


\section{Koordination}

\label{sec:koor}

Das erste Schema beschreibt Strukturen mit Konjunktionen wie \textit{und} oder \textit{oder}.

\index{Koordination}
\index{Konjunktion}

\Phrasenschema{Koordination}{\label{str:koor}
  \begin{tabular}{c|c|c|c|}
    \cline{2-4}
    $\kappa$ = & $\kappa$ & ( Konj ) & $\kappa$ \\
    \cline{2-4}
  \end{tabular}
}

Der Koordinationstest (Abschnitt~\ref{sec:konstituententestsimeinzelnen}) beruht auf der hier angenommenen Flexibilität dieses Schemas.
Zwei Wortformen oder zwei Phrasen können jederzeit mit einer Konjunktion verbunden werden, vorausgesetzt, dass beide zu derselben Klasse gehören (N und N oder AP und AP usw.).
$\kappa$ steht hier stellvertretend für N, NP, A usw.
Das Schema erlaubt die Verbindung von irgendeiner syntaktischen Einheit vom Typ $\kappa$ mit einer anderen Einheit derselben Klasse $\kappa$, wobei wieder eine Einheit der Klasse $\kappa$ herauskommt.
Wie in einer mathematischen Formel zeigt die Variable $\kappa$ also an, dass statt ihrer irgendetwas, aber immer das Gleiche eingesetzt werden muss.
Es werden außerdem für die Konjunktion keine konkreten Wortformen wie \textit{und} explizit im Schema genannt, sondern das Wortklassensymbol Konj, so dass irgendeine Konjunktion eingesetzt werden kann.
Das Schema erlaubt damit die Analyse der Koordinationen in (\ref{ex:grp2357}) wie in den Abbildungen~\ref{fig:koorsubst}--\ref{fig:koorpraep}
Die koordinierten Teilstrukturen werden an dieser Stelle dabei jeweils nicht weiter analysiert und mit einem Dreieck abgekürzt.%
\footnote{Das Symbol S für \textit{Satz} wird in Kapitel~\ref{sec:saetze} genauer eingeführt.}

\begin{exe}
  \ex\label{ex:grp2357}
  \begin{xlist}
    \ex{Ihre Freundin möchte [Kuchen und Sahne].}
    \ex{[[Es ist Sonntag] und [die Zeit wird knapp]].}
    \ex{Hast du das Teepulver [auf oder neben] den Tatami-Matten verstreut?}
  \end{xlist}
\end{exe}

\begin{figure}
  \centering
  \Tree{
    & \K{\textbf{N}}\B{dl}\B{d}\B{dr} & \\
    \K{\textbf{N}}\B{d} & \K{Konj}\B{d} & \K{\textbf{N}}\B{d} \\
  \K{\textit{Kuchen}} & \K{\textit{und}} & \K{\textit{Sahne}}
  }
  \caption{Koordination von Substantiven}
  \label{fig:koorsubst}
\end{figure}

\begin{figure}
  \centering
  \Tree{
    &&& \K{S}\B{dll}\B{d}\B{drr} \\
    & \K{S}\TRi && \K{Konj}\B{d} && \K{S}\TRi[3] & \\
    & \K{\textit{Es ist Sonntag}} && \K{\textit{und}} && \K{\textit{die Zeit wird knapp}} & \\
  }
  \caption{Koordination von Sätzen}
  \label{fig:koorsatz}
\end{figure}

\begin{figure}
  \centering
  \Tree{
    & \K{\textbf{P}}\B{dl}\B{d}\B{dr} \\
    \K{\textbf{P}}\B{d} & \K{Konj}\B{d} & \K{\textbf{P}}\B{d} \\
  \K{\textit{auf}} & \K{\textit{oder}} & \K{\textit{neben}}
  }
  \caption{Koordination von Präpositionen}
  \label{fig:koorpraep}
\end{figure}

Die Optionalität der Konjunktion (im Schema eingeklammert) trägt Fällen wie in (\ref{ex:grp1935}) Rechnung, in denen die Konjunktion weggelassen werden kann.
Hierfür gelten aber normalerweise zusätzliche Bedingungen, wie zum Beispiel, dass die letzte Koordination in einer längeren Kette auf jeden Fall mit einer Konjunktion verbunden werden muss.
Diese Bedingungen können aus Platzgründen und wegen der eingeschränkten Ausdrucksfähigkeit der Phrasenschemata hier nicht detailliert formuliert werden.
Zu den Kommaschreibungen in solchen Konstruktionen s.\ aber auch Abschnitt~\ref{sec:koordinschreib}.

\begin{exe}
  \ex{\label{ex:grp1935} Ich sehe [[Wörter, Sätze] und Texte].}
\end{exe}

\section{Nominalphrase}

\label{sec:ngr}

\subsection{Die Struktur der NP}

Das Phrasenschema für die NP, wie es jetzt hier vorgestellt wird, unterscheidet sich deutlich von den bisher definierten provisorischen NPs (vor allem Definition~\ref{def:vollstaendigesnominal} auf S.~\pageref{def:vollstaendigesnominal} und Abbildung~\ref{fig:strschembsp} auf S.~\pageref{fig:strschembsp}), weil es der Versuch ist, alle möglichen NPs in einem Schema zu beschreiben.
Die NP hat eine potentiell große Komplexität, und daher wird hier mit dem Begriff \textit{innere Rechtsattribute} ein Sammelbegriff eingeführt, der im Folgenden dann aufgelöst und im Einzelnen erläutert wird.
Zunächst folgt die Definition des \textit{Attributs} in Definition~\ref{def:attribut}.

\Definition{Attribut}{
\label{def:attribut}
Attribute sind alle unmittelbaren Konstituenten der Nominalphrase außer dem Artikelwort und dem nominalen Kopf.
\index{Attribut}
}

Die Relativsätze (und Adjektive) zählen also auch zu den Attributen, und als \textit{innere Rechtsattribute} bezeichnen wir besonders die Attribute, die rechts vom nominalen Kopf aber links von Relativsätzen stehen.
Einige der inneren Rechtsattribute sind als Ergänzungen aufzufassen, was im weiteren Verlauf diskutiert wird.
In unserer Auffassung ist der Attributsbegriff also ein struktureller und kein funktionaler, und Attribute in der NP können sowohl (regierte) Ergänzungen sein als auch nicht. 

\index{Nominalphrase}

\Phrasenschema{Nominalphrase (NP)}{\label{str:ngr}
    NP = \\
    \begin{tabular}{|c|c|c|c|c|}
      \cline{1-5}
      ( Art\ \slash\ NP[\textsc{Kas}:\textit{gen}] ) & ( AP )* & \textbf{N} & ( innere Rechtsattribute )* & ( RS )* \\
      \cline{1-5}
    \end{tabular}
  }

Die NP ist in ihrer Komplexität dem vollständigen Satz durchaus ebenbürtig.
Die folgende Ausformulierung des Schemas in Worten kann daher bei der Verdeutlichung eine erhebliche Hilfe sein.
Eine NP besteht aus:

\index{Artikel!Position}
\index{Adjektivphrase}
\index{Relativsatz}
\index{Genitiv!pränominal}

\begin{enumerate}\Lf
  \item einem \textit{Nomen} (Substantiv oder Pronomen) als Kopf,
  \item keinem, einem oder mehreren rechts stehenden \textit{inneren Attributen} (Ergänzungen oder Angaben von heterogener Form, vgl.\ Abschnitt~\ref{sec:innererechtsattribute}),
  \item keiner, einer oder mehreren links stehenden \textit{Adjektivphrasen} (AP, vgl.\ Abschnitt~\ref{sec:agr}),
  \item keinem, einem oder mehreren rechts von den rechten Attributen stehenden \textit{Relativsätzen} (RS, vgl.\ Abschnitt~\ref{sec:relativsaetze}),
  \item einem optionalen links (von allen Adjektiven) stehenden \textit{Artikelwort} oder einer \textit{Nominalphrase im Genitiv} (dem sogenannten \textit{pränominalen Genitiv}).
\end{enumerate}

\index{Nomen!vs.\ Substantiv}

Zunächst ist noch die Frage offen, warum überhaupt das Symbol N für den Kopf solcher Phrasen steht, und nicht Subst.
Immerhin wurde in Kapitel~\ref{sec:wortklassen} besonders betont, dass Substantive und Nomina nicht dasselbe sind.
Die Klassifikation der Nomina als Überklasse aus Kapitel~\ref{sec:wortklassen} (Wortartenfilter \ref{wfilt:verbennomina}, S.~\pageref{wfilt:verbennomina}) zahlt sich hier aus.
Einerseits sind das Kategoriensystem und die Flexion der verschiedenen Nomina einheitlich bzw.\ ähnlich.
Andererseits fungieren sie als Kopf von Phrasen, die sich im weiteren Strukturaufbau gleich verhalten -- und daher auch gleich bezeichnet werden sollten.
Insbesondere sind die Pronomina genauso wie die Substantive typische Köpfe von Nominalphrasen.
Als wichtiger Unterschied gilt, dass bei Pronomina als NP-Kopf sämtliche Positionen links vom Kopf (also Artikelwörter und APs) nicht besetzt werden können.
Im Bereich der inneren Rechtsattribute und der Relativsätze gibt es auch Einschränkungen, wenn ein Pronomen der Kopf ist, aber wir belassen es bei dieser Feststellung und geben die Abbildungen~\ref{fig:pronngr1} und \ref{fig:pronngr2} als Analysen der NP aus (\ref{ex:pronngr}).

\begin{exe}
  \ex\label{ex:pronngr}
  \begin{xlist}
    \ex{\label{ex:pronngr1} [Dieser] schmeckt besonders lecker mit Sahne.}
    \ex{\label{ex:pronngr2} [Einen [mit Sahne]] würde ich schon noch essen.}
  \end{xlist}
\end{exe}

\begin{figure}
  \centering
  \Tree{
    \K{NP}\B{d} \\
    \K{\textbf{N}} \B{d} \\
    \K{\textit{dieser}} \\
  }
  \caption{NP mit pronominalem Kopf}
  \label{fig:pronngr1}
\end{figure}

\begin{figure}
  \centering
  \Tree{
    \K{NP}\B{d}\B{drr} \\
    \K{\textbf{N}}\B{d} && \K{PP}\TRi[-4] \\ 
    \K{\textit{einen}} && \K{\textit{mit Sahne}} & \\
  }
  \caption{NP mit pronominalem Kopf}
  \label{fig:pronngr2}
\end{figure}

Die einfachste Struktur einer NP mit einem substantivischen Kopf wird in Abbildung~\ref{fig:ngreinf} gezeigt, und sie ähnelt Abbildung~\ref{fig:pronngr1}.
Wenn dies aber die minimale Struktur einer substantivhaltigen NP ist, ist unsere Annahme, dass das Substantiv der Kopf ist, sehr naheliegend.%
\footnote{Wie bereits auf S.~\pageref{abs:kopfnichtweglassbar} argumentiert wurde, sind Elemente einer Phrase, die niemals weggelassen werden können, sehr gute Kandidaten für den Kopf-Status.
Wenn es nur ein solches Element gibt, ist es in der hier vertretenen Art der Grammatikschreibung nahezu zwangsläufig als Kopf aufzufassen.}
In den folgenden Unterabschnitten werden die einzelnen Attribute und die Artikel diskutiert.
Die Relativsätze werden später in Abschnitt~\ref{sec:relativsaetze} besprochen.

\begin{figure}
  \centering
  \Tree{
    \K{NP}\B{d} & \\
    \K{\textbf{N}}\B{d} & \\
    \K{\textit{Zahnbürsten}} & \\
  }
  \caption{Minimale NP mit substantivischem Kopf}
  \label{fig:ngreinf}
\end{figure}

\subsection{Innere Rechtsattribute}

\label{sec:innererechtsattribute}

Die inneren Rechtsattribute treten in vielen verschiedenen Formen auf.
Im einfachen Fall steht eine PP oder eine NP im Genitiv rechts direkt neben dem nominalen Kopf.
Die Sätze in (\ref{ex:grp9146}) werden in den Abbildungen~\ref{fig:ngrmitprpgr} und \ref{fig:ngrmitngr} analysiert.
Da wir PP-Strukturen noch nicht beschrieben haben (s.\ Abschnitt~\ref{sec:prpgr}), wird die PP mit einem Dreieck abgekürzt.

\begin{exe}
  \ex\label{ex:grp9146}
  \begin{xlist}
    \ex{Man hat [Zahnbürsten [vom König]] im Hotel gefunden.}
    \ex{Man hat [Zahnbürsten [des Königs]] im Hotel gefunden.}
  \end{xlist}
\end{exe}

\begin{figure}
  \centering
  \Tree[0.5]{
    \K{NP}\B{d}\B{drr} & \\
    \K{\textbf{N}}\B{d} && \K{PP}\TRi[-2] \\
    \K{\textit{Zahnbürsten}} && \K{\textit{von dem König}} & \\
  }
  \caption{NP mit PP}
  \label{fig:ngrmitprpgr}
\end{figure}

\begin{figure}
  \centering
  \Tree{
    \K{NP}\B{d}\B{drr} & \\
    \K{\textbf{N}}\B{d} && \K{NP}\TRi[-4] \\
    \K{\textit{Zahnbürsten}} && \K{\textit{des Königs}} & \\
  }
  \caption{NP mit Genitiv-NP}
  \label{fig:ngrmitngr}
\end{figure}

\index{Genitiv!postnominal}

Die dem Kopf nachgestellte Genitiv-NP und die \textit{von}-PP konkurrieren in einer gewissen Weise, weil sie funktional ähnlich sind.
Dies sieht man deutlich in (\ref{ex:grp9146}), wo beide eine Besitzrelation zwischen den Zahnbürsten und dem König kodieren.
Neben der PP mit \textit{von} können aber viele andere Arten von PPs (und auch mehrere) eingesetzt werden, die nicht wie die \textit{von}-PP dem Genitiv funktional ähnlich sind, vgl.\ (\ref{ex:grp2497}).

\begin{exe}
  \ex\label{ex:grp2497}
  \begin{xlist}
    \ex{Ich sammle [Zahnbürsten [aus Silber]].}
    \ex{Ich sammle [Zahnbürsten [von Königen]].}
    \ex{\label{ex:grp2497c} Ich sammle [Zahnbürsten [von Königen] [aus Silber]].}
    \ex{Ich sammle [Zahnbürsten [in Reise-Etuis]].}
  \end{xlist}
\end{exe}

\begin{figure}
  \centering
  \Tree{
    \K{NP}\B{d}\B{drr}\B{drrrr} \\
    \K{\textbf{N}}\B{d} && \K{PP}\TRi[-2] && \K{PP}\TRi[-4] \\
    \K{\textit{Zahnbürsten}} && \K{\textit{von Königen}} && \K{\textit{aus Silber}} & \\
  }
  \caption{NP mit zwei inneren PP-Rechtsattributen}
  \label{fig:grp2497c}
\end{figure}

In den Sätzen in (\ref{ex:grp2497}) besteht definitiv keine Valenz- oder Rektionsanforderung seitens des N-Kopfes.
Diese PPs können mit beliebigen Substantiven kombiniert werden, solange PP und Substantiv semantisch kompatibel sind.

\subsection{Rektion und Valenz in der NP}

\label{sec:rektioninderngr}

\index{Valenz!Substantiv}
\index{Nominalisierung}

Es gibt einige Konstruktionen, in denen Valenz und Rektion von nominalen Köpfen angenommen werden muss.
Einerseits gibt es den Fall, dass Ergänzungen von Verben bei Nominalisierungen als Ergänzung des Substantivs wieder auftauchen.
Dies ist in den Sätzen (b) in (\ref{ex:grp1590}) und (\ref{ex:grp1592}) der Fall.
Die fraglichen Konstituenten sind sämtlich innere Rechtsattribute.

\begin{exe}
  \ex\label{ex:grp1590}
  \begin{xlist}
    \ex{\label{ex:grp1590a} Wir glauben [an das Gute im Menschen].}
    \ex{\label{ex:grp1590b} [Unser Glaube [an das Gute im Menschen]] hält uns am Leben.}
  \end{xlist}
  \ex\label{ex:grp1592}
  \begin{xlist}
    \ex{\label{ex:grp1592a} Liv vermutet, [dass der Marmorkuchen sehr lecker ist].}
    \ex{\label{ex:grp1592b} [Livs Vermutung, [dass der Marmorkuchen sehr lecker ist]], liegt nah.}
  \end{xlist}
\end{exe}

In (\ref{ex:grp1590b}) wird die Valenz des Verbs \textit{glauben} zu seiner Substantivierung \textit{Glaube} überführt, indem die PP [\textit{an das Gute im Menschen}] in beiden Fällen erscheint.
Die Rektionsanforderung, dass die PP die Präposition \textit{an} enthalten muss, wird ebenfalls vom Verb zur Substantivierung mitgenommen.
Ähnlich verhält es sich mit dem \textit{dass}-Satz (einem Komplementsatz, vgl.\ Abschnitt~\ref{sec:komplementsaetze}), der sowohl von dem Verb \textit{vermuten} als auch der Substantivierung \textit{Vermutung} regiert wird.
Dafür, dass hier Valenz bzw.\ Rektion am Werk sind, spricht, dass diese Attribute nicht mit beliebigen Substantiven kombinierbar sind.
An den nicht akzeptablen NPs in (\ref{ex:grp1479}) ist dies deutlich zu sehen.
Die Substantive in der PP wurden hier ausgetauscht, um zu zeigen, dass nicht etwa die Bedeutung das Problem ist.

\begin{exe}
  \ex\label{ex:grp1479}
  \begin{xlist}
    \ex[*]{[der Kuchen [an den Kuchenteller]]}
    \ex[*]{[die Überzeugung [an das Regierungsprogramm]]}
    \ex[*]{[die Verlässlichkeit [an meine Freunde]]}
  \end{xlist}
\end{exe}

\index{Genitiv!postnominal}
\index{Genitiv!pränominal}
\index{Objektsgenitiv}
\index{Akkusativ}

Bei der Substantivierung transitiver Verben (solche, die einen Nominativ und einen Akkusativ regieren) wird der Akkusativ beim Verb regelhaft in einen Genitiv beim Substantiv umgesetzt, vgl.\ (\ref{ex:grp8970}).
Der Genitiv tritt hier als inneres Rechtsattribut auf, als sogenannter postnominaler Genitiv.
Alternativ ist auch wie in (\ref{ex:grp8970c}) die \textit{von}-PP möglich (ggf.\ umgangssprachlich).
Die Genitive, die auf diese Weise auf Akkusative bezogen sind, nennt man \textit{Objektsgenitive}.
Wir nennen hier den Objektsgenitiv und die ähnliche \textit{von}-PP zusammen \textit{Objektsattribute}.
Der Nominativ des Verbs kann dabei als PP mit \textit{durch} beim Substantiv realisiert werden, ähnlich wie bei einem verbalen Passiv (s.\ Abschnitt~\ref{sec:passiv}).

\begin{exe}
  \ex\label{ex:grp8970}
  \begin{xlist}
    \ex{\label{ex:grp8970a} Sarah verziert [den Kuchen].}
    \ex{\label{ex:grp8970b} [Die Verzierung [des Kuchens] [durch Sarah]] erfolgt unter höchster Geheimhaltung.}
    \ex{\label{ex:grp8970c} [Die Verzierung [von dem Kuchen] [durch Sarah]] erfolgt unter höchster Geheimhaltung.}
  \end{xlist}
\end{exe}

\index{Genitiv!Subjekts--}

Statt den Nominativ des Verbs in eine \textit{durch}-PP zu überführen, kann er eingeschränkt auch als pränominaler Genitiv in der Artikelposition auftauchen wie in (\ref{ex:grp1940}).
Ausführliches zur Artikelposition folgt in Abschnitt~\ref{sec:adjektiveundartikelwoerter}.

\begin{exe}
  \ex\label{ex:grp1940}
  \begin{xlist}
    \ex{\label{ex:grp1940a} [Sarah] rettet [den Kuchen] [vor dem Anbrennen].}
    \ex{\label{ex:grp1940b} [Sarahs Rettung [des Kuchens] [vor dem Anbrennen]] war heldenhaft.}
  \end{xlist}
\end{exe}

\begin{figure}
  \centering
  \Tree{
    &&& \K{NP}\B{dll}\B{d}\B{drr}\B{drrrr} \\
    & \K{NP}\TRi[-5] && \K{\textbf{N}}\B{d} && \K{NP}\TRi[-5] && \K{PP}\TRi[-4] \\
    & \K{\textit{Sarahs}} && \K{\textit{Rettung}} && \K{\textit{des}}\Below{\textit{Kuchens}} && \K{\textit{vor dem}}\Below{\textit{Anbrennen}} & \\
  }
  \caption{NP mit pränominalem Subjektsgenitiv}
  \label{fig:grp1940b}
\end{figure}

Der Nominativ von intransitiven Verben kann ebenfalls als Genitiv (pränominal oder postnominal) bei einer Nominalisierung des Verbs auftauchen, vgl.\ (\ref{ex:grp8234}).
In ihrer Verwendung eingeschränkt, aber möglich ist auch hier die \textit{von}-PP (\ref{ex:grp8234bb}).
Man nennt diesen Genitiv den \textit{Subjektsgenitiv}, zusammen mit der entsprechenden \textit{von}-PP sprechen wir vom \textit{Subjektsattribut}.

\begin{exe}
  \ex\label{ex:grp8234}
  \begin{xlist}
    \ex[]{\label{ex:grp8234a} [Die Schokolade] wirkt gemütsaufhellend.}
    \ex[]{\label{ex:grp8234b} [Die Wirkung [der Schokolade]] ist gemütsaufhellend.}
    \ex[?]{\label{ex:grp8234bb} [Die Wirkung [von der Schokolade]] ist gemütsaufhellend.}
    \ex[?]{\label{ex:grp8234c} [[Der Schokolade] Wirkung] ist gemütsaufhellend.}
  \end{xlist}
\end{exe}

Die eventuellen Probleme mit Beispiel (\ref{ex:grp8234c}), die durch das Fragezeichen angedeutet werden, rühren daher, dass der pränominale Genitiv in Artikelposition nur noch eingeschränkt verwendet wird.
Uneingeschränkt verwendbar ist er nur noch mit \textit{s}-Genitiven von Eigennamen.
Auch (\ref{ex:grp8234bb}) wird von manchen Sprechern als ungewöhnlich eingestuft.

Wenn bei der Nominalisierung eines transitiven Verbs ein Genitiv oder eine \textit{von}-PP realisiert wird, die den Akkusativ des Verbs vertritt, ohne dass der Nominativ des Verbs in der NP repräsentiert wird, entspricht die gesamte NP eher einem Passiv, da beim Passiv der Nominativ des Aktivs ebenfalls nicht realisiert wird (vgl.\ zum Passiv Abschnitt~\ref{sec:passiv}).
In (\ref{ex:grp8090}) wird dies durch einen Passivsatz und eine entsprechende NP illustriert.

\begin{exe}
  \ex\label{ex:grp8090}
  \begin{xlist}
    \ex{\label{ex:grp8090a} Sarah wurde gerettet.}
    \ex{\label{ex:grp8090b} [Sarahs Rettung] war erfolgreich.}
  \end{xlist}
\end{exe}

Zur abschließenden Illustration der funktionalen Nähe des Genitivs und der \textit{von}-PP dienen nun (\ref{ex:grp6200})--(\ref{ex:grp6211}), in denen nochmals gezeigt wird, dass beide zur Kodierung der Subjekts- und Objektsfunktion verwendet werden können, auch wenn beim Subjekts- und Objektsattribut von einigen Sprechern die \textit{von}-PP als umgangssprachlich eingestuft wird.
Tabelle~\ref{tab:verbnomenvalenz} fasst die typischen Korrespondenzen zusammen, die zwischen den Valenzen eines Verbs und seiner Nominalisierung bestehen.

\begin{exe}
  \ex\label{ex:grp6200} 
  \begin{xlist}
    \ex{\label{ex:grp6200a}[Mein Hund] bellt.}
    \ex{\label{ex:grp6200b}[das Bellen [meines Hundes]]}
    \ex{\label{ex:grp6200c}[das Bellen [von meinem Hund]]}
  \end{xlist}
  \ex\label{ex:grp6201} 
  \begin{xlist}
    \ex{\label{ex:grp6201a}Pavel verziert [den Kuchen].}
    \ex{\label{ex:grp6201b}[die Verzierung [des Kuchens]]}
    \ex{\label{ex:grp6201c}[die Verzierung [von dem Kuchen]]}
  \end{xlist}
  \ex\label{ex:grp6211} 
  \begin{xlist}
    \ex{\label{ex:grp6211a}Pavel rettet Sarah.}
    \ex{\label{ex:grp6211b}[Pavels Rettung [von Sarah]]}
  \end{xlist}
\end{exe}

\begin{table}
  \resizebox{\textwidth}{!}{
    \begin{tabular}{lll}
      \lsptoprule
      \textbf{beim Verb} & \textbf{in der NP} & \textbf{Beispiel} \\
      \midrule
      Nominativ-NP & postnominaler Genitiv & \textit{die Schokolade wirkt} \\
      && $\Rightarrow$ \textit{die Wirkung der Schokolade} \\
      & pränominaler Genitiv & \textit{Pavel rettet} (\textit{den Kuchen}) \\
      & (Eigennamen) & $\Rightarrow$ \textit{Pavels Rettung} (\textit{des Kuchens}) \\
      && \textit{Sarah läuft.} \\
      && $\Rightarrow$ \textit{Sarahs Lauf} \\
      & postnominale \textit{von}-PP & \textit{die Schokolade wirkt} \\
      & (bei manchen Sprechern) & $\Rightarrow$ \textit{die Wirkung von der Schokolade} \\
      & postnominale \textit{durch}-PP & \textit{Sarah rettet den Kuchen} \\
      & (zusammen mit Objektsgenitiv) & $\Rightarrow$ \textit{die Rettung des Kuchens durch Sarah} \\
      \midrule
      Akkusativ-NP & postnominaler Genitiv & (\textit{jemand}) \textit{backt den Kuchen} \\
      (Passiv-Lesart) && $\Rightarrow$ \textit{das Backen des Kuchens} \\
      & pränominaler Genitiv & \textit{Pavel rettet Sarah} \\
      & (Eigennamen) & $\Rightarrow$ \textit{Sarahs Rettung} (\textit{durch Pavel}) \\
      & \textit{von}-PP & \textit{jemand backt den Kuchen} \\
      && $\Rightarrow$ \textit{das Backen von dem Kuchen} \\
      \midrule
      PP & PP (unverändert) & (\textit{jemand}) \textit{glaubt an das Gute} \\
      && $\Rightarrow$ \textit{der Glaube an das Gute} \\
      \midrule
      Satz & Satz (unverändert) & (\textit{jemand}) \textit{vermutet, dass es lecker ist} \\
      && $\Rightarrow$ \textit{die Vermutung, dass es lecker ist} \\
      \lspbottomrule
    \end{tabular}
  }
  \caption{Valenz-Korrespondenzen bei substantivierten Verben}
  \label{tab:verbnomenvalenz}
\end{table}

\subsection{Adjektivphrasen und Artikelwörter}

\label{sec:adjektiveundartikelwoerter}

Eine AP links vom Kopf ist niemals eine Ergänzung und niemals regiert.%
\footnote{Hier werden nur APs gezeigt, die aus einem einfachen A ohne weitere Konstituenten bestehen, da die Struktur der AP erst in Abschnitt~\ref{sec:agr} erklärt wird.
Das Abkürzungsdreieck muss trotzdem stehen, da wir die einzelnen Adjektive nicht als Wörter in die größere Phrase einbinden, sondern als vollständige Phrase.
Würden wir statt des Dreiecks hier eine einfache Linie nehmen, käme das der Behauptung gleich, die Adjektive seien schon im Lexikon AP.
Das kann man durchaus so modellieren, aber für die praktische grammatische Analyse ist es nach der hier vertretenen Auffassung besser, zwischen Wortart im Lexikon und Phrase im grammatischen Aufbau auf diese Weise zu trennen.}
Dies kann man leicht daran erkennen, dass beliebig viele von ihnen vorkommen können, vgl. (\ref{ex:grp0718}) und die Analysen dazu in den Abbildungen~\ref{fig:grp0718a} und \ref{fig:grp0718b}.
In den Beispielen sieht man auch, dass das Vorhandensein der PP oder NP rechts keinen Einfluss auf die Anwesenheit der AP hat oder umgekehrt.
Beide sind völlig unabhängig voneinander.

\begin{exe}
  \ex\label{ex:grp0718}
  \begin{xlist}
    \ex{\label{ex:grp0718a} Man hat [rote Zahnbürsten [des Königs]] im Hotel gefunden.}
    \ex{\label{ex:grp0718b} Man hat [freundliche ehemalige Präsidenten] geehrt.}
  \end{xlist}
\end{exe}

\begin{figure}
  \centering
  \Tree{
    &&& \K{NP}\B{dll}\B{d}\B{drr} \\
    & \K{AP}\TRi[-8] && \K{\textbf{N}}\B{d} && \K{NP}\TRi[-3] \\
    & \K{\textit{rote}} && \K{\textit{Zahnbürsten}} && \K{\textit{des Königs}} & \\
  }
  \caption{NP mit AP und Rechtsattribut}
  \label{fig:grp0718a}
\end{figure}

\begin{figure}
  \centering
  \Tree{
    &&&&& \K{NP}\B{dllll}\B{dll}\B{d} \\
    & \K{AP}\TRi[-3] && \K{AP}\TRi[-3] && \K{\textbf{N}}\B{d} \\
    & \K{\textit{freundliche}} && \K{\textit{ehemalige}} && \K{\textit{Präsidenten}} \\
  }
  \caption{NP mit mehreren Adjektiven}
  \label{fig:grp0718b}
\end{figure}

In Abbildung~\ref{fig:grp0718b} wurden die APs getrennt in den NP-Baum eingefügt.
Man kann alternativ auch versuchen, das Vorhandensein mehrerer APs in der NP als Koordination von APs ohne Konjunktion zu analysieren, so dass im Phrasenschema der NP die Wiederholbarkeit der AP nicht extra angezeigt werden müsste.
Diese Struktur ist in Abbildung~\ref{fig:ngradjkonj} zu sehen, wo informell die fehlende Konjunktion von einem Komma vertreten wird.
Das Adjektiv \textit{ehemalig} wurde bewusst gegen \textit{nett} ausgetauscht.

\begin{figure}
  \centering
  \Tree{
    &&&&&&&& \K{NP}\B{dllll}\B{dd} \\
    &&&& \K{AP}\B{dll}\B{d}\B{drr} \\
    && \K{AP}\TRi[-3] && \K{Konj}\B{d} && \K{AP}\TRi[-6] && \K{\textbf{N}}\B{d} \\
    && \K{\textit{freundliche}} && \K{\textit{,}} && \K{\textit{nette}} && \K{\textit{Präsidenten}}\\
  }
  \caption{NP mit koordinierten Adjektiven}
  \label{fig:ngradjkonj}
\end{figure}

Wenn die Analyse in Abbildung~\ref{fig:ngradjkonj} plausibel sein soll, dann sollte zumindest im Prinzip eine Konjunktion stehen können.
Wie in (\ref{ex:intextadj}) gezeigt wird, ist dies abhängig von der Wahl der Adjektive manchmal, aber eben nicht immer der Fall.

\begin{exe}
  \ex\label{ex:intextadj}
  \begin{xlist}
    \ex[*]{[Die [freundlichen und ehemaligen] Präsidenten] wurden geehrt.}
    \ex[]{[Die [freundlichen und netten] Präsidenten] wurden geehrt.}
  \end{xlist}
\end{exe}

In Fällen, wo eine Konjunktion einsetzbar ist, steht prototypischerweise ein Komma.
Wir nehmen nur in Fällen, wo eine Konjunktion einsetzbar ist, eine Koordinationsstruktur wie in Abbildung~\ref{fig:ngradjkonj} an.
In allen anderen Fällen kommt nur die Struktur in Abbildung~\ref{fig:grp0718b} als Analyse in Frage.

Schließlich können optional noch Artikelwörter oder NPs im Genitiv ganz links eingesetzt werden.

\begin{exe}
  \ex\label{ex:grp7000}
  \begin{xlist}
    \ex{Man hat [die roten Zahnbürsten [des Königs], [die benutzt waren]], im Hotel gefunden.}
    \ex{Man hat [Elins Zahnbürste, die benutzt war], im Hotel gefunden.}
  \end{xlist}
\end{exe}

\begin{figure}
  \centering
  \Tree{
    &&&& \K{NP}\B{dllll}\B{dll}\B{d}\B{drr}\B{drrrr} \\
    \K{Art}\B{d} && \K{AP}\TRi[-8] && \K{\textbf{N}}\B{d} && \K{NP}\TRi[-5] && \K{RS}\TRi[-3] \\
    \K{\textit{die}} && \K{\textit{roten}} && \K{\textit{Zahnbürsten}} && \K{\textit{des}}\Below{\textit{Königs}} && \K{\textit{die benutzt}}\Below{\textit{waren}} & \\
  }
  \caption{NP mit Artikelwort und Relativsatz}
  \label{fig:grp7000}
\end{figure}

Dank der Weglassbarkeit aller Teile der Phrase zwischen Artikel und Substantiv werden aber auch viel einfachere Strukturen vom Phrasenschema beschrieben, \zB (\ref{ex:artsubst}).

\begin{exe}
  \ex{\label{ex:artsubst} Man hat [einige Zahnbürsten] im Hotel gefunden.}
\end{exe}

\begin{figure}
  \centering
  \Tree{
    && \K{NP}\B{dll}\B{d} \\
    \K{Art}\B{d} && \K{\textbf{N}}\B{d} \\
    \K{\textit{einige}} && \K{\textit{Zahnbürsten}}
  }
  \caption{NP mit N und Art}
  \label{fig:artsubst}
\end{figure}

\index{Stoffsubstantiv}
\index{Artikel!NP ohne}

Der Artikel kann bei sogenannten \textit{Stoffsubstantiven} (Substantiven, die nicht-zählbare Substanzen bezeichnen) im Singular (\ref{ex:grp0018a}) und bei Indefinita im Plural (\ref{ex:grp0018b}) unterbleiben.
Dem Artikel \textit{ein} im Singular entspricht im Plural also die Artikellosigkeit, wenn nicht auf Pronomina wie \textit{einige} ausgewichen wird (\ref{ex:grp0018c}).

\begin{exe}
  \ex\label{ex:grp0018} 
  \begin{xlist}
    \ex{\label{ex:grp0018a} Wir trinken [Tee].}
    \ex{\label{ex:grp0018b} Wir lesen [Bücher].}
    \ex{\label{ex:grp0018c} Wir lesen [einige Bücher].}
  \end{xlist}
\end{exe}

Ein pränominaler Genitiv (s.\ schon Abschnitt~\ref{sec:rektioninderngr}) und ein Artikelwort können nicht gleichzeitig vorkommen.
Daher ist die Annahme plausibel, dass sie dieselbe nicht wiederholbare Position in der Struktur einnehmen.
Genau deswegen besetzen Art und NP[\textsc{Kas}:\textit{gen}] im Phrasenschema dasselbe Feld.

\begin{exe}
  \ex\label{ex:grp9001}
  \begin{xlist}
    \ex[]{\label{ex:grp9001a} [Elins große Kaffeetasse]}
    \ex[]{\label{ex:grp9001b} [die große Kaffeetasse]}
    \ex[*]{\label{ex:grp9001c} [Elins die große Kaffeetasse]}
    \ex[*]{\label{ex:grp9001d} [die Elins große Kaffeetasse]}
  \end{xlist}
\end{exe}

\section{Adjektivphrase}

\label{sec:agr}

Wir besprechen in diesem Abschnitt nur Fälle, in denen die AP attributiv verwendet wird.
Prädikative APs werden in Abschnitt~\ref{sec:kopulakonstruktionen} diskutiert.
Eine attributive AP besteht aus einem Adjektiv und einem optionalen davor stehenden \textit{Intensivierer} (\zB \textit{sehr}).
Davor stehen wiederum optional verschiedene Arten von Ergänzungen und Angaben, und zwar gegebenenfalls mehrere. 
Was genau für Ergänzungen vorkommen, wird (im Sinn der Definition von Valenz und Rektion, s.\ Kapitel~\ref{sec:grundbegriffe}) durch das jeweilige Adjektiv festgelegt.
Es ist notwendig, zwei verschiedene strukturelle Positionen in der AP links vom Kopf anzunehmen, weil Intensivierer wie \textit{sehr} immer direkt vor dem Adjektiv stehen und sich niemals unter die anderen Ergänzungen und Angaben mischen.
Im weiteren Verlauf dieses Abschnitts werden genau diese Verhältnisse illustriert.
Zunächst folgt aber das Phrasenschema.

\index{Adjektivphrase}

\Phrasenschema{Adjektivphrase (AP), nur attributiv}{\label{str:agr}

    \begin{tabular}{c|c|c|c|}
      \cline{2-4}
      AP = & ( NP / PP / AdvP )* & ( Intensivierer ) & \textbf{A} \\
      \cline{2-4}
    \end{tabular}

}

Die Intensivierer treten in Form von Partikeln (\zB \textit{sehr}, \textit{voll} oder \textit{ziemlich}), PPs (\zB \textit{über alle Maßen}), AdvPs (\zB \textit{hochgradig}) auf.
Die Bezeichnung \textit{Intensivierer} in Schema~\ref{str:agr} ist damit eine Sammelbezeichnung ähnlich wie \textit{innere Rechtsattribute} in Schema~\ref{str:ngr}.
Typische Angaben, die links von der Intensiviererposition stehen, sind \zB temporale oder lokale wie \textit{seit gestern} in \textit{seit gestern berühmt} oder \textit{in Hessen} wie in \textit{in Hessen beliebt}. 
Einige APs werden in (\ref{ex:adjgrnormal}) bebeispielt.
In (\ref{ex:adjgrnormalc}) und (\ref{ex:adjgrnormald}) zeigt sich auch die Nichtvertauschbarkeit der Modifikatoren und Intensivierer.
Ein Analyse von (\ref{ex:adjgrnormalc}) in Baumform findet sich in Abbildung~\ref{fig:adgrnormalana}.

\begin{exe}
  \ex\label{ex:adjgrnormal}
  \begin{xlist}
    \ex[]{\label{ex:adjgrnormala} die [angenehme] Stimmung}
    \ex[]{\label{ex:adjgrnormalb} die [sehr angenehme] Stimmung}
    \ex[]{\label{ex:adjgrnormalc} die [[seit gestern] sehr angenehme] Stimmung}
    \ex[*]{\label{ex:adjgrnormald}die [sehr [seit gestern] angenehme] Stimmung}
  \end{xlist}
\end{exe}

\begin{figure}
  \centering
  \Tree{
    &&&&& \K{AP}\B{dll}\B{dllll}\B{d} \\
    & \K{PP}\TRi[-3] && \K{Ptkl}\B{d} && \K{\textbf{A}}\B{d} \\
    & \K{\textit{seit gestern}} && \K{\textit{sehr}} && \K{\textit{angenehme}} \\
  }
  \caption{AP mit Modifizierer und Intensivierer}
  \label{fig:adgrnormalana}
\end{figure}

Wie erwähnt sind nur einige der Elemente in der ersten Position innerhalb der AP regierte Ergänzungen, und
nur wenige Adjektive haben eine besondere Valenz (vgl.\ auch Abschnitt~\ref{sec:adjektivklassifikation}).
Einige Adjektive regieren eine NP im Genitiv, \zB \textit{ähnlich}, \textit{gewahr}, \textit{müde}, \textit{überdrüssig}.
Andere regieren eine PP mit einer bestimmten Präposition, \zB \textit{verwundert} (\textit{über}), \textit{stolz} (\textit{auf}), \textit{ansässig} (irgendeine lokale PP oder AdvP, die eventuelle nicht im engeren Sinne regiert ist).
Die Beispiele (\ref{ex:adjgrvaleinfach}) zeigen, dass in attributiven APs die Ergänzungen genau wie die Angaben immer vor dem Adjektiv stehen.
Eine Konstituentenanalyse von (\ref{ex:adjgrvaleinfacha}) liefert Abbildung~\ref{fig:erginadjgr}.
Eine maximale AP, die eine Angabe in Form einer PP (\textit{seit gestern}), eine Ergänzung in Form einer PP (\textit{auf ihre Tochter}) und einen Intensivierer (\textit{sehr}) enthält, ist in Abbildung~\ref{fig:adjgrziemlichvoll} dargestellt.
In prädikativen APs sind die Stellungsmöglichkeiten etwas anders (Abschnitt~\ref{sec:kopulakonstruktionen}).

\begin{exe}
  \ex\label{ex:adjgrvaleinfach}
  \begin{xlist}
    \ex[]{\label{ex:adjgrvaleinfacha} die [[auf ihre Tochter] stolze] Frau}
    \ex[*]{die [stolze [auf ihre Tochter]] Frau}
    \ex[]{die [[über ihre Tochter] verwunderte] Frau}
    \ex[*]{die [verwunderte [über ihre Tochter]] Frau}
    \ex[]{die [[ihres Lieblingseises] überdrüssige] Frau}
    \ex[*]{die [überdrüssige [ihres Lieblingseises]] Frau}
  \end{xlist}
\end{exe}

\begin{figure}
  \centering
  \Tree{
    &&& \K{AP}\B{dll}\B{d} \\
    & \K{PP}\TRi[-6] && \K{\textbf{A}}\B{d} \\
    & \K{\textit{auf ihre}}\Below{\textit{Tochter}} && \K{\textit{stolze}} \\
  }
  \caption{AP mit Ergänzung}
  \label{fig:erginadjgr}
\end{figure}

\begin{figure}
  \centering
  \Tree{
    &&&&&&& \K{AP}\B{dllllll}\B{dllll}\B{dll}\B{d} \\
    & \K{PP}\TRi[-6] && \K{PP}\TRi[-6] && \K{Ptkl}\B{d} && \K{\textbf{A}}\B{d} \\
    & \K{\textit{seit}}\Below{\textit{gestern}} && \K{\textit{auf ihre}}\Below{\textit{Tochter}} && \K{\textit{sehr}} && \K{\textit{stolze}} \\
  }
  \caption{AP, in der alle Positionen gefüllt sind}
  \label{fig:adjgrziemlichvoll}
\end{figure}

\section{Präpositionalphrase}

\label{sec:prpgr}

\subsection{Normale PP}

\index{Präpositionalphrase}

\Phrasenschema{Präpositionalphrase (PP)}{\label{str:prpgr}
  \begin{tabular}{c|c|c|c|}
    \cline{2-4} 
    PP = & ( NP / PP / AP / AdvP / Ptkl ) & \textbf{P} & NP \\
    \cline{2-4} 
  \end{tabular}
}

Alle Präpositionen haben eine strikt einstellige Valenz.
Sie nehmen genau eine niemals fakultative NP als Ergänzung und regieren ihren Kasus.
Außerdem können sie von Modifizierern (darunter Maßangaben im Akkusativ) modifiziert werden, die immer links stehen und nie regiert sind.
Entsprechende Beispiele und Strukturen finden sich in (\ref{ex:prpgrallg}).
Einen Modifikator in Form eines Adjektivs (\textit{weit}) zeigt (\ref{ex:prpgrintensiv}).

\begin{exe}
  \ex\label{ex:prpgrallg}
  \begin{xlist}
    \ex{[Auf [dem Tisch]] steht Ischariots Skulptur.}
    \ex{[[Einen Meter] unter [der Erde]] ist die Skulptur versteckt.}
  \end{xlist}
  \ex{\label{ex:prpgrintensiv} Seit den London Masters spielt Carter [weit unter [seinem Niveau]].}
\end{exe}

\begin{figure}
  \centering
  \Tree{
    &&& \K{PP}\B{dll}\B{d}\B{drr} \\
    & \K{NP}\TRi[-6] && \K{\textbf{P}}\B{d} && \K{NP}\TRi[-4] \\
    & \K{\textit{einen}}\Below{\textit{Meter}} && \K{\textit{unter}} && \K{\textit{der Erde}} & \\
  }
  \caption{PP mit Maßangabe (NP)}
  \label{fig:prpgrmitakkmod}
\end{figure}

\begin{figure}
  \centering
  \Tree{
    & \K{PP}\B{dl}\B{d}\B{drr} \\
    \K{Ptkl}\B{d} & \K{\textbf{P}}\B{d} && \K{NP}\TRi[-1] \\
    \K{\textit{weit}} & \K{\textit{unter}} && \K{\textit{seinem Niveau}} &\\
  }
  \caption{PP mit Intensivierer}
  \label{fig:prpgrmitadvmod}
\end{figure}

\index{Postposition}

Für die im Deutschen sehr seltenen \textit{Postpositionen} (\zB \textit{wegen}, \textit{halber}) wird aus Platzgründen kein gesondertes Schema angegeben.
Die regierte NP steht bei ihnen vor dem Kopf statt nach dem Kopf.
Ein Beispiel ist (\ref{ex:postpos}), wobei bei \textit{wegen} sowohl eine präpositionale als auch eine postpositionale Variante hat.

\begin{exe}
  \ex{\label{ex:postpos} [[Der Sprechstunde] wegen] habe ich mein Training verpasst.}
\end{exe}

\begin{figure}
  \centering
  \Tree{
    &&& \K{PP}\B{dll}\B{d} \\
    & \K{NP}\TRi[1] && \K{\textbf{Psp}}\B{d} \\
    & \K{\textit{der Sprechstunde}} && \K{\textit{wegen}}
  }
  \caption{Postpositionsstruktur}
  \label{fig:postpos}
\end{figure}

\subsection{PP mit flektierbaren Präpositionen}

\label{sec:syntaxflektierbareprp}

\index{Präposition!flektierbar}

Einen Problemfall stellen die flektierbaren Präpositionen wie \textit{zur}, \textit{am} dar, die historisch Verbindungen aus einer Präposition und einem Artikelwort sind.
Die einfachste Möglichkeit wäre, sie nicht als eine, sondern zwei Wortformen zu analysieren.
Das sähe dann aus wie in Abbildung~\ref{fig:flektprpreanalyse}.

\begin{figure}
  \centering
  \Tree{
    \K{PP}\B{dd}\B{drr} \\
    && \K{NP}\B{dl}\B{d} \\
    \K{\textbf{P}}\B{d} & \K{Art}\B{d} & \K{\textbf{N}}\B{d} \\
    \K{\textit{zu}} & \K{\textit{'m}} & \K{\textit{Training}} \\
  }
  \caption{Flektierbare Präpositionen als zwei Wortformen}
  \label{fig:flektprpreanalyse}
\end{figure}

Die zweite Möglichkeit ist die Analyse der flektierbaren Präposition als eine Wortform.
Für diese muss man dann annehmen, dass sie eine NP ohne Artikelwort regiert.
Außerdem werden in dieser die \textsc{Genus}- und \textsc{Numerus}-Werte der NP von der Präposition regiert, denn \textit{zum} kann \zB nur mit Maskulina und Neutra im Dativ Singular kombiniert werden.
Eine solche Analyse wird in \ref{fig:flektprpalswortform} gezeigt.
Wir entscheiden uns hier für diese Variante.

\begin{figure}
  \centering
  \Tree{
    \K{PP}\B{dd}\B{dr} \\
    & \K{NP}\B{d} \\
    \K{\textbf{P}}\B{d} & \K{\textbf{N}}\B{d} \\
    \K{\textit{zum}} & \K{\textit{Training}} \\
  }
  \caption{Flektierbare Präpositionen als eine Wortform (präferiert)}
  \label{fig:flektprpalswortform}
\end{figure}

\section{Adverbphrase}

\label{sec:advgr}

\index{Adverbphrase}

\Phrasenschema{Adverbphrase (AdvP)}{\label{str:advgr}
  \begin{tabular}{c|c|c|}
    \cline{2-3}
    AdvP = & ( NP / PP / AdvP / Ptkl ) & \textbf{Adv} \\
    \cline{2-3} 
  \end{tabular}
}

\begin{figure}
  \centering
  \Tree{
    &&& \K{AdvP}\B{dll}\B{d} \\
    & \K{Ptkl}\B{d} && \K{\textbf{Adv}}\B{d} \\
    & \K{\textit{etwas}} && \K{\textit{schnell}}
  }
  \caption{AdvP mit Modifizierer}
  \label{fig:advgrintensiv}
\end{figure}

Adverben verhalten sich bezüglich der Phrasenbildung wie Präpositionen ohne Valenz, also ohne die NP-Ergänzung.
Auch bei ihnen sind Maßangaben und andere Modifikatoren möglich, wie in (\ref{ex:advgrintensiv}).

\begin{exe}
  \ex\label{ex:advgrintensiv} 
  \begin{xlist}
    \ex{\label{ex:advgrintensiva} Ischariot malt [etwas schnell].}
    \ex{\label{ex:advgrintensivb} Ischariot malt [irrsinnig schnell].}
    \ex{\label{ex:advgrintensivc} Ischariot malt [über alle Maßen schnell].}
  \end{xlist}
\end{exe}

\section{Komplementiererphrase}

\label{sec:subjgr}
\index{Komplementierer}
\index{Komplementiererphrase}

\Phrasenschema{Komplementiererphrase (KP)}{\label{str:subjgr}
  \begin{tabular}{c|c|c|}
    \cline{2-3}
    KP = & \textbf{K} & VP \\
    \cline{2-3} 
  \end{tabular}
}

Die \textit{Komplementiererphrase} (KP) entspricht im Wesentlichen dem sogenannten \textit{Nebensatz} (s.\ auch Abschnitte~\ref{sec:komplementiererwortklassen} und \ref{sec:nebensaetze}), und sie ist denkbar einfach strukturiert.
Jeder Komplementierer (K) verlangt als Valenzanforderung immer eine VP (s.\ Abschnitt~\ref{sec:vgr}), die rechts von ihm steht.
Die grammatischen und ungrammatischen Beispielsätze in (\ref{ex:substval}) illustrieren dies.

\begin{exe}
  \ex\label{ex:substval}
  \begin{xlist}
    \ex[]{Der Arzt möchte, [dass [der Kassenpatient geht]].}
    \ex[*]{Der Arzt möchte, [dass].}
    \ex[*]{Der Arzt möchte, [dass [ein Kassenpatient]].}
  \end{xlist}
\end{exe}

\begin{figure}
  \centering
  \Tree{
    \K{KP}\B{d}\B{drr} \\
    \K{\textbf{K}}\B{d} && \K{VP}\TRi[-2] \\
    \K{\textit{dass}} && \K{\textit{der Kassen-}}\Below{\textit{patient geht}} & \\
  }
  \caption{KP}
  \label{fig:substval}
\end{figure}

\index{Verb-Letzt-Satz}
\index{Verb-Zweit-Satz}
\index{Verb-Erst-Satz}

Innerhalb der Verbphrase, die von einem Komplementierer regiert wird, steht das finite Verb immer ganz rechts.
Es liegt also die sogenannte \textit{Verb-Letzt-Stellung} (s.\ Kapitel~\ref{sec:saetze}) vor wie in (\ref{ex:subjvea}).
Dies ist im unabhängigen Aussagesatz (ohne Konjunktion) nicht der Fall.
Dort liegt die sogenannte \textit{Verb-Zweit-Stellung} vor, wobei das finite Verb nach einem anderen Satzglied an zweiter Stelle steht.
Diese Stellung ist in einer VP innerhalb einer KP aber ausgeschlossen, wie (\ref{ex:subjveb}) zeigt.
Auch die im Ja\slash Nein-Fragesatz obligatorische Verb-Erst-Stellung ist innerhalb der KP nicht möglich, s.\ (\ref{ex:subjvec}).

\begin{exe}
  \ex\label{ex:subjve}
  \begin{xlist}
    \ex[]{\label{ex:subjvea} Der Arzt möchte, [dass [der Privatpatient die Rechnung bezahlt]].}
    \ex[*]{\label{ex:subjveb} Der Arzt möchte, [dass [der Privatpatient bezahlt die Rechnung]].}
    \ex[*]{\label{ex:subjvec} Der Arzt möchte, [dass [bezahlt der Privatpatient die Rechnung]].}
  \end{xlist}
\end{exe}

Die KP wurde hier vor der VP eingeführt, weil die Einbettung der VP in die KP eine besondere Ordnung innerhalb der VP (Verb-Letzt-Stellung) sicherstellt, die es in allen anderen Satzarten nicht gibt.
Im nächsten Abschnitt wird zunächst nur die Variante der VP, die im eingeleiteten Nebensatz (der KP) vorkommt, besprochen, ohne auf andere mögliche Konstituentenstellungen einzugehen.
Wie alle anderen Satzgliedstellungsmuster in Sätzen erklärt werden, ist das Thema von Kapitel~\ref{sec:saetze}.

\section{Verbphrase und Verbalkomplex}

\label{sec:vgr}

Die Verbphrase enthält das Verb als Kopf und links davon die Ergänzungen des Verbs.
Die Ergänzungen können vergleichsweise beliebig mit Angaben (PP, AdvP, seltener auch NP) gemischt werden.
Im Deutschen gibt es darüber hinaus für die Kombination von finiten und infiniten Verbformen (s.\ Abschnitt~\ref{sec:finit}) innerhalb der VP spezielle Regularitäten, die hier gesondert im Schema des \textit{Verbalkomplexes} kodiert werden.
Wir stellen Verbalkomplexe hier so dar, dass die Kombination aus dem finiten und den von ihm direkt oder indirekt abhängenden infiniten Verben immer das Symbol V erhalten.
Zahlreiche Beispiele folgen nach den Phrasenschemata.

\index{Verbphrase}

\Phrasenschema{Verbphrase (VP)}{\label{str:vgr}
  \begin{tabular}{c|c|c|}
    \cline{2-3}
    VP = & ( NP / PP / AdvP / Ptkl )* & \textbf{V} \\
    \cline{2-3}
  \end{tabular}
}

\index{Verbalkomplex}

\Phrasenschema{Verbalkomplex (V)}{\label{str:vk}
  \begin{tabular}{c|c|c|}
    \cline{2-3}
    V = & ( V ) & \textbf{V} \\
    \cline{2-3}
  \end{tabular}
}

\subsection{Verbphrase}

\label{sec:verbphrase}

\index{Valenz!Verb}

Verben nehmen eine bestimmte Anzahl von Ergänzungen (im Wesentlichen eine, zwei oder drei), deren relevante Merkmale sie regieren (vgl.\ Abschnitt~\ref{sec:valenz}).
Da Valenz und Rektion typische Kopfeigenschaften sind, ist es plausibel, alle Strukturen, in denen Verben ihre Valenz und Rektion entfalten, als Verbphrasen zu bezeichnen.

\begin{exe}
  \ex{\label{ex:prpgrval} dass [Ischariot sich [an der Eiskrem] vergeht]}
\end{exe}

In Abschnitt~\ref{sec:komplementsaetze} wird ausführlich gezeigt, dass nicht nur NPs und PPs sondern auch satzförmige Ergänzungen von Verben regiert werden.
Ein Beispiel für ein Verb, das einen \textit{dass}-Satz regiert, findet sich in (\ref{ex:dassval}), wobei die Stellung dieser Ergänzung rechts vom Verb zunächst ignoriert wird.
Es werden also auf keinen Fall nur NPs von Verben regiert, und eine Fixierung auf \textit{intransitive} Verben -- die eine NP im Nominativ regieren -- und \textit{intransitive} Verben -- die eine NP im Nominativ und eine im Akkusativ regieren -- würde dem deskriptiven Befund nicht gerecht.

\begin{exe}
  \ex{\label{ex:dassval} dass Ischariot glaubt, [dass seine Bilder viel wert sind]}
\end{exe}

Es geht hier zunächst nur um VPs in eingeleiteten Nebensätzen, also solche, die in eine KP eingebettet sind.
Wenn wir fürs Erste auch den Verbalkomplex nicht genauer betrachten, sondern nur Nebensätze mit einem einfachen finiten Verb, dann erhalten wir Beispiele für VPs wie in (\ref{ex:phr634561}), die mit dem Schema für die VP (Schema~\ref{str:vgr}) einfach analysierbar sind.

\begin{exe}
\ex\label{ex:phr634561}\begin{xlist}
  \ex{dass [Ischariot malt]}
  \ex{dass [Ischariot [das Bild] malt]}
  \ex{dass [Ischariot [dem Arzt] [das Bild] verkauft]}
  \ex{dass [Ischariot wahrscheinlich [dem Arzt] heimlich [das Bild] schnell verkauft]}
\end{xlist} 
\end{exe}

\begin{figure}
  \centering
  \Tree{
    &&& \K{VP}\B{dll}\B{d} \\
    & \K{NP}\TRi[-4] && \K{\textbf{V}}\B{d} \\
    & \K{\textit{Ischariot}} && \K{\textit{malt}} & \\
  }
  \caption{VP mit einstelliger Valenz}
  \label{fig:vgreinstellig}
\end{figure}

\begin{figure}
  \centering
  \Tree{
    &&&&& \K{VP}\B{dllll}\B{dll}\B{d} \\
    & \K{NP}\TRi[-4] && \K{NP}\TRi[-5] && \K{\textbf{V}}\B{d} \\
    & \K{\textit{Ischariot}} && \K{\textit{das Bild}} && \K{\textit{malt}} & \\
  }
  \caption{VP mit zweistelliger Valenz}
  \label{fig:vgrzweistellig}
\end{figure}

\begin{figure}
  \centering
  \Tree{
    &&&&&&& \K{VP}\B{dllllll}\B{dllll}\B{dll}\B{d} \\
    & \K{NP}\TRi[-3] && \K{NP}\TRi[-3] && \K{NP}\TRi[-4] && \K{\textbf{V}}\B{d} \\
    & \K{\textit{Ischariot}} && \K{\textit{dem Arzt}} && \K{\textit{das Bild}} && \K{\textit{verkauft}} & \\
  }
  \caption{VP mit dreistelliger Valenz}
  \label{fig:vgrdreistellig}
\end{figure}

\begin{figure}
  \resizebox{\textwidth}{!}{
    \Tree[0.1]{
      &&&&&&&&&&&&& \K{VP}\B{ddllllllllllll}\B{ddllllllllll}\B{ddllllllll}\B{ddllllll}\B{ddllll}\B{ddll}\B{dd} \\
      \\
      & \K{NP}\TRi[-3] && \K{AdvP}\TRi[-2] && \K{NP}\TRi[-3] && \K{AdvP}\TRi[-4] && \K{NP}\TRi[-4] && \K{AdvP}\TRi[-4] && \K{\textbf{V}}\B{d} \\
      & \K{\textit{Ischariot}} && \K{\textit{wahrscheinlich}} && \K{\textit{dem Arzt}} && \K{\textit{heimlich}} && \K{\textit{das Bild}} && \K{\textit{schnell}} && \K{\textit{verkauft}} & \\
    }
  }
  \caption{VP mit dreistelliger Valenz und Adverbialen}
  \label{fig:vgrdreistelligmitadv}
\end{figure}

In welcher Abfolge die Ergänzungen und Angaben in der VP stehen, ist nicht nach sonderlich transparenten Regularitäten festgelegt.
Wir haben hier nur der Einfachheit halber immer die Reihenfolge Nominativ -- Dativ -- Akkusativ gewählt.
Sätze wie in (\ref{ex:vgrorder}) mit Dativ -- Nominativ -- Akkusativ oder Akkusativ -- Nominativ -- Dativ usw.\ sind allerdings genauso grammatisch.
Es gibt durchaus Regularitäten, die diese Reihenfolge in konkreten Sätzen beeinflussen, aber diese sind komplex und innerhalb einer eher formbezogenen Grammatik nicht gut beschreibbar.
Man nimmt üblicherweise Bezug auf semantische Eigenschaften der Einheiten (\zB ob es sich um Pronomina oder NPs mit substantivischem Kopf handelt) oder die \textit{Informationsstruktur} des Satzes (\zB ob die von NPs bezeichneten Gegenstände im Diskurs schon einmal erwähnt wurden oder nicht).

\newcommand{\ThePhrasenExOne}{dass [[dem Jungen] [die Mutter] [ein Eis] [geschenkt hat]]}
\newcommand{\ThePhrasenExTwo}{dass [[ein Eis] [die Mutter] [dem Jungen] [geschenkt hat]]}

\begin{exe}
  \ex\label{ex:vgrorder}
  \begin{xlist}
    \ex{\ThePhrasenExOne}
    \ex{\ThePhrasenExTwo}
  \end{xlist}
\end{exe}

Die rein strukturelle Eigenschaft, dass die Abfolge der Ergänzungen und Angaben in der VP nicht strikt festgelegt ist, nennt man \textit{Scrambling} (engl.\ wörtlich \textit{Verrühren}).

\Definition{Scrambling}{
\label{def:scrambling}
Scrambling bezeichnet die Eigenschaft, dass innerhalb der VP die Ergänzungen in keiner strikt festgelegten Reihenfolge realisiert werden.
\index{Scrambling}
}

\subsection{Verbalkomplex}

\label{sec:verbkomplexe}

\index{Tempus!analytisch}

Wenden wir uns nun dem \textit{Verbalkomplex} zu.
Alle bisherigen Beispiele zur VP enthielten immer genau ein Verb, und zwar ein finites Vollverb (vgl.\ Abschnitte~\ref{sec:finit} und \ref{sec:vunterklass}), wie in (\ref{ex:auxinf0}).
Bei analytischen Tempora (Perfekt, Futur), Passiven und einer großen Menge von weiteren Konstruktionen haben wir es jedoch mit einem oder mehreren infiniten Verben und meist einem finiten Verb zu tun.
Beispiele dafür sind (\ref{ex:auxinfa})--(\ref{ex:auxinfc}).

\begin{exe}
  \ex{\label{ex:auxinf0} dass der Junge ein Eis [isst]}
  \ex\label{ex:auxinf}
  \begin{xlist}
    \ex{\label{ex:auxinfa} dass der Junge ein Eis [essen wird]}
    \ex{\label{ex:auxinfb} dass das Eis [gegessen wird]}
    \ex{\label{ex:auxinfc} dass die Mutter das Eis [kaufen wollen wird]}
  \end{xlist}
\end{exe}

\begin{figure}
  \centering
  \Tree{
    \K{\textbf{V}}\B{d} \\
    \K{\textit{isst}} \\
  }
  \caption{Maximal einfacher Verbalkomplex}
  \label{fig:vkomplex0}
\end{figure}

Normalerweise steht das finite Verb, wenn es eines gibt, in VPs innerhalb von Nebensätzen immer am Ende.%
\footnote{Um systematische Abweichungen von dieser Generalisierung geht es in Abschnitt~\ref{sec:ersiofu}.}
Da man zwei oder mehr Verben im Verbalkomplex kombinieren kann, ist das Schema für V so angegeben, dass es rekursiv angewendet werden kann.
In den Abbildungen~\ref{fig:vkomplex0}--\ref{fig:vkomplex3} finden sich Analysen der Verbalkomplexe aus (\ref{ex:auxinf}) mit einer zusätzlichen Numerierung, die im Folgenden sofort erklärt wird.
Außerdem zeigen Pfeile die Rektionsbeziehungen an.

\begin{figure}
  \centering
  \Tree{
    &&& \K{\textbf{V}}\B{dll}\B{d} \\
    & \K{\textbf{V}}\B{d} && \K{\textbf{V}}\B{d} \\ 
    & \K{\textit{essen}}\Below{2} && \K{\textit{wird}}\Below{1} \POS[]-(0,9)\ar@/^/[ll]-(0,9) \\
  }~\Tree{
    &&& \K{\textbf{V}}\B{dll}\B{d} \\
    & \K{\textbf{V}}\B{d} && \K{\textbf{V}}\B{d} \\ 
    & \K{\textit{gegessen}}\Below{2} && \K{\textit{wird}}\Below{1}\POS[]-(0,9)\ar@/^/[ll]-(0,9)
  }
  \vspace{0.3cm}
  \caption{Verbalkomplexe für Futur (links) und Passiv (rechts)}
  \label{fig:vkomplex2}
\end{figure}

\begin{figure}
  \centering
  \Tree{
    &&&&& \K{\textbf{V}}\B{dll}\B{dd} \\
    &&& \K{\textbf{V}}\B{dll}\B{d} \\
    & \K{\textbf{V}}\B{d} && \K{\textbf{V}}\B{d} && \K{\textbf{V}}\B{d} \\
    & \K{\textit{kaufen}}\Below{3} && \K{\textit{wollen}}\Below{2}\POS[]-(0,9)\ar@/^/[ll]-(0,9) && \K{\textit{wird}}\Below{1}\POS[]-(0,9)\ar@/^/[ll]-(0,9) & \\
  }
  \vspace{0.3cm}
  \caption{Futur-Modalverb-Verbalkomplex}
  \label{fig:vkomplex3}
\end{figure}

\index{Status}

Die hierarchische Ordnung im Verbalkomplex ist davon geprägt, dass jeder der Verbköpfe seine eigene Valenz hat.
Bereits in Abschnitt~\ref{sec:infinflex} wurde gesagt, dass bestimmte Verben andere Verben als Ergänzungen nehmen können, und dass sie dabei deren Status regieren (vgl.\ Abschnitt~\ref{sec:infinflex}).
Die in Frage kommenden Formen sind der reine Infinitiv (der \textit{erste Status}), der Infinitiv mit \textit{zu} (der \textit{zweite Status}) und das Partizip (der \textit{dritte Status}).
Wenn wie in (\ref{ex:auxinfa}) bzw.\ Abbildung~\ref{fig:vkomplex2} (linker Baum) das Futurhilfsverb \textit{werden} verwendet wird, muss das von \textit{werden} abhängige Vollverb immer im ersten Status stehen.
Konstruktionen wie in (\ref{ex:futursterr}) sind nicht möglich (\ref{ex:futursterra}) bzw.\ sind keine Futurformen (\ref{ex:futursterrb}).

\begin{exe}
  \ex\label{ex:futursterr}
  \begin{xlist}
    \ex[*]{\label{ex:futursterra} dass der Junge ein Eis zu essen wird}
    \ex[*]{\label{ex:futursterrb} dass der Junge ein Eis gegessen wird}
  \end{xlist}
\end{exe}

\index{Hilfsverb}
\index{Modalverb}
In (\ref{ex:auxinfb}) bzw.\ Abbildung~\ref{fig:vkomplex2} (rechter Baum) handelt es sich um das Passivhilfsverb \textit{werden}, das durch seine Rektion (dritter Status) klar vom Futur-Hilfsverb (erster Status) unterscheidbar ist.
Hier steht \textit{gegessen} im dritten Status und erfüllt damit die Rektionsanforderung des Passivhilfsverbs.

Genauso haben Modalverben wie \textit{wollen} eine Valenz und Status-Rektion.
Alle Modalverben regieren den ersten Status.
In (\ref{ex:auxinfc}) bzw.\ Abbildung~\ref{fig:vkomplex3} bilden drei die Verben dadurch eine von hinten nach vorne zu lesende Kette von Rektionsbeziehungen.
Das Futur-Hilfsverb \textit{werden} regiert den ersten Status der Modalverbs \textit{wollen}, und \textit{wollen} regiert den ersten Status des Vollverbs \textit{kaufen}.
Verben, die einen Status regieren (wie \textit{werden} und \textit{wollen} im gegebenen Beispiel) haben also Kopfcharakter.
Dass in (\ref{ex:auxinfc}) bzw.\ Abbildung~\ref{fig:vkomplex3} \textit{kaufen} auch Kopfeigenschaften haben muss, ergibt sich aus dem restlichen Aufbau der VP.
Das Verb \textit{kaufen} ist maßgeblich für die Ergänzungen und Angaben, die in der VP stehen.
Im Vergleichssatz in (\ref{ex:vkrekexta}) besteht der Verbalkomplex nur aus einem finiten \textit{kaufen}, aber der Rest der VP ist genauso gebaut wie in (\ref{ex:auxinfc}).
Nach außen wird die Valenz des Verbalkomplexes also immer vom lexikalischen Verb bestimmt.
Die Kongruenzeigenschaften (Numerus und Person) werden am Verb, das in der Rektionshierarchie am höchsten steht realisiert.
In (\ref{ex:auxinfc}) ist das \textit{wird} und in (\ref{ex:vkrekexta}) \textit{kauft}.

\begin{exe}
  \ex{\label{ex:vkrekexta} dass die Mutter das Eis kauft}
\end{exe}

\begin{figure}
  \centering
  \Tree{
    &&&&&&&&& \K{VP}\B{ddddllllllll}\B{ddddllllll}\B{dd} \\
    \\
    &&&&&&&&& \K{\textbf{V}}\B{dll}\B{dd} \\
    &&&&&&& \K{\textbf{V}}\B{dll}\B{d} \\
    & \K{NP}\TRi[-4] && \K{NP}\TRi[-4] && \K{\textbf{V}}\B{d} && \K{\textbf{V}}\B{d} && \K{\textbf{V}}\B{d} \\
    & \K{\textit{die Mutter}} && \K{\textit{das Eis}} && \K{\textit{kaufen}}\POS[]-(0,4)\ar@/^/[ll]-(0,4)\POS[]-(0,4)\ar@/^{1.5pc}/[llll]-(0,4) && \K{\textit{wollen}}\POS[]-(0,4)\ar@/^/[ll]-(0,4) && \K{\textit{wird}}\POS[]-(0,4)\ar@/^/[ll]-(0,4) & \\
    \vspace{1cm}
  }
  \caption{VP mit komplexem Verbalkomplex und Rektionsbeziehungen}
  \label{fig:vkrekext}
\end{figure}

Die Zahlen 1, 2 und 3 unter den Verben führen dabei lediglich eine in der germanistischen Linguistik übliche Art und Weise ein, über die Reihenfolge und die Rektionsabhängigkeiten der Verben im Verbalkomplex zu sprechen.
Die Zahlen geben die Rektionsabfolge wieder und dürfen nicht mit Bestimmungen des Status verwechselt werden.
Die Reihenfolge, in der sie stehen, entspricht der linearen Abfolge (und eben nicht der Rektionshierarchie) der korrespondierenden Elemente im Verbalkomplex.
Dabei gilt, dass das Verb mit Nummer 1 das Verb mit Nummer 2 regiert, welches wiederum das Verb mit Nummer 3 regiert.
Die 3 entfällt selbstverständlich, wenn der Komplex nur aus zwei Verben besteht, wie in (\ref{ex:auxinfa}) und (\ref{ex:auxinfb}).
Ein sogenannter \textit{321-Komplex} ist ein Verbalkomplex wie in (\ref{ex:auxinfc}), also einer, in dem zwei statusregierende Verben und ein Vollverb vorkommen, die in der Rektionshierarchie von hinten nach vorne geordnet sind.
Die wichtigste Abweichung vom 321-Komplex wird in Abschnitt~\ref{sec:ersiofu} besprochen.

\section{Konstruktion von Konstituentenanalysen}

\label{sec:syntaxana}
\index{Baumdiagramm}

Abschließend wird kurz darauf eingegangen, wie man Analysen von Konstituentenstrukturen konstruktiv angehen kann.
Je komplexer die Strukturen werden, umso schwerer ist es erfahrungsgemäß, durch einfache Intuition eine angemessene Analyse zu finden.
Einige wichtige bereits bekannte syntaktische Prinzipien helfen aber bei einem systematischen Vorgehen.
Einerseits wird der Aufbau der syntaktischen Struktur durch die Wortklasse (und weitere grammatische Merkmale) der Wörter bestimmt.
Außerdem bestimmen Valenz und Rektion in großem Maß die Syntax.
Schließlich gilt, dass wir nur Strukturen annehmen wollen, für die Phrasenschemata definiert wurden.
Satz~\ref{satz:syntaxana} gibt darauf aufbauend die wesentlichen Schritte der konstruktiven Analyse von Konstituentenstrukturen an.

\Satz{Konstruktive Analyse von Konstituentenstrukturen}{
\label{satz:syntaxana}
Bei der Analyse von Konstituentenstrukturen sind folgende Schritte durchzuführen:
\begin{enumerate}\Lf
  \item\label{it:sya01} Bestimmung der Wortklassen für alle Wörter,
  \item\label{it:sya02} Suche nach und Analyse von Teilkonstituenten, die nur nach einem bestimmten Schema analysierbar sind, vor allem erkennbar an Kongruenzmarkierungen,
  \item\label{it:sya03} Suche nach Valenznehmern zu Köpfen, die Valenz haben,
  \item\label{it:sya04} Suche der Köpfe zu den verbleibenden Angaben
\end{enumerate}
}

Der Ablauf einer solchen Analyse wird jetzt anhand der KP in (\ref{ex:phrasen23734}) Schritt für Schritt illustriert.

\begin{exe}
  \ex{\label{ex:phrasen23734} dass Frida den heißen Kaffee gerne trinken möchte}
\end{exe}

Zunächst muss gemäß Punkt~\ref{it:sya01} aus Satz~\ref{satz:syntaxana} für jedes Wort die Wortklasse bestimmt werden, s.\ Abbildung~\ref{fig:sya01}.

\begin{figure}
%  \resizebox{\textwidth}{!}{
    \Tree[1.25]{
      \K{K}\B{d} & \K{N}\B{d} & \K{Art}\B{d} & \K{A}\B{d} & \K{N}\B{d} & \K{Adv}\B{d} & \K{V}\B{d} & \K{V}\B{d} \\
      \K{\textit{dass}} & \K{\textit{Frida}} & \K{\textit{den}} & \K{\textit{heißen}} & \K{\textit{Kaffee}} & \K{\textit{gerne}} & \K{\textit{trinken}} & \K{\textit{möchte}} \\
    }
%  }
  \caption{Konstruktion einer syntaktischen Analyse, Schritt 1}
  \label{fig:sya01}
\end{figure}

Im nächsten Schritt müssen nach Punkt \ref{it:sya02} aus Satz~\ref{satz:syntaxana} bereits die Strukturschemata berücksichtigt werden.
Es fällt zunächst auf, dass \textit{Frida} ein Eigenname ist, und dass Eigennamen immer der Kopf einer NP sind, die keine weiteren Konstituenten enthält.
Es muss also eine vollständige NP sein.
Als nächstes fällt auf, dass ein Adjektiv wie \textit{heißen} immer eine AP bilden muss.
Gemäß Strukturschema \ref{str:agr} kann vor einem Adjektiv in einer AP dabei nur ein Intensivierer und bzw.\ oder eine NP, PP oder AdvP stehen.
Da vor dem Adjektiv ein Artikel steht, können wir sicher sein, dass die AP hier nur aus dem Adjektiv besteht.
Eine ähnliche Schlussfolgerung kann für das Adverb \textit{gerne} getroffen werden, und wir erhalten Abbildung~\ref{fig:sya02}.

\begin{figure}
%  \resizebox{\textwidth}{!}{
    \Tree[1.25]{
      & \K{NP}\B{d} && \K{AP}\B{d} && \K{AdvP}\B{d} \\
      \K{K}\B{d} & \K{\textbf{N}}\B{d} & \K{Art}\B{d} & \K{\textbf{A}}\B{d} & \K{N}\B{d} & \K{\textbf{Adv}}\B{d} & \K{V}\B{d} & \K{V}\B{d} \\
      \K{\textit{dass}} & \K{\textit{Frida}} & \K{\textit{den}} & \K{\textit{heißen}} & \K{\textit{Kaffee}} & \K{\textit{gerne}} & \K{\textit{trinken}} & \K{\textit{möchte}} \\
    }
%  }
  \caption{Konstruktion einer syntaktischen Analyse, Schritt 2}
  \label{fig:sya02}
\end{figure}

Jetzt ergibt sich in der Mitte des Satzes eine Abfolge aus Art, AP und N.
Vor allem, weil Artikel nur in NPs vorkommen, kann die NP \textit{den heißen Kaffee} zusammengesetzt werden.
Dass das Symbol Art nur in dem Schema für die NP (Schema~\ref{str:ngr}) vorkommt, kann leicht durch Durchsicht aller definierten Schemata verifiziert werden.%
\footnote{Hier wird das Verfahren ein bisschen verkürzt, weil man natürlich nicht ohne Betrachtung des Kontexts sehen kann, ob es der Artikel \textit{den} oder das Pronomen \textit{den} ist.
Würde man probieren, den Satz mit einem Pronomen \textit{den} zu analysieren, würde man aber am Ende nicht bei einer wohlgeformten Struktur enden und müsste die zweite Option mit \textit{den} als Artikel ausprobieren.
In der Praxis muss man das selten tun, weil die Intuition und die Kenntnis der Bedeutung des Satzes hilft, die richtigen Phrasengrenzen zu finden.}
Außerdem stehen \textit{den}, \textit{heißen} und \textit{Kaffee} alle im Akkusativ, so dass die NP-interne Kongruenzanforderung damit  auch erfüllt ist.
Man erhält also im nächsten Schritt Abbildung~\ref{fig:sya03}.%
\footnote{Einige Symbole, \zB Art, werden hier aus rein grafischen Gründen angehoben.
Die Struktur des Baumes verändert sich dadurch nicht.}

\begin{figure}
%  \resizebox{\textwidth}{!}{
    \Tree[1.25]{
      &&&& \K{NP}\B{dll}\B{dl}\B{dd} \\
      & \K{NP}\B{d} & \K{Art}\B{dd} & \K{AP}\B{d} && \K{AdvP}\B{d} \\
      \K{K}\B{d} & \K{\textbf{N}}\B{d} && \K{\textbf{A}}\B{d} & \K{\textbf{N}}\B{d} & \K{\textbf{Adv}}\B{d} & \K{V}\B{d} & \K{V}\B{d} \\
      \K{\textit{dass}} & \K{\textit{Frida}} & \K{\textit{den}} & \K{\textit{heißen}} & \K{\textit{Kaffee}} & \K{\textit{gerne}} & \K{\textit{trinken}} & \K{\textit{möchte}} \\
    }
%  }
  \caption{Konstruktion einer syntaktischen Analyse, Schritt 3}
  \label{fig:sya03}
\end{figure}

Bei der Suche nach valenzgebundenen Phrasen und ihren Köpfen gemäß Punkt \ref{it:sya03} aus Satz~\ref{satz:syntaxana} fällt als nächstes auf, dass mit \textit{trinken möchte} eine Folge aus einem infiniten Vollverb im ersten Status und einem finiten Modalverb vorliegt.
Da sonst keine ersten Status im Satz vorkommen, saturiert \textit{trinken} offensichtlich die Valenzanforderung des Modalverbs, und der Verbalkomplex aus beiden kann gebildet werden, vgl.\ Abbildung~\ref{fig:sya04}.

\begin{figure}
%  \resizebox{\textwidth}{!}{
    \Tree[1.25]{
      &&&& \K{NP}\B{dll}\B{dl}\B{dd} \\
      & \K{NP}\B{d} & \K{Art}\B{dd} & \K{AP}\B{d} && \K{AdvP}\B{d} && \K{\textbf{V}}\B{dl}\B{d} \\
      \K{K}\B{d} & \K{\textbf{N}}\B{d} && \K{\textbf{A}}\B{d} & \K{\textbf{N}}\B{d} & \K{\textbf{Adv}}\B{d} & \K{\textbf{V}}\B{d} & \K{\textbf{V}}\B{d}\\
      \K{\textit{dass}} & \K{\textit{Frida}} & \K{\textit{den}} & \K{\textit{heißen}} & \K{\textit{Kaffee}} & \K{\textit{gerne}} & \K{\textit{trinken}} & \K{\textit{möchte}} \\
    }
%  }
  \caption{Konstruktion einer syntaktischen Analyse, Schritt 4}
  \label{fig:sya04}
\end{figure}

Da \textit{trinken} ein transitives Verb ist und damit typischerweise eine NP im Nominativ und eine im Akkusativ regiert, ist die weitere Strukturbildung vorgezeichnet.
Solche NPs liegen in Form von \textit{Frida} (Nominativ) und \textit{den heißen Kaffee} (Akkusativ) vor.
Wir können daher die VP in Abbildung~\ref{fig:sya05} annehmen.

\begin{figure}
%  \resizebox{\textwidth}{!}{
    \Tree[1.25]{
      &&&&&&& \K{VP}\B{dllllll}\B{dlll}\B{dd} \\
      & \K{NP}\B{dd} &&& \K{NP}\B{dll}\B{dl}\B{dd} \\
      && \K{Art}\B{dd} & \K{AP}\B{d} && \K{AdvP}\B{d} && \K{\textbf{V}}\B{dl}\B{d} \\
      \K{K}\B{d} & \K{\textbf{N}}\B{d} && \K{\textbf{A}}\B{d} & \K{\textbf{N}}\B{d} & \K{\textbf{Adv}}\B{d} & \K{\textbf{V}}\B{d} & \K{\textbf{V}}\B{d}\\
      \K{\textit{dass}} & \K{\textit{Frida}} & \K{\textit{den}} & \K{\textit{heißen}} & \K{\textit{Kaffee}} & \K{\textit{gerne}} & \K{\textit{trinken}} & \K{\textit{möchte}} \\
    }
%  }
  \caption{Konstruktion einer syntaktischen Analyse, Schritt 5}
  \label{fig:sya05}
\end{figure}

Es bleiben nur noch der Komplementierer \textit{dass} und die AdvP \textit{gerne} unverbunden.
Gemäß Punkt \ref{it:sya04} können wir die AdvP als Angabe in die VP eingliedern.
Da gemäß Schema~\ref{str:subjgr} jede KP aus einem K und einer VP besteht, kann die Analyse abgeschlossen werden.
Der Baum in Abbildung~\ref{fig:sya06} ist fertig.
Dass er fertig ist, erkennt man daran, dass es einen einzigen Wurzelknoten gibt, nämlich den KP-Knoten, und dass alle anderen Knoten genau einen Mutterknoten haben.
Für jeden Knoten gibt es außerdem ein Strukturschema, das ihn beschreibt.

\begin{figure}
 % \resizebox{\textwidth}{!}{
    \Tree[1.25]{
      \K{KP}\B{dddd}\B{drrrrrrr} \\
      &&&&&&& \K{VP}\B{dllllll}\B{dlll}\B{ddll}\B{dd} \\
      & \K{NP}\B{dd} &&& \K{NP}\B{dll}\B{dl}\B{dd} \\
      && \K{Art}\B{dd} & \K{AP}\B{d} && \K{AdvP}\B{d} && \K{\textbf{V}}\B{dl}\B{d} \\
      \K{\textbf{K}}\B{d} & \K{\textbf{N}}\B{d} && \K{\textbf{A}}\B{d} & \K{\textbf{N}}\B{d} & \K{\textbf{Adv}}\B{d} & \K{\textbf{V}}\B{d} & \K{\textbf{V}}\B{d}\\
      \K{\textit{dass}} & \K{\textit{Frida}} & \K{\textit{den}} & \K{\textit{heißen}} & \K{\textit{Kaffee}} & \K{\textit{gerne}} & \K{\textit{trinken}} & \K{\textit{möchte}} \\
    }
%  }
  \caption{Konstruktion einer syntaktischen Analyse, Schritt 6}
  \label{fig:sya06}
\end{figure}

Es gibt in der Praxis sicher viele Fälle, in denen die Analyse nicht so eindeutig verläuft, und auch bei diesem Verfahren ist daher oft ein bisschen Ausprobieren vonnöten.
Neben einer guten Intuition ist das Verfahren zusammen mit einer strengen Beachtung der Strukturschemata aber die erfolgversprechendste Methode, Syntaxbäume zu konstruieren.

\Zusammenfassung

\begin{enumerate}\Lf
  \item Koordinationsstrukturen haben in unserer Analyse keinen Kopf, sondern sie verbinden lediglich zwei Einheiten der gleichen Kategorie (\zB NP und NP), wobei sich das Ergebnis wieder wie eine Einheit dieser Kategorie (also \zB NP) verhält.
  \item Die NP hat einen besonders komplexen Aufbau und wird links vom Artikelwort begrenzt, das genauso wie eventuell zwischen Artikel und Kopf stehende APs mit dem Kopf in Genus, Numerus und Kasus kongruiert.
  \item Statt eines Artikelworts kann auch ein pränominaler Genitiv stehen, der aber nur noch bei Eigennamen wirklich produktiv ist.
  \item Rechts vom NP-Kopf stehen Genitiv-NPs, PPs, Nebensätze, Relativsätze.
  \item Nominalisierungen von Verben behalten die Valenz des Verbs, wobei Nominative und Akkusative des Verbs zu Subjekts- und Objektsgenitiven des Substantivs werden.
  \item Ergänzungen in der attributiven AP stehen immer links vom Kopf.
  \item PPs sind ähnlich wie AdvPs strukturiert, nur dass Präpositionen immer eine obligatorische einstellige Valenz haben, Adverben aber niemals eine Valenz haben.
  \item Komplementierer bilden mit einer VP eine KP, wobei in der VP das finite Verb im Normalfall ganz rechts steht.
  \item Die Abfolge der Ergänzungen und Angaben innerhalb der VP ist nicht nur durch rein formale Prinzipien geregelt (Scrambling).
  \item Finite und infinite Verben, die in einer Rektionskette stehen (Statusrektion), bilden einen Verbalkomplex, der sich in gewisser Hinsicht wie ein einziges Verb verhält.
\end{enumerate}

\Uebungen

\begin{sloppypar}

\Uebung \label{u111} Analysieren Sie die eingeklammerten NPs und APs als Bäume.
Alle eingebetteten Phrasen können Sie mit Dreiecken abkürzen.%
\footnote{Siglen der Belege im DeReKo: NON09\slash OKT.10157, A00\slash SEP.61566, A00\slash OKT.71958, A00\slash OKT.71958, M09\slash JAN.03401, M09\slash JUL.58769}

\begin{enumerate}\Lf
  \item Nach [dem Führungstreffer durch Winkler] entschied der Schiedsrichter nach einem Tor der Felixdorfer völlig unverständlicherweise auf Abseits.
  \item Rimensberger kam zur [Überzeugung, dass dies ein Weisungstraum sei].
  \item Der Allradantrieb sorgt für exzellente, zuverlässige Fahreigenschaften auf allen Strassen und zwar vom [trockenen, glatten Asphalt] bis hin zu engen, gewundenen Pfaden mit steilen Kurven.
  \item Der Allradantrieb sorgt für exzellente, zuverlässige Fahreigenschaften auf allen Strassen und zwar vom [trockenen, glatten] Asphalt bis hin zu engen, gewundenen Pfaden mit steilen Kurven.
  \item Durch die dortige Bautätigkeit bestehe [Unsicherheit, ob entsprechender Platz zur Verfügung stehe].
  \item Doch schnell ist die Kaiserin auch hier [der Wiederholung überdrüssig].
\end{enumerate}

\Uebung \label{u112} Analysieren Sie die eingeklammerten PPs und AdvPs als Bäume.
Alle eingebetteten Phrasen können Sie mit Dreiecken abkürzen.%
\footnote{Siglen der Belege im DeReKo: WPD\slash SSS.00147, BRZ07\slash OKT.07777, NON09\slash AUG.06672, NUZ06\slash JAN.01852, NON09\slash OKT.10157}

\begin{enumerate}\Lf
  \item Außerdem dringen die Ausläufer des Pfälzer Waldes [weit in das Land] ein.
  \item Wir stehen sogar [sehr unter Druck].
  \item Daher glaube ich nicht, dass die Mannschaft [ganz vorne] zu finden sein wird.
  \item Gut [zwei Stunden nach dem Diebstahl] meldete sich der reuige Sünder bei der Polizei.
  \item Nach dem Führungstreffer durch Winkler entschied der Schiedsrichter nach einem Tor der Felixdorfer [völlig unverständlicherweise] auf Abseits.
\end{enumerate}

\Uebung[\tristar] \label{u113} Diskutieren Sie, was angesichts der bisherigen Analyse problematisch ist, wenn man Konstruktionen wie die eingeklammerte in Satz (\ref{ex:ex8123}) hinzunimmt (Sigle im DeReKo: NUN07\slash AUG.03639).

\begin{exe}
  \ex{\label{ex:ex8123} Philly reichte ihr [von unter dem Sitz] die Spucktüte.}
\end{exe}

\Uebung \label{u114} Analysieren Sie die eingeklammerten KPs als Bäume.
Die VPs analysieren Sie ebenfalls.
Alle in die VP eingebetteten Phrasen können Sie mit Dreiecken abkürzen, ebenso den Verbalkomplex.%
\footnote{Siglen der Belege im DeReKo: A00\slash JAN.05123, A09\slash AUG.03243, A99\slash DEZ.84493, A09\slash JUN.08933, RHZ06\slash APR.03188, A09\slash DEZ.00327}

\begin{enumerate}\Lf
  \item Es ist das erste Mal seit 1962, [dass ein griechischer Aussenminister zu einem offiziellen Besuch in der Türkei weilt].
  \item Und ich hoffe, [dass die schweren Fehler so nicht mehr passieren].
  \item Unklar sei, [ob ein Casino zum Gesamtkonzept passen würde].
  \item Die Kinder waren die grosse Konstante, [obwohl sich auch diese über die Jahrzehnte verändert haben].
  \item Eine teure Angelegenheit für die Verursacher, [falls sie ermittelt werden].
  \item Ja, haben denn Kleintierhalter 30 000 Franken oder mehr in der Porto-Kasse, nur [weil der Staat ihnen Wölfe und Bären schenkt]?
\end{enumerate}

\Uebung \label{u115} Analysieren Sie die eingeklammerten Verbalkomplexe als Baumdigramme.
Setzen Sie die Nummern der Rektionsfolge.
Bestimmen Sie den Status der einzelnen Verbformen.%
\footnote{Siglen der Belege im DeReKo: RHZ09\slash NOV.13071, RHZ07\slash AUG.05275, X99\slash JUL.25407, M09\slash DEZ.00596}

\begin{enumerate}\Lf
  \item Ich frage mich, welches Chaos bei uns ausbricht, wenn wir mit wirklichen Katastrophen [konfrontiert werden sollten].
  \item Wehen gegen Stuttgart, da steckte zwar eine ganz große Portion Aufregung drin, eine Stunde nach Spielende aber hätten sich die Gemüter doch so langsam [beruhigt haben müssen].
  \item{} [\ldots]für jene Person, die beim Zigeunerfest in Allhaming einer Frau 40 Rosen entwendete, die diese von einem Verehrer [geschenkt bekommen hatte].
  \item Eine Fachgruppe im Rathaus sucht nach Lösungen, wie das Grillen auf der Mannheimer Rheinwiese doch wieder [erlaubt werden kann].
\end{enumerate}

\end{sloppypar}
