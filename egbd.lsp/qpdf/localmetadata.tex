\title{Einführung in die grammatische Beschreibung des Deutschen}
%\subtitle{Zweite, überarbeitete Auflage}
\BackTitle{Einführung in die\newline grammatische Beschreibung des Deutschen}
\BackBody{\begin{sloppypar}\noindent
		\textit{Einführung in die grammatische Beschreibung des Deutschen} ist eine Einführung in die deskriptive Grammatik am Beispiel des gegenwärtigen Deutschen in den Bereichen Phonetik, Phonologie, Morphologie, Syntax und Graphematik.
		Das Buch ist für jeden geeignet, der sich für die Grammatik des Deutschen interessiert, vor allem aber für Studierende der Germanistik bzw.\ Deutschen Philologie.
		Im Vordergrund steht die Vermittlung grammatischer Erkenntnisprozesse und Argumentationsweisen auf Basis konkreten sprachlichen Materials.
		Es wird kein spezieller theoretischer Rahmen angenommen, aber nach der Lektüre sollten Leser in der Lage sein, sowohl deskriptiv ausgerichtete Forschungsartikel als auch theorienahe Einführungen lesen zu können.
		Trotz seiner Länge ist das Buch für den Unterricht in BA-Studiengängen geeignet, da grundlegende und fortgeschrittene Anteile getrennt werden und die fünf Teile des Buches auch einzeln verwendet werden können.
		Das Buch enthält zahlreiche Übungsaufgaben, die im Anhang gelöst werden.
		
		Die zweite Auflage ist vor allem auf Basis von Rückmeldungen aus Lehrveranstaltungen entstanden und enthält neben zahlreichen kleineren Korrekturen größere Überarbeitungen im Bereich der Phonologie, Wortbildung und Graphematik.
	\end{sloppypar}}
	\dedication{Für Mausi und so.}
	\typesetter{Roland Schäfer}
	\proofreader{Thea Dittrich}
	\author{Roland Schäfer}

\renewcommand{\lsISBNdigital}{978-3-944675-53-4}
\renewcommand{\lsISBNhardcover}{978-3-944675-82-4}
\renewcommand{\lsISBNsoftcover}{978-3-944675-69-5}
\renewcommand{\lsISBNsoftcoverus}{978-1-523743-37-7}
\renewcommand{\lsSeries}{tbls} % use lowercase acronym, e.g. sidl, eotms, tgdi
\renewcommand{\lsSeriesNumber}{2} %will be assigned when the book enters the proofreading stage
\renewcommand{\lsURL}{http://langsci-press.org/catalog/book/46} % contact the coordinator for the right number